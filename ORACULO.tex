\chapter[O Oráculo]{O Oráculo\subtitulo{Comédia em um ato}}
\hedramarkboth{O Oráculo}{Artur Azevedo}

\textit{Representada pela primeira vez no Rio de Janeiro, no Teatro
Recreio Dramático, pela Companhia Dias Braga,
em 2 de abril de 1870.}
%luis checar data

\hfill A Eduardo Vitorino

\hfill Que me fez escrever esta comediazinha

\castpage

\cast{Helena}{viúva}
\cast{Nélson}{advogado}
\cast{Frederico Pontes}{solteirão}
\cast{José}{criado de Nélson}

\vfil
A cena passa"-se na cidade do Rio de Janeiro.

Atualidade.

\pagebreak

\newactnamed{Ato único}


\stagedir{Sala e, ao mesmo tempo, consultório do doutor Nélson. Porta ao fundo.
Duas janelas à esquerda e duas portas à direita. Estantes de livros, consolos
etc. À direita, perto do primeiro plano, mesa carregada de livros, papéis, pena,
tímpano, tinteiro, uma caixa de charutos etc. Perto da mesa, quase ao centro,
uma poltrona.}

\newscenenamed{Cena I}
\stagedir{\textsc{José}, só.}

\repl{José} \paren{Ao levantar o pano, José está refestelado
na poltrona com um espanador na mão, a
saborear um charuto.}  Digam lá o que disserem: não há vida
melhor que a de um criado de um advogado rico e sem causas. Passo os
dias numa beatitude invejável, sem ter absolutamente o que fazer,
comendo e bebendo do melhor, e fumando magníficos charutos! O amo nunca
está em casa, e eu faço de conta que tudo isto é nosso. Permita Deus
que tão cedo não acabem os seus amores com a tal viúva das Laranjeiras.
Enquanto aquilo durar, durará também a minha beatitude. E por que não
há de durar? A viúva é bonita a valer, e não deve custar grandes
sacrifícios por ser senhora abonada. \paren{Sinal de dinheiro.} É
esquisito que não se casem\ldots{} ela, viúva\ldots{} ele, solteiro. Mas Deus me
livre de se lembrarem disso. Entrando uma mulher nesta casa, adeus
beatitude! \paren{Toque de campainha. José
levanta"-se.} Quem será? Algum cliente? Duvido! Seria o mesmo
que aparecer uma violeta em dezembro. \paren{Indo espiar pelo buraco
da fechadura da porta do fundo.} Mas não me engano, é ela, é a viúva
de Laranjeiras! Ora esta! É a primeira vez que aqui vem. Dar"-se"-á caso
que\ldots{} \paren{Novo toque de campainha.} Lá vou! Lá vou!
\paren{Abre a porta. Entra Helena elegantemente vestida. Toalete
clara.}

\newscenenamed{Cena II}
\stagedir{\textsc{José, Helena}}

\repl{José} \paren{Inclinando"-se diante de Helena.} Minha
senhora.

\repl{Helena}  Boa tarde. \paren{Procura alguém com os
olhos.}

\repl{José}  Ele não está em casa, minha senhora.

\repl{Helena}  Demora"-se?

\repl{José}  Não sei, porque não tem horas certas.

\repl{Helena} \paren{Encarando"-o.} Conhece"-me?

\repl{José}  Pois não, minha senhora. Mais de uma vez tive a honra de
ir à casa de Vossa Excelência, a mandado do s'or
doutor.

\repl{Helena}  Sim\ldots{} é verdade\ldots{}

\repl{José}  E, quando assim não fosse, bastava todos os dias ver o
retrato de Vossa Excelência à cabeceira do leito do
s'or doutor\ldots{} \paren{Apontando para a porta da
direita, primeiro plano.} Ali naquele quarto.

\repl{Helena}  O meu retrato?

\repl{José}  Está parecidíssimo. Só lhe falta falar.

\repl{Helena}  Ele saiu há muito tempo?

\repl{José}  Logo depois do almoço.

\repl{Helena}  Tem estado doente?

\repl{José}  Não, minha senhora; está de perfeita saúde.

\repl{Helena} \paren{Arrebatadamente.}  Então por que há quatro
dias não me aparece?

\repl{José}  Não sei, minha senhora.

\repl{Helena}  Está visto\ldots{} não pode saber\ldots{} não é da sua conta\ldots{} Mas como
estou nervosa e agitada!

\repl{José} \paren{Oferecendo"-lhe a poltrona.} Por que
não se senta, minha senhora? \paren{Helena senta"-se.} Vossa
Excelência quer que lhe vá buscar um copo d'água com
um pouco de açúcar e uma gota de flor de laranja?\footnote{
A água, ou infusão, de flor de laranjeira é utilizada como calmante e contra enxaqueca.}

\repl{Helena}  Para quê?

\repl{José}  Como Vossa Excelência disse que estava nervosa\ldots{}

\repl{Helena}  Pois sim, aceito. \paren{José inclina"-se e
sai. Helena ergue"-se e percorre a cena.} Não há que ver: está farto de
mim! Desfez"-se o encanto! Tudo acabou. Já o esperava: há muitos meses
noto a mudança do seu entusiasmo de outrora. Melhor seria que nos
houvéssemos casado. E dizer que fui eu que não quis! Dei"-me tão mal com
o casamento, que não me sorriu experimentá"-lo de novo. Era bem
independente para não me importar com o que dissessem.
\paren{Senta"-se e ergue"-se logo em seguida, cada vez mais agitada.}
Mas não! É impossível que Nélson seja ingrato. Há três anos
pertenço"-lhe, e nunca tive outro amor, nunca pensei noutro homem.
\paren{José volta trazendo um copo d'água
numa salva de prata, que apresenta a Helena. Ela bebe alguns goles\ldots{}}
Obrigada. \paren{José vai colocar a salva com o copo sobre
um consolo\ldots{}} Diga"-me, José. \paren{Ele aproxima"-se.} Chama"-se
José, não é assim?

\repl{José}  José Tralhota, para servir a Vossa Excelência.

\repl{Helena} Diga"-me. \paren{Arrependendo"-se.} Não, não me diga
nada! \paren{À parte\ldots{}} Que ia eu fazer? Um criado!

\repl{José}  Vossa Excelência pode confiar cegamente em mim. Há dois anos
estou a serviço do s'or doutor Nélson e ele aprecia
muito a minha discrição.

\repl{Helena}  Não; não seria correto interrogá"-lo. Não quero que o seu amo
possa acusar"-me da mais leve incorreção.

\repl{José}  Sou um simples criado de servir, mas\ldots{} possuo alguma
penetração.

\repl{Helena}  Que tenho eu com isso?

\repl{José}  Julgo ser agradável a Vossa Excelência afiançando"-lhe
que nada observei nesta casa que pudesse causar a Vossa Excelência a
menor inquietação.

\repl{Helena}  Bom.

\repl{José}  Entretanto, se Vossa Excelência quiser, observarei daqui
em diante ainda com mais cuidado, e comunicarei a Vossa Excelência.

\repl{Helena}  Cale"-se! Por quem me toma? Espiá"-lo? Nunca!
\paren{Toque de campainha; sobressaltada\ldots{}} Será ele?

\repl{José}  Não, minha senhora. O toque de campainha do
s'or doutor é mais enérgico, mais de dono da casa.

\repl{Helena}  Então algum cliente?

\repl{José}  Seria um fenômeno, mas\ldots{} quem sabe? Tudo acontece. Não
fizeram a avenida?\footnote{
Referência à abertura da avenida Central, atual Rio Branco, pelo prefeito Pereira Passos, em 1906, no Rio de Janeiro.} 
\paren{Indo ver pelo buraco da fechadura\ldots{}} Não,
senhora, não é um fenômeno. \paren{Descendo\ldots{}} É um cavalheiro do
meu conhecimento, que nunca vi cá em casa: o comendador Frederico
Pontes.

\repl{Helena}  Frederico Pontes? Não quero que me veja! É um velho
amigo de minha família.

\repl{José} \paren{Indo abrir a porta do quarto da direita\ldots{}}
Queira Vossa Excelência entrar para cá enquanto o despacho.

\repl{Helena} \paren{Hesitando\ldots{}} No quarto
dele?\ldots{}

\repl{José} \paren{Quase malicioso\ldots{}} Que tem isso? Vossa Excelência já lá
está em fotografia. O original não será de mais.

\repl{Helena} \paren{Ao entrar.} Se ele aparecer, não lhe diga
que estou no seu quarto.

\repl{José}  Sim, minha senhora.

\repl{Helena}  Quero causar"-lhe uma surpresa.

\repl{José}  E muito agradável. \paren{Helena sai.} Parece"-me que a água de
flor de laranja lhe fez bem. \paren{Novo toque de campainha.} Lá
vou! Lá vou! \paren{Vai abrir a porta do fundo.}

\newscenenamed{Cena III}
\stagedir{\textsc{José, Frederico}}

\repl{José} \paren{Inclinando"-se.}  Queira entrar,
s'or comendador Frederico Pontes. \paren{Entra
Frederico. Homem quase septuagenário, bem conservado e
elegante. Cabelos brancos. Monóculo. Polainas.
Veste um fato claro da última moda, um pouco
impróprio, talvez, da sua idade. Traz um pacote na
mão.}

\repl{Frederico}  Então você conhece"-me?

\repl{José}  Se o conheço! Olhe bem para mim, s'or
comendador: sou o José, o José Tralhota, que Vossa Excelência trouxe de
Lisboa.

\repl{Frederico} \paren{Assestando o monóculo.}  Ah!
sim\ldots{} o meu criado de quarto do Hotel Central. Eras tão esperto, tão
vivo, tão inteligente, que resolvi trazer"-te comigo quando saí de
Lisboa\ldots{} Chegando, porém, ao Rio de Janeiro, arrependi"-me, e pus"-te no
olho da rua. \paren{Senta"-se na poltrona.}

\repl{José}  Ainda estou por saber o motivo dessa desgraça.

\repl{Frederico}  Convenci"-me de que tinhas espírito demais para um
simples criado. Os Scapins\footnote{ Nome de um criado, na comédia \textit{Les Fourberies de Scapin} (1671), de Molière.} 
e Frontins\footnote{ Nome de um criado, na comédia \textit{Marton et Frontin} (1804), de Jean"-Baptiste Dubois.} 
só me agradam na \textit{Comédie}\footnote{ \textit{Comédie Française}: 
grande companhia de teatro francesa, criada em 1680 e subvencionada pelo governo.}
ou no Odéon.\footnote{
Teatro francês inaugurado em 1782.}
Fora dali acho"-os detestáveis. Entretanto, ao saíres de
minha casa, poderias aspirar a coisa melhor\ldots{} Por que não te
arranjaste no comércio?

\repl{José}  Não sou ambicioso\ldots{} Agrada"-me esta situação\ldots{} considero"-me
colocado melhor que o meu amo.

\repl{Frederico}  És filósofo\ldots{} e mandrião.

\repl{José}  Mais mandrião que filósofo.

\repl{Frederico}  Estás então ao serviço do doutor Nélson?

\repl{José}  Sim, senhor, e afianço"-lhe que o doutor Nélson está satisfeito.

\repl{Frederico}  Se ele fosse tão espirituoso como tu, não te
poderia aturar.

\repl{José}  Nem eu o aturaria.

\repl{Frederico}  Ele fuma charutos tão bons como os que eu fumava?

\repl{José}  Os charutos que ele fuma não se comparam com os de Vossa
Excelência. Os de Vossa Excelência eram baianos; os dele são de Havana.

\repl{Frederico}  Tanto melhor para ti. Eu gosto dos meus, e não
quero de outros. \paren{Mostrando o pacote.} Ainda agora aqui trago
provisão para um mês. \paren{Erguendo"-se.} Vai pôr isto sobre um
móvel qualquer. \paren{José coloca o pacote sobre um
consolo.} Pelo que vejo, teu amo não está em casa?

\repl{José}  Não, senhor.

\repl{Frederico}  Se é bem criado, não deve tardar. Escreveu"-me,
pedindo"-me que desse um pulo até cá quando viesse à cidade, porque
desejava fazer"-me uma consulta.

\repl{José}  Logo vi que Vossa Excelência vinha para ser consultado.
Para consultar ainda está para ser o primeiro que aqui venha.

\repl{Frederico}  Respondi"-lhe dizendo que hoje às duas horas o
procuraria. \paren{Consultando o relógio.} Já são duas e
cinco.

\newscenenamed{Cena IV}
\stagedir{\textsc{Os mesmos, Nélson}, depois \textsc{Helena}, escondida.}

\repl{Nélson} \paren{Entrando do fundo.}  O seu relógio
está cinco minutos adiantado, comendador. O meu está certo pelo balão.

\repl{Helena} \paren{Entreabrindo a porta, à
parte.}  É a sua voz! É ele!\ldots{} 

\repl{Frederico}  Mais minutos, menos minutos não quer dizer nada.
\paren{Depois de apertar a mão de Nélson.} Estou
ao seu dispor.

\repl{Nélson} \paren{A José.} Vá lá dentro.
\paren{José sai pela direita, olhando para a porta do quarto
onde Helena está escondida; leva a salva e o copo.}
Desculpe"-me tê"-lo incomodado, mas o senhor mora tão longe, na Gávea\ldots{} 
para lá ir é preciso perder um dia inteiro\ldots{} por isso pedi"-lhe que
quando viesse à cidade\ldots{}

\repl{Frederico}  Fez muito bem, não tem de que se desculpar. Sou um
solteirão ocioso. Vivo dos rendimentos que escaparam à minha mocidade
tempestuosa, e tornei"-me um contemplativo, sem outra ocupação que não
seja fumar e ler Balzac.

\repl{Nélson} \paren{Oferecendo"-lhe uma cadeira perto da
mesa.} É o seu
autor favorito?  

\repl{Frederico}  O favorito não, o único: Balzac é suficiente para a
existência de um leitor. Na sua obra estão compendiados, não só toda a
sociedade moderna, como todo o gênero humano. Tenho relido aqueles cem
volumes não sei quantas vezes. Sempre que chego ao último, sinto
saudades do primeiro, e atiro"-me a ele com curiosidade e sofreguidão.
Bastaram a Balzac vinte anos para escrever tudo aquilo; aos simples
mortais como nós, meu caro Nélson, são necessários cinquenta para ler
aquilo tudo. Mas vamos lá, que deseja de mim? \paren{Sentam"-se,
devendo Nélson ficar o mais perto possível de Helena, que continua
escondida.}

\repl{Nélson}  Eu sei que o comendador é um dos brasileiros
que mais têm viajado\ldots{} Sei que, na sua mocidade, que o senhor é o
primeiro a classificar de tempestuosa, teve um número considerável de
aventuras galantes, e é tido como oráculo em questões de amor.
Sei também que muitos rapazes inexperientes recorreram aos seus
conselhos, e tais e tão discretos foram estes, que eles alcançaram tudo
quanto pretendiam. Pois bem; fiado na velha amizade que o ligou a meu
pai e na bondade com que sempre me tratou, quero também eu consultá"-lo
sobre um caso melindroso.

\repl{Frederico}  Um caso de amor?

\repl{Nélson}  Sim, um caso de amor.

\repl{Frederico}  Exagerou quem lhe disse que sou oráculo. Alguma
experiência, isso tenho, porque toda a minha vida rescende a
\textit{odor di femina.} As mulheres me custaram muito para que não me
deixassem, pelo menos, o orgulho e a consolação de as ficar conhecendo.
Entretanto, não foram elas, foi esse grande psicólogo, Balzac, quem fez
de mim, em questão de amor, não um oráculo, mas um conselheiro modesto,
embora avisado. Exponha"-me o seu caso.

\repl{Nélson}  Mas de antemão perdoe a maçada.

\repl{Frederico}  Não é maçada. Estes assuntos, para mim, têm mais
interesse que a navegação aérea e a telegrafia sem fios.

\repl{Nélson}  Então um charutinho, para me ouvir com mais
paciência. \paren{Oferece"-lhe a caixa de charutos.}

\repl{Frederico} \paren{Tirando um charuto.}  Aceito, mesmo porque sei que
só fuma havanos.

\repl{Nélson}  Sabe?

\repl{Frederico}  Pelo seu criado.

\repl{Nélson}  Ah! \paren{Acendem os charutos e fumam.}

\repl{Frederico}  Vamos lá.

\repl{Nélson}  Há três anos sou o amante de uma senhora
viúva, distinta, bem educada. Quero acabar com essa ligação. Que devo
fazer?

\repl{Helena} \paren{À parte.} Oh!

\repl{Frederico}  É a primeira vez que sou consultado neste sentido.
Ordinariamente recorrem à minha experiência os que desejam, não acabar,
mas principiar. É indispensável, antes de mais nada, conhecer o motivo
que o desgostou. Tem ciúmes dela?

\repl{Nélson}  Ciúmes? Oh! se a conhecesse! É um modelo de
meiguice, fidelidade e constância.

\repl{Frederico}  Existe alguma particularidade que o afaste desse
modelo?\ldots{} Quero dizer: alguma enfermidade\ldots{} Algum defeito físico\ldots{}
por exemplo: o mau hálito?

\repl{Nélson}  Por amor de Deus! É uma mulher sadia, limpa,
cheirosa!

\repl{Frederico}  Então é feia?

\repl{Nélson}  Feia? Uma das caras mais bonitas do Rio de
Janeiro!

\repl{Frederico}  Tem mau gênio?

\repl{Nélson}  Uma pombinha sem fel.

\repl{Frederico}  Então é tola, vaidosa, presumida, afetada,
asneirona?\ldots{}

\repl{Nélson} \paren{Interrompendo"-o.}  Nada
disso. É uma mulher de espírito e, como já lhe disse, perfeitamente
educada.

\repl{Frederico}  É devota? Anda metida nas igrejas? Passa
horas esquecidas a rezar diante de um oratório?

\repl{Nélson}  Apenas vai ouvir missas aos domingos.

\repl{Frederico}  Talvez abuse do piano, ou cante
desafinado\ldots{}

\repl{Nélson}  Não canta. Toca piano, mas não abusa.
Digo"-lhe mais: é uma boa intérprete de Chopin.

\repl{Frederico}  O senhor gosta de outra mulher?

\repl{Nélson}  Juro"-lhe que não.

\repl{Frederico}  Bom. Já sei o que é. O meu amigo
aborreceu"-se dela, porque não lhe descobriu defeitos. É boa demais.

\repl{Nélson}  Quem sabe?

\repl{Helena} \paren{À parte.} Oh!

\repl{Nélson}  O caso é que esta ligação já durou mais tempo
do que devia. Urge acabar com ela. A viúva tem uma filhinha que ainda
está na idade em que se olha sem ver, mas a menina cresce a olhos
vistos, e é conveniente fazer com que mais tarde não obrigue a mãe a
corar.

\repl{Frederico}  Isso agora é um pouco de hipocrisia. Que
lhe importaria a filha se o senhor gostasse deveras da mãe? O amor não
conhece escrúpulos nem conveniências.

\repl{Nélson}  Demais, sou moço\ldots{} tenho um grande horizonte
diante de mim\ldots{} enceto agora a minha carreira de advogado\ldots{} Esta
ligação pode prejudicar seriamente o meu futuro.

\repl{Frederico}  Vá por aí. O que o inquieta é o seu
futuro, e não o da menina. Mas diga"-me: tem certeza, certeza absoluta
de que essa mulher possui todas as perfeições?

\repl{Nélson}  Se não é a mais perfeita, é a menos
imperfeita que ainda conheci.

\repl{Frederico}  Cuidado, meu amigo! Muitas vezes tem a gente certeza de uma
coisa, e a coisa é outra, muito diversa. Por exemplo: este charuto, que
o senhor pagou como sendo de Havana, é um rio"-grandense que não troco
pelo pior dos meus baianos. \paren{Levanta"-se e vai atirar o charuto
pela janela.}

\repl{Nélson} \paren{Erguendo"-se.}  Pois olhe, paguei"-o
bem caro.

\repl{Frederico}  E as mulheres enganam mais facilmente que os
charutos.

\repl{Nélson}  Afirmo"-lhe que a mulher de quem se trata é
excepcional.

\repl{Frederico}  E o senhor quer se ver livre dela?

\repl{Nélson}  Quero!

\repl{Frederico}  E a sua resolução é inabalável? 

\repl{Nélson}  Inabalável.

\repl{Frederico}  Que esquisitice! Enfim, só há um meio de conseguir o que
deseja\ldots{} um meio violento, mas único.

\repl{Nélson}  Qual?

\repl{Frederico}  Suma"-se! Desapareça!

\repl{Nélson}  Ela irá procurar"-me onde quer que eu vá.

\repl{Frederico}  Boa dúvida; mas faça"-se invisível, meta"-se no mato e volte
ao cabo de oito dias. Naturalmente ela aparece e pergunta em termos
ásperos, ou sentidos, o motivo do seu procedimento. Muna"-se então de um
pouco de coragem, e responda o seguinte: ``à vista de um
fato que chegou ao meu conhecimento, nada mais pode haver de comum
entre nós. Não me peça explicações: meta a mão na consciência, e meça a
extensão do meu ressentimento.''

\repl{Nélson}  E se ela aparecer antes que eu desapareça? Há
quatro dias não a procuro. Espero que de um momento para outro surja
por aí. Admira"-me até que ainda não tivesse vindo.

\repl{Frederico}  Ela não lhe escreveu?

\repl{Nélson}  Não há nada neste mundo que a obrigue a
escrever uma carta, nem mesmo um simples bilhete ao amante. É um
sistema que adotou e ao qual não cede, haja o que houver.

\repl{Frederico} Decididamente essa mulher é uma fênix. Eu,
no seu caso, metia"-a numa redoma.

\repl{Nélson}  Mas diga"-me\ldots{} se ela aparecer?

\repl{Frederico}  Atire"-lhe a tal frase: ``À
vista de um fato\ldots{}''

\repl{Nélson} \paren{Interrompendo"-o.}  Mas que fato?
Pois não lhe disse já que ela é um modelo de fidelidade?

\repl{Frederico} \paren{Sorrindo.} Meu jovem
amigo, devo parecer"-lhe implacável para com o belo sexo; mas creia: não
há mulher, por mais virtuosa, por mais amante, que não tenha alguma
coisa de que a acuse a consciência. A sua bela viúva, em que pese às
aparências, não deve, não pode escapar à lei comum. Desde que o senhor
se refira positivamente, categoricamente, a um fato, embora não declare
que fato seja, ela ficará persuadida de que o seu amante veio ao
conhecimento de alguma coisa que se passou, e que a pobrezinha julgava
encoberta no véu de impenetrável mistério.

\repl{Nélson}  Mas quando mesmo ela tenha algum pecadilho na
consciência (juro"-lhe que o não tem), com certeza protestará
energicamente e exigirá que eu ponha os pontos nos \textit{ii}; há de
querer que eu declare a que fato aludo e\ldots{} Vamos e venhamos! Como
acusá"-la sem consentir que ela se defenda?

\repl{Frederico} Ah! meu doutor! se pretende aplicar razões
jurídicas ao caso, está bem arranjado! A jurisprudência do amor é
absurda. Acuse, retire"-se e não entre em explicações. Afianço"-lhe que o
êxito é seguro, tanto mais, perdoe"-me este pequenino ataque ao seu
amor\ldots{}  tanto mais que receio seja ela tão inocente como os seus
charutos são de Havana. \paren{Indo buscar o chapéu
e a bengala.} E com esta, adeus! Siga o meu conselho e dê"-me
notícias suas. \paren{Estende a mão.}

\repl{Nélson} \paren{Apertando"-lha.}  Adeus, comendador,
e muito obrigado. Vou acompanhá"-lo até a escada.

\repl{Frederico}  Por quem é, não se incomode!

\repl{Nélson}  Ora essa é boa! \paren{Saem ambos pela
porta do fundo.}

\repl{Helena} \paren{Vindo à cena.} 
Agora nós!\ldots{} é preciso que ele não me veja\ldots{} quero mostrar a estes
senhores que eu também li a \textit{Comédia Humana}. \paren{Esconde"-se
atrás de uma das portas do fundo.}

\repl{Nélson} \paren{No corredor.} Adeus,
comendador, e ainda uma vez obrigado! \paren{Volta sem ver Helena, e
esta sai rapidamente pela porta do fundo.}

\newscenenamed{Cena V}
\stagedir{\textsc{Nélson}, depois \textsc{José}}

\repl{Nélson}  ``À vista de um fato que chegou
ao meu conhecimento, nada mais pode haver de comum entre nós! Não me
peça explicações. Meta a mão na consciência e meça a extensão do meu
ressentimento!'' Assim, sozinho, sem ela diante de mim, é
fácil; mas dizer coisas destas a uma senhora de quem não se suspeita\ldots{} 
Mas, se realmente\ldots{} Qual! Pode lá ser! Decididamente há de faltar"-me o
ânimo. \paren{Com uma ideia.} Se eu lhe escrevesse? O
efeito seria o mesmo. \paren{Senta"-se à mesa, dispondo"-se para
escrever e toca um tímpano. Molha a pena, prepara o papel etc.
Entra José.} Ninguém me procurou enquanto estive fora?

\repl{José} \paren{Depois de lançar uma olhadela
à porta do quarto.} Ninguém.

\repl{Nélson}  Feche aquela porta. \paren{Aponta para o
fundo.}

\repl{José} \paren{Depois de fechar a porta, reparando no
pacote que o Comendador deixou ficar.} Oh! o
s'or comendador deixou ficar aqui os charutos!

\repl{Nélson}  Como sabe que são charutos?

\repl{José}  Ele disse"-me.

\repl{Nélson}  Conhecem"-se?

\repl{José}  Pois se foi ele quem me trouxe de Lisboa.

\repl{Nélson}  É um bom tipo.

\repl{José}  Magnífico.

\repl{Nélson}  E atirado às mulheres, hein?

\repl{José}  Faziam dele gato"-sapato.

\repl{Nélson}  Deveras?

\repl{José}  E foi uma delas que o fez comendador.

\repl{Nélson}  Como assim?

\repl{José}  Foi a condição que impôs aos seus favores. Parece"-me estar ainda
a ouvi"-la: {``meu Frederiquinho, enquanto não fores
comendador, não serei tua!''. Dai a quinze dias ele tinha a
comenda de Cristo.

\repl{Nélson}  Bom. Basta de dar à língua. Veja se o apanha
no largo da Carioca. Provavelmente foi tomar o bonde da Gávea. Esses
charutos devem fazer"-lhe falta.

\repl{José}  É já. \paren{Vai abrir a porta do fundo.}

\repl{Nélson}  Por aí não. Vá pela porta da sala de
jantar\ldots{} \paren{José sai pela direita, segundo plano.}

\newscenenamed{Cena VI}
\stagedir{\textsc{Nélson}, depois \textsc{Helena}}

\repl{Nélson} \paren{Tomando a pena e escrevendo.}
%Jorge: checar essas aspas!
``Minha senhora, à vista de um fato\ldots{}''
\paren{Toque de campainha.} Deve ser o comendador, que vem buscar
os charutos\ldots{} E eu lhos mandei levar! \paren{Levanta"-se e vai abrir
a porta. Entra Helena.} Helena!

\repl{Helena} \paren{Com ímpeto.} Meu Nélson, meu amor, que quer
isto dizer? Há quatro dias não me apareces! É a primeira vez em três
anos que a tua ausência foi tão prolongada!\ldots{} Dize\ldots{} que tens tu?\ldots{} 
Que te fiz eu?\ldots{} por que me recebes com tanta frieza?\ldots{} que se
passou?\ldots{} disseram"-te mal de mim?\ldots{} fui vítima de alguma intriga?\ldots{} 
Já me não amas? Dize! \paren{Pausa.} Este silêncio\ldots{} \paren{Com
um grito.} Ah! Tudo adivinho! amas outra!\ldots{}

\repl{Nélson} \paren{Com um grande esforço.} À
vista de um fato que chegou ao meu conhecimento, nada mais pode haver
de comum entre nós.

\repl{Helena}  Que fato?

\repl{Nélson}  Não me peça explicações.

\repl{Helena}  Tenho, me parece, o direito não de pedi"-las,
mas de exigi"-las.

\repl{Nélson}  Meta a mão na consciência, e meça a extensão
do meu ressentimento. \paren{Afasta"-se.}

\repl{Helena}  Estou perdida! O miserável não guardou
segredo! \paren{Cai sentada numa cadeira e cobre o rosto com as
mãos.}

\repl{Nélson} \paren{Com um sobressalto.} O
miserável?! Que miserável?!

\repl{Helena}  Bem sabes quem é, pois vejo que nada
ignoras. \paren{Erguendo"-se.} Tens razão, Nélson: nada mais pode
haver de comum entre nós. Aprecio e respeito a delicadeza dos teus
sentimentos. \paren{Dirige"-se para a porta do fundo.}

\repl{Nélson}  Ouve, Helena!

\repl{Helena}  Nada mais quero ouvir. Peço"-te, como um
último favor, que me não insultes. Eu estava na doce persuasão de que
tudo ignorarias, de que jamais virias ao conhecimento de uma fraqueza
que tão desgraçada me faz, porque cava um abismo entre nós. Vejo que o
infame foi indiscreto e fez chegar aos teus ouvidos a notícia de uma
vergonhosa aventura a que fui arrastada num momento de desvario, e da
qual me arrependi amargamente. Que fatalidade! \paren{Finge que chora
e soluça.} Oh! Eu devia ter adivinhado que tudo sabias!\ldots{} A
tua ausência foi significativa, e eu, louca, na suposição estúpida de
que poderia esconder a minha ignomínia! \paren{Com um
soluço.} Adeus!

\repl{Nélson}  Mas vem cá\ldots{} quero saber\ldots{}

\repl{Helena}  Saber o que, se tudo sabes? Que resultaria de
qualquer explicação entre os dois? O teu perdão?\ldots{} Oh! não! não me
perdoes, Nélson, porque o teu perdão deporia contra o teu caráter de
homem de bem! \paren{Com outro soluço.} Adeus!
\paren{Encaminha"-se para a porta.}

\repl{Nélson} \paren{Tomando"-lhe a passagem.}  Já te
disse que quero saber.

\repl{Helena}  Se alguma coisa queres saber que não saibas,
sabe que foi a tua frieza, o teu desprendimento, o pouco caso com que
afinal começaste a tratar"-me, que me determinaram a dar o mau passo que
dei, e que tantas lágrimas me vai custar. Tu nunca me compreendeste\ldots{} 
nunca estimaste o incomparável tesouro que havia aqui. \paren{Bate no
peito.}

\repl{Nélson} \paren{Enfurecido.}  Então era certo?
Pertenceste a outro homem?

\repl{Helena} \paren{Com doçura.}  Se já tão
fria, tão tranquilamente mo disseste, por que o repetes agora com tanta
veemência? Não fiquemos irritados um com o outro\ldots{} separemo"-nos como
dois bons amigos\ldots{} com um aperto de mão. \paren{Enquanto lhe aperta
a mão.} Adeus! Lembra"-te sempre da infeliz Helena, que te ama
ainda como sempre te amou, mas não procures nunca mais tornar a vê"-la:
não é digna de ti. \paren{Aproximando"-se mais de Nélson, sem lhe
largar a mão.} Se algum dia te recordares, com pena, da nossa
ventura passada, console"-te a certeza de que a minha vida vai ser de
agora em diante um inferno de remorsos e saudades. Adeus para sempre!

\repl{Nélson} \paren{Enlaçando"-a.}  Não! Não sairás daqui
sem me dizer o nome desse homem!

\repl{Helena} \paren{Tranquilamente.}  Pois se o sabes\ldots{}

\repl{Nélson} \paren{Furioso.}  Não sei! Queria
experimentar"-te\ldots{} e não imaginava\ldots{}

\repl{Helena} \paren{Fugindo"-lhe dos braços.}  Experimentar"-me! Não
compreendo! Se de nada sabias, como e por que me lançaste em rosto a
minha culpa? E culpa foi? Pergunto agora. Tens acaso mais direito sobre
mim que qualquer outro homem? Não sou eu livre como os pássaros? Não
recusei a mão de esposo que me ofereceste? Sabes tu se nesse homem
encontrei mais solicitude, mais carinho, mais amor do que em ti? Quem é
aqui o credor? Que me deste em troca de quanto te dei? Por ti
segreguei"-me da sociedade, sacrifiquei o futuro de minha filha,
enterrei a minha mocidade, porque imaginei que o teu amor compensasse
tudo isso! Qual foi a compensação? Esse ardil infame de inventar um
homem! Pois bem, Nélson, esse homem existe e nunca saberás quem é!
Adeus!

\repl{Nélson} \paren{Agarrando"-a.} Helena!
Helena! Dize"-me o nome do teu amante!

\repl{Helena}  Cala"-te! Não desças mais!

\repl{Nélson} \paren{Frenético e apaixonado.}
Desço! Desço! Quero descer, descer muito, contanto que te
encontre lá embaixo!\ldots{} Faze de mim o juízo que quiseres\ldots{} despreza"-me
como ao mais abjeto dos homens\ldots{} Mas essa terrível confissão fez com
que o meu amor extinto despertasse mais violento, mais impetuoso que
nunca!

\repl{Helena} \paren{Tentando desvencilhar"-se dos
braços de Nélson.} Deixa"-me! Deixa"-me.

\repl{Nélson}  Ao meu amor faltou isto --- o ciúme! Eu amo"-te!
Amo"-te mais do que te amei, porque nunca me pareceste mais bela, nunca
me seduziste assim!

\repl{Helena}  Não! Deixa"-me! Não sou digna de ti!

\repl{Nélson}  Cala"-te, meu amor, minha amante, minha doce
Helena! Perdoo"-te! Amo"-te! Adoro"-te!

\repl{Helena}  Se realmente me amas, se me adoras, então és
tu que não és digno de mim! \paren{Desprende"-se dos braços
dele e corre para a porta do fundo.}

\repl{Nélson} \paren{Indo buscá"-la.}  Vem cá\ldots{} 
ouve\ldots{} Não sou eu que te perdoo\ldots{} És tu que me perdoas a mim, porque
tens razão: o indigno sou eu. \paren{Helena finge que chora.} Não
chores\ldots{} Senta"-te aqui\ldots{} ao pé de mim\ldots{} e conversemos
tranquilamente. \paren{Fá"-la sentar"-se na poltrona e
senta"-se numa cadeira.}

\repl{Helena} \paren{Enxugando as lágrimas fingidas.} 
Nada disto sucederia se nos tivéssemos casado.

\repl{Nélson}  Tu não quiseste\ldots{}

\repl{Helena}  Se eu fosse tua mulher não te enganaria\ldots{}

\repl{Nélson}  Ainda estás em tempo de o ser.

\repl{Helena}  Oh! Nélson!

\repl{Nélson}  Amo"-te! Amas"-me! Que nos importa o resto?

\repl{Helena}  Não, tu não me podes amar como outrora\ldots{}

\repl{Nélson}  Amo"-te com mais paixão, com mais fogo!
\paren{Enche"-a de beijos; entra José e cobre os
olhos com as mãos.}

\newscenenamed{Cena VII}
\stagedir{\textsc{Nélson, Helena, José}, que logo sai.}

\repl{José}  Ah!

\repl{Nélson e Helena}  Ah!

\repl{Nélson} \paren{Erguendo"-se.} Que é? Tire
a mão dos olhos!

\repl{José}  Não encontrei o comendador no largo da Carioca. Voltei
com os charutos.

\repl{Nélson}  Pois guarde"-os lá dentro. Logo à tardinha irá
levá"-los à Gávea.

\repl{José} \paren{À parte.}  Um passeio à Gávea! Oh! beatitude!\ldots{} 
\paren{Sai pela direita, segundo plano.
Nélson volta a sentar"-se onde estava, ao lado de
Helena.}

\repl{Helena}  Queres então que eu seja tua mulher?

\repl{Nélson}  Esse é o único meio de sermos felizes; essa é
a maior prova de amor que podemos dar um ao outro.

\repl{Helena}  Imponho apenas uma condição.

\repl{Nélson}  Dize.

\repl{Helena}  Jamais e sob pretexto algum me pedirás
explicações sobre o passado\ldots{} nenhum nome procurarás saber\ldots{}

\repl{Nélson}  Persistes então em me ocultar\ldots{}

\repl{Helena} \paren{Erguendo"-se.} Persisto.

\repl{Nélson} \paren{Erguendo"-se.} Seja!

\newscenenamed{Cena VIII}
\stagedir{\textsc{Nélson, Helena, Frederico}}

\repl{Frederico} \paren{Entrando.}  Com licença. Deixei
ficar aqui os meus charutos. \paren{Vendo Helena, surpreso.} Oh! a
senhora Dona Helena aqui!

\repl{Nélson}  Conhecem"-se?

\repl{Helena}  Há muitos anos\ldots{} O senhor comendador foi
muito amigo de meu pai.

\repl{Nélson}  E também do meu. Que coincidência.

\repl{Frederico}  Coincidência, por quê?

\repl{Nélson}  Porque somos noivos.

\repl{Frederico}  Noivos?

\repl{Helena}  Acabamos de ajustar o nosso casamento. 

\repl{Frederico}  Parabéns, muitos
parabéns\ldots{} Mas os meus charutos? Tenho um bonde daqui a meia hora.

\repl{Nélson}  Vou buscá"-los. Estão com o criado.
\paren{Sai pela direita, segundo plano.}

\newscenenamed{Cena IX}
\stagedir{\textsc{Helena}, \textsc{Frederico}, depois \textsc{Nélson} e \textsc{José}}

\repl{Helena}  Aí está em que deram os seus conselhos,
senhor oráculo!

\repl{Frederico}  Os meus conselhos?

\repl{Helena}  Eu sou a fênix, a mulher ideal, de quem ele se queria ver
livre, e ouvi tudo dali, onde estava escondida. Creia, não obstante a
sua implacabilidade para com as pobres mulheres, que nunca tive outro
amante\ldots{} Mas disse"-lhe o contrário\ldots{} Confessei"-lhe uma culpa que não
tinha, porque só assim poderia reconquistá"-lo.

\repl{Frederico}  Mas agora que o casamento está tratado, é
preciso dissuadir o pobre rapaz.

\repl{Helena}  Mais tarde, ou talvez nunca. Esse homem, que
ele não sabe quem é\ldots{} essa aventura misteriosa\ldots{} essa ignóbil mentira
é a garantia da minha felicidade. Enquanto ele supuser que não fui dele
só, será só meu.

\repl{Frederico}  Que mulher! Aquele idiota não a merece!

\repl{Helena}  Merece\ldots{} Hei de prová"-lo. Tenho a minha ideia.

\repl{Frederico} \paren{À parte.} 
Hum!

\repl{Nélson} \paren{Voltando com o pacote e acompanhado por José.}
 Comendador, aqui tem seus baianos.

\repl{Frederico}  Obrigado. \paren{Apertando a
mão a Nélson.} Meu amigo, renovo os meus parabéns,
e, uma vez que se vai casar, recomendo"-lhe que leia a \textit{Fisiologia do
casamento}.

\repl{Helena}  De Balzac?

\repl{Frederico}  De Balzac, sim. É uma fantasia licenciosa,
mas genial, que corre mundo desde 1829, minha senhora\ldots{} 
\paren{Aperta a mão a Helena.}

\repl{José} \paren{A parte.} Ele casa"-se!\ldots{} Adeus, beatitude!\ldots{}

\bigskip

\begin{center}
\textsc{Cai o pano}
\end{center}



