\chapter[O Cordão]{O CORDÃO\subtitulo{Burleta em um ato e cinco quadros}}
\hedramarkboth{O Cordão}{Artur Azevedo}

\textit{Representada pela primeira vez no Teatro
Carlos Gomes, do Rio de Janeiro, em fevereiro de 1908}

\castpage


\cast{Alfredo}{namorado de Florinda}

\cast{Gastão}{namorado de Rosa}

\cast{Remígio}{praça reformado do Exército}

\cast{Florinda}{filha de Remígio}

\cast{Rosa}{irmã de Florinda}

\cast{Salustiano}{presidente do Cordão}

\cast{Emerenciana}{esposa de Salustiano}

\cast{Zeca}{filho de Salustiano e Emerenciana}

\cast{Cazuza}{malandro}

\cast{Zé}{português, tocador de bumbo}

\cast{Joaquina}{esposa de Zé}

\cast{Gaudêncio}{tocador de harmônica}

\cast{Conselheiro}{chefe de repartição dos
dois primeiros}

\vfill
\pagebreak

\newactnamed{Ato único} 
\newscenenamed{\textit{Quadro 1}}

\stagedir{A cena representa uma
travessa em Catumbi. À direita baixa, uma casa
modesta com porta e janela; ao fundo, um muro e, ao centro da cena, uma
amendoeira tendo por baixo um banco.}

\newscenenamed{Cena I}
\stagedir{\textsc{Alfredo} e \textsc{Gastão}}

\repl{Alfredo} \paren{Entrando com Gastão.} Aí tens! É naquela casinha que ela mora.

\repl{Gastão}  Salve Dinorá, casta e pura!

\repl{Alfredo} Casta e pura, dizes muito bem.

\repl{Gastão} Mas a aparência é medíocre.
Apesar de estarem os aluguéis\ldots{} de casas pela hora
da morte, aquilo é casa para uns cinquenta mil réis.

\repl{Alfredo} Já te disse que
Florinda é paupérrima e de condição
humilde. É filha de um velho soldado
reformado, hoje servente ou coisa que o valha, numa repartição
qualquer.

\repl{Gastão} Vê lá em que te vais meter!

\repl{Alfredo} As minhas intenções são as mais puras.

\repl{Gastão} Tira o cavalo da chuva. Não é,
decerto, para te casares com ela que a namoras?

\repl{Alfredo} Por que não, se é honesta?

\repl{Gastão} Mas pertence a uma sociedade que não é a tua.

\repl{Alfredo} E a minha sociedade qual é?

\repl{Gastão} Mas\ldots

\repl{Alfredo} Deixa"-te
disso. No Brasil não há hierarquias sociais; somos
todos da mesma massa.

\repl{Gastão} Essa rapariga não tem,
não pode ter a tua educação.

\repl{Alfredo} Enganas"-te, meu
amigo, não é mal"-educada. O pai é analfabeto. Ela
porém e a irmã\ldots

\repl{Gastão} Ah! Tem uma irmã?\ldots

\repl{Alfredo} Uma irmã
que também é bonita. Ambas sabem ler e são muito
prendadas, porque um velho general muito bom e de quem o pai
foi ordenança se incumbiu de educá"-las.

\repl{Gastão} E que fim levou
esse velho general?

\repl{Alfredo} Morreu como todos
os velhos generais. Morreu e deixou uns cobres às
raparigas. Ah, se ainda vivesse, elas não estariam
em casa do pai, onde vivem sozinhas porque perderam a
mãe há muito tempo! Vê que perigo!

\repl{Gastão} E onde fizeste esse
conhecimento?

\repl{Alfredo} Pois ainda
não te disse? Num cinematógrafo.

\repl{Gastão} Num cinematógrafo?

\repl{Alfredo} No
cinematógrafo Pathé. Bem sabes que não há
nada que se preste a um namoro como um
cinematógrafo! Encontramo"-nos primeiramente na sala
de espera. Aí começou uma correspondência
ativa entre os nossos olhos. Parece que fomos fulminados
simultaneamente, pelo mesmo raio vibrado do amor!

\repl{Gastão} Bonito! Mas
não confunda Cupido com Júpiter!

\repl{Alfredo} Cala"-te e ouve. Ao
atravessarmos aquela portinha estreita, ficamos apertadinhos um contra
o outro. Na sala encontrei meio de sentar"-me a seu lado. O bonito
é que naquela balbúrdia ela separou"-se do pai e da
irmã. Quando a sala ficou escura, as nossas
mãos encontraram"-se. Estavam mortas para isso\ldots{} E os
nossos corações entenderam"-se\ldots

\repl{Gastão} Estás completamente perdido!

\repl{Alfredo} Ela, a pobrezinha,
tremia que nem a fita do cinematógrafo\ldots{} uma fita
muito comprida\ldots{} tão comprida que tivemos tempo de
contar um ao outro a nossa vida. Ela disse"-me onde morava,
perguntei"-lhe se poderia falar aqui; respondeu"-me que sim, à
hora em que o velho saísse para a repartição.
E aqui estou. E como somos velhos amigos e moramos juntos,
pedi"-te que me acompanhasses para o que desse e viesse.

\repl{Gastão} Isto é,
para apanhar contigo a provável sova de pau!
Cá estou! Os amigos conhecem"-se nas
ocasiões. Mas vê lá que o morarmos
na mesma casa não é razão para apanharmos juntos.

\repl{Alfredo} Descansa que
não haverá sangue! \paren{Vê o relógio.}
São nove horas, o pai já saiu ou está por
sair. Ah, meu Gastão\ldots{} isto é amor.
\paren{Canta.}

{\smallskip\raggedleft\itshape Romança\par}
\begin{verse}
\hspace{17mm} I\\
\smallskip
Não imaginas, meu amigo,\\
Que amor violento eu sinto aqui!\\
Já pregar olho não consigo, \\
Inda esta noite não dormi! \\
E quando, pela madrugada, \\
Ardendo em febre, adormeci, \\
A minha bela namorada \\
Aparecer"-me em sonhos vi!

\hspace{17mm} II\\
\smallskip
Tal qual fizera no bondoso \\
Cinematógrafo Pathé, \\
Tomei"-lhe a mão, e um beijo ansioso\\
Lhe dei gostoso como quê\ldots{}\\
A minha amada, com meiguice, \\
A soluçar não sei por que, \\
Medrosa e trêmula me disse: \\
-- Eu gosto tanto de você!
\end{verse}



\repl{Gastão} Só? 

\repl{Alfredo} Só!

\repl{Gastão} Ainda bem. Para sonho, as
consequências não foram terríveis.  

\repl{Alfredo} \paren{Sério.} 
Mesmo em sonhos eu seria incapaz de lhe faltar ao respeito.

\repl{Gastão} Pois sim!

\repl{Alfredo} \paren{Olhando para a porta.} Olha, abre"-se a porta. É
talvez o velho que vai sair.

\repl{Gastão} Cuidado com o lombo!

\repl{Alfredo} Bico!
Escondamo"-nos atrás desta amendoeira.
\paren{Escondem"-se atrás da árvore.}

\newscenenamed{Cena II}
\stagedir{\textsc{Os mesmos, Remígio} \textit{e depois} \textsc{Florinda} e \textsc{Rosa.} \textsc{Remígio} 
\textit{sai de casa, fecha a porta com chave e guarda no bolso.}}


\repl{Remígio} \paren{Que ia saindo, voltando.} Não se esqueçam de ver se a galinha
já tirou os pintos!

\repl{As duas} Sim, papai!

\repl{Remígio} Meninas, não se esqueçam
de limpar a minha espingarda! A lixa está por trás
do alguidar grande, na despensa.

\repl{Rosa e Florinda} \paren{Dentro.} Sim, papai.

\repl{Remígio} Aprontem o jantar mais cedo. Olhem que hoje há ensaio.

\repl{Florinda} \paren{Dentro.} Ainda não se sabe. 

\repl{Rosa} \paren{Dentro.} Disse que o delegado ia proibir. 

\repl{Florinda} Se houver, seu Salustiano virá prevenir a gente. 

\repl{Remígio} Pois se seu
Salustiano vier dizer que há ensaio, aprontem o
jantar mais cedo. \paren{Vai sair e volta.}

\repl{Rosa e Florinda} Sim, papai!

\repl{Remígio}  E janelas fechadas, hein? Não abram, venha quem vier. Falem com
seu Salustiano mesmo de dentro. 

\repl{Rosa e Florinda} Sim, papai.

\repl{Remígio} \paren{À parte.} Da porta não tenho medo, porque
levo a chave comigo. \paren{Alto.} Até logo. \paren{Sai.}

\repl{Rosa e Florinda} Sim, papai.

\newscenenamed{Cena III}


\stagedir{\textsc{Os mesmos}, \textit{menos} \textsc{Remígio}.}


\repl{Florinda} \paren{Abre a janela e aparecem os dois.} Janelas fechadas?
Não vê! E o meu Alfredo?

\repl{Alfredo} \paren{A Gastão.} O seu Alfredo! Oh! que
delícia! 

\repl{Rosa} Ah! Florinda!
Você não pensa noutra coisa!

\repl{Florinda} Pudera!

\repl{Gastão} \paren{À parte, a Alfredo.}
A irmã é mais bonita!

\repl{Alfredo} \paren{Idem.}
São opiniões!

\repl{Florinda} Ele deve estar
perto\ldots{} Vamos até à esquina espiar?

\repl{Rosa} Papai pode voltar.

\repl{Florinda} Não volta, não.
\paren{Fecham a janela e vêm à cena pela porta, com muita cautela. Alfredo e Gastão saem do esconderijo, as duas
assustam"-se.}

\repl{Florinda e Rosa} \paren{Com medo.} Ah!		\EP[]

{\smallskip\raggedleft\itshape Quartetino\par}
\begin{verse}

\fala{Alfredo} \paren{Canta.}\\
Não tenhas medo!\\
É o teu Alfredo\\
Que te aparece!\\
Trouxe comigo\\
Fiel amigo!\\
Para o que desse\\
E viesse


\fala{Alfredo}\\
Não tenhas medo! etc.


\fala{Gastão}\\
Não tenhas medo! etc.


\fala{Rosa}\\
Não tenhas medo! etc.


\fala{Alfredo}\\
Por detrás da amendoeira\\
Ardendo como uma brasa,\\
Esperei uma hora inteira\\
Saísse o velho de casa.

\fala{Florinda} \paren{Mostrando Gastão.}\\
Este moço é moço sério?

\fala{Alfredo}\\
É de toda confiança.

\fala{Gastão}\\
Confie no meu critério.

\fala{Rosa}\\
Tenho medo à vizinhança\ldots

\fala{Alfredo}\\
Ninguém passa a esta hora\\
Nesta lôbrega travessa;\\
Nem sequer um cão agora\\
Este deserto atravessa;

Mas se julgas que na rua\\
Não 'stamos em segurança,\\
Entremos na casa tua!\\
Lá não vai a vizinhança!

\fala{As duas}\\
Isso não!\\ 
Isso não!


\fala{Todos}\\
Aquela porta não passarão!\\
É o teu Alfredo etc.\\
É o meu  Alfredo etc.\\
É o seu Alfredo etc.
\end{verse}

\repl{Alfredo} \paren{Puxa Florinda para a extrema esquerda, e Gastão e Rosa sentam"-se no banco.}
Então como tens passado, desde aquela noite feliz?

\repl{Florinda} \paren{Baixa os olhos.} Com muitas saudades\ldots

\repl{Alfredo} De quem?

\repl{Florinda} De quem há de ser?!

\repl{Alfredo} Gostas muito de
mim?

\repl{Florinda} Se não gostasse, não estaria aqui. Mas peço"-lhe,
Alfredo, que não abuse da minha
condescendência e trate quanto antes de ser meu
marido, porque sou digna de você, juro"-lhe por essa
luz que nos alumia!

\repl{Alfredo} Que beleza!

\repl{Florinda} Mas olhe: minha
irmã irá para a nossa companhia! É
preciso que saiamos ambas daqui\ldots{} desta casa\ldots{} deste
bairro\ldots{} não posso sozinha com papai, que\ldots{} que\ldots{} 
\paren{Chora.}

\repl{Alfredo} Então que é isso? Choras, meu amor?

\repl{Florinda} Papai é
bom, muito bom, coitado, mas não mede o
alcance de sua responsabilidade e julga que fechar a porta e
levar a chave no bolso é quanto basta!

\repl{Alfredo} É verdade! Como
conseguiste sair, se ele está com a chave?

\repl{Florinda} As chaves
são duas e papai supõe que
é uma só\ldots{} \paren{Mudando de tom.} Mas
vamos ao que importa: o nosso casamento?

\repl{Alfredo} O nosso casamento
é todo meu desejo\ldots{} 
mas os meus recursos são limitados\ldots{} Já te
disse que não passo de um simples amanuense de
secretaria\ldots{} Não te posso dar luxo\ldots

\repl{Florinda} Luxo?! Basta que
me dê o que tenho em casa de meu pai. E, demais, eu
trabalho, e coso perfeitamente, posso ajudá"-lo\ldots{} E já
lhe disse que dindinho, o general, nos deixou alguma coisa a
mim e à minha irmã\ldots{} Mas eu
preciso sair daqui\ldots{} Imagine que papai nos leva a um cordão
carnavalesco!
%luis é preciso checar se dindinho é tratamento, como padrinho, ou apelido, que pede caixa alta

\repl{Alfredo} Que encanto eu
acho na tua voz! Como me agradam os teus modos!

\repl{Florinda} Sou como as
outras. Você é que me vê e ouve com
olhos de namorado\ldots{} Também eu gosto tanto de
você!

\repl{Alfredo} Ah, meu Deus! A
mesma frase do sonho.

\repl{Florinda} Que sonho?

\repl{Alfredo} Um sonho que eu
tive. Sonhei\ldots{} que me dizias isso mesmo.

\repl{Florinda} Também eu sonhei
com você\ldots{} Você de casaca preta e eu de vestido
branco! Uma grinalda de flores de laranjeira na cabeça\ldots{} 
subíamos os degraus de um altar muito bonito\ldots{} com muitas
velas\ldots{} muitas flores\ldots{} um padre muito velhinho à
nossa espera lá em cima. \paren{A Rosinha.} Não foi, Rosinha?
\paren{Rosa, que conversa com Gastão, não
ouve.} Rosinha\ldots

\repl{Rosa} \paren{Como despertando.}  Que é?

\repl{Florinda} Eu não te disse que sonhei que estava vestida de noiva?

\repl{Rosa} E seu Alfredo de
casaca, foi mesmo. E eu creio que esta noite vou também
sonhar com o outro\ldots

\repl{Gastão} \paren{A Alfredo.} Ah!
meu amigo! Que ingenuidade! Que candura! Que rapariga deliciosa! E olha
que é muito mais bonita do que a irmã.

\repl{Alfredo} São opiniões.

\repl{Gastão} Estou apaixonado,
sabes?

\repl{Alfredo} Apaixonado, já?

\repl{Gastão} E então, tu?!

\repl{Alfredo} Sim, mas eu, ao
menos fui ao cinematógrafo. A eletricidade influiu
talvez para a rapidez das minhas impressões. Pois,
ainda bem, os quatro nos entenderemos perfeitamente. Estou
resolvido a casar.

\repl{Gastão} Também eu. Já lho
disse, \paren{alto} não foi Rosinha?

\repl{Rosa} O quê?

\repl{Gastão} Não lhe disse que
desejava ser seu marido?

\repl{Rosa} Disse, mas eu
não sei se foi de bobagem.  

\repl{Gastão} Que
bobagem que nada. Sou teu. És minha. Acabou"-se.

\repl{Florinda} Xi! Lá
vem seu Salustiano!

\repl{Alfredo e Gastão} Quem é?

\repl{Florinda} O presidente dos Foliões do Itapiru.

\repl{Alfredo} Ah, o tal cordão?

\repl{Florinda} Vem prevenir a
gente de que hoje há ensaio. Adeus, adeus.

\repl{Alfredo} Assim, sem um
beijo?

\repl{Florinda} Não, os beijos
ficam para os maridos. Os namorados que se contentem
com um bom aperto de mão. \paren{Saem para casa e fecham a porta.}

\repl{Alfredo} Que diabo de homem
será esse presidente dos Foliões do Itapiru?

\repl{Gastão} Vamos para
trás daquela amendoeira e observemos.

\repl{Alfredo} Bem lembrado.
\paren{Escondem"-se.}

%\pagebreak 
\newscenenamed{Cena IV}


\stagedir{\textsc{Os mesmos} e \textsc{Salustiano}. Salustiano, pernóstico, pardavasco, grande carapinha, pretensa elegância, procurando os termos e sibilando \textit{ss}. Entra e vai bater à porta.}


\repl{Alfredo} A figura
é de um verdadeiro cafajeste.

\repl{Gastão} Cala a boca.

\repl{Florinda} e {Rosinha} \paren{Dentro.} 
Quem bate?

\repl{Salustiano} Eu, gentis
donzelas! Salustiano Barradas\ldots{} Vosso ilustre
progenitor está em casa?

\repl{Rosa} Não, senhor.
Papai saiu.

\repl{Florinda} Não abrimos a janela, porque ele
não quer.

\repl{Salustiano} E faz muito
bem. As donzelas são melindrosas flores, que devem
ser guardadas com o mais espontâneo recato. Tenham a
bondade, senhoritas, de dizer ao vosso honrado progenitor, quando ele
voltar do labor cotidiano e sintomático, que hoje
há ensaio.

\repl{Rosa e Florinda} Sim,
senhor.

\repl{Salustiano} A autoridade
policial desta circunscrição permitiu, contanto que o
ensaio não passe das onze horas.

\repl{Rosa e Florinda} Sim,
senhor.

\repl{Salustiano} Até logo, gentilíssimas
flores.

\repl{As vozes} Adeus, seu Salustiano.

\repl{Alfredo} \paren{A Gastão.} Vou falar àquele tipo.

\repl{Gastão} Para quê?

\repl{Alfredo} Tenho uma
ideia. \paren{Aproximando"-se de Salustiano
e cumprimentando"-o com muita
cortesia.} Oh! meu caro senhor Salustiano Barradas.

\repl{Salustiano} Cavalheiro.
Não tenho a honra de lhe
conhecer.

\repl{Alfredo} Isto não quer
dizer nada. Basta que eu o conheça. E quem o
não conhece?

\repl{Salustiano} Favores\ldots{} s\ldots{} s\ldots

\repl{Alfredo} Sei que estou em
presença do ilustre presidente do grupo de
Foliões do Itapiru

\repl{Salustiano} Efetivamente\ldots{}  
Oh, um modesto grupo destinado a arrastar as almas sensíveis
na elucubração dos folguedos carnavalescos\ldots{} s\ldots{} s\ldots

\repl{Alfredo} Pois eu e aqui o amigo\ldots{} 
dá licença que lho apresente? É o
senhor\ldots{} senhor\ldots{} Vasconcelos\ldots

\repl{Salustiano} Senhor Vasconcelos\ldots{} Tem um
humilde servo preponderante às suas ordens\ldots

\repl{Alfredo} O amigo e eu somos
doidos pelo carnaval\ldots{} mas o verdadeiramente popular, o carnaval bem
entendido, o carnaval de cordão.

\repl{Salustiano} Percebo, chefe.
\paren{Faz uns passes de velho, quando dançam.}

\repl{Alfredo e Gastão} Isso! Isso!

{\smallskip\raggedleft\itshape Coplas\par}
\begin{verse}
%I

\fala{salustiano}\\
Deixem lá falar quem fala,\\
Pois o melhor carnaval\\
Não é carnaval de sala\\
Nem da Avenida Central.\\
O verdadeiro carioca\\
Nascido nesse torrão\\
Por nenhum carnaval troca


\fala{Os três} \paren{Dançando.}\\
O do cordão!\\
O do cordão!\\
Dão, dão, dão, dão!

%II

\fala{Salustiano}\\
Ninguém nestes belos dias\\
Se diverte como nós,\\
Que odiamos as sombrias\\
Figuras dos dominós!\\
Carnaval que, como vinho,\\
Torna alegre o coração,\\
É o carnaval do povinho,


\fala{Os três} \paren{Dançando.}\\
O do cordão!\\
O do cordão!\\
Dão, dão, dão, dão
\end{verse}

\repl{Alfredo} \paren{Fala.} 
Oh! esse carnaval\ldots{} Só esse nos
entusiasma e alvoroça!

\repl{Salustiano} Vejo que falo
com pessoal escovado. Mas não percebo o motivo desta
intervenção jurídica e jubilosa para o meu eu.

\repl{Alfredo} Teríamos muito
prazer em assistir a um ensaio dos Foliões
do Itapiru.

\repl{Gastão} Aí está!

\repl{Salustiano} Com mil
vontades, meu chefe!\ldots{} Não me furto a esse
regozijamento\ldots{} Lá estamos todas as noites na sede
da nossa exuberante sociedade, isto é, na nossa
humilde choupana\ldots{} naquela rua\ldots{} avenida\ldots{} Ah!~eu também
moro numa avenida! Avenida Xavier, número
17, casa F, porta 11.

\repl{Alfredo} Pois hoje mesmo lá
iremos.

\repl{Salustiano} Mas com
franqueza: Vossas Senhorias não são do nosso pessoal.
Vão encontrar naquela vivenda certas
incoerências fatais\ldots{} uns em chinelos, outros em
trajes menores; vulgo\ldots{} mangas de camisas\ldots{} e ouvir palavras pouco
amenas e abstratas!

\repl{Alfredo} Já me lembrei
disso, não por nós, mas pelo pessoal. Como desejamos
que estejam todos à vontade, iremos
disfarçados. E o senhor nos apresentará
como velhos amigos do outro bairro, que os vão
visitar.

\repl{Gastão} O que nós queremos é
apreciar o ensaio.

\repl{Salustiano} Estamos
entendidos\ldots{} Direi que o chefe é quitandeiro e aqui
o senhor\ldots

\repl{Gastão} \paren{Distraído.} 
Barbosa\ldots

\repl{Salustiano} Perdão,
Vasconcelos\ldots{} s\ldots

\repl{Alfredo} \paren{Emendando.}
Pois é, Barbosa de
Vasconcelos.

\repl{Salustiano} Vossa Senhoria,
será\ldots{} será\ldots

\repl{Gastão} Vendedor de fósforos baratos.

\repl{Salustiano} Irá com vestes
sacerdotais?

\repl{Gastão} Não, senhor, a secular\ldots

\repl{Salustiano} Secular,
gosto!\ldots{} O moleque é sarado! \paren{Despede"-se.}
Creiam, cidadães conspícuos,
que têm neste seu criado um servo
preponderante, pródigo e observante de todas as
pragmáticas ultra"-sociais\ldots{} Avenida Xavier\ldots

\repl{Alfredo} Dezessete!

\repl{Gastão} Casa F.

\repl{Salustiano} Porta 11. \paren{Sai
cantando.}

\newscenenamed{Cena V}

\stagedir{\textsc{Alfredo} e \textsc{Gastão}}


\repl{Gastão} 
Mas que linguagem tão esquisita! Dez homens
assim são capazes de reformar a língua portuguesa.

\repl{Alfredo} No teatro,
pareceria um exagero. Entretanto, o tipo existe, é comum.
\paren{Olhando para a janela.} 
Ah! Florinda, agora compreendo as tuas lágrimas.

\repl{Gastão}  É
preciso arrancá"-las a esta sociedade.

\repl{Alfredo} E fazer delas as
companheiras das nossas vidas. Vamos arranjar os nossos disfarces.

\repl{Gastão} 
Mas que ideia a tua!

\repl{Alfredo} Estamos na
época de carnaval. Portanto, vamos.

\repl{Gastão} Vamos para
a casa. Adeus, Rosinha!\ldots

\repl{Alfredo} Adeus, Florinda. Coitadinhas! Devem
estar limpando a espingarda do pai. \paren{Saem atirando
beijos à casa. Mutação.}


\newscenenamed{\textit{Quadro 2}}

\stagedir{Em casa de Salustiano. Sala pobre característica. Ao fundo, 
sobre uma velha cômoda, o
estandarte do Grupo Carnavalesco Foliões 
do Itapiru. Bancos de madeira etc.}


\newscenenamed{Cena I}


\stagedir{\textsc{Salustiano} \textit{e depois} \textsc{Emerenciana}
Salustiano entra, e deixando perceber
que está um tanto bebido.}

\repl{Salustiano} Emerenciana.
Emerenciana\ldots

\repl{Emerenciana} \paren{Entra do interior.} 
Que é, seu Salustiano?

\repl{Salustiano} \paren{Sem mover"-se.} 
Hoje o ensaio é aqui na sala.

\repl{Emerenciana} Ora! A sala é tão pequena!

\repl{Salustiano} Não faz mal.

\repl{Emerenciana} O Zeca
já tinha levado as \textit{cadeira}
e a mesinha pro \textit{quintá}.

\repl{Salustiano} Espero dois amigos. Não
quero que eles suponham que o Grupo Carnavalesco
Foliões do Itapiru não
tenha um teto.

\repl{Emerenciana} Quem
são esses amigos?

\repl{Salustiano} Você não conhece eles.

\repl{Emerenciana} \paren{Chama.} Zeca!

\repl{Zeca} \paren{Dentro.} 
Senhora\ldots{} \paren{Percebe"-se que o menino responde
de má vontade.}

\repl{Emerenciana} Torna a
\textit{trazê} pra dentro a mesinha e as
\textit{cadeira}, que tu levou pro
\textit{quintá}. \paren{A Salustiano.} Se
você esperava dois \textit{amigo},
pra que foi \textit{bebê}?

\repl{Salustiano} \paren{No mesmo lugar.}
Percebe"-se no meu semblante que eu bebi?

%Jorge: Tirei itálicos abaixo.
\repl{Emerenciana} No seu
semblante, não sei. Mas quando você põe o
chapéu pra \textit{trais}, pisca muito os
\textit{olho} e fica parado no mesmo lugar com medo
de dar um passo e cambalear\ldots{} já se sabe que esteve
na venda\ldots{} Você quando tem visita não
devia \textit{bebê}!

\repl{Salustiano} Por
quê?

\repl{Emerenciana} Acaba sempre dando bordoada
nelas! Dando ou levando\ldots

\repl{Salustiano} Hoje não haverá
motivo para essa demonstração típica.
\paren{Fala para dentro.} Oh,
menino\ldots{} não ouviste\ldots{} traz as
\textit{cadeira}\ldots

\repl{Emerenciana} Você é muito
bom \textit{home}, seu Salustiano, é um
\textit{home} inteligente que até
fala \textit{dificel}, mas tem um
defeito\ldots

\repl{Salustiano} Que defeito?!
\paren{Emerenciana faz gesto de beber.}
Minha esposa, beber não envergonha ninguém.

\repl{Emerenciana} Que
não envergonha, que nada! E quando vai preso?

\repl{Salustiano} Minha esposa:
as crianças bebem, as virgens donzelas em casa de
seus extremosos pais bebem! Todos bebem; eu é que
não hei de beber?

\repl{Emerenciana} Beba
água!

\repl{Salustiano} Na culta Europa
ninguém bebe água. O rio Sena não
presta pra nada. Ali só se bebe vinho,
cerveja, água de \textit{Seltes} e
água de Caxambu.\footnote{ 
Marcas de água mineral vendidas em garrafa.}

\newscenenamed{Cena II}

\stagedir{\textsc{Os mesmos} e \textsc{Zeca}}


\repl{Zeca} \paren{Vem do interior arrastando umas
cadeiras de pau. Doze anos, descalço e
esfarrapado.} Se a senhora
não queria as \textit{cadeira no
quintá}, por que me \textit{mandô
levá elas?} \paren{Coloca as cadeiras ao fundo
à esquerda.}

\repl{Emerenciana} Bota aí as
\textit{cadeira} e cala a tua boca, diabo. Foi teu
pai que mandou.

\repl{Salustiano} \paren{A Zeca, que vai saindo.} 
Então não tomas a bênção ao autor dos teus
dias e das tuas noites?

\repl{Zeca} \paren{Com arremesso.} 
Ah! \paren{Sai.}

\repl{Emerenciana} Este menino
é a minha desgraça!

\repl{Salustiano} Deixe. Isso
é falta de \textit{instruquição} primária e
secundária. Vai \textit{fazê} doze anos e devia já
estar na escola.

\repl{Emerenciana} Devia. Mas
\textit{quedê} roupa? \textit{quedê}
sapato? Você não se importa\ldots{} 
\paren{Vendo Zeca, que entra arrastando mais duas
cadeiras.} Veja como está sujo e
esfarrapado!

\repl{Zeca} Isto parece casa de
malucos!

\repl{Emerenciana} Vai tomar a
bênção a teu pai.

\repl{Zeca} Não vou\ldots{} Não vou!
\paren{Coloca as cadeiras.}

\repl{Emerenciana} É assim que tu
\textit{arrespondes} à tua mãe?
Coisa ruim\ldots

\repl{Zeca} É.

\repl{Emerenciana} \paren{Correndo
atrás de Zeca, que foge pela frente
de Salustiano.} Eu te mostro, coisa ruim,
desavergonhado.

\repl{Zeca} \paren{Continua a correr,
Emerenciana persegue"-o.} Não mostra
nada.

\repl{Emerenciana} Patife! Bêbado!

\repl{Zeca} Se eu sou
bêbado, foi meu pai que me ensinou a
\textit{bebê}. \paren{Foge pela direita baixa.}

\repl{Emerenciana} Está ouvindo,
seu Salustiano?

\repl{Salustiano} Estou. Aquilo
é o fruto sazonado de uma educação inóspita
e desbragada! Hei de \textit{corrigi ele}.

\repl{Emerenciana} Isso diz
você há muito tempo. O pequeno acabará
feito vagabundo.

\repl{Zeca} \paren{Entra com uma mesinha
de pinho, muito suja.} Com esta vida que
eu levo aqui, não posso \textit{dá} senão
pra ladrão.

\repl{Salustiano} Deixe"-se de
impertinências esdrúxulas e vá buscar um
castiçal com uma vela acesa, porque estamos quase no
escuro!

\repl{Zeca} \paren{Com maus modos.} 
Ah! \paren{Sai.}

\repl{Emerenciana} Esse capeta ainda há de
ser a causa da minha morte\ldots{} Mais eu \textit{amostro}
a  ele o que é uma pernambucana\ldots{} \paren{Ouvem"-se palmas.}

\repl{Salustiano} São talvez os
amigos que espero. \paren{Grita para dentro.}
Menino\ldots{} Olha esse
castiçal!\ldots{} \paren{Vai à porta, faz entrar Alfredo e Gastão, que
vêm disfarçados. Zeca entra com uma
vela metida no gargalo de uma garrafa e coloca"-a sobre a mesa.}

\newscenenamed{Cena III}
\stagedir{\textsc{Os mesmos, Alfredo} e \textsc{Gastão}}


\repl{Salustiano} Entrai, meus nobres amigos\ldots{} 
\paren{Apresentando"-os a Emerenciana.} 
O senhor\ldots

\repl{Alfredo} Pereira\ldots

\repl{Salustiano} O senhor Pereira, meu velho
camarada, negociante matriculado, estabelecido com comércio
de quitanda, aves domésticas e louça do
país. \paren{A Zeca.} Vá buscar outro castiçal,
que ainda está escuro.

\repl{Zeca} \paren{Com maus modos.}
Ah! \paren{Sai.}

\repl{Salustiano} \paren{Continuando.}
O senhor\ldots

\repl{Gastão} Abdul Abdala!\ldots{} 
Sou turco.

\repl{Salustiano} Vendedor
ambulante de fósforos \textit{barato}. \textit{Surge et
ambula}\ldots{} footnote{
``Levanta"-te e anda'', em latim. Frase tirada do Evangelho de São João.}

\repl{Gastão} \textit{Fosfe barata}. \paren{Zeca entra com
garrafa e vela, coloca"-as sobre a mesa.}

\repl{Salustiano} Dona
Emerenciana Barradas\ldots{} Minha legítima esposa.

\repl{Alfredo e Gastão} Minha senhora\ldots

\repl{Emerenciana}  \paren{Com mesura.}
Seu Pereira\ldots{} Seu como se chama\ldots{} \textit{sejem} bem
aparecidos nesta sua casa\ldots

\repl{Salustiano} \paren{A Zeca.}
Menino, vai buscar outro
castiçal, mas sem vela! \paren{Zeca sai
com arremesso.} Com licença.
\paren{Toma uma vela e vai iluminar o estandarte.}
Admirem!

\repl{Alfredo e Gastão} \paren{Admirando.} 
Oh!\ldots

\repl{Salustiano} O nosso
estandarte\ldots{} \paren{Lendo.} Grupo
Carnavalesco. Foliões do Itapiru. A primeira ideia foi
filhos, e não foliões. Filhos do Itapiru. Mas filhos
prestava"-se a anfibologia fantástica e preliminar!
Ficou foliões! Que tal acham a pintura?

\repl{Alfredo e Gastão} Soberba! Magnífica!

\repl{Alfredo} Um Ticiano.

\repl{Gastão} Um Rafael!

\repl{Salustiano} O pintor é um
gênio. Infelizmente só pinta casas, e lá uma vez por outra
alguma tabuleta adventícia e tópica. \paren{Põe
a vela sobre a mesa.} Oh! As artes, neste
país, não têm o sufrágio das iminências
plásticas!

\repl{Alfredo} Isso não
têm. \paren{Baixo a
Gastão.} Percebeste?

\repl{Gastão} \paren{Baixo a Alfredo.}
Ele está que não se pode lamber.

\repl{Emerenciana} \paren{Vai buscar as cadeiras e
coloca"-as junto à mesa.}
Façam o favor de se abancar.

\repl{Alfredo e Gastão} Obrigado. \paren{Sentam"-se.}

\repl{Alfredo} Viemos um pouco cedo\ldots

\repl{Salustiano} Sim, os ensaios
nunca principiam antes das sete e meia, porque os nossos companheiros
têm ocupações variadas e problemáticas\ldots{} 
\paren{Toma de Zeca, que entra, uma garrafa.} 
Se os meus nobres amigos querem provar desta abrideira
simbólica e providencial\ldots{} \paren{A
Zeca.} \textit{Quedê} os copos?

\repl{Alfredo e Gastão} Não se incomode, depois\ldots

\repl{Emerenciana} Vai buscar os
copos, burro.

\repl{Zeca} Sim, minha
mãe. \paren{Sai e volta com os copos.}

\repl{Salustiano} Desculpai essa
\textit{ofragiologia}\ldots{} bem lhes preveni que ouviriam
palavras abstratas e incoerentes. \paren{Deita
cachaça em um copo.}
Então não vai uma gota virtuosa?

\repl{Alfredo e Gastão} Agora não. Depois\ldots

\repl{Salustiano} Então, à nossa!
\paren{Bebe.}

\repl{Emerenciana} Não beba mais,
Salustiano.

\newscenenamed{Cena IV}

\stagedir{\textsc{Os mesmos} e \textsc{Cazuza}}


\repl{Cazuza} \paren{Entra esbaforido,
como que perseguido por alguém. Gastão
e Alfredo assustam"-se.} 
Aqui estou seguro\ldots{} Que sangangu de maçadas!
Que sangangu onça, seu Salustiano!
\paren{Vendo Alfredo e Gastão,
interroga"-o com o olhar.}

\repl{Salustiano} São de paz.

\repl{Cazuza} Foi no botequim de
seu Chico. O Espanta quebrou uma cadeira na sinagoga
do trouxa. Tudo por causa daquela mulata do outro dia.

\repl{Salustiano} \textit{Cherchê la fama}.\footnote{
Deturpação do francês \textit{cherchez la femme}, ``procurai a mulher'', com
o sentido de que, achada esta, será descoberta a razão da desavença.}
E foi preso? \paren{A Gastão e
Alfredo.} O Espanta, de quem se trata,
é o tesoureiro dos Foliões do Itapiru.

\repl{Alfredo} Ah!

\repl{Cazuza} Foi preso.

\repl{Salustiano} Lastimável incidente!

\repl{Cazuza} Não houve tempo de
fugir da canoa. O Miudinho, o Pan"-americano e eu
\textit{arresistimo}. Eu dei um banho de
fumaça num praça de polícia, que coriscou rente na
alegria do tombo; mas veio o reforço e o
Pan"-americano foi pegado.

\repl{Salustiano} Que pena, o nosso
porta"-estandarte!

\repl{Cazuza} O Miudinho abriu o chambre pelo Nheco
acima e eu abri esta menina. \paren{Abre uma navalha e faz
passos de capoeiragem.} E brinquei uns
cinco minutos assim. Depois meti a cabeça num praça.
\paren{Quer meter a cabeça
em Alfredo, que foge.}

\repl{Alfredo} Pra lá!

\repl{Cazuza} Passei uma rasteira noutro\ldots{} \paren{Passa uma rasteira em
Gastão, que cai.} Azulei pela travessa do Navarro, passei pelo túnel, caí no
Rio Comprido e era uma vez seu Cazuza. \paren{Vai à garrafa e bebe por ela.}

\repl{Salustiano} \paren{Ajuda Gastão a levantar"-se.}
 Isto não é nada.  Simples brincadeira. \paren{Apresenta.} Nobres
amigos, apresento"-lhes o grande Cazuza, terror do Catumbi, e secretário dos
Foliões do Itapiru.

\repl{Cazuza} Viva!

\repl{Gastão} \paren{A Alfredo.}  Acho prudente irmos embora.

\repl{Alfredo} Deixá"-las com essa cáfila, nunca. \paren{Ouve"-se tocar
harmônica.}

\repl{Emerenciana} \paren{À porta.} É seu Gaudêncio.

\repl{Salustiano} É a nossa orquestra. Por falar em orquestra, Zeca, vai
buscar os pandeiros.  \paren{Zeca sai e volta com os pandeiros. Entra
Gaudêncio, embriagado, tocando harmônica. Salustiano vai
recebê"-lo à porta.}

\newscenenamed{Cena V}

\stagedir{\textsc{Os mesmos, Gaudêncio}, \textit{depois} \textsc{Zé} e \textsc{Joaquina}}


\repl{Salustiano} \paren{Apresenta.} O maestro Gaudêncio, que quanto mais
bêbado está, melhor toca. Vejam que expressão! Que alma! Que fulgurância divina
e harmoniosa!

\repl{Emerenciana} Mais não toca outra coisa senão aquela
porca. \paren{Gaudêncio senta ao fundo depois de beber na
garrafa.}

\repl{Gastão} \paren{A Alfredo.}  Os copos são
inúteis. Bebem todos pelas garrafas. \paren{Entram Zé da
Carroça e Joaquina, tipos de portugueses de cortiço.}

\repl{Zé} \paren{Na porta.} Ora \textit{biba} lá o \textit{S'or} Salustiano e \textit{mal} a companhia.

\repl{Joaquina} Como está, \textit{S'ora} Emerenciana?

\repl{Salustiano} Oh, o estimável vizinho, Senhor Zé das Carroças\ldots{} Como vai
essa bizarria?

\repl{Zé} \textit{Bamos indo, bamos indo}, graças a deus!  \textit{Bim bere o}
ensaio do \textit{grúpio}.

\repl{Salustiano} Ainda bem.  \paren{Apresenta"-os.} Este é o primeiro bombo
do Zé Pereira.

\repl{Joaquina} Mas não \textit{debes abusare} porque ano passado,
depois do \textit{carnabal, lebaste} uma s'mana inteira que nem
podias \textit{muber} o braço.

\repl{Zé} Cá a Joaquina também quis \textit{bir bere o ensaio}. Que isto de
mulheres, em lhe dando pra \textit{avelhudas} não há quem possa co elas.
\paren{Bebe pela garrafa.}

\repl{Salustiano} Receio que não tenhamos ensaio. Ou que tenhamos um ensaio
perfunctório e dogmático; o Espanta e o Pan"-americano estão engaiolados.

\repl{Joaquina} Ai!

\repl{Zé} Já \textit{tardaba}! Há oito dias que não iam ao xilindró!

\repl{Cazuza} \paren{Formalizando.}  Que tem você com isso,
seu trouxa?

\repl{Joaquina} Eh, lá!\ldots{} cá co meu \textit{home} não quero que se
\textit{ingracem}.

\repl{Cazuza} Eu\ldots

\repl{Salustiano} Cazuza, amigo, este é o meu lar doméstico, e é 
também a sede do nosso cordão. Comporte"-se com toda a dignidade e prosopopeia.

\repl{Cazuza} Não há novidade, seu Salustiano.

\repl{Salustiano} Não! Agora não me chama seu Salustiano; me chama seu
presidente.

\repl{Cazuza} Não há novidade, seu presidente.

\repl{Gastão} \paren{A Alfredo.}  Isto acaba mal!

\repl{Alfredo} Nada receies.

\repl{Gastão} Um trambolhão já levei.

\repl{Salustiano} Mas como ia dizendo: o Espanta e o Pan"-americano estão na
gaiola\ldots{} e o Miudinho homiziado no morro do Nheco.

\repl{Emerenciana} \paren{Na porta.} Aí vem seu Remígio e traz as moças.

\repl{Alfredo e Gastão} São elas!\ldots

\newscenenamed{Cena VI}
\stagedir{\textsc{Os mesmos, Remígio, Rosa} e \textsc{Florinda}}


\repl{Salustiano} \paren{Indo recebê"-lo.} Oh, o grande, o
incomensurável, o célebre Remígio e suas adoráveis filhas, as flores mais
odorantes e embrionárias dos jardins de Catumbi. Vão entrando, vão entrando.
\paren{Entram os três citados. Apresenta"-o.}  Este é o
ensaiador do grupo. Não há outro para combinar aquelas evoluções prismáticas e
sintéticas que dão tanta graça a um cordão, tornando"-o, se assim me posso
exprimir, um cordão sanitário\ldots{} \paren{Alfredo e Gastão esforçam"-se
para que Florinda e Rosa os vejam, porém elas conversam
de cabeça baixa, parecendo vencidas.}

\repl{Joaquina} Este seu Salustiano fala como um \textit{libro averto}.

\repl{Remígio} Antes de mais nada, um beijo nesta negra. \paren{Vai à mesa
e bebe pela garrafa. Depois, Salustiano examina se ainda há
alguma coisa na garrafa, bebe o resto e chama Zeca.}

\repl{Salustiano} Oh, Zeca\ldots{} Vai acender este castiçal. O garrafão está
debaixo da cama. \paren{Zeca sai e volta depois com a garrafa cheia.}

\repl{Remígio} Então o ensaio hoje não é no quintal?! Aqui faz muito calor.

\repl{Salustiano} Remígio, apresento"-lhes estes camaradas: o Pereira e o\ldots{} 
turco.

\repl{Gastão} Abdul Abdala. \paren{Ouvindo a voz de Gastão, Rosa
ergue
a cabeça e reconhece"-o, e reconhece também
Alfredo. Chama atenção de Florinda e começa
então um jogo de cena entre os quatro.}

\repl{Salustiano} O senhor Remígio, praça reformado do exército e ex"-bravo
da Pátria!

\repl{Gaudêncio} \paren{Despertando.}  Bravo da Pátria?

\repl{Salustiano} Ex.

\repl{Gaudêncio} Então com licença! \paren{Toca a polca na harmônica. Todos
se põem a dançar.  Alfredo e Gastão
aproveitam para dançar com Florinda e Rosa.}

\repl{Remígio}
\paren{Depois da dança.} 
Bravo da Pátria, sim. Mas de que serve isso? De que serviu ter arriscado tantas
vezes a pele? \paren{Vai pegar na garrafa e quebra um copo.}

\repl{Emerenciana} Zeca, leva esses copo lá para dentro, antes que
se quebrem os outro.

\repl{Salustiano} Oh!  \textit{copos hic, que labor est}!
\paren{Zeca carrega os copos.}

\repl{Remígio}
Sabe Deus o que me tem custado a educar
estas duas meninas!

\repl{Zé} \textit{Mais} também vossemecê pode \textit{gavar"-se} de que estão
pr'aí duas senhoras de truz!

\repl{Joaquina} Isso estão! \textit{Calquer} das duas é capaz de dar
\textit{boltas} ao miolo de um \textit{doitor}.

\repl{Cazuza} \textit{Mais}, seu \textit{Remijo}, você foi um herói. Foi cabra
valente, hein?

\repl{Remígio} Se fui valente\ldots{} Na batalha de 24 de Maio, eu
sozinho matei dez paraguaios e meio, digo meio porque um deles não morreu, foi
apenas ferido.

\repl{Gaudêncio} Dez paraguaios e meio?

\repl{Remígio} Sim, senhor!

\repl{Gaudêncio} Sozinho?

\repl{Remígio} Sim, senhor.

\repl{Gaudêncio} Então com sua licença! \paren{Toca harmônica e todos
dançam.}

\repl{Remígio} Lhes garanto que não \textit{houveram} muitos soldados como
eu.

\repl{Cazuza} Admira que não voltasse pelo menos alferes.

\repl{Remígio} Que quer você? Eu era valente, mas não
sabia ler nem escrever, e depois não tinha proteção. Ah! se eu fosse um cabra
escovado como é o nosso presidente, nem o Caxias me passava a perna\ldots{} Depois de
velho ainda mostrei que valia alguma coisa quando marchei para o Paraná. No
cerco da Lapa combati 27 dias e 27 noites sem parar.

\repl{Gaudêncio} 27 dias e 27 noites sem parar?!

\repl{Remígio} Sim, senhor.

\repl{Gaudêncio} Então com sua licença.  \paren{Toca a harmônica. Dançam.}

\repl{Remígio} Mas isso não foi nada. Na batalha de Tuiuti, dez de julho,
essa sim. Foi um delírio. Aí lutamos braço a braço. Nós éramos \textit{inferior}
em número, lutávamos um contra vinte. Os combates a armas brancas são
sensacionais. Vocês não imaginam! Os campos cobertos de cadáveres. Lutamos até
dentro de um rio, com água até o pescoço\ldots{} Foi tanto o sangue, que a água ficou
encarnada. Os cadáveres eram tantos, que fizeram uma ponte para nossas tropas.

\repl{Gaudêncio} Os cadáveres fizeram uma ponte?!

\repl{Remígio} Sim, senhor.

\repl{Gaudêncio} E vocês passaram por cima?!

\repl{Remígio} Sim, senhor.

\repl{Gaudêncio} Então com sua licença.  \paren{Toca a harmônica. Dançam.}

\repl{Cazuza} Mas há, ou não há ensaio?

\repl{Salustiano} Falta muita gente, e eu queria fazer um ensaio completo,
sugestivo e saliente para ser agradável àqueles nobres camaradas.

\repl{Cazuza} Vamos ensaiar com quem está.

\repl{Salustiano} Faltam tantas figuras. O Miudinho é insubstituível.

\repl{Remígio} Podemos dar amostra de uma cantiga. Forma o cordão.
\paren{Forma"-se um cordão com todos, menos os namorados.}
Vamos cantar\ldots

\begin{verse}
\fala{Todos} \paren{Repetem muitas vezes.}\\
Ai, ai, ai! eu ai!\\
Deixa as \textit{cadeira}\\
negra \textit{boli}.
\end{verse}

\repl{Remígio} E não se esqueçam das palminhas. Vamos\ldots{} 
lá\ldots{} Um, dois, três!  \paren{Cantam em evoluções.}

\begin{verse}
\fala{Todos} \paren{Repetem muitas vezes.}\\
Ai, ai, ai! eu ai!\\
Deixa as \textit{cadeira}\\
da negra \textit{boli}.
\end{verse}

\paren{Tanto para começar como para terminar, Remígio toca um
apito, que coloca no pescoço.}

\repl{Remígio} Mas isto aqui na sala não presta. Não há espaço e faz calor.
Vamos para o quintal?

\repl{Todos} Apoiado! Vamos!

\repl{Remígio} Levemos a mesa a as cadeiras.

\repl{Todos} Levemos. \paren{Saem levando as cadeiras e a mesa.}

\repl{Salustiano} Entrai, nobres amigos\ldots{} Disponham deste tugúrio tangente
e plácido.

\repl{Alfredo} Vão indo\ldots{} Nós acompanhamos\ldots{} \paren{Formam o
cordão e saem cantando. Ficam por último Rosa e Florinda, que
são retidas por Alfredo e Gastão, que
agarram"-nas e as trazem para o meio da cena, que fica escura. Orquestra em
surdina rápida, ai, ai, ai etc.}

%\vfill\pagebreak 
\newscenenamed{Cena VII}

\stagedir{\textsc{Alfredo, Gastão, Rosa} e \textsc{Florinda}}


\repl{Alfredo} Não há tempo a perder\ldots{} As senhoras não podem ficar aqui nem
mais um momento.

\repl{Rosa e Florinda} Meu Deus!

\repl{Florinda} E tudo escuro?

\repl{Alfredo} Venham conosco, vamos depositá"-las em casa de uma família
respeitável.

\repl{Florinda} Mas papai?\ldots

\repl{Alfredo} Seu pai é indigno das filhas que tem. Vamos.

\repl{Florinda} Não, Alfredo, não. Tudo, menos isso.

\repl{Alfredo} Nesse caso, adeus! Adeus para sempre! Porque se continuam
aqui, não serão dignas de nós. Vamos, Gastão.

\repl{Rosa} Meu Deus!  

\repl{Alfredo} Temos um automóvel ali na esquina. Venham.

\repl{Florinda} Juram que serão nossos maridos?

\repl{Alfredo} Juro por minha mãe.

\repl{Gastão} Juro por minha irmã.

\repl{Florinda} \paren{Resolutamente.} Vamos.

\repl{Rosa} Florinda!

\repl{Florinda} Vamos! É melhor assim!

\repl{Alfredo e Gastão} Vamos, vamos. \paren{Saem os quatro.}

\repl{Salustiano} \paren{Fora.}  Nobres amigos\ldots

\repl{Remígio} \paren{Fora.} Florinda\ldots{} Rosa\ldots{}  \paren{Entra Emerenciana, que
vai à porta e ainda os vê fugir.}

%\pagebreak 
\newscenenamed{Cena VIII}
\stagedir{\textsc{Todos}, \textit{menos os quatro namorados.} Entram todos com velas acesas.}


\repl{Remígio} \paren{Entra.}  Florinda.\ldots{} Rosa\ldots{} Onde estão
minhas filhas?

\repl{Emerenciana} Fugiram com o turco e o quitandeiro.

\repl{Salustiano} \paren{Entra.} Será possível?\ldots{} \paren{Falam ao mesmo
tempo. Grande confusão.}

\repl{Remígio} Quem é esse quitandeiro? Quem é esse turco?

\repl{Salustiano} Não conheço eles!

\repl{Remígio} Minhas filhas! Minhas filhas!

\repl{Cazuza} Bem me pareceram que não eram do pessoal. Demais, ali, na
esquina, estava um automóvel.

\repl{Remígio} Um automóvel?! Estão perdidas! Minhas filhas!\ldots{} Minhas
filhas!\ldots{} \paren{Saem todos correndo para a rua.}

\repl{Gaudêncio} \paren{Que entrou um pouco antes.}  Que é isso?\ldots{} As
pequenas fugiram? Então, com sua licença\ldots{} \paren{Toca a harmônica.}


\newscenenamed{\textit{Quadro 3}}

\stagedir{Jardim em casa do Conselheiro. Ao fundo, grade com portão.
À esquerda alta, grade com um banco de jardim.
Alvorada na orquestra, 
com pássaros.}


\newscenenamed{Cena I}
\stagedir{\textsc{Alfredo} e \textsc{Gastão}. Aparecem ao fundo por trás das grades.}


\repl{Alfredo} Não está ninguém\ldots{} vamos entrando. \paren{Empurram a
grade, entram.}

\repl{Gastão} Como terão passado a noite?

\repl{Alfredo} Naturalmente não dormiram\ldots{} Mas devem estar satisfeitas, por
terem reconhecido que somos dois rapazes sérios. O automóvel trouxe"-nos
diretamente do Itapiru para esta casa\ldots

\repl{Gastão} \ldots{} onde reside com sua excelentíssima
família o nosso ilustre e venerando diretor geral\ldots

\repl{Alfredo} \ldots{} que nos acolheu paternalmente\ldots

\repl{Gastão} \ldots{} depois de nos haver passado tremenda descalçadeira em
estilo oficial.

\repl{Alfredo} O conselheiro não tem outro estilo, mesmo quando não passa
descalçadeiras. Sua Excelência está saturado de burocracia.

\repl{Gastão} Estas moças, disse"-nos ele, só daqui sairão para a igreja\ldots

\repl{Alfredo} \ldots{} com escala pela pretoria.

\repl{Gastão} Elas aí vêm.

\repl{Alfredo} Ah!

\newscenenamed{Cena II} 
\stagedir{\textsc{Os mesmos, Florinda} e \textsc{Rosa}.}

{\smallskip\raggedleft\itshape Quarteto\par}
\begin{verse}
\fala{Rosa}\\
Gastão!

\fala{Alfredo}\\
Florinda!

\fala{Florinda}\\
Alfredo!

\fala{Gastão}\\
Rosinha!

\fala{Gastão e Alfredo}\\
Como passaram?

\fala{Rosa e Florinda}\\
Perfeitamente.

\fala{Florinda}\\
Fiquei tranquila.

\fala{Rosa}\\
Não tive medo.

\fala{Florinda}\\
Lençóis de linho!

\fala{Rosa e Florinda}\\
Cama excelente!

\fala{Os quatro}\\
Que bela coisa é estar ao lado\\
Do bem"-amado!\\
Sentir na mão, a mão querida\\
De quem nos dá prazer na vida,\\
Que bela coisa é estar ao lado\\
Do bem"-amado!

\fala{Florinda}\\
Quando o receio do perigo\\
Do meu peito varri,\\
Adormeci\ldots{} sonhei contigo\ldots

\fala{Rosa}\\
Eu cá também adormeci!

\fala{Florinda}\\
Pela manhã muito cedinho,\\
O sol no quarto penetrou,\\
E então a gente de mansinho,\\
Se levantou.

\fala{Rosa}\\
E, sem receio,\\
Florinda veio,\\
Eu também vim\\
Dar um passeio\\
Pelo jardim.

\fala{Florinda}\\
O que me dá cuidado\\  
Agora é só papai!

\fala{Alfredo}\\
Está tudo arranjado!

\fala{Gastão}\\
Tudo arranjar"-se vai!

\fala{Os quatro}\\
Que bela coisa é estar ao lado etc.
\end{verse}

\newscenenamed{Cena III}

\stagedir{\textsc{Os Mesmos} e \textsc{o Conselheiro}.}  

\stagedir{O Conselheiro é um velho
de suíças brancas, vestido com trajes matinais e boné; aparece ao fundo com um
Diário Oficial.}


\repl{Conselheiro} Oh!\ldots

\repl{Os quatro} \paren{Que estavam abraçados separam"-se num grito.} Ah!

\repl{Conselheiro} Declaro"-vos que me não conformo com esta infração das cláusulas do
nosso ajuste. Vós não podeis estar juntos, senão na presença de minha mulher ou
na minha. Foi essa a condição \textit{sine qua non} do recolhimento e manutenção
destas senhoritas em minha casa. E é o princípio administrativo que\ldots

\repl{Alfredo} \paren{Interrompe.} Chegamos neste instante\ldots{} Íamos entrar\ldots

\repl{Gastão} Encontramo"-las no jardim. \paren{Elas sobem.}

\repl{Conselheiro} Já fostes ter com o pai, a fim de que ele, com a urgência
possível, se dirigisse à nossa residência?!

\repl{Alfredo} Não, senhor Conselheiro, não fomos ter com ele, porque receamos que
nos recebesse mal\ldots{} O homem não é para graças! Matou de uma só vez dez
paraguaios e meio. Não fomos ter com ele, mas, logo pela manhã, mandamos"-lhe um
telegrama.

\repl{Gastão} ``Filhas depositadas casa conselheiro Faria, rua Escobar, 150.
São Cristóvão. Vá lá às nove horas.''

\repl{Conselheiro} Podíeis ter acrescentado: ``manhã''.

\repl{Alfredo} Não acrescentamos para não exceder de vinte palavras.

\repl{Gastão} Não queríamos gastar mais de cinco tostões.

\repl{Conselheiro} Ele é capaz de só aparecer aqui às nove horas da noite.

\repl{Gastão} Não creia, senhor conselheiro.

\repl{Florinda} \paren{Olhando para fora.} Meu Deus! Lá vem papai!

\repl{Gastão} Que dizia eu?

\repl{Conselheiro} Nesse caso tende por muito recomendado afastar"-vos até deliberação
ulterior.

\repl{Alfredo} \paren{À parte.} Saúde e fraternidade!

\repl{Conselheiro} Na ocasião oportuna chamar"-vos"-ei. Ide para debaixo daquele
caramanchão, mas espero do vosso critério, procedais com toda a circunspeção e
dignidade.

\repl{Alfredo} Esteja descansado, senhor Conselheiro\ldots{} Nós\ldots{} somos quatro,
fiscalizamo"-nos uns aos outros. \paren{Saem os quatro. Remígio aparece ao
portão.}

\newscenenamed{Cena IV}

\stagedir{\textsc{Conselheiro} e \textsc{Remígio}}


\repl{Remígio} \paren{Fora.} Minhas filhas\ldots{} Onde estão minhas filhas?!

\repl{Conselheiro} Empurre o portão e entre.

\repl{Remígio} Não tem cachorro?

\repl{Conselheiro} Não. \paren{Remígio entra.} Vossemecê chama"-se Remígio?

\repl{Remígio} Para servir a Sua Senhoria.

\repl{Conselheiro} É o próprio e autêntico pai de dona Florinda e dona
Rosa?

\repl{Remígio} Sim, senhor, e quero\ldots

\repl{Conselheiro} Já lá vamos. Sente"-se naquele banco.

\repl{Remígio} Não estou cansado, não, senhor.

\repl{Conselheiro} Sente"-se.  \paren{Remígio senta"-se.} Vossemecê está em
presença do Conselheiro Faria, velho funcionário, com quarenta anos de
serviço\ldots{} Conselheiro e Oficial da Rosa pelo Império, e tenente"-coronel
honorário do Exército pela República.  \paren{Remígio levanta"-se, perfila"-se e
faz continência militar.} Obrigado. Sente"-se [que] vossemecê é um praça
reformado\ldots{} Militou nas fileiras dos defensores da Pátria e, como tal, merecia
ser respeitado\ldots{} Se se desse ao respeito\ldots

\repl{Remígio} Então eu não me dou ao respeito?

\repl{Conselheiro} Não, senhor, vossemecê não se dá ao respeito.

\repl{Remígio} Por quê?

\repl{Conselheiro} Em primeiro lugar, porque mente e é fanfarrão.

\repl{Remígio} \paren{Levantando"-se.} Fanfarrão, eu?

\repl{Conselheiro} Sim, senhor. Sente"-se.

\repl{Remígio} Mas\ldots

\repl{Conselheiro} Sente"-se.  \paren{Remígio senta.} Vossemecê gaba"-se de
ter matado uma infinidade de paraguaios\ldots{} Ora, a guerra do Paraguai acabou há
38 anos\ldots{} Vossemecê tem cinquenta, quando muito\ldots{} Para lá ter
estado, era preciso que combatesse com idade de 12 anos.  \paren{Remígio
abaixa a cabeça.} Vossemecê\ldots{} tem duas filhas, menos mal"-educadas, não pelo
pai, mas pelo padrinho, um general de quem vossemecê foi ordenança. Em vez de
resguardar suas filhas e afastá"-las do mal, vossemecê leva"-as a um desses antros
denominados cordões carnavalescos, em casa de um homem de má vida, onde se
reúnem bêbedos e desordeiros.

\repl{Remígio} É uma casa de família.

\repl{Conselheiro} Não duvido, mas há família e família.  \paren{Remígio
abaixa a cabeça.} Vossemecê foi feliz. Dois moços honestos e morigerados,
empregados na mesma repartição onde exerço as funções de diretor geral,
enamoraram"-se de suas filhas e imediatamente resolveram subtraí"-las ao
pernicioso contato da sociedade onde vossemecê tão imprudentemente as levava.
Penetraram disfarçadamente no referido cordão, raptaram"-nas e trouxeram"-nas para
minha casa, de onde só sairão para se casarem. Para tais consórcios precisamos
do consentimento de vossemecê.

\repl{Remígio} \paren{Levantando"-se.} Mas, seu
tenente"-coronel\ldots

\repl{Conselheiro} Chame"-me Conselheiro.

\repl{Remígio} Mas, seu Conselheiro, não era preciso nada disso. Bastava que
esses moços me aparecessem e dissessem: ``Seu Remígio, nós queremos
casar com suas filhas''. Se elas quisessem, eu não empatava.

\repl{Conselheiro} Então, está satisfeito?

\repl{Remígio} Se estou satisfeito? Sua Excelência não imagina que peso
tirou de riba de mim\ldots{} Passei uma noite dos diabos! Podia lá pensar que as
meninas estivessem aqui? Quer saber onde fui procurar elas? Na Copacabana.

\repl{Conselheiro} Pois estão em São Cristóvão, no seio de uma família respeitável.

\repl{Remígio} Muito obrigado a Sua Senhoria.

\repl{Conselheiro} Vou chamá"-las e também os noivos, que se acham presentes\ldots
\paren{Acena para fora com o Diário Oficial.} Quanto a vossemecê,
esforce"-se de agora em diante por ser digno daquelas filhas, e deixe aos rapazes
os prazeres impróprios da sua idade.  \paren{Entram os quatro, ressabiados.
Remígio abre os braços à Florinda e à Rosa, sorrindo. Elas vão a ele e
abraçam"-no.}


\newscenenamed{Cena V}

\stagedir{\textsc{Os mesmos} e \textsc{Rosa, Florinda, Alfredo} e \textsc{Gastão}}

\repl{Remígio} Oh, minhas filhas! Que susto!

\repl{Florinda} Perdoe, papai!

\repl{Remígio} Não tenho nada que perdoar.  Pelo contrário, vocês é que
devem perdoar seu pai! Este bom homem é que me abriu os olhos, e me deu bons
conselhos.

\repl{Conselheiro} Não fosse eu conselheiro.

\repl{Remígio} \paren{A Alfredo e Gastão.}  Olha o quitandeiro, olha o
turco.

\repl{Gastão} Abdul Abdala!

\repl{Conselheiro} Bem, vamos almoçar, que são horas. O serviço público nos
reclama a mim e a estes senhores. Venha almoçar conosco, senhor Remígio.

\repl{Remígio} Oh, senhor Conselheiro, eu\ldots

\repl{Conselheiro} Venha!

\repl{Alfredo} É o nosso almoço de núpcias. \paren{Saem.}

\repl{Remígio} Seu Conselheiro\ldots{} Sua Senhoria de hoje em diante pode contar
com um homem pra lhe servir como um escravo. \paren{Saem todos.}


\newscenenamed{\textit{Quadro 4}}

\stagedir{A mesma cena do segundo quadro}


\newscenenamed{Cena I}
\stagedir{\textsc{Gaudêncio}, vestido de princês, com a máscara atirada para
as costas: vem bêbedo e tocando a sua harmônica,
depois \textsc{Salustiano}, depois \textsc{Zeca}.}


\repl{Gaudêncio} Seu Salustiano\ldots{} Seu Salustiano\ldots{} \paren{Salustiano
ricamente vestido de veludo e ouro, mas de um modo híbrido, sem inteligência,
propriedade e gosto. Vem de botinas.}

\repl{Salustiano} Aqui estou! Admire, maestro. Admire esta imponência,
surpreendente e circuncisfláutica! Tenho cinquenta mil réis em cima de mim\ldots{} 
Mas que quer? Sou o presidente e o porta"-estandarte dos Foliões do
Itapiru. \textit{Nobresse obrige} como dizia o grande
Chateaubriand. \paren{Chamando.} Ó Zeca, traz as cadeiras para dentro. \paren{A
Gaudêncio.} As cadeiras ficaram esta noite no quintal.

\repl{Gaudêncio} Mas não houve ensaio\ldots{}  Foi proibido.

\repl{Salustiano} Sim, porque o Comendador, meu vizinho, praticou uma
insídia incongruente.  Aproveitando a emergência das imposições policiais,
apresentou na delegacia um atestado médico, provando que a sogra estava
gravemente enferma\ldots{} \paren{Zeca entra vestido de diabinho.} \textit{Quedê} as
cadeiras? 
 \paren{Zeca faz um arremesso e sai.} Pois bem, hoje, logo ao romper
da roxa aurora, fui ao quintal refrescar os humores aquáticos do cérebro
alagadiço\ldots

\repl{Gaudêncio} Sim, você ontem apanhou uma camoeca turuna\ldots

\repl{Salustiano} E você, duas; mas lancemos um véu sobre o passado\ldots{} Vou ao
quintal e quem hei de ver à janela do Comendador?

\repl{Gaudêncio} A sogra?

\repl{Salustiano} A sogra, que, na véspera, estava gravemente enferma. Foi
ou não foi uma insídia incongruente?

\repl{Gaudêncio} Incongruentíssima!

\repl{Salustiano} \paren{A Zeca, que entra.} \textit{Quedê} sua mãe?

\repl{Zeca} Não sei\ldots{} Tá se vestindo. \paren{Sai.}

\repl{Gaudêncio} Mas deixe lá\ldots{} estes ensaios\ldots{} estas cantorias\ldots{} estes
pandeiros devem atordoar a vizinhança, mesmo sem haver sogra doente\ldots

\repl{Salustiano} Meu caro artista: o carnaval é uma época de loucuras
anormais e casuísticas em que há o direito de procrastinar a vizinhança.

\vfil\newscenenamed{Cena II}

\stagedir{\textsc{Os mesmos} e \textsc{Emerenciana}}


\repl{Emerenciana} \paren{Vestida de baiana.} Boas tardes, seu Gaudêncio.

\repl{Gaudêncio} Bravos à baiana!

\repl{Emerenciana} Eu cá não quero outra fantasia\ldots{} É a mais barata\ldots{} E
esta há quatro anos que serve. Quando chega quarta"-feira de Cinzas vai pro fundo
do baú, e só sai no sábado gordo do outro ano.

\repl{Gaudêncio} Por isso o cheiro do bafio, chega até cá.

\repl{Emerenciana} Não sou como seu Salustiano, que gasta um dinheirão com
aquela roupa de princês, e o pequeno anda coberto de trapos e, por isso, não vai
à escola. \paren{Ouve"-se ao longe o cordão cantar o ai, ai, ai etc.}

\repl{Salustiano} Sendo eu o presidente, o porta"-estandarte dos
Foliões do Itapiru, sou obrigado a esta suntuosidade híbrida e
analítica! Mas não me engano, aí chega a bela rapaziada.

\newscenenamed{Cena III}
\stagedir{\textsc{Os mesmos, Cazuza, Zé das Carroças, Joaquina, Espanta,
Pan"-americano} e outros foliões. Todos fantasiados. Cazuza entra à frente,
vestido de velho, dançando, Zé das Carroças com bombo e outros com pandeiros,
entram cantando e fazendo uma volta pela cena.}

\begin{verse}
Eis aqui os Foliões\\
\quad Do Itapiru,\\
Venham palmas e ovações,\\
Pois não há, na realidade,\\
Tão bonita sociedade\\
Desde a Gávea até o Caju!\ldots{}
Venham palmas e ovações\\
\quad Pra os Foliões\\
\quad Do Itapiru!\ldots
\end{verse}

\repl{Cazuza} \paren{Tirando a cabeçorra.} O diabo é que esta cabeçorra
pesa muito e faz lá dentro um calor de assar passarinho!

\repl{Salustiano} Cazuza amigo, o sacrifício é o apanágio do carnaval.

\repl{Zé} São horas. Tenho comichões no \textit{vraço}.  Este diabo deste bombo está a
me desafiar.

\repl{Joaquina} Ah, \textit{mê home}! Vê lá se, passada a festa,
\textit{nam} podes \textit{mober} o \textit{vraço} como no ano passado.

\repl{Zé} Deixa lá, mulher!

\repl{Salustiano} Estão presentes todos os foliões?

\repl{Todos} Falta seu Remígio!

\repl{Salustiano} Sim, falta o nosso Remígio. Mas esse pertence agora a
outro pessoal que não é o nosso! Casou as filhas com aqueles dois heróis
românticos e anda agora de sobrecasaca e cartola\ldots{} Não pensemos em coisas
tristes\ldots{} \paren{Vai buscar o estandarte.} Chegou o momento solene! Vou
empunhar o brioso estandarte e 
pôr"-me à frente do cordão! Estão todos em ordem?

\repl{Todos} Estamos!

\repl{Salustiano} Marche!  \paren{Todos cantando e marchando.}


\newscenenamed{\textit{Quadro 5}}


\stagedir{Avenida Central, em noite de Carnaval.}


\newscenenamed{Cena I}


\stagedir{\textsc{Alfredo, Gastão, Rosa, Florinda} e \textsc{Remígio}. Entram todos com roupas leves. Remígio, de barba
feita, de cartola e sobrecasaca; os quatro vêm de braço.}


\repl{Alfredo} \paren{A Florinda.} Não achas melhor irmos para casa?

\repl{Florinda} Decerto. Estas festas não se inventaram para os noivos.
Depois, já vimos passar as sociedades.

\repl{Gastão} Florinda diz bem. O nosso carnaval é em casa. Não achas,
Rosinha?

\repl{Rosa} Acho, meu amor.

\repl{Alfredo} Sogro e amigo, fique a divertir"-se. Nós vamos tomar um carro.
Vamos para casa.

\repl{Remígio} Pois vão com Deus.  \paren{Saem os quatro.}

\repl{Rosa e Florinda} Adeus, papai.

\newscenenamed{Cena II}

\stagedir{\textsc{Remígio}, \textit{povo e todos os personagens do cordão.}}


\repl{Remígio} Tirei a sorte grande\ldots{} Mas que vejo!\ldots{} O meu ex"-cordão!

\repl{Salustiano} Olhem, é ele! O nosso incomensurável Remígio! Viva o
Remígio! \paren{O cordão entra cantando e dançando.}

\repl{Todos} Viva! Entra! Entra! Fecha!  \paren{Põem Remígio no centro e
dançam todos.}

\repl{Remígio} Não \textit{arresisto}! Oh, o cordão! O cordão do povo!

\paren{Dança.}


\bigskip

\begin{center}
\textsc{Cai o pano}
\end{center}


