%%%%%%%%%%%%%%%%%%%%%%%%%%%%%%%%%%%%%%%%%%%%%%%%%%%%

\newcommand{\quadro}[1]{\section{#1}}

\chapter{A capital federal}


%\castpagenamed{Personagens e Seus Criadores}

\cast{Lola}-\cast{Pepa Ruiz}
\cast{D. Fortunata}-\cast{Clélia}
\cast{Benvinda}-\cast{Olímpia Amoedo}
\cast{Quinota}-\cast{Estefânia Louro}
\cast{Juquinha}-\cast{Adelaide Lacerda}
\cast{Mercedes}-\cast{Maria Mazza}
\cast{Dolores}-\cast{Marieta Aliverti}
\cast{Blanchette}-\cast{Madalena Vallet}
\cast{Um Literato}-\cast{Maria Granada}
\cast{Uma Senhora Maria Granada}
\cast{Uma Hóspede do Grande Hotel da Capital Federal}-\cast{Olívia}
\cast{Eusébio}-\cast{Brandão}
\cast{Figueiredo}-\cast{Colás}
\cast{Gouveia}-\cast{H. Machado}
\cast{Lourenço}-\cast{Leonardo}
\cast{Duquinha}-\cast{Zeferino}
\cast{Rodrigues}-\cast{Portugal}
\cast{Pinheiro}-\cast{Portugal}
\cast{Um Proprietário}-\cast{Pinto}
\cast{Um Frequentador do Belódromo}-\cast{Pinto}
\cast{Outro Literato}-\cast{Lopes}
\cast{O Gerente do Grande}
\cast{Hotel da Capital Federal}-\cast{Lopes}
\cast{S’il Vous Plaît, amador de bicicleta}-\cast{Louro}
\cast{Mota}-\cast{Azevedo}
\cast{Lemos}-\cast{Azevedo}
\cast{Um Convidado}-\cast{Oliveira}
\cast{Guedes}-\cast{Oliveira}
\cast{Um Inglês}-\cast{Peppo}
\cast{Um Fazendeiro}-\cast{Montani}
\cast{O Chasseur}-\cast{N.N.}

Hóspedes e criados do Grande Hotel da Capital Federal, vítimas de uma
agência de alugar casa, amadores de bicicleta, convidados, pessoas do povo, soldados
etc.

\newact

 \quadro{Quadro I}

\paren{Suntuoso vestíbulo do Grande Hotel da Capital Federal. Escadaria ao fundo.
Ao levantar o pano, a cena está cheia de hóspedes de ambos os sexos, com malas
nas mãos, e criados e criadas que vão e vêm. O gerente do hotel anda daqui para
ali na sua faina.}

\newscenenamed{Cena I}
\stagedir{O Gerente, um Inglês, uma Senhora, um Fazendeiro e um Hóspede}

 Coro e Coplas

 Os Hóspedes

 De esperar estamos fartos
 Nós queremos descansar!
 Sem demora aos nossos quartos
 Faz favor de nos mandar!

 Os Criados

 De esperar estamos fartos!
 Precisamos descansar!
 Um hotel com tantos quartos
 O topete faz suar!

\repl{Um Hóspede} Um banho quero!

\repl{Um Inglês} Aoh! Mim quer come!

\repl{Uma Senhora} Um quarto espero!

\repl{Um Fazendeiro} Eu estou com fome!

 O Gerente

 Um poucochinho de paciência!
 Servidos todos vão ser, enfim!
 Eu quando falo, fala a gerência!
 Fiem-se em mim!

 Coro

 Pois paciência, uma vez que assim quer a gerência!

 Coplas

 O Gerente

 -I-

 Este hotel está na berra!
 Coisa é muito natural!
 Jamais houve nesta terra
 Um hotel assim mais tal!
 toda a gente, meus senhores,
 Toda a gente, ao vê-lo, diz:
 Que os não há superiores
 Na cidade de Paris!
 Que belo hotel excepcional
 O Grande Hotel da Capital Federal!

 Coro

 Que belo hotel excepcional, etc\ldots{}

 O Gerente

 - II -

 Nesta casa não é raro
 Protestar algum freguês:
 Acha bom, mas acha caro
 Quando chega o fim do mês.
 Por ser bom precisamente,
 Se o freguês é do bom-tom
 Vai dizendo a toda a gente
 Que isto é caro mas é bom.
 Que belo hotel excepcional!
 O Grande Hotel da Capital Federal!

 Coro

 Que belo hotel excepcional, etc\ldots{}

\repl{O Gerente} \paren{Aos criados.} -- Vamos! Vamos! Aviem-se! Tomem as malas e
encaminhem estes senhores! Mexam-se! Mexam-se!\ldots{} \paren{Vozeria. Os hóspedes
pedem quartos, banhos, etc\ldots{} Os criados respondem. Tomam as malas, saem
todos, uns pela escadaria, outros pela direita.}

\newscenenamed{Cena II}
\stagedir{O Gerente, depois Figueiredo}

\repl{O Gerente} \paren{Só.} Não há mãos a medir! Pudera! Se nunca houve no Rio de
Janeiro um Hotel assim! Serviço elétrico de primeira ordem! Cozinha
esplêndida, música de câmara durante as refeições da mesa-redonda! Um relógio
pneumático em cada aposento! Banhos frios e quentes, duchas, sala de natação,
ginástica e massagem! Grande salão com um plafond pintado pelos nossos primeiros
artistas! Enfim, uma verdadeira novidade! -- Antes de nos estabelecermos aqui, era uma
vergonha! Havia hotéis em S. Paulo superiores aos melhores do Rio de
Janeiro! Mas em boa hora foi organizada a Companhia do Grande Hotel da Capital Federal,
que dotou esta cidade com um melhoramento tão reclamado! E o caso é que a
empresa está dando ótimos dividendos e as ações andam por empenhos! \paren{Figueiredo
aparece no topo da escada e começa a descer.} Ali vem o Figueiredo. Aquele
é o verdadeiro tipo do carioca: nunca está satisfeito. Aposto que vem fazer
alguma reclamação.

\newscenenamed{Cena III}
\stagedir{O Gerente, Figueiredo}

\repl{Figueiredo} Ó seu Lopes, olhe que, se isto continuar assim, eu mudo-me!

\repl{O Gerente} \paren{À parte.} Que dizia eu?

\repl{Figueiredo} Esta vida de hotel é intolerável! Eu tinha recomendado ao
criado que me levasse o café ao quarto às sete horas, e hoje\ldots{}

\repl{O Gerente} O meliante lhe apareceu um pouco mais tarde.

\repl{Figueiredo} Pelo contrário. Faltavam dez minutos para as sete\ldots{} Você
compreende que isto não tem lugar.

\repl{O Gerente} Pois sim, mas\ldots{}

\repl{Figueiredo} Perdão; eu pedi o café para as sete e não para as seis e
cinquenta!

\repl{O Gerente} Hei de providenciar.

\repl{Figueiredo} E que ideia foi aquela ontem de darem lagostas ao almoço?

\repl{O Gerente} Homem, creio que lagosta\ldots{}

\repl{Figueiredo} É um bom petisco, não há dúvida, mas faz-me mal!

\repl{O Gerente} Pois não coma!

\repl{Figueiredo} Mas eu não posso ver lagostas sem comer!

\repl{O Gerente} Não é justo por sua causa privar os demais hóspedes.

\repl{Figueiredo} Felizmente até agora não sinto nada no estômago\ldots{} É um
milagre! E sexta-feira passada? Apresentaram-me ao jantar maionese. -- Maionese! Quase
atiro com o prato à cara do criado!

\repl{O Gerente} Mas comeu!

\repl{Figueiredo} Comi, que remédio! Eu posso lá ver maionese sem comer? Mas foi
uma coisa extraordinária não ter tido uma indigestão!\ldots{}

\newscenenamed{Cena IV}
\stagedir{Os mesmos, Lola}

\repl{Lola} \paren{Entrando arrebatadamente da esquerda.} Bom dia! \paren{Ao gerente.} Sabe
me dizer se o Gouveia está?

\repl{O Gerente} O Gouveia?

\repl{Lola} Sim, o Gouveia -- um cavalheiro que está aqui morando desde a semana
passada.

\repl{O Gerente} \paren{Indiscretamente.} Ah! o jogador\ldots{} \paren{Tapando a boca.} Oh!\ldots{}
Desculpe!\ldots{}

\repl{Lola} O jogador, sim, pode dizer! Porventura o jogo é hoje um vício
inconfessável?

\repl{O Gerente} Creio que esse cavalheiro está no seu quarto; pelo menos ainda
o não vi descer.

\repl{Lola} Sim, o Gouveia é jogador, e essa é a única razão que me faz gostar
dele.

\repl{O Gerente} Ah! A senhora gosta dele?

\repl{Lola} Se gosto dele? Gosto, sim, senhor! Gosto, e hei de gostar, pelo
menos enquanto der a primeira dúzia!

\repl{O Gerente} \paren{Sem entender.} Enquanto der\ldots{}

\repl{Lola} Ele só aponta nas dúzias -- ora na primeira, ora na segunda, ora na
terceira, conforme o palpite. Há perto de um mês que está apontando na
primeira.

\repl{Figueiredo} \paren{À parte.} É um jogador das dúzias!

\repl{Lola} Enquanto der a primeira, amá-lo-ei até o delírio!

\repl{Figueiredo} A senhora é franca!

\repl{Lola} Fin de siècle, meu caro senhor, fin de siècle.

 Valsa

 Eu tenho uma grande virtude:
 Sou franca, não posso mentir!
 Comigo somente se ilude
 Quem mesmo se queira iludir!
 Porque quando apanho um sujeito
 Ingênuo, simplório, babão,
 Necessariamente aproveito,
 Fingindo por ele paixão!
 
 Engolindo a pílula,
 Logo esse imbecil
 Põe-se a fazer dívidas
 E loucuras mil!
 Quando enfim, o mísero
 Já nada mais é,
 Eu sem dó aplico-lhe
 Rijo pontapé!
 
 Eu tenho uma linha traçada,
 E juro que não me dou mal\ldots{}
 Desfruto uma vida folgada
 E evito morrer no hospital.
 
 Descuidosa,
 Venturosa,
 Com folias
 Sem amar,
 Passo os dias
 A folgar!
 
 Só conheço as alegrias,
 Sem tristezas procurar!
 Eu tenho uma grande virtude, etc\ldots{}

Mas vamos, faça o favor de indicar-me o quarto do Gouveia.

\repl{O Gerente} Perdão, mas a senhora não pode lá ir.

\repl{Lola} Por quê?

\repl{O Gerente} Aqui não há disso\ldots{}

\repl{Figueiredo} \paren{À parte.} Toma!

\repl{O Gerente} Os nossos hóspedes solteiros não podem receber nos quartos
senhoras que não estejam acompanhadas.

\repl{Lola} Caracoles! Sou capaz de chamar o Lourenço para acompanhar-me.

\repl{O Gerente} Quem é o Lourenço?

\repl{Lola} O meu cocheiro. Ah! Mas que lembrança a minha! Ele não pode
abandonar a caleça!

\repl{O Gerente} O que a senhora deve fazer é esperar no salão. Um belo salão,
vai ver, com um plafond pintado pelos nossos primeiros artistas!

\repl{Lola} Onde é?

\repl{O Gerente} \paren{Apontando para a direita.} Ali.

\repl{Lola} Pois esperá-lo-ei. Oh! Estes prejuízos! Isto só se vê no Rio de
Janeiro!\ldots{} \paren{Vai a sair e lança um olhar brejeiro a Figueiredo.}

\repl{Figueiredo} Deixe-se disso, menina! Eu não jogo na primeira dúzia! \paren{Lola
sai pela direita.}

\newscenenamed{Cena V}
\stagedir{O Gerente, depois o Chasseur}

\repl{O Gerente} Oh! Sr. Figueiredo! Não se trata assim uma mulher bonita!\ldots{}

\repl{Figueiredo} Não ligo importância a esse povo.

\repl{O Gerente} Sim, eu sei\ldots{} é como a lagosta\ldots{} Faz-lhe mal, talvez, mas
atira-se-lhe que\ldots{}

\repl{Figueiredo} Está enganado. Essas estrangeiras não têm o menor encanto para
mim.

\repl{O Gerente} Não conheço ninguém mais pessimista que o senhor.

\repl{Figueiredo} Falem-me de uma trigueira\ldots{} bem trigueira, bem carregada\ldots{}

\repl{O Gerente} Uma mulata?

\repl{Figueiredo} Uma mulata, sim! Eu digo trigueira por ser menos rebarbativo.
Isso é que é nosso, é que vai com o nosso temperamento e o nosso sangue! E quanto
mais dengosa for a mulata, melhor! Ioiô, eu posso? Entrar de caixeiro, sair
como sócio?\ldots{} Você já esteve na Bahia, seu Lopes?

\repl{O Gerente} Ainda não. Mas com licença: vou mandar chamar o tal Gouveia.
\paren{Chamando.} Chasseur. \paren{Entra da direita um menino fardado.} Vá ao quarto no
135 e diga ao hóspede que está uma senhora no salão à sua espera. \paren{O menino sai a
correr pela escada.}

\repl{Figueiredo} Chasseur! Pois não havia uma palavra em português para\ldots{}

\repl{O Gerente} Não havia, não senhor. Chasseur não tem tradução.

\repl{Figueiredo} Ora essa! Chasseur é\ldots{}

\repl{O Gerente} É caçador, mas chasseur de hotel não tem equivalente. O Grande
Hotel da Capital Federal é o primeiro no Brasil que se dá ao luxo de ter um
chasseur! -- Mas como ia dizendo\ldots{} a Bahia?\ldots{}

\repl{Figueiredo} Foi lá que tomei predileção pelo gênero. Ah, meu amigo! É
preciso conhecê-las! Aquilo é que são mulatas! No Rio de Janeiro não as há!

\repl{O Gerente} Perdão, mas eu tenho visto algumas que\ldots{}

\repl{Figueiredo} Qual! Não me conte histórias. -- Nós não temos nada! Mulatas na
Bahia!\ldots{}

 Coplas

 -I-

 As mulatas da Bahia
 Têm de certo a primazia
 No capítulo mulher;
 O sultão lá na Turquia
 Se as apanha um belo dia,
 De outro gênero não quer!
 Ai gentes! Que bela,
 Que linda não é
 A fada amarela
 De trunfa enroscada,
 De manta traçada,
 Mimosa chinela
 Levando calçada
 Na ponta do pé!\ldots{}

 - II -

 As formosas georgianas,
 As gentis circassianas
 São as flores dos haréns;
 Mas, seu Lopes, tais sultanas,
 Comparadas às baianas,
 Não merecem dois vinténs!
 Ai! gentes! Que bela, etc\ldots{}

Seu Lopes, você já viu a Mimi Bilontra?

\repl{O Gerente} Isso vi, mas a Mimi Bilontra não é mulata.

\repl{Figueiredo} Não, não é isso. Na Mimi Bilontra há um tipo que gosta de
lançar mulheres. Você sabe o que é lançar mulheres?

\repl{Lopes} Sei, sei.

\repl{Figueiredo} Pois eu também gosto de lançá-las! Mas só mulatas! Tenho
lançado umas poucas!

\repl{Lopes} Deveras?

\repl{Figueiredo} Todas as mulatas bonitas que têm aparecido por aí arrastando
sedas foram lançadas por mim. É a minha especialidade.

\repl{O Gerente} Dou-lhe os meus parabéns.

\repl{Figueiredo} Que quer? Sou solteiro, aposentado, independente: não tenho
que dar satisfações a ninguém. \paren{Outro tom.} Bom: vou dar uma volta antes de
jantar. Não se esqueça de providenciar para que o criado não continue a levar-me café
às seis e cinquenta!

\repl{O Gerente} Vá descansado. A reclamação é muito justa.

\repl{Figueiredo} Até logo! \paren{Sai}.

\repl{O Gerente} \paren{Só.} Gabo-lhe o gosto de lançar mulatas! Imaginem se um tipo
assim tem capacidade para apreciar o Grande Hotel da Capital Federal!

\newscenenamed{Cena VI}
\stagedir{O Gerente, Lola, depois Gouveia, depois O Gerente}

\repl{Lola} \paren{Entrando.} Então? Estou esperando há uma hora!\ldots{}

\repl{O Gerente} Admirou o nosso plafond?

\repl{Lola} Não admirei nada! O que eu quero é falar ao Gouveia!

\repl{O Gerente} Já o mandei chamar. \paren{Vendo o Gouveia que desce a escada.} E ele
aí vem descendo a escada. \paren{À parte.} Pois a esta não se me dava de lançá-la.
\paren{Sai.}

\repl{Gouveia} \paren{Que tem descido.} Que vieste fazer? Não te disse que não me
procurasses aqui? Este hotel\ldots{}

\repl{Lola} Bem sei: não admite senhoras que não estejam acompanhadas; mas tu
não me apareceste ontem nem anteontem, e quando tu não me apareces, dir-se-ia
que eu enlouqueço! Como te amo, Gouveia! \paren{Abraça-o.}

\repl{Gouveia} Pois sim, mas não dês escândalo! Olha o chasseur. \paren{O chasseur tem
efetivamente descido a escada, desaparecendo por qualquer um dos lados.}

\repl{Lola} Então? A primeira dúzia?

\repl{Gouveia} Tem continuado a dar que faz gosto! 5\ldots{} 11\ldots{} 9\ldots{} 5\ldots{} Ontem
saiu o 5 três vezes seguidas!

\repl{Lola} Continuas então em maré de felicidade?

\repl{Gouveia} Uma felicidade brutal!\ldots{} Tanto assim, que tinha já preparado
este envelope para ti\ldots{}

\repl{Lola} Oh! dá cá! dá cá!\ldots{}

\repl{Gouveia} Pois sim, mas com uma condição: vai para casa, não estejas aqui.

\repl{Lola} \paren{Tomando o envelope.} Oh! Gouveia, como eu te amo! Vais hoje jantar
comigo, sim?

\repl{Gouveia} Vou, contanto que saia cedo. É preciso aproveitar a sorte! Tenho
certeza de que a primeira dúzia continuará hoje a dar!

\repl{Lola} \paren{Com entusiasmo.} Oh! Meu amor!\ldots{} \paren{Quer abraçá-lo.}

\repl{Gouveia} Não! Não!\ldots{} Olha o gerente!\ldots{}

\repl{Lola} Adeus! \paren{Sai muito satisfeita.}

\repl{O Gerente} \paren{Que tem entrado, à parte.} Vai contente! Aquilo é que deu a
tal primeira dúzia! \paren{Inclinando-se diante de Gouveia.} Doutor\ldots{}

\repl{Gouveia} Quando aqui vier esta senhora, o melhor é dizer-lhe que não
estou. É uma boa rapariga, mas muito inconveniente.

\repl{O Gerente} Vou transmitir essa ordem ao porteiro, porque eu posso não
estar na ocasião. \paren{Sai.}

\newscenenamed{Cena VII}

\repl{Gouveia} \paren{Só.} É adorável esta espanhola, isso é\ldots{} não choro uma boa
dúzia de contos de réis gastos com ela, e que, aliás, não me custaram a ganhar\ldots{}
mas tem um defeito: é muito colante\ldots{} Estas ligações são o diabo\ldots{} Mas como
acabar com isto? Ah! Se a Quinota soubesse! Pobre Quinota! Deve estar queixosa de
mim\ldots{} Oh! Os tempos mudaram\ldots{} Quando estive em Minas era um simples caixeiro de
cobranças\ldots{} É verdade que hoje nada sou, porque um jogador não é coisa
nenhuma\ldots{} mas ganho dinheiro, sou feliz, muito feliz! A Quinota, no final
das contas, é uma roceira\ldots{} mas tão bonita! E daí, quem sabe? -- talvez já se tivesse
esquecido de mim.

\newscenenamed{Cena VIII}
\stagedir{Gouveia, Pinheiro, depois O Gerente}

\repl{Pinheiro} \paren{Entrando.} Oh! Gouveia!

\repl{Gouveia} Oh! Pinheiro! Que andas fazendo?

\repl{Pinheiro} Venho a mandado do patrão falar com um sujeito que mora neste
hotel\ldots{} Mas que luxo! Como estás abrilhantado! Vejo que as coisas têm te
corrido às mil maravilhas!

\repl{Gouveia} \paren{Muito seco.} Sim\ldots{} deixei de ser caixeiro\ldots{} Embirrava com isso
de ir a qualquer parte a mandado do patrão\ldots{} Atirei-me a umas tantas
especulações\ldots{} Tenho arranjado para aí uns cobres\ldots{}

\repl{Pinheiro} Vê-se\ldots{} Estás outro, completamente outro!

\repl{Gouveia} Devo lembrar-te que nunca me viste sujo.

\repl{Pinheiro} Sujo não digo\ldots{} mas vamos lá, já te conheci pau de laranjeira!
Por sinal que\ldots{}

\repl{Gouveia} Por sinal que uma vez me emprestaste dez mil-réis. Fazes bem em
lembrar-me essa dívida.

\repl{Pinheiro} Eu não te lembrei coisa nenhuma!

\repl{Gouveia} Aqui tens vinte mil-réis. Dou-te dez de juros.

\repl{Pinheiro} Vejo que tens a esmola fácil, mas -- que diabo! -- guarda o teu
dinheiro e não o dês a quem to não pede. Fico apenas com os dez mil-réis que te
emprestei com muita vontade -- e sem juros. Quando precisares deles, vem buscá-los. Cá
ficam.

\repl{Gouveia} Oh! Não hei de precisar, graças a Deus!

\repl{Pinheiro} Homem, quem sabe? O mundo dá tantas voltas!

\repl{Gouveia} Adeus, Pinheiro. \paren{Sai pela esquerda.}

\repl{Pinheiro} Adeus, Gouveia. \paren{Só.} Umas tantas especulações\ldots{} Bem sei quais
são elas\ldots{} Pois olha, meu figurão, não te desejo nenhum mal, mas conto que
ainda hás de vir buscar estes dez mil-réis, que ficam de prontidão.

\repl{O Gerente} \paren{Entrando.} Deseja alguma coisa?

\repl{Pinheiro} Sim, senhor, falar a um hóspede\ldots{} Eu sei onde é, não se
incomode. \paren{Sobe a escada e desaparece.}

\repl{O Gerente} \paren{Só.} E lá vai sem dar mais cavaco! Esta gente há de custar-lhe
habituar-se a um hotel de primeira ordem como é o Grande Hotel da Capital
Federal!

\newscenenamed{Cena IX}
\stagedir{Criadas.}

\paren{A família traz maletas, trouxas, embrulhos, etc.}

\repl{O Gerente} Olá! Temos hóspedes! \paren{Chamando.} Chasseur! Vá chamar gente! \paren{O
chasseur aparece e desaparece, e pouco depois volta com alguns criados e
criadas.}

\repl{Eusébio} \paren{Entrando à frente da família, fechando uma enorme carteira.} Ave
Maria! Trinta mil-réis pra nos trazê da estação da estrada de ferro até
aqui. Esta gente pensa que dinheiro se cava! \paren{Aperta a mão do gerente. O resto da
família imita-o, apertando também a mão ao chasseur e à criadagem.} Deus Nosso
Sinhô esteje nesta casa!\ldots{} \paren{Vai pagar aos carregadores, que saem.}

\repl{Fortunata} É um casão!

\repl{Quinota} Um palácio!

\repl{Juquinha} Eu tou com fome! Quero jantá!

\repl{Benvinda} Espera, nhô Juquinha!

\repl{Fortunata} Menino, não começa a reiná!

\repl{O Gerente} Desejam quartos?

\repl{Eusébio} Sim sinhô!\ldots{} Mas antes disso deixe lhe dizê quem sô.

\repl{O Gerente} Não é preciso. O seu nome será escrito no registro dos
hóspedes.

\repl{Eusébio} Pois sim, sinhô, mas ouça\ldots{}

 Coplas-Lundu

 Eusébio

 -I-

 Sinhô, eu sou fazendeiro
 Em São João do Sabará,
 E venho ao Rio de Janeiro
 De coisas grave tratá.

 Ora aqui está!

 Tarvez leve um ano inteiro
 Na Capitá Federá!

 Coro

 Ora aqui está! etc\ldots{}

 Eusébio

 - II -

 Apareceu um janota
 Em São João do Sabará;
 Pediu a mão de Quinota
 E vei’ se embora pra cá.

 Ora aqui está!

 Hei de achá esse janota
 Na Capitá Federá!

 Coro

 Ora aqui está, etc\ldots{}

Esta é minha muié, Dona Fortunata.

\repl{Fortunata} Uma sua serva. \paren{Faz uma mesura.}

\repl{O Gerente} Folgo de conhecê-la, minha senhora. E esta moça? É sua
filha?\ldots{}

\repl{Eusébio} Nossa.

\repl{Fortunata} Nome dela é Quinota\ldots{} Joquina\ldots{} mas a gente chama ela de
Quinota.

\repl{Quinota} Cala a boca, mamãe. O senhor não perguntou nada.

\repl{Eusébio} É muito estruída. Teve três professô\ldots{} Este é meu fio\ldots{}
\paren{Procurando Juquinha.} Onde tá ele? Juquinha! \paren{Vai buscar pela mão o fio, que
traquinava ao fundo.} Tá aqui ele. Tem cabeça -- qué vê? Diz um verso, Juquinha!

\repl{Juquinha} Ora, papai!

\repl{Fortunata} Diz um verso, menino! Não ouve teu pai tá mandando?

\repl{Juquinha} Ora, mamãe!

\repl{Quinota} Diz o verso, Juquinha! Você parece tolo!\ldots{}

\repl{Juquinha} Não digo!

\repl{Benvinda} Nhô Juquinha, diga aquele de lá vem a lua saindo!

\repl{Juquinha} Eu não sei verso!

\repl{Fortunata} Diz o verso, diabo! \paren{Dá-lhe um beliscão. Juquinha faz grande
berreiro.}

\repl{Eusébio} \paren{Tomando o filho e acariciando-o.} Tá bão! não chora! não chora!
\paren{Ao gerente.} Tá muito cheio de vontade\ldots{} Ah! mas eu hei de endireitá ele!

\repl{O Gerente} Não será melhor subirem para os seus quartos?

\repl{Eusébio} Sim, sinhô. \paren{Examinando em volta de si.} O hotezinho parece bem
bão.

\repl{O Gerente} O hotelzinho? Um hotel que seria de primeira ordem em qualquer
parte do mundo! O Grande Hotel da Capital Federal!

\repl{Fortunata} E diz que é só de família.

\repl{O Gerente} Ah! Por esse lado podem ficar tranquilos.

\newscenenamed{Cena X}
\stagedir{Os mesmos, Figueiredo}

\paren{Figueiredo volta; examina os circunstantes e mostra-se impressionado por
Benvinda, que repara nele.}

\repl{O Gerente} \paren{Aos criados.} Acompanhem estas senhoras e estes senhores\ldots{}
para escolherem os seus quartos à vontade. \paren{Vai saindo e passa por perto de
Figueiredo.}

\repl{Figueiredo} \paren{Baixinho.} Que boa mulata, seu Lopes! \paren{O gerente sai.}

\repl{Os Criados e Criadas} \paren{Tomando as malas e embrulhos.} Façam favor!\ldots{}
Venham!\ldots{} Subam!\ldots{}

\repl{Eusébio} \paren{Perto da escada.} Suba, Dona Fortunata! Sobe, Quinota! Sobe,
Juquinha! \paren{Todos sobem.} Vamo! \paren{Sobe também.} Sobe, Benvinda! \paren{Quando
Benvinda vai subindo, Figueiredo dá-lhe um pequeno beliscão no braço.}

\repl{Figueiredo} Adeus, gostosura!

\repl{Benvinda} Ah! Seu assanhado! \paren{Sobe.}

\repl{O Gerente} \paren{Que entrou e viu.} Então, que é isso, Sr. Figueiredo? Olhe que
está no Grande Hotel da Capital Federal!

\repl{Figueiredo} Ah! Seu Lopes, aquela hei de eu lançá-la! \paren{Sobe a escada.}

\repl{O Gerente} \paren{Só.} Queira Deus não vá arranjar uma carga de pau do
fazendeiro!

\paren{Sai, Mutação.}

\quadro{Quadro II}

\paren{Corredor. Na parede uma mão pintada, apontando para este letreiro:
“Agência de alugar casas. Preço de cada indicação, R\$ 5.000, pagos adiantados.” Ao
fundo um banco, encostado à parede.}

\newscenenamed{Cena I}
\stagedir{Vítimas, entrando furiosas da esquerda, depois, Mota, Figueiredo}

 Coro

 Que ladroeira!
 Que maroteira!
 Que bandalheira!
 Pasmado estou!
 Viu toda a gente
 Que o tal agente
 Cinicamente
 Nos enganou!

\repl{Mota} \paren{Entrando da esquerda também muito zangado.} Cinco mil-réis deitados
fora!\ldots{} Cinco mil-réis roubados!\ldots{} Mas deixem estar que\ldots{} \paren{Vai saindo e
encontra-se com Figueiredo, que entra da direita.}

\repl{Figueiredo} Que é isto, seu Mota? Vai furioso!

\repl{Mota} Se lhe parece que não tenho razão! Esta agência indica onde há casas
vazias por cinco mil-réis.

\repl{Figueiredo} Casas por cinco mil-réis? Barata feira!

\repl{Mota} Perdão; indica por cinco mil-réis\ldots{}

\repl{Figueiredo} \paren{Sorrindo.} Bem sei, e é isso justamente o que aqui me traz.
Resolvi deixar o Grande Hotel da Capital Federal e montar casa. Esgotei todos os
meios para obter com que naquele suntuoso estabelecimento me levassem o café ao
quarto às sete horas em ponto. Como não estou para me zangar todas as
manhãs, mudo-me. O diabo é que não acho casa que me sirva. Dizem-me que nesta
agência\ldots{}

\repl{Mota} Volte, seu Figueiredo, volte, se não quer que lhe aconteça o mesmo
que me sucedeu e tem sucedido a muita gente! Indicaram-me uma casa no morro do
Pinto, com todas as acomodações que eu desejava\ldots{} Você sabe o que é subir ao
morro do Pinto?

\repl{Figueiredo} Sei. Já lá subi uma noite por causa de uma trigueira.

\repl{Mota} Pois eu subi ao morro do Pinto e encontrei a casa ocupada.

\repl{Figueiredo} Foi justamente o que me aconteceu com a trigueira.

\repl{Mota} Volto aqui, faço ver que a indicação de nada me serviu e peço que me
restituam os meus ricos cinco mil-réis. Respondem-me que a agência nada me
restitui, porque não tem culpa de que a casa se tivesse alugado.

\repl{Figueiredo} E não lhe deram outra indicação?

\repl{Mota} Deram. Cá está. \paren{Tira um papel.}

\repl{Figueiredo} \paren{À parte.} Vou aproveitá-la!

\repl{Mota} Mas provavelmente vale tanto como a outra!

\repl{Figueiredo} \paren{Depois de ler.} Oh!

\repl{Mota} Que é?

\repl{Figueiredo} Esta agora não é má! Rua dos Arcos no 100. Indicaram a casa da
Minervina!

\repl{Mota} Que Minervina?

\repl{Figueiredo} Uma trigueira.

\repl{Mota} A do morro do Pinto?

\repl{Figueiredo} Não. Outra. Outra que eu lancei há quatro anos. Mudou-se para
a Rua dos Arcos não há oito dias.

\repl{Mota} Então? Quando lhe digo!

\repl{Figueiredo} Oh! As trigueiras têm sido o meu tormento!

\repl{Mota} As trigueiras são\ldots{}

\repl{Figueiredo} As mulatas. Eu digo trigueiras por ser menos rebarbativo\ldots{}
Ainda agora está lá no hotel uma família de Minas que trouxe consigo uma
mucama\ldots{} Ah, seu Mota\ldots{}

\repl{Mota} Pois atire-se!

\repl{Figueiredo} Não tenho feito outra coisa, mas não me tem sido possível
encontrá-la a jeito. Só hoje consegui meter-lhe uma cartinha na mão, pedindo-lhe que
vá ter comigo ao Largo da Carioca. Quero lançá-la!

\repl{Mota} Mas vamos embora! Estamos numa caverna!

\repl{Figueiredo} E é tudo assim no Rio de Janeiro\ldots{} Não temos nada, nada,
nada\ldots{} Vamos\ldots{}

\newscenenamed{Cena II}
\stagedir{Os mesmos, uma Senhora, depois um Proprietário}

\repl{A Senhora} \paren{Vindo da esquerda.} Um desaforo! Uma pouca vergonha!

\repl{Mota} Foi também vítima, minha senhora?

\repl{A Senhora} Roubaram-me cinco mil-réis!

\repl{Figueiredo} Também -- justiça se lhes faça -- eles nunca roubam mais do que
isso!

\repl{A Senhora} Indicaram-me uma casa\ldots{} Vou lá, e encontro um tipo que me
pergunta se quero um quarto mobiliado! Vou queixar-me\ldots{}

\repl{Mota} Ao bispo, minha senhora! Queixemo-nos todos ao bispo!\ldots{} \paren{O
Proprietário entra e vai atravessando a cena da direita para a esquerda, cumprimentando
as pessoas presentes.}

\repl{Figueiredo} \paren{Embargando-lhe a passagem.} Não vá lá, não vá lá, meu caro
senhor! Olhe que lhe roubam cinco mil-réis.

\repl{O Proprietário} Nada! Eu não pretendo casa. O que eu quero é alugar a
minha.

\repl{Os Três} Ah! \paren{Cercam-no.}

\repl{A Senhora} Talvez não seja preciso ir à agência. Eu procuro uma casa.

\repl{Mota} E eu.

\repl{Figueiredo} E eu também.

\repl{A Senhora} A sua onde é?

\repl{O Proprietário} Se querem a indicação, venham cinco mil-réis de cada um!

\repl{Os Três} Hein?

\repl{O Proprietário} Ora essa! Por que é que a agência há de cobrar e eu não?

\repl{Mota} A agência paga impostos e é, apesar dos pesares, um estabelecimento
legalmente autorizado.

\repl{O Proprietário} Bem; como eu não sou um estabelecimento legalmente
autorizado, dou a indicação por três mil-réis.

\repl{Mota} Guarde-a!

\repl{Figueiredo} Dispenso-a!

\repl{A Senhora} Aqui tem os três mil-réis. A necessidade é tão grande que me
submeto a todas as patifarias!

\repl{O Proprietário} \paren{Calmo.} Patifaria é forte, mas como a senhora paga\ldots{}
\paren{Guarda o dinheiro.}

\repl{A Senhora} Vamos!

\repl{O Proprietário} A minha casa é na Praia Formosa.

\repl{Mota e Figueiredo} Que horror!

\repl{O Proprietário} Um sobrado com três janelas de peitoril. Os baixos estão
ocupados por um açougue.

\repl{Mota e Figueiredo} Xi!

\repl{A Senhora} Deve haver muito mosquito!

\repl{O Proprietário} Mosquitos há em toda a parte. Sala, três quartos, sala de
jantar, despensa, cozinha, latrina na cozinha, água, gás, quintal, tanque de lavar
e galinheiro\ldots{}

\repl{A Senhora} Não tem banheiro?

\repl{O Proprietário} Terá, se o inquilino o fizer. A casa foi pintada e forrada
há dez anos; está muito suja. Aluguel, duzentos e cinquenta mil-réis por mês.
Carta de fiança passada por negociante matriculado, trezentos mil-réis de posse e
contrato por três anos. O imposto predial e de pena d’água é pago pelo inquilino.

\repl{A Senhora} Com os três mil-réis que me surrupiou compre uma corda e
enforque-se! \paren{Sai.}

\repl{Figueiredo} \paren{Enquanto ela passa.} Muito bem respondido, minha senhora!

\repl{Mota} Com efeito!

\repl{O Proprietário} Mas os senhores\ldots{}

\repl{Figueiredo} \paren{Tirando um apito do bolso.} Se diz mais uma palavra, apito
para chamar a polícia.

\repl{O Proprietário} Ora vá se catar! \paren{Vai saindo.}

\repl{Figueiredo} Que é? Que é?\ldots{} \paren{Segue-o.}

\repl{O Proprietário} Largue-me!

\repl{Figueiredo} Este tipo merecia uma lição! \paren{Empurrando-o.} Vamos embora!
Deixá-lo!

\repl{Mota} Vamos!

\repl{O Proprietário} \paren{Voltando e avançando para eles.} Mas eu\ldots{}

\repl{Os Dois} Hein? \paren{Atiram-se ao Proprietário, que foge, desaparecendo pela
esquerda. Mota e Figueiredo encolhem os ombros e saem pela direita,
encontrando-se à porta com Eusébio, que entra. O Proprietário volta e, enganado, dá com
o guarda-chuva em Eusébio, e foge. Eusébio tira o casaco para persegui-lo.}

\newscenenamed{Cena III}
\stagedir{Eusébio, só; depois, Fortunata, Quinota, Juca, Benvinda}

\repl{Eusébio} Tratante! Se eu te agarro, tu havia de vê o que é purso de
mineiro! Que terra esta, minha Nossa Senhora, que terra esta em que um home apanha sem
sabê por quê? Mas onde ficô esta gente? Aquela Dona Fortunata não presta pra
subi escada! \paren{Indo à porta da direita.} Entra! É aqui! \paren{Entra a família.}

\repl{Fortunata} \paren{Entrando apoiada no braço de Quinota.} Deixe-me arrespirá um
bocadinho! Virge Maria! Quanta escada!

\repl{Eusébio} E ainda é no outro andá! Olhe! \paren{Aponta para o letreiro.}

\repl{Juca} \paren{Vendo Eusébio a vestir o casaco.} Mamãe, papai se despiu!

\repl{As Três} É verdade!

\repl{Eusébio} Tirei o casaco pra brigá! Não foi nada.

\repl{Fortunata} Não posso mais co’esta história de casa!

\repl{Quinota} É um inferno!

\repl{Benvinda} Uma desgraça!

\repl{Eusébio} Paciência. Nós não podemo ficá naquele hoté\ldots{} Aquilo é luxo
demais e custa os oio da cara! Como temo que ficá argum tempo na Capitá Federá, o
mió é precurá uma casa. A gente compra uns traste e alguma louça\ldots{} Benvinda vai
pra cozinha\ldots{}

\repl{Benvinda} \paren{À parte.} Pois sim!

\repl{Eusébio} E Quinota trata dos arranjo da casa.

\repl{Quinota} Mas a coisa é que não se arranjá casa.

\repl{Eusébio} Desta vez tenho esperança de arranjá. Diz que essa agência é
munto séria. Vamo!

\repl{Fortunata} Eu não subo mais escada! Espero aqui no corredô.

\repl{Eusébio} Tudo fica! Eu vou e vorto! \paren{Vai saindo.}

\repl{Juca} \paren{Chorando e batendo o pé.} Eu quero i cum papai! eu quero i cum
papai!

\repl{Fortunata} Pois vai, diabo!\ldots{}

\repl{Eusébio} Vem! vem! não chora! Dá cá a mão! \paren{Sai com o filho pela
esquerda.}

\newscenenamed{Cena IV}
\stagedir{Fortunata, Quinota e Benvinda}

\repl{Quinota} Mamãe, por que não se senta naquele banco?

\repl{Fortunata} Ah! é verdade! não tinha arreparado. Estou moída. \paren{Senta-se e
fecha os olhos.}

\repl{Benvinda} Sinhá vai dromi.

\repl{Quinota} Deixa.

\repl{Benvinda} \paren{Em tom confidencial.} Ó nhanhã?

\repl{Quinota} Que é?

\repl{Benvinda} Nhanhã arreparou naquele home que ia descendo pra baixo quando a
gente vinha subindo pra cima?

\repl{Quinota} Não. Que homem?

\repl{Benvinda} Aquele que mora lá no hoté em que a gente mora\ldots{}

\repl{Quinota} Olha mamãe! \paren{D. Fortunata ressona.}

\repl{Benvinda} Já está dromindo. Nhanhã arreparou?

\repl{Quinota} Reparei, sim.

\repl{Benvinda} Sabe o que ele fez hoje de menhã? Me meteu esta carta na mão!

\repl{Quinota} Uma carta? E tu ficaste com ela? Ah! Benvinda! \paren{Pausa.} É para
mim?

\repl{Benvinda} Pra quem havera de sê?

\repl{Quinota} Não está sobrescritada.

\repl{Benvinda} \paren{À parte, enquanto Quinota se certifica de que Fortunata dorme.} --
Bem sei que a carta é minha\ldots{} O que eu quero é que ela leia pra eu ouvi.

\repl{Quinota} Dá cá. \paren{Toma a carta e vai abri-la, mas arrepende-se.} Que
asneira ia eu fazendo!

 Duetino

 Quinota

 Eu gosto do seu Gouveia;
 Com ele quero casar;
 O meu coração anseia
 Pertinho dele pulsar;

 Portanto a epístola
 Não posso abrir!
 Sérios escrúpulos
 Devo sentir!

 Benvinda

 Está longe seu Gouveia;
 Aqui agora não vem\ldots{}
 Abra a carta, a carta leia\ldots{}
 Não digo nada a ninguém!
 
 Quinota
 
 Não! não! a epístola
 Não posso abrir!
 Sérios escrúpulos
 Devo sentir!
 Entretanto, é verdade
 Que tenho tal ou qual curiosidade,
 Mamãe -- eu tremo! --
 Dormindo está?

 Benvinda

 Sim, e ela memo
 Respondeu já. \paren{Fortunata tem ressonado.}

 Quinota

 É feio,
 Mas que importa? Abro e leio! \paren{Abre a carta.}

 Juntas

 Quinota Benvinda

 Eu sou curiosa! É bem curiosa!
 Não sei me conter! Não há que dizê!
 A carta amorosa A carta amorosa
 Depressa vou ler! Depressa vai lê!\ldots{}

\repl{Ambas} Uê!\ldots{}

\repl{Quinota} \paren{Lendo a carta.} “Minha bela mulata.”

\repl{Ambas} Uê!\ldots{}

\repl{Quinota} \paren{Lendo.} “Minha bela mulata. Desde que está morando neste hotel,
tenho debalde procurado falar-te. Tu não passas de uma simples mucama\ldots{}”
\paren{Dá a carta a Benvinda.} A carta é para ti. \paren{À parte.} Fui bem castigada.

\repl{Benvinda} Leia pra eu ouvi, nhanhã.

\repl{Quinota} \paren{Lendo.} “Se queres ter uma posição independente e uma casa
tua\ldots{}”

\repl{Benvinda} Gentes!

\repl{Quinota} “\ldots{}deixa o hotel, e vai ter comigo terça-feira, às quatro horas
da tarde, no Largo da Carioca, ao pé da charutaria do Machado.”

\repl{Benvinda} \paren{À parte.} Terça-feira\ldots{} quatro hora\ldots{}

\repl{Quinota} “Nada te faltará. Eu chamo-me Figueiredo.”

\repl{Benvinda} Rasga essa carta, nhanhã! Veja só que sem-vergonhice de home!

\repl{Quinota} \paren{Rasgando a carta.} Se papai soubesse\ldots{}

\repl{Benvinda} \paren{À parte.} Figueiredo\ldots{}

\newscenenamed{Cena V}
\stagedir{As mesmas, Eusébio, Juquinha}

\repl{Eusébio} Já tenho uma indicação!

\repl{D. Fortunata} \paren{Despertando.} Ah! quase pego no sono! \paren{Erguendo-se.} Já
temo casa?

\repl{Eusébio} Parece. O dono dela é o home com quem eu briguei indagorinha.
Tinha me tomado por outro. Vamo à Praia Fermosa pra vê se a casa serve.

\repl{D. Fortunata} Ora graça!

\repl{Benvinda} \paren{À parte.} Perto da charutaria.

\repl{Eusébio} \paren{Que ouviu.} Não sei se é perto da charutaria, mas diz que o logá
é aprazive; a casa munto boa\ldots{} Fica pro cima de um açougue, o que qué dizê
que nunca fartará carne! Vamo!

\repl{Quinota} É muito longe?

\repl{Eusébio} É; mas a gente vai no bonde\ldots{}

\repl{Benvinda} \paren{À parte.} Largo da Carioca\ldots{}

\repl{Eusébio} \paren{Que ouviu.} Que Largo da Carioca! É os bondinho da Rua Direita!
Vamo!

\repl{Juquinha} Eu quero i co Benvinda!

\repl{Fortunata} Vai vai co Benvinda! É perciso munta paciência para aturá este
demônio deste menino! \paren{Saem todos.}

\repl{Benvinda} \paren{Saindo por último, com Juquinha pela mão.} Terça-feira\ldots{}
quatro hora\ldots{} Figueiredo\ldots{}

\newscenenamed{Cena VI}

\repl{O Proprietário} \paren{Vindo da esquerda.} Queira Deus que o mineiro fique com a
casa\ldots{} mas não lhe dou dois meses para apanhar uma febre palustre! \paren{Sai
pela direita. Mutação.}

\quadro{Quadro III}

\paren{O Largo da Carioca. Muitas pessoas estão à espera de bonde. Outras
passeiam.}

\newscenenamed{Cena I}
\stagedir{Figueiredo, Rodrigues, Pessoas do Povo}

 Coro

 À espera do bonde elétrico
 Estamos há meia hora!
 Tão desusada demora
 Não sabemos explicar!
 Talvez haja algum obstáculo,
 Algum descarrilamento,
 Que assim possa o impedimento
 Da linha determinar!

\paren{Figueiredo e Rodrigues vêm ao proscênio. Rodrigues está carregado de
pequenos embrulhos.}

\repl{Rodrigues} Que estopada, hein?

\repl{Figueiredo} É tudo assim no Rio de Janeiro! Este serviço de bondes é
terrivelmente malfeito! Não temos nada, nada, absolutamente nada!

\repl{Rodrigues} Que diabo! Não sejamos tão exigentes! Esta companhia não serve
mal. Não é por culpa dela esse atraso. Ali na estação me disseram. Na Rua
do Passeio está uma fila de bondes parados diante de um enorme caminhão, que
levava uma máquina descomunal não sei para onde, e quebrou as rodas. É ter
um pouco de paciência.

\repl{Figueiredo} Eu felizmente não estou à espera de bonde, mas de coisa
melhor. \paren{Consultando o relógio.} Estamos na hora.

\repl{Rodrigues} Ah! Seu maganão\ldots{} alguma mulher\ldots{} Você nunca há de tomar
juízo!

\repl{Figueiredo} Uma trigueira\ldots{} uma deliciosa trigueira!

\repl{Rodrigues} Continua então a ser um grande apreciador de mulatas?

\repl{Figueiredo} Continuo, mas eu digo trigueiras por ser menos rebarbativo.

\repl{Rodrigues} Pois eu cá sou o homem da família, porque entendo que a família
é a pedra angular de uma sociedade bem organizada.

\repl{Figueiredo} Bonito!

\repl{Rodrigues} Reprovo incondicionalmente esses amores escandalosos, que
ofendem a moral e os bons costumes.

\repl{Figueiredo} Ora, não amola! Eu sou solteiro\ldots{} não tenho que dar
satisfações a ninguém.

\repl{Rodrigues} Pois eu sou casado, e todos os dias agradeço a Deus a santa
esposa e os adoráveis filhinhos que me deu! Vivo exclusivamente para a família.
Veja como eu vou para casa cheio de embrulhos! E é isto todos os dias! Vão aqui
empadinhas, doces, queijo, chocolate andaluza, sorvetes de viagem, o diabo!\ldots{} Tudo
gulodices!\ldots{}

\repl{Figueiredo} \paren{Que, preocupado, não lhe tem prestado grande atenção.} Não
imagina você como estou impaciente! É curioso! Não varia aos quarenta anos
esta sensação esquisita de esperar uma mulher pela primeira vez! Note-se que não
tenho certeza de que ela venha, mas sinto uns formigueiros subirem-me pelas
pernas! \paren{Vendo Benvinda.} Oh! Diabo! Não me engano! Afaste-se, afaste-se, que lá
vem ela!\ldots{}

\repl{Rodrigues} Seja feliz. Para mim não há nada como a família. \paren{Afasta-se e
fica observando de longe.}

\newscenenamed{Cena II}
\stagedir{Os mesmos, Benvinda}

\repl{Benvinda} \paren{Aproximando-se com uma pequena trouxa na mão.} Aqui estou.

\repl{Figueiredo} \paren{Disfarçando a olhar para o céu.} Disfarça, meu bem. \paren{Pausa.}
- Estás pronta a acompanhar-me?

\repl{Benvinda} \paren{Disfarçando e olhando também para o céu.} Sim, sinhô, mas eu
quero sabê se é verdade o que o sinhô disse na sua carta\ldots{}

\repl{Figueiredo} \paren{Disfarçando por ver um conhecido que passa e o cumprimenta.} -
Como passam todos lá por casa? As senhoras estão boas?

\repl{Benvinda} \paren{Compreendendo.} Boas, muito obrigado\ldots{} Sinhá Miloca é que tem
andado com enxaqueca.

\repl{Figueiredo} \paren{À parte.} Fala mal, mas é inteligente.

\repl{Benvinda} O sinhô me dá memo casa pra mim morá?

\repl{Figueiredo} Uma casa muito chique, muito bem mobiliada, e uns vestidos
muito bonitos. \paren{Passa outro conhecido. O mesmo jogo de cena.} Mas por que esta
demora com a minha roupa lavada?

\repl{Benvinda} É porque choveu munto\ldots{} não se pôde corá\ldots{} \paren{Outro tom.} Não me
fartará nada?

\repl{Figueiredo} Nada! Não te faltará nada! Mas aqui não podemos ficar. Passa
muita gente conhecida, e eu não quero que me vejam contigo enquanto não tiveres
outra encadernação. Acompanha-me e toma o mesmo bonde que eu. \paren{Vai se afastando
pela direita e Benvinda também.} Espera um pouco, para não darmos na vista.
\paren{Passa um conhecido.} Adeus, hein? Lembranças à Baronesa.

\repl{Benvinda} Sim, sinhô, farei presente. \paren{Figueiredo afasta-se, disfarçando,
e desaparece pela direita. Durante a fala que se segue, Rodrigues pouco a
pouco se aproxima de Benvinda.} Ora! Isto sempre deve sê mió que aquela vida enjoada
lá da roça! Ah! seu Borge! seu Borge! Você abusou porque era feitô lá da fazenda;
fez o que fez e me prometeu casamento\ldots{} Mas casará ou não? Sinhá e nhanhã ondem
ficá danada\ldots{} Pois que fique! Quero a minha liberdade! \paren{Vai afastar-se na
direção que tomou Figueiredo e é abordada pelo Rodrigues, que não a tem perdido de
vista um momento.}

\repl{Rodrigues} Adeus, mulata!

\repl{Benvinda} Viva!

\repl{Rodrigues} \paren{Disfarçando.} Dá-me uma palavrinha?

\repl{Benvinda} Agora não posso.

\repl{Rodrigues} Olhe, aqui tem o meu cartão\ldots{} Se precisar de um homem sério\ldots{}
De um homem que é todo família\ldots{}

\repl{Benvinda} \paren{Tomando disfarçadamente o cartão.} Pois sim. \paren{Saindo, à parte}
- O que não farta é home\ldots{} Assim queira uma muié\ldots{} \paren{Sai.}

\repl{Rodrigues} \paren{Consigo.} Sim\ldots{} lá de vez em quando\ldots{} para variar\ldots{} não
quero dizer que\ldots{} \paren{Outro tom.} E o maldito bonde que não chega! \paren{Afasta-se pela
direita e desaparece.}

\newscenenamed{Cena III}
\stagedir{Lola, Mercedes, Blanchette, Dolores, Gouveia, Pessoas do Povo}

\paren{As quatro mulheres entram da esquerda, trazendo Gouveia quase à força.}

 Quinteto

 As Mulheres
 Ande pra frente,
 Faça favor!
 Está filado,
 Caro senhor!
 Queira ou não queira,
 Daqui não sai!
 Janta conosco!
 Conosco vai!

 Lola

 Há tantos dias
 Tu não me vias,
 E agora qu’rias
 Deixar-me só!
 A tua Lola,
 Meu bem, consola!
 Dá-me uma esmola!
 De mim, tem dó!

 As Outras

 Há tantos dias
 Tu não a vias,
 E agora qu’rias
 Deixá-la só!
 A tua Lola,
 Meu bem, consola!
 Dá-lhe uma esmola!
 tem dó, tem dó!

 Gouveia

 Não me aborreçam!
 Não me enfureçam!
 Desapareçam!
 Quero estar só!
 Isto me amola!
 Perco esta bola!
 Querida Lola,
 De mim tem dó!

 Lola

\repl{Ingrato} já não me queres!
 Tu já não gostas de mim!

 Gouveia

 São terríveis as mulheres!
 Gosto de ti, gosto, sim!
 Mas não serve este lugar
 Pra tais assuntos tratar!

 Lola

 Então daqui saiamos!
 Vamos!

 Todas

 Vamos!
 Há tantos dias, etc\ldots{}

\repl{Lola} Vamos a saber: por que não tens aparecido?

\repl{Gouveia} Tu bem sabes por quê.

\repl{Lola} A primeira dúzia falhou?

\repl{Gouveia} Oh! não! Ainda não falhou, graças a Deus, e por isso mesmo é que
não a tenho abandonado noite e dia! Não vês como estou pálido? como tenho as
faces desbotadas e os olhos encovados? É porque já não durmo, é porque já me não
alimento, é porque não penso noutra coisa que não seja a roleta!

\repl{Lola} Mas é preciso que descanses, que te distraias, que espaireças o
espírito. Por isso mesmo exijo que venhas jantar hoje comigo, quero dizer, conosco,
porque, como vês, terei à mesa estas amigas, que tu conheces: a Dolores, a Mercedes
e a Blanchette.

\repl{As Três} Então, Gouveia? Venha, venha jantar!\ldots{}

\repl{Gouveia} Já deve ter começado a primeira banca!

\repl{Lola} Deixa lá a primeira banca! Tenho um pressentimento de que hoje não
dá a primeira dúzia.

\repl{As Três} Então, Gouveia, então? \paren{Querem abraçá-lo.}

\repl{Gouveia} \paren{Esquivando-se.} Que é isto? Vocês estão doidas! Reparem que
estamos no Largo da Carioca!

\repl{Lola} Vem! Não te faças rogado!

\repl{As Três} \paren{Implorando.} Gouveia!\ldots{}

\repl{Gouveia} Pois sim, vamos lá! Vocês são o diabo!

\repl{Lola} Ai! E o meu leque?! Trouxeste-o, Dolores?

\repl{Dolores} Não.

\repl{Blanchette} Nem eu.

\repl{Mercedes} Tu deixaste-o ficar sobre a mesa, no Braço de Ouro.

\repl{Gouveia} Que foi?

\repl{Lola} Um magnífico leque, comprado, não há uma hora, no Palais-Royal.
Querem ver que o perdi?

\repl{Gouveia} Se queres, vou procurá-lo ao Braço de Ouro.

\repl{Lola} Pois sim, faze-me esse favor. \paren{Arrependendo-se.} Não! se tu vais à
Rua do Ouvidor, és capaz de encontrar lá algum amigo que te leve para o jogo.

\repl{Mercedes} E esta é a hora do recrutamento.

\repl{Lola} Vamos nós mesmas buscar o leque. Fica tu aqui muito quietinho à
nossa espera. É um instante.

\repl{Gouveia} Pois vão e voltem.

\repl{Lola} Vamos! \paren{Sai com as três amigas.}

\newscenenamed{Cena IV}
\stagedir{Gouveia, depois, Eusébio, Fortunata, Quinota e Juquinha}

\repl{Gouveia} Com esta não contava eu. Daí -- quem sabe? -- como ando em maré
de felicidade, talvez seja uma providência lá não ir hoje. \paren{Eusébio entra
descuidado acompanhado pela família, e, ao ver Gouveia, solta um grande grito.}

\repl{Eusébio} Oh! seu Gouveia! \paren{Chamando.} Dona Fortunata! \ldots{} Quinota!\ldots{}
\paren{Cercam Gouveia.}

\repl{As Senhoras e Juquinha} Oh! seu Gouveia! \paren{Apertam-lhe a mão.}

\repl{Eusébio} Seu Gouveia! \paren{Abraça-o.}

\repl{Gouveia} \paren{Atrapalhado.} Sr. Eusébio\ldots{} Minha Senhora\ldots{} Dona Quinota\ldots{} \paren{À
parte.} Maldito encontro!\ldots{}

 Quarteto

 Eusébio, Fortunata, Quinota e Juquinha

 Seu Gouveia, finalmente,
 Seu Gouveia apareceu!
 Seu Gouveia tá presente!
 Seu Gouveia não morreu!

 Eusébio

 Andei por todas as rua,
 Toda a cidade bati;
 Mas de tê notícias sua
 As esperança perdi!

 Quinota

 Mas ao meu anjo da guarda
 Em sonhos dizer ouvi:
 Sossega, que ele não tarda
 A aparecer por aí!

\repl{Todos} Seu Gouveia, finalmente, etc\ldots{}

\repl{Fortunata} Ora, seu Gouveia! o sinhô chegou lá na fazenda feito cometa, e
começou a namorá Quinota. Pediu ela em casamento, veio se embora dizendo
que vinha tratá dos papé, e nunca mais deu siná de si! Isto se faz, seu
Gouveia?

\repl{Quinota} Mamãe\ldots{}

\repl{Eusébio} Cumo Quinota andava apaixonada, coitadinha! que não comia, nem
bebia, nem dromia, nem nada, nós arresorvemo vi le procurá\ldots{} porque le
escrevi três carta que ficou sem resposta\ldots{}

\repl{Gouveia} Não recebi nenhuma.

\repl{Eusébio} Então entreguei a fazenda a seu Borge, que é home em que a gente
pode confiá, e aqui estemo!

\repl{Fortunata} O sinhô sabe que cum moça de família não se brinca\ldots{} Se seu

Eusébio não soubé sê pai, aqui tô eu que hei de sabê sê mãe!

\repl{Quinota} Mamãe, tenha calma\ldots{} seu Gouveia é um moço sério\ldots{}

\repl{Gouveia} Obrigado, Dona Quinota. Sou, realmente, um moço sério, e hei de
justificar plenamente o meu silêncio. Espero ser perdoado.

\repl{Quinota} Eu há muito tempo lhe perdoei.

\repl{Gouveia} \paren{À parte.} Está ainda muito bonita! \paren{Alto.} Onde moram?

\repl{Eusébio} No Grande Hoté da Capitá Federá.

\repl{Gouveia} \paren{À parte.} Oh! diabo! no meu hotel!\ldots{} Mas eu nunca os vi!

\repl{Quinota} Mas andamos à procura de casa: não podemos ficar ali.

\repl{Fortunata} É munto caro.

\repl{Gouveia} Sim, aquilo não convém.

\repl{Eusébio} Mas é munto difice achá casa. Uma agência nos indicou uma, na
Praia Fermosa\ldots{}

\repl{Fortunata} Que chiqueiro, seu Gouveia!

\repl{Eusébio} Paguemo cinco mil-réis pra nos enchê de purga!

\repl{Quinota} E era muito longe.

\repl{Gouveia} Descansem, há de se arranjar casa. \paren{À parte.} E a Lola que não
tarda!

\repl{Eusébio} Cumo diz?

\repl{Gouveia} Nada\ldots{} Mas, ao que vejo, veio toda a família?

\repl{Eusébio} Toda! -- Dona Fortunata\ldots{} Quinota\ldots{} o Juquinha\ldots{}

\repl{Juquinha} A Benvinda.

\repl{Eusébio} Ah! é verdade! nos aconteceu uma desgraça!

\repl{Fortunata} Uma grande desgraça!

\repl{Gouveia} Que foi? Ah! já sei\ldots{} o senhor foi vítima do conto do vigário!

\repl{Eusébio} Eu?!\ldots{} Então eu sô argum matuto? Não sinhô, não foi isso.

\repl{Juquinha} Foi a Benvinda que fugiu!

\repl{Quinota} Cale a boca!

\repl{Juquinha} Fugiu c’um home!

\repl{Eusébio} Cala a boca, menino!

\repl{Juquinha} Foi Quinota que disse!

\repl{Fortunata} Cala a boca, diabo!

\repl{Eusébio} O sinhô se alembra da Benvinda?

\repl{Fortunata} Aquela mulatinha? cria da fazenda?

\repl{Gouveia} Lembra-me.

\repl{Eusébio} Hoje de menhã, a gente se acorda-se\ldots{} precura\ldots{}

\repl{Fortunata} Quê dê Benvinda?

\repl{Gouveia} Pode ser que ainda a encontrem.

\repl{Fortunata} Mas em que estado, seu Gouveia!

\repl{Eusébio} E seu Borge já estava arresorvido a casá cum ela\ldots{} Mas não
fiquemo aqui\ldots{}

\repl{Gouveia} \paren{Inquieto.} Sim, não fiquemos aqui.

\repl{Eusébio} Temo munto que conversá, seu Gouveia. Não quero que Dona
Fortunata diga que não sei sê pai\ldots{} Quero sabê se o sinhô tá ou não tá
disposto a cumpri o que tratô!

\repl{Gouveia} Certamente. Se Dona Quinota ainda gosta de mim\ldots{}

\repl{Quinota} \paren{Baixando os olhos.} Eu gosto.

\repl{Gouveia} Mas vamos! Em caminho conversaremos. São contos largos!

\repl{Eusébio} Vamo jantá lá no hoté.

\repl{Gouveia} No hotel? Não! A linha está interrompida. \paren{À parte.} Era o que
faltava! Ela lá iria! \paren{Alto.} Vamos ao Internacional.

\repl{Eusébio} Onde é isso?

\repl{Gouveia} Em Santa Teresa. Toma-se aqui o bonde elétrico.

\repl{Fortunata} O tá que vai pro cima do arco?

\repl{Gouveia} Sim, senhora.

\repl{Fortunata} Xi!

\repl{Gouveia} Não há perigo. Mas vamos! Vamos! \paren{Dá o braço a Quinota.}

\repl{Fortunata} \paren{Querendo separá-los.} Mas\ldots{}

\repl{Eusébio} Deixe. Isto aqui é moda. A senhora se alembre que não estamo em
São João do Sabará.

\repl{Juquinha} Eu quero i co Quinota!

\repl{Fortunata} Principia! principia! Que menino, minha Nossa Senhora!

\repl{Gouveia} \paren{Vendo Lola.} E lá vamos! Vamos! \paren{Retira-se precipitadamente.}

\repl{Eusébio} Espere aí, seu Gouveia! Ande, Dona Fortunata!

\repl{Juquinha} \paren{Chorando.} Eu quero i co Quinota! \paren{Saem todos a correr pela
direita.}

\newscenenamed{Cena V}
\stagedir{Lola, Mercedes, Dolores, Blanchette, Rodrigues, Pessoas do Povo}

\repl{Lola} Então? O Gouveia? Não lhes disse? Bem me arrependi de o ter deixado
ficar! Não teve mão em si e lá se foi para o jogo!

\repl{Mercedes} Que tratante!

\repl{Dolores} Que malcriado!

\repl{Blanchette} Que grosseirão!

\repl{Lola} E nada de bondes!

\repl{Mercedes} Que fizeste do teu carro?

\repl{Lola} Pois não te disse já que o meu cocheiro, o Lourenço, amanheceu hoje
com uma pontinha de dor de cabeça?

\repl{Blanchette} \paren{Maliciosa.} Poupas muito o teu cocheiro.

\repl{Lola} Coitado! é tão bom rapaz! \paren{Vendo Rodrigues que se tem aproximado aos
poucos.} Olá, como vai você?

\repl{Rodrigues} \paren{Disfarçando.} Vou indo, vou indo\ldots{} Mas que bonito ramilhete
franco-espanhol! A Dolores\ldots{} a Mercedes\ldots{} a Blanchette\ldots{} Viva la gracia!

\repl{Lola} \paren{Às outras.} Uma ideia, uma fantasia: vamos levar este tipo para
jantar conosco?

\repl{As Outras} Vamos! Vamos!

\repl{Blanchette} Substituirá o Gouveia! Bravo!

\repl{Lola} \paren{A Rodrigues.} Você faz-nos um favor? Venha jantar com ramilhete
franco-espanhol!

\repl{Rodrigues} Eu?! Não posso, filha: tenho a família à minha espera.

\repl{Lola} Manda-se um portador à casa com esses embrulhos.

\repl{Mercedes} Os embrulhos ficam, se é coisa que se coma.

\repl{Rodrigues} Vocês estão me tentando, seus demônios!

\repl{Lola} Vamos! anda! um dia não são dias!

\repl{Rodrigues} Eu sou um chefe de família!

\repl{Todas} Não faz mal!

\repl{Rodrigues} Ora, adeus! Vamos! \paren{Olhando para a esquerda.} Ali está um
carro. O próprio cocheiro levará depois um recado à minha santa esposa\ldots{}
disfarcemos\ldots{} Vou alugar o carro. \paren{Sai.}

\repl{Todas} Vamos! \paren{Acompanham-no.}

\repl{Pessoas do Povo} Lá vem afinal um bonde! Tomemo-lo! Avança! \paren{Correm todos.
Música na orquestra até o fim do ato. Mutação.}

\quadro{Quadro IV}

\paren{A passagem de um bonde elétrico sobre os arcos. Vão dentro do bonde entre
outros passageiros, Eusébio, Gouveia, D. Fortunata, Quinota e Juquinha. Ao
passar o bonde em frente ao público, Eusébio levanta-se entusiasmado pela beleza
do panorama.}

\repl{Eusébio} Oh! a Capitá Federá! a Capitá Federá!\ldots{}

\begin{center}
\textsc{pano}
\end{center}

\newact

\quadro{Quadro V}

O Largo de São Francisco

\newscenenamed{Cena I}
\stagedir{Benvinda, Pessoas do Povo, depois Figueiredo}

\paren{Benvinda está exageradamente vestida à última moda e cercada por muitas
pessoas do povo, que lhe fazem elogios irônicos.}

 Coro

 Ai, Jesus! Que mulata bonita!
 Como vem tão janota e faceira!
 Toda a gente por ela palpita!
 Ninguém há que adorá-la não queira!
 Ai, mulata!
 Não há peito que ao ver-te não bata!

 Benvinda

 Vão andando seu caminho,
 Deixe a gente assossegada!

 Coro

 Para ao menos um instantinho!
 Não te mostres irritada!

 Benvinda

 Gentes! meu Deus! que maçada!

 Coro

 Dize o teu nome, benzinho

 Coplas

 Benvinda

 -I-

 Meu nome não digo!
 Não quero, aqui tá!
 Não bulam comigo!
 Me deixem passá!
 Jesus! quem me acode?
 Já vejo que aqui
 As moças não pode
 Sozinha saí!
 Sai da frente,
 Minha gente!
 Sai da frente pro favô!
 Tenho pressa!
 Vou depressa!
 Vou pra Rua do Ouvidô!

 Coro

 Sai da frente!
 Minha gente!
 Sai da frente por favor!
 Vai com pressa!
 Vai depressa!
 Vai à Rua do Ouvidor!

 Benvinda

 - II -

 Não digo o meu nome!
 Não tou de maré!
 Diabo dos home
 Que insurta as muié!
 Quando eu vou sozinha,
 Só ouço, dizê:
 “Vem cá, mulatinha,
 Que eu vou com você!”
 Sai da frente, etc\ldots{}

 Coro

 Sai da frente, etc\ldots{}
 
\paren{Figueiredo aparece e coloca-se ao lado de Benvinda.}

 Figueiredo

 Meus senhores, que é isto?
 Perseguição assim é caso nunca visto!\ldots{}
 Mas saibam que esta fazenda
 Tem um braço que a defenda!

 Benvinda

 Seu Figueiredo
 -Eu tava aqui com munto medo!

 Coro

\repl{} \paren{À meia voz.}
 Este é o marchante\ldots{}
 Deixá-los, pois, no mesmo instante!
 Provavelmente o tipo é tolo,
 E há querer armar um rolo!

\paren{A toda voz, cumprimentando ironicamente Figueiredo.}

 Feliz mortal, parabéns
 Pelo tesouro que tens!
 Ah! ah! ah! ah! ah! ah! ah! ah!
 Mulher mais bela aqui não há!

\paren{Todos se retiram. Durante as cenas que seguem, até o fim do quadro, passam
pessoas do povo.}

\newscenenamed{Cena II}
\stagedir{Figueiredo, Benvinda}

\repl{Figueiredo} \paren{Repreensivo.} Já vejo que há de ser muito difícil fazer
alguma coisa de ti!

\repl{Benvinda} Eu não tenho curpa que esses diabo\ldots{}

\repl{Figueiredo} \paren{Atalhando.} Tens culpa, sim! Em primeiro lugar, essa toalete
é escandalosa! Esse chapéu é descomunal!

\repl{Benvinda} Foi o sinhô que escolheu ele!

\repl{Figueiredo} Escolhi mal! Depois, tu abusas do face-en-main!

\repl{Benvinda} Do\ldots{} do quê?

\repl{Figueiredo} Disto, da luneta! Em francês chama-se face-en-main. Não é
preciso estar a todo o instante\ldots{} \paren{Faz o gesto de quem leva aos olhos o
face-en-main.} Basta que te sirvas disso lá uma vez por outra, e assim, olha, assim, com certo
ar de sobranceria. \paren{Indica.} E não sorrias a todo instante, como uma bailarina\ldots{}
A mulher que sorri sem cessar é como o pescador quando atira a rede: os homens vêm
aos cardumes, como ainda agora! -- E esse andar? Por que gingas tanto? Por que
te remexes assim?

\repl{Benvinda} \paren{Chorosa.} Oh! meu Deus! Eu ando bem direitinha\ldots{} não olho pra
ninguém\ldots{} Estes diabo é que intica comigo. -- Vem cá, mulatinha! Meu bem,
ouve aqui uma coisa!

\repl{Figueiredo} Pois não respondas! Vai olhando sempre para a frente! Não
tires os olhos de um ponto fixo, como os acrobatas, que andam na corda bamba\ldots{}
Olha, eu te mostro\ldots{} Faze de conta que eu sou tu e estou passando\ldots{} Tu és um
gaiato, e me dizes uma gracinha quando eu passar por ti. \paren{Afasta-se, e passa pela frente
de Benvinda muito sério.} Vamos, dize alguma coisa!\ldots{}

\repl{Benvinda} Dizê o quê?

\repl{Figueiredo} \paren{À parte.} Não compreendeu! \paren{Alto.} Qualquer coisa! Adeus, meu
bem! Aonde vai com tanta pressa! Olha o lenço que caiu!

\repl{Benvinda} Ah! bem!

\repl{Figueiredo} Vamos, outra vez. \paren{Repete o movimento.}

\repl{Benvinda} Adeus, seu Figueiredo.

\repl{Figueiredo} Que Figueiredo! Eu agora sou Benvinda! E a propósito: hei de
arranjar-te um nome de guerra.

\repl{Benvinda} De guerra? Uê!\ldots{}

\repl{Figueiredo} Sim, um nome de guerra. É como se diz. Benvinda é nome de
preta velha. Mas não se trata agora disso. Vou passar de novo. Não te esqueças de
que eu sou tu. Já compreendeste?

\repl{Benvinda} Já, sim sinhô.

\repl{Figueiredo} Ora, muito bem! -- Lá vou eu. \paren{Repete o movimento.}

\repl{Benvinda} \paren{Enquanto ele passa.} Ouve uma coza, mulata! Vem cá, meu
coração!\ldots{}

\repl{Figueiredo} \paren{Que tem passado imperturbável.} Viste? Não se dá troco!
Arranja-se um olhar de mãe de família! E diante desse olhar, o mais atrevido se
desarma! -- Vamos! anda um bocadinho até ali! Quero ver se aprendeste alguma coisa!

\repl{Benvinda} Sim, sinhô. \paren{Anda.}

\repl{Figueiredo} Que o quê! Não é nada disso! Não é preciso fazer projeções do
holofote para todos os lados! Assim, olha\ldots{} \paren{Anda.} Um movimento gracioso
e quase imperceptível dos quadris\ldots{}

\repl{Benvinda} \paren{Rindo.} Que home danado!

\repl{Figueiredo} É preciso também corrigir o teu modo de falar, mas a seu tempo
trataremos desse ponto, que é essencial. Por enquanto o melhor que tens a
fazer é abrir a boca o menor número de vezes possível, para não dizeres home em vez
de homem e quejandas parvoíces\ldots{} Não há elegância sem boa prosódia. -- Aonde
ias tu?

\repl{Benvinda} Ia na Rua do Ouvidô.

\repl{Figueiredo} \paren{Emendando.} Ouvidorr\ldots{} Ouvidorr\ldots{} Não faças economia nos
erres, porque apesar da carestia geral, eles não aumentarão de preço. E sibila bem
os esses -- Assim\ldots{} Bom. Vai e até logo! Mas vê lá: nada de olhadelas, nada de
respostas! Vai!

\repl{Benvinda} Inté logo.

\repl{Figueiredo} Que inté logo! Até logo é que é! Olha, em vez de inté logo,
dize: Au revoir! Tem muita graça de vez em quando uma palavra ou uma expressão
francesa.

\repl{Benvinda} Ó revoá!

\repl{Figueiredo} Antes isso! \paren{Benvinda afasta-se.} Não te mexas tanto,
rapariga! Ai! Ai! Isso! Agora foi demais! Ai! \paren{Benvinda desaparece.} De quantas tenho
lançado, nenhuma me deu tanto trabalho! Há de ser difícil coisa lapidar este
diamante! É uma vergonha! Não pode estar ao pé de gente! \paren{Lola vai atravessando a cena;
vendo Figueiredo, encaminha-se para ele.}

\newscenenamed{Cena III}
\stagedir{Figueiredo, Lola}

\repl{Lola} Oh! estimo encontrá-lo! Pode dar-me uma palavra?

\repl{Figueiredo} Pois não, minha filha!

\repl{Lola} Não o comprometo?

\repl{Figueiredo} De forma alguma! Vossemecê já está lançada!

\repl{Lola} Como?

\repl{Figueiredo} Vossemecês só envergonham a gente antes de lançadas.

\repl{Lola} Não entendo.

\repl{Figueiredo} Nem é preciso entender. Que desejava?

\repl{Lola} Lembra-se de mim?

\repl{Figueiredo} Perfeitamente. Encontramo-nos um dia no vestíbulo do Grande
Hotel da Capital Federal.

\repl{Lola} \paren{Apertando-lhe a mão.} Nunca mais me esqueci da sua fisionomia. O
senhor não é bonito\ldots{} oh! não! mas é muito insinuante.

\repl{Figueiredo} \paren{Modestamente.} Oh! filha!\ldots{}

\repl{Lola} Lembra-se do motivo que me levava àquele hotel?

\repl{Figueiredo} Lembra-me. Vossemecê ia à procura de um moço que apontava na
primeira dúzia.

\repl{Lola} Vejo que tem boa memória. Pois é na sua qualidade de hóspede do
Grande Hotel da Capital Federal que me atrevo a pedir-lhe uma informação.

\repl{Figueiredo} Mas eu há muitos dias já lá não moro! Era um bom hotel, não
nego, mas que quer? -- Não me levavam o café ao quarto às sete horas em ponto! --
Entretanto, se for coisa que eu saiba\ldots{}

\repl{Lola} Queria apenas que me desse notícias do Gouveia.

\repl{Figueiredo} Do Gouveia?

\repl{Lola} O tal da primeira dúzia.

\repl{Figueiredo} Mas eu não o conheço!

\repl{Lola} Deveras?

\repl{Figueiredo} Nunca o vi mais gordo!

\repl{Lola} Que pena! Supus que o conhecesse!

\repl{Figueiredo} Pode ser que o conheça de vista, mas não ligo o nome à pessoa.

\repl{Lola} Tenho-o procurado inúmeras vezes no hotel\ldots{} e não há meio! Não
está! Saiu! Há três dias não aparece cá! Um inferno!\ldots{}

\repl{Figueiredo} Continua a amá-lo?

\repl{Lola} Sim, continuo, porque a primeira dúzia, pelo menos até a última vez
que lhe falei, não tinha ainda falhado; mas como não o vejo há muitos dias, receio
que a sorte afinal se cansasse.

\repl{Figueiredo} Então o seu amor regula-se pelos caprichos da bola da roleta?

\repl{Lola} É como diz. Ah! eu cá sou franca!

\repl{Figueiredo} Vê-se!

Coplas

 Lola

 -I-

 Este afeto incandescente
 Pela bola se regula
 Que vertiginosamente
 Na roleta salta e pula!

 Figueiredo

 Vossemecê o moço estima
 Dando a bola de um a doze;
 Mas de treze para cima
 Ce n’est pas la même chose!

 Lola

 - II -

 É Gouveia um bom pateta
 Se supõe que inda o quisesse
 Quando a bola da roleta
 A primeira já não desse!

 Figueiredo

 A mulata brasileira
 De carinhos é fecunda,
 Embora dando a primeira,
 Embora dando a segunda!

\repl{Lola} E, por outro lado, ando apreensiva\ldots{}

\repl{Figueiredo} Por quê?

\repl{Lola} Porque\ldots{} O senhor não estranhe estas confidências por parte de uma
mulher que nem ao menos sabe o seu nome.

\repl{Figueiredo} Figueiredo\ldots{}

\repl{Lola} Mas, como já disse, a sua fisionomia é tão insinuante\ldots{} simpatizo
muito com o senhor.

\repl{Figueiredo} Creia que lhe pago na mesma moeda. Digo-lhe mais: se eu não
tivesse a minha especialidade\ldots{} \paren{À parte.} Deixem lá! Se o moreno fosse
mais carregado\ldots{}

\repl{Lola} Ando apreensiva porque a Mercedes me contou que há dias viu o
Gouveia no teatro com uma família que pelos modos parecia gente da roça\ldots{} e ele
conversava muito com uma moça que não era nada feia\ldots{} Tenho eu que ver se
o tratante se apanha com uma boa bolada arranja casório e eu fico a chuchar
no dedo!

\repl{Figueiredo} \paren{À parte.} Ela exprime-se com muita elegância!

\repl{Lola} Dos homens tudo há que esperar!

\repl{Figueiredo} Tudo, principalmente quando dá a primeira dúzia.

\repl{Lola} \paren{Estendendo a mão que ele aperta.} Adeus, Figueiredo.

\repl{Figueiredo} Adeus\ldots{} Como te chamas?

\repl{Lola} Lola.

\repl{Figueiredo} Adeus, Lola.

\repl{Lola} \paren{Com uma ideia.} Ah! uma coisa: você é homem que vá a uma festa?

\repl{Figueiredo} Conforme.

\repl{Lola} Eu faço anos sábado\ldots{}

\repl{Figueiredo} Este agora?

\repl{Lola} Não; o outro.

\repl{Figueiredo} Sábado de aleluia?

\repl{Lola} Sábado de aleluia, sim. Faço anos e dou um baile à fantasia.

\repl{Figueiredo} Bravo! Não faltarei!

\repl{Lola} Contanto que vá fantasiado! Se não vai, não entra!

\repl{Figueiredo} Irei fantasiado.

\repl{Lola} Aqui tem você a minha morada. \paren{Dá-lhe um cartão.}

\repl{Figueiredo} Aceito com muito prazer, mas olhe que não vou sozinho\ldots{}

\repl{Lola} Vai com quem quiseres.

\repl{Figueiredo} Levo comigo uma trigueira que estou lançando, e que precisa
justamente de ocasiões como essa para civilizar-se.

\repl{Lola} Aquela casa é tua, meu velho! \paren{Vendo Gouveia que entra do outro
lado, cabisbaixo, e não repara nela.} Olha quem vem ali!

\repl{Figueiredo} Quem?

\repl{Lola} Aquele é que é o Gouveia.

\repl{Figueiredo} Ah! é aquele?\ldots{} Conheço-o de vista\ldots{} É um moço do comércio.

\repl{Lola} Foi. Hoje não faz outra coisa senão jogar. Mas como está cabisbaixo
e pensativo! Querem ver que a primeira dúzia\ldots{}

\repl{Figueiredo} Adeus! Deixo-te com ele. Até sábado de aleluia!

\repl{Lola} Não faltes, meu velho! \paren{Apertam-se as mãos.}

\repl{Figueiredo} \paren{À parte.} Dir-se-ia que andamos juntos na escola! \paren{Sai.}

\newscenenamed{Cena IV}
\stagedir{Lola, Gouveia}

\repl{Gouveia} \paren{Descendo cabisbaixo ao proscênio.} Há três dias dá a segunda
dúzia\ldots{} Consultei hoje a escrita: perdi em noventa e cinco bolas o que tinha ganho
em perto de mil e duzentas! Decididamente aquele famoso padre do Pará tinha razão
quando dizia que não se deve apontar a roleta nem com o dedo, porque o próprio
dedo pode lá ficar!

\repl{Lola} \paren{À parte, do outro lado.} Fala sozinho!

\repl{Gouveia} Hei de achar a forra! O diabo é que fui obrigado a pôr as joias
no prego. Venho neste instante da casa do judeu. É sempre pelas joias que começa a
esbodegação\ldots{}

\repl{Lola} \paren{À parte.} Continua\ldots{} Aquilo é coisa\ldots{}

\repl{Gouveia} Com certeza vão dar por falta dos meus brilhantes\ldots{} Pobre
Quinota! Se ela soubesse! Ela, tão simples, tão ingênua, tão sincera!

\repl{Lola} \paren{Aproximando-se inopinadamente.} Tu estás maluco?

\repl{Gouveia} Hein?\ldots{} Eu\ldots{} Ah! és tu? Como vais?\ldots{}

\repl{Lola} Estavas falando sozinho?

\repl{Gouveia} Fazendo uns cálculos\ldots{}

\repl{Lola} Aconteceu-te alguma coisa desagradável? Tu não estás no teu natural!

\repl{Gouveia} Sim\ldots{} aconteceu-me\ldots{} fui roubado\ldots{} um gatuno levou as minhas
joias\ldots{} e eu estava aqui planejando deixar hoje a primeira dúzia e atacar dois
esguichos, o esguicho de 7 a 12 e o esguicho de 25 a 30, a dobrar, a dobrar!

\repl{Lola} \paren{Com ímpeto.} A primeira dúzia falhou?

\repl{Gouveia} Falhou\ldots{} \paren{A um gesto de Lola.} Mas descansa: eu já a tinha
abandonado antes que ela me abandonasse.

\repl{Lola} Tens então continuado a ganhar?

\repl{Gouveia} Escandalosamente!

\repl{Lola} Ainda bem, porque sábado de aleluia faço anos\ldots{}

\repl{Gouveia} É verdade\ldots{} fazes anos no sábado de aleluia\ldots{}

\repl{Lola} É preciso gastar muito dinheiro! Tenho te procurado um milhão de
vezes! No hotel dizem-me que lá nem apareces!

\repl{Gouveia} Exageração.

\repl{Lola} E outra coisa: quem era uma família com quem estavas uma noite
destas no S. Pedro? Uma família da roça?

\repl{Gouveia} Quem te disse?

\repl{Lola} Disseram-me. Que gente é essa?

\repl{Gouveia} Uma família muito respeitável que eu conheci quando andei por
Minas.

\repl{Lola} Gouveia, Gouveia, tu enganas-me!

\repl{Gouveia} Eu? Oh! Lola! Nunca te autorizei a duvidares de mim!\ldots{}

\repl{Lola} Nessa família há uma moça que\ldots{} Oh! o meu coração adivinha uma
desgraça, e\ldots{} \paren{Desata a chorar.}

\repl{Gouveia} \paren{À parte.} É preciso, realmente, que ela me ame muito, para ter
um pressentimento assim! \paren{Alto.} Então? Que é isso? Não chores! Vê que estamos
na rua!\ldots{}

\repl{Lola} \paren{À parte.} Pedaço d’asno!

\repl{Gouveia} Eu irei logo lá à casa, e conversaremos.

\repl{Lola} Não! não te deixo! Hás de ir agora comigo, hás de acompanhar-me,
senão desapareces como aquela vez, no Largo da Carioca!

\repl{Gouveia} Mas\ldots{}

\repl{Lola} Ou tu me acompanhas, ou dou um escândalo!

\repl{Gouveia} Bom, bom, vamos. Tens aí o carro?

\repl{Lola} Não, que o Lourenço, coitado, foi passar uns dias em Caxambu. Vamos
a pé. Bem sei que tu tens vergonha de andar comigo em público, mas isso são
luxos que deves perder!

\repl{Gouveia} Vamos! \paren{À parte.} Hei de achar meio de escapulir\ldots{}

\repl{Lola} Vamos! \paren{À parte.} Ou eu me engano, ou está liquidado! \paren{Afastam-se.
Entram pelo outro lado Eusébio, Fortunata e Quinota, que os veem sem serem vistos
por eles.}

\newscenenamed{Cena V}
\stagedir{Eusébio, Fortunata, Quinota}

\repl{Fortunata} Olhe. Lá vai! É ele! É seu Gouveia com a mema espanhola cum
quem tava aquela nôite no jardim do Recreio! \paren{Correndo a gritar.} Seu Gouveia!
seu Gouveia!\ldots{}

\repl{Eusébio} \paren{Agarrando-a pela saia.} Ó senhora! não faça escândalo! Que
maluquice de muié!\ldots{}

\repl{Quinota} \paren{Abraçando o pai, chorosa.} Papai, eu sou muito infeliz!

\repl{Eusébio} Aqui está! É o que a senhora queria!

\repl{Fortunata} Aquilo é um desaforo que eu não posso admiti! O diabo do home é
noivo de nossa fia e anda por toda a parte cuma pelintra!

\repl{Eusébio} Que pelintra, que nada!\ldots{} Não acredita, fia da minha bença. É
uma prima dele. Coitadinha! Chorando! \paren{Beija-lhe os olhos.}

\repl{Quinota} Eu gosto tanto daquele ingrato!

\repl{Eusébio} Ele também gosta de ti\ldots{} e há de casá contigo\ldots{} e há de sê um
bão marido!

\repl{Fortunata} \paren{Puxando Eusébio de lado.} É perciso que você tome uma
porvidência quaqué, seu Eusébio -- senão, faço uma estralada!\ldots{}

\repl{Eusébio} \paren{Baixo.} Descanse\ldots{} Eu já tomei informação\ldots{} Já sei onde mora
essa espanhola\ldots{} Agora memo vô procurá ela. Vá as duas já pra casa! Eu já vô.

\repl{Fortunata} E Juquinha? Por onde anda aquele menino?

\repl{Eusébio} Deixe, que o pequeno não se perde\ldots{} Está lá no tar Belódromo,
aprendendo a andá naquela coza\ldots{} Cumo chama?

\repl{Quinota} Bicicleta.

\repl{Eusébio} É. -- Diz que é bom pra desenvorvê os músquios!

\repl{Fortunata} Desenvorvê a vadiação é que é!

\repl{Quinota} Ele é tão criança!

\repl{Eusébio} Deixa o menino se adiverti. -- Vão pra casa.

\repl{Quinota} Lá vamos para aquele forno!

\repl{Eusébio} Tem paciência, Quinota! Enquanto não se arranja coza mió, a gente
deve se contentá c’aquele sote.

\repl{Fortunata} Vamo, Quinota!

\repl{Quinota} Não se demore, papai!

\repl{Eusébio} Não.

\repl{Fortunata} \paren{Saindo.} Eu tô mas é doida pra me apanhá na fazenda! \paren{Eusébio
leva as senhoras até o bastidor e, voltando-se, vê pelas costas Benvinda.}

\newscenenamed{Cena VI}
\stagedir{Eusébio, Benvinda}

\repl{Benvinda} \paren{Consigo.} Parece que assim o meu andá tá direito\ldots{}

\repl{Eusébio} \paren{Consigo.} Xi que tentação! \paren{Seguindo Benvinda.} Psiu!\ldots{} Ó
Dona\ldots{} Dona!\ldots{}

\repl{Benvinda} \paren{À parte.} Esta voz\ldots{} \paren{Volta-se.} Sinhô Eusébio!

\repl{Eusébio} Benvinda!!\ldots{}

\repl{Benvinda} \paren{Assestando o face-en-main.} Ó revoá.

\repl{Eusébio} A mulata de luneta, minha Nossa Senhora! Este mundo tá
perdido!\ldots{}

\repl{Benvinda} \paren{Dando-se ares e sibilando os esses.} Deseja alguma coisa? Estou
as suas ordes!

\repl{Eusébio} Ah! ah! ah! que mulata pernóstica! Quem havia de dizê! Vem cá,
diabo, vem cá; me conta tua vida!

\repl{Benvinda} \paren{Mudando de tom.} Vam’cê não tá zangado comigo?

\repl{Eusébio} Eu não! Tu era senhora do teu nariz! O que tu podia tê feito era
se despedi da gente\ldots{} Dona Fortunata não te perdoa! E seu Borge, quando
soubé, há de ficá danado, porque ele gosta de ti.

\repl{Benvinda} Se ele gostasse de mim, tinha se casado comigo.

\repl{Eusébio} Ele um dia me deu a entendê que se eu te desse um dote\ldots{}

\repl{Benvinda} Vamcês ainda mora no hoté?

\repl{Eusébio} Não. Nos mudemo para um sote da rua dos Inválio. Paguemo sessenta
mi-réis.

\repl{Benvinda} Seu Gouveia já apareceu?

\repl{Eusébio} Apareceu e tudo tá combinado\ldots{} \paren{À parte.} O diabo é a espanhola!

\repl{Benvinda} Sinhá? nhanhã? nhô Juquinha? tudo tá bom?

\repl{Eusébio} Tudo! Tudo tá bom?

\repl{Benvinda} Nhô Juquinha eu vejo ele às vez passá na Rua do Lavradio\ldots{} cum
outros menino\ldots{}

\repl{Eusébio} Tá aprendendo a andá no\ldots{} n\ldots{} nesses carro de duas roda, uma
atrás outra adiante, que a gente trepa em cima e tem um nome esquisito\ldots{}

\repl{Benvinda} Eu sei.

\repl{Eusébio} E tu, mulata?

\repl{Benvinda} Eu tô cum seu Figueiredo.

\repl{Eusébio} Sei lá quem é seu Figueiredo!

\repl{Benvinda} Tou morando na Rua do Lavradio, canto da Rua da Relação.
\paren{Assestando o face-en-main.} Se quisé aparecê não faça cerimônia. \paren{Sai
requebrando-se.} Ó revoá!

\repl{Eusébio} Aí, mulata!

\newscenenamed{Cena VII}
\stagedir{Eusébio, depois Juquinha}

\repl{Eusébio} O curpado fui eu\ldots{} Quando me alembro que seu Borge queria casá
com ela\ldots{} Bastava um dote, quaqué coza\ldots{} dois ou três conto de réis\ldots{} Mas
deixa está: ele não sabe de nada, e tarvez que a coza ainda se arranje. Quem não sabe é
cumo quem não vê. \paren{Vendo passar Juquinha montado numa bicicleta.} Eh!
Juquinha\ldots{} Menino, vem cá!

\repl{Juquinha} Agora não posso, não, sinhô! \paren{Desaparece.}

\repl{Eusébio} Ah! menino! Espere lá! \paren{Corre atrás do Juquinha. Gargalhada dos
circunstantes. Mutação.}

\quadro{Quadro VI}

Saleta em casa de Lola

\newscenenamed{Cena I}
\stagedir{Lola e Gouveia}

\paren{Lola entra furiosa. Traz vestida uma elegante bata. Gouveia acompanha-a.
Vem vestido de Mefistófeles.}

\repl{Lola} Não! Isto não se faz! E o senhor escolheu o dia dos meus anos para
me fazer essa revelação! Devia esperar pelo menos que acabasse o baile! Com
que mau humor vou agora receber os meus convidados! \paren{Caindo numa cadeira.} Oh!
os meus pressentimentos não me enganavam!\ldots{}

\repl{Gouveia} Esse casamento é inevitável; quando estive em S. João do Sabará,
comprometi-me com a família de minha noiva e não posso faltar à minha
palavra!

\repl{Lola} Mas por que não me disse nada? Por que não foi franco?

\repl{Gouveia} Supus que essa dívida tivesse caído em exercícios findos; mas a
pequena teve saudades minhas, e tanto fez, tanto chorou, que o pai se viu
obrigado a vir procurar-me! Como vês, é uma coisa séria!

\repl{Lola} Mas o senhor não pode procurar um subterfúgio qualquer para evitar
esse casamento? Que ideia é essa de se casar agora que está bem, quem tem sido
feliz no jogo? E eu? que papel represento eu em tudo isto?

\repl{Gouveia} \paren{Puxando uma cadeira.} Lola, vou ser franco, vou dizer-te toda a
verdade. \paren{Senta-se.} Há muito tempo não faço outra coisa senão perder\ldots{} O
outro dia tive uma aragem passageira, um sopro de fortuna, que serviu apenas para
pagar as despesas da tua festa de hoje e mandar fazer esta roupa de Mefistófeles!
Estou completamente perdido! As minhas joias não foram roubadas, como eu te
disse. Deitei-as no prego e vendi as cautelas. Para fazer dinheiro, eu, que aqui
vês coberto de seda, tenho vendido até a roupa do meu uso\ldots{} Nessas casas de jogo já
não tenho a quem pedir dinheiro emprestado. Os banqueiros olham-me por cima dos
ombros, porque eu tornei-me um piaba\ldots{} Sabes o que é um piaba? É um
sujeito que vai jogar com muito pouco bago. Estou completamente perdido!

\repl{Lola} \paren{Erguendo-se.} Bom. Prefiro essa franqueza. É muito mais razoável.

\repl{Gouveia} \paren{Erguendo-se.} Esse casamento é a minha salvação; eu\ldots{}

\repl{Lola} Não precisa dizer mais nada. Agora sou eu a primeira a aconselhar-te
que te cases, e quanto antes melhor.

\repl{Gouveia} Mas, minha boa Lola, eu sei que com isso vais padecer bastante,
e\ldots{}

\repl{Lola} Eu? Ah! ah! ah!\ldots{} Só esta me faria rir!\ldots{} Ah! ah! ah! ah!\ldots{}
Sempre me saíste um grande tolo! Pois entrou-te na cabeça que eu algum dia quisesse de ti
outra coisa que não fosse o teu dinheiro?

\repl{Gouveia} \paren{Horrorizado.} Oh!

\repl{Lola} E realmente supunhas que eu te tivesse amor?

\repl{Gouveia} \paren{Caindo em si.} Compreendo e agradeço o teu sacrifício, minha boa
Lola. Tu está a fingir uma perversidade e um cinismo que não tens, para que
eu saia desta casa sem remorsos! Tu és a Madalena, de Pinheiro Chagas!

\repl{Lola} E tu és um asno! -- O que te estou dizendo é sincero! Estava eu bem
aviada se me apaixonasse por quem quer que fosse!

\repl{Gouveia} Dar-se-á caso que te saíssem do coração todos aqueles horrores?

\repl{Lola} Do coração? Sei lá o que isso é! O que afianço é que sou tão
sincera, que me comprometo a amar-te ainda com mais veemência que da primeira vez, no
dia em que resolveres dar cabo do dote da tua futura esposa!

\repl{Gouveia} \paren{Com uma explosão.} Cala-te, víbora danada! Olha que nem o jogo,
nem os teus beijos me tiraram totalmente o brio! Eu posso fazer-te pagar
bem caro os teus insultos!

\repl{Lola} Ora, vai te catar! Se julgas amedrontar-me com esses ares de galã de
dramalhão, enganas-te redondamente! Depois, repara que estás vestido de
Mefistófeles! Esse traje prejudica os teus efeitos dramáticos! Vai, vai ter
com a tua roceira. Casem-se, sejam muito felizes, tenham muitos Gouveiazinhos, e não
me amoles mais! \paren{Gouveia avança, quer dizer alguma coisa, mas não acha uma
palavra. Encolhe os ombros e sai.}

\newscenenamed{Cena II}
\stagedir{Lola, depois Lourenço}

\repl{Lola} \paren{Só.} Faltou-lhe uma frase, para o final da cena -- coitado! A
respeito da imaginação, este pobre rapaz foi sempre uma lástima! -- Os homens não
compreendem que o seu único atrativo é o dinheiro! Este pascácio devia ser
o primeiro a fazer uma retirada em regra, e não se sujeitar a tais
sensaborias! Bastavam quatro linhas pelo correio. -- Oh! também a mim, quando eu ficar
velha e feia, ninguém me há de querer! Os homens têm o dinheiro, nós temos a
beleza; sem aquele e sem esta, nem eles nem nós valemos coisa nenhuma. \paren{Entra Lourenço,
trajando uma libré de cocheiro. Vem a rir-se.}

\repl{Lourenço} Que foi aquilo?

\repl{Lola} Aquilo quê?

\repl{Lourenço} O Gouveia! Veio zunindo pela escada abaixo e, no saguão, quando
eu me curvei respeitosamente diante dele, mandou-me ao diabo, e foi pela rua
fora, a pé, vestido de satanás de mágica! Ah! ah! ah!

\repl{Lola} Daquele estou eu livre!

\repl{Lourenço} Eu não dizia a você? Aquilo é bananeira que já deu cacho!

\repl{Lola} Que vieste fazer aqui? Não te disse que ficasses lá embaixo?

\repl{Lourenço} Disse, sim, mas é que está aí um matuto, pelos modos fazendeiro,
que deseja falar a você.

\repl{Lola} A ocasião é imprópria. São quase horas, ainda tenho que me vestir!

\repl{Lourenço} Coitado! o pobre-diabo já aqui veio um ror de vezes a semana
passada, e parece ter muito interesse nesta visita. Demais\ldots{} você bem sabe
que nunca se manda embora um fazendeiro.

\repl{Lola} Que horas são?

\repl{Lourenço} Oito e meia. Já estão na sala alguns convidados.

\repl{Lola} Bem! num quarto de hora eu despacho esse matuto. Faze-o entrar.

\repl{Lourenço} É já. \paren{Sai assoviando.}

\repl{Lola} \paren{Só.} Como anda agora lépido o Lourenço! Voltou de Caxambu que nem
parece o mesmo! -- Ele tem razão: um fazendeiro nunca se manda embora.

\repl{Lourenço} \paren{Introduzindo Eusébio muito corretamente.} Tenha V. Ex.a a
bondade de entrar. \paren{Eusébio entra muito encafifado e Lourenço sai fechando a
porta.}

\newscenenamed{Cena III}
\stagedir{Lola, Eusébio}

\repl{Eusébio} Boa nôte, madama! Deus esteje nesta casa!

\repl{Lola} Faz favor de entrar, sentar-se e dizer o que deseja. \paren{Oferece-lhe
uma cadeira. Sentam-se ambos.}

\repl{Eusébio} Na sumana passada eu precurei a madama um bandão de vez sem
conseguir le falá\ldots{}

\repl{Lola} E por que não veio esta semana?

\repl{Eusébio} Dona Fortunata não quis, por sê sumana santa\ldots{} Eu então esperei
que rompesse as aleluia! \paren{Uma pausa.} Eu pensei que a madama embrulhasse
língua comigo, e eu não entendesse nada que a madama dissesse, mas tô vendo que
fala munto bem o português\ldots{}

\repl{Lola} Eu sou espanhola e\ldots{} o senhor sabe\ldots{} o espanhol parece-se muito
com o português; por exemplo: hombre, homem; mujer, mulher.

\repl{Eusébio} \paren{Mostrando o chapéu que tem na mão.} E como é chapéu, madama?

\repl{Lola} Sombrero.

\repl{Eusébio} E guarda-chuva?

\repl{Lola} Paraguas.

\repl{Eusébio} É! Parece quase a mema coza! -- E cadeira?

\repl{Lola} Silla.

\repl{Eusébio} E janela?

\repl{Lola} Ventana.

\repl{Eusébio} Munto parecido!

\repl{Lola} Mas, perdão, creio que não foi para aprender espanhol que o senhor
veio à minha casa\ldots{}

\repl{Eusébio} Não, madama, não foi para aprendê espanhol: foi para tratá de
coza munto séria!

\repl{Lola} De coisa séria? Comigo! É esquisito!\ldots{}

\repl{Eusébio} Não é esquisito, não madama; eu sou o pai da noiva de seu
Gouveia!\ldots{}

\repl{Lola} Ah!

\repl{Eusébio} Cumo minha fia anda munto desgostosa pru via da madama, eu me
alembrei de vi na sua casa para sabê\ldots{} sim, para sabê se é possive a
madama se separá de seu Gouveia. Se fô possive, munto que bem; se não fô, paciência:
a gente arruma as mala, e amenhã memo vorta pra fazenda. Minha fia é bonita e é
rica; não há de sê defunto sem choro!\ldots{}

\repl{Lola} Compreendo: o senhor vem pedir a liberdade de seu futuro genro!

\repl{Eusébio} Sim, madama; eu quero o moço livre e desembaraçado de quaqué
ônus! \paren{Lola levanta-se, fingindo uma comoção extraordinária; quer falar, não
pode, e acaba numa explosão de lágrimas. Eusébio levanta-se.} Que é isso? A madama tá
chorando?!\ldots{}

\repl{Lola} \paren{Entre lágrimas.} Perder o meu adorado Gouveia! Oh! o senhor pede-me
um sacrifício terrível! \paren{Pausa.} Mas eu compreendo\ldots{} Assim é necessário\ldots{}
Entre a mulher perdida e a menina casta e pura. Entre o vício e a virtude, é o
vício que deve ceder\ldots{} Mas o senhor não imagina como eu amo aquele moço e quantas
lágrimas preciso verter para apagar a lembrança do meu amor desgraçado! \paren{Abraça
Eusébio, escondendo o rosto nos ombros dele, e soluça.} Sou muito infeliz!

\repl{Eusébio} \paren{Depois de uma pausa, em que faz muitas caretas.} Então,
madama?\ldots{} sossegue\ldots{} A madama não perde nada\ldots{} \paren{À parte.} Que cangote cheiroso!\ldots{}

\repl{Lola} \paren{Olhando para ele, sem tirar a cabeça do ombro.} Não perco nada? que
quer o senhor dizer com isso?

\repl{Eusébio} Quero dizê que\ldots{} sim\ldots{} quero dizê\ldots{} Home, madama, tira a
cabeça daí, porque assim eu não acerto cas palavras!

\repl{Lola} \paren{Sem tirar a cabeça.} Sim, a minha porta se fechará ao Gouveia\ldots{}
Juro-lhe que nunca mais o verei\ldots{} Mas onde irei achar consolação?\ldots{} Onde
encontrarei uma alma que me compreenda, um peito que me abrigue, um coração que vibre
harmonizado com o meu?

\repl{Eusébio} Nós podemo entrá num ajuste.

\repl{Lola} \paren{Afastando-se dele com ímpeto.} Um ajuste?! Que ajuste?! O senhor
quer talvez propor-me dinheiro!\ldots{} Oh! por amor dessa inocente menina, que é sua
filha, não insulte, senhor, os meus sentimentos, não ofenda o que eu tenho de mais
sagrado!\ldots{}

\repl{Eusébio} \paren{À parte.} É um pancadão! Seu Gouveia teve bom gosto!\ldots{}

\repl{Lola} O senhor quer que eu deixe o Gouveia porque sua filha o ama e é
amada por ele, não é assim? Pois bem: é seu o Gouveia; dou-lho, mas dou-lho de
graça, não exijo a menor retribuição!

\repl{Eusébio} Mas o que vinha propô à madama não era um pagamento, mas uma\ldots{}
Cumo chama aquilo que se falou cando foi o 13 de Maio? Uma\ldots{} Ora, sinhô!
\paren{Lembrando-se.} Ah! uma indenização! O caso muda muito de figura!

\repl{Lola} Não! -- nenhuma indenização pretendo! Mas de ora em diante fecharei o
meu coração aos mancebos da capital, e só amarei \paren{Enquanto fala vai
arranjando o laço da gravata e a barba de Eusébio.} algum homem sério\ldots{} de
meia-idade\ldots{} filho do campo\ldots{} ingênuo\ldots{} sincero\ldots{} incapaz de um embuste\ldots{} \paren{Alisando-lhe o
cabelo.} Oh! Não exigirei que ele seja belo\ldots{} Quanto mais feio for, menos ciúmes
terei! \paren{Eusébio cai como desfalecido numa cadeira, e Lola senta-se no colo dele.}
A esse hei de amar com frenesi\ldots{} com delírio!\ldots{} \paren{Enche-o de beijos.}

\repl{Eusébio} \paren{Resistindo e gritando.} Eu quero i me embora! \paren{Ergue-se.}

\repl{Lola} Cala-te, criança louca!\ldots{}

\repl{Eusébio} Criança louca! Uê!\ldots{}

\repl{Lola} \paren{Com veemência.} Desde que transpuseste aquela porta, senti que uma
força misteriosa e magnética me impelia para os teus braços! Ora, o
Gouveia! Que me importa a mim o Gouveia se és meu, se estás preso pela tua Lola, que não
te deixará fugir?

\repl{Eusébio} Isso tudo é verdade?

\repl{Lola} Estes sentimentos não se fingem! Eu adoro-te!

\repl{Eusébio} Eu me conheço\ldots{} já sou um home de idade\ldots{} não sei falá como os
doutô da Capitá Federá\ldots{}

\repl{Lola} Mas é isso mesmo o que mais me encanta na tua pessoa!

\repl{Eusébio} Cuando a esmola é munta, o pobre desconfia.

\repl{Lola} Põe à prova o meu amor! Já te não sacrifiquei o Gouveia?

\repl{Eusébio} Isso é verdade.

\repl{Lola} Pois sacrifico-te o resto!\ldots{} Queres que me desfaça de tudo quanto
possuo, e que vá viver contigo numa ilha deserta?\ldots{} Oh! bastam-me o teu amor e uma
choupana! \paren{Abraça-o.} Dá-me um beijo! Dá-mo como um presente do céu!
\paren{Eusébio limpa a boca com o braço e beija-a.} Ah! \paren{Lola fecha os olhos e fica como
num êxtase.}

\repl{Eusébio} \paren{À parte.} Seu Eusébio tá perdido! \paren{Dá-lhe outro beijo.}

\repl{Lola} \paren{Sem abrir os olhos.} Outro\ldots{} outro beijo ainda\ldots{} \paren{Eusébio beija-a
e ela afasta-se, esfregando os olhos.} Oh! Não será isto um sonho?

\repl{Eusébio} Bom, madama, com sua licença: eu vou mimbora\ldots{}

\repl{Lola} Não; não consinto! Faço hoje anos e dou uma festa. A minha sala já
está cheia de convidados.

\repl{Eusébio} Ah! por isso é que, cuando eu entrei, subia uns mascarado\ldots{}

\repl{Lola} Sim; é um baile à fantasia. Precisas de um vestuário.

\repl{Eusébio} Que vestuário, madama?

\repl{Lola} Espera. Tudo se arranjará. \paren{Vai à porta.} Lourenço!

\repl{Eusébio} Que vai fazê, madama?

\repl{Lola} Vais ver.

\newscenenamed{Cena IV}
\stagedir{Os mesmos, Lourenço}

\repl{Lola} \paren{A Lourenço que se apresenta muito respeitosamente.} Vá com este
senhor a uma casa de alugar vestimentas à fantasia a fim de que ele se prepare
para o baile.

\repl{Eusébio} Mas\ldots{}

\repl{Lola} \paren{Súplice.} Oh! não me digas que não! \paren{A Lourenço.} Dê ordem ao
porteiro para não deixar entrar o Sr. Gouveia. Esse moço morreu para mim!

\repl{Lourenço} \paren{À parte.} Que diabo disto será aquilo?

\repl{Lola} \paren{Baixo a Eusébio.} Estás satisfeito? \paren{Antes que ele responda.} Vou
preparar-me também. Até logo! \paren{Sai pela direita.}

\newscenenamed{Cena V}
\stagedir{Eusébio, Lourenço}

\repl{Eusébio} \paren{Consigo.} Sim, sinhô; isto é o que se chama vi buscá lã e saí
tosquiado! -- Se Dona Fortunata soubesse\ldots{} \paren{Dando com o Lourenço.} Vamos lá, seu\ldots{}
cumo o sinhô se chama?

\repl{Lourenço} Lourenço, para servir a V. Ex.a

\repl{Eusébio} Vamo lá, seu Lourenço\ldots{} \paren{Sem arredar pé de onde está.} Isto é o
diabo! Enfim!\ldots{} Mas que espanhola danada! \paren{Encaminha-se para a porta e faz lugar
para Lourenço passar.} Faz favô!

\repl{Lourenço} \paren{Inclinando-se.} Oh! meu senhor\ldots{} isso nunca\ldots{} eu, um
cocheiro!\ldots{} Então? Por obséquio!

\repl{Eusébio} Passe, seu Lourenço, passe, que o sinhô é de casa e tá fardado!
\paren{Lourenço passa e Eusébio acompanha-o. Mutação.}

\quadro{Quadro VII}

\paren{Rico salão de baile 
profusamente iluminado.}

\newscenenamed{Cena I}

\repl{Rodrigues, Dolores, Mercedes, Blanchette, convidados.} \paren{Estão todos vestidos
à fantasia.}

 Coro

 Que lindo baile! que bela festa!
 Luzes e flores em profusão!
 A nossa Lola não é modesta!
 Eu sinto aos pulos o coração!

 Mercedes, Dolores e Blanchette

 Senhores e senhoras,
 Divirtam-se a fartar!
 Alegremente as horas
 Vejamos deslizar!
 A mocidade é sonho
 Esplêndido e risonho
 Que rápido se esvai;
 Portanto, a mocidade
 Com voluptuosidade
 Depressa aproveitai!

 Blanchette

 Dancemos, que a dança,
 Se o corpo nos cansa,
 A alma nos lança
 Num mundo melhor!

 Dolores

 Bebamos, que o vinho,
 Com doce carinho,
 Nos mostra o caminho
 Fulgente do amor!

 Mercedes

 Amemos, embora
 Chegadas à hora
 Da fúlgida aurora,
 Deixemos de amar!
 Que em nós os amores,
 Tal como nas flores
 Perfumes e cores,
 Não possam durar!

 As Três

 Dancemos! Bebamos! Amemos!

\repl{Rodrigues} \paren{Que está vestido de Arlequim.} Então? Que me dizem desta
fantasia? Vocês ainda não me disseram nada!\ldots{}

\repl{Mercedes} Deliciosa!

\repl{Dolores} Magnífica.

\repl{Blanchette} É patante!

\repl{Rodrigues} Saiu baratinha, porque foi feita em casa pelas meninas. Como
sabem, sou o homem da família.

\repl{Mercedes} Você confessou em casa que vinha ao baile da Lola?

\repl{Rodrigues} Não, que isso talvez aborrecesse minha senhora. Eu disse-lhe
que ia a um baile dado em Petrópolis pelo Ministro Inglês\ldots{}

\repl{Todas} Ah! ah! ah!\ldots{}

\repl{Rodrigues} \paren{Continuando.} \ldots{}baile a que não podia faltar por amor de uns
tantos interesses comerciais\ldots{}

\repl{Blanchette} Ah! seu patife!

\repl{Dolores} De modo que, neste momento, a sua pobre senhora julga-o em
Petrópolis.

\repl{Rodrigues} \paren{Confidencialmente, muito risonho.} Saí hoje de casa com a
minha bela fantasia dentro de uma mala de mão, e fingi que ia tomar a barca das
quatro horas. Tomei mas foi um quarto do hotel, onde o austero negociante jantou e
onde à noite se transformou no polícromo arlequim que estão vendo -- e depois,
metendo-me num carro fechado, voei a esta deliciosa mansão de encantos e prazeres.
Tenho por mim toda a noite e parte do dia de amanhã, pois só tenciono voltar à
tardinha. Ah! não imaginam vocês com que saudade estou da família, e com que
satisfação abraçarei a esposa e os filhos quando vier de Petrópolis!

\repl{Mercedes} Você é na realidade um pai de família modelo!

\repl{Dolores} Um exemplo de todas as virtudes!

\repl{Blanchette} Esse vestuário de Arlequim não lhe fica bem! Você devia
vestir-se de Catão!

\repl{Rodrigues} Trocem à vontade, mas creiam que não há no Rio de Janeiro um
chefe de família mais completo do que eu. \paren{Afastando-se.} Em minha casa não
falta nada. \paren{Afasta-se.}

\repl{Mercedes} Nada, absolutamente nada, a não ser o marido.

\repl{Dolores} É um grande tipo.

\repl{Blanchette} E a graça é que a senhora paga-lhe na mesma moeda!

\repl{Mercedes} É mais escandalosa que qualquer de nós.

\repl{Dolores} Não quero ser má língua, mas há dias encontrei-a num bonde da
Vila Isabel muito agarradinha ao Lima Gama!

\repl{Blanchette} Aqueles bondes da Vila Isabel são muito comprometedores.

\repl{Rodrigues} \paren{Voltando.} Que estão vocês aí a cochichar?

\repl{Mercedes} Falávamos da vida alheia.

\repl{Blanchette} Dolores contava que há dias encontrou num bonde da Vila Isabel
uma senhora casada que mora em Botafogo.

\repl{Rodrigues} Isso não tira! Talvez fosse ao Jardim Zoológico.

\repl{Dolores} Talvez; mas o leão ia ao lado dela no bonde\ldots{}

\repl{Rodrigues} Há, efetivamente, senhoras casadas que se esquecem do decoro
que devem a si e à sociedade!

\repl{As Três} \paren{Com convicção.} Isso há\ldots{}

\repl{Rodrigues} Por esse lado posso levantar as mãos para o céu! Tenho uma
esposa virtuosa!

\repl{Mercedes} Deus lha conserve tal qual tem sido até hoje.

\repl{Rodrigues} Amém.

\repl{Blanchette} E Lola que não aparece?

\repl{Dolores} Está se vestindo: não tarda.

\repl{Um Convidado} Oh! que bonito par vem entrando!

\repl{Todos} É verdade!

\repl{O Convidado} Façamos alas para recebê-lo!

\repl{Rodrigues} Propomos que o recebamos com um rataplã!

\repl{Todos} Apoiado! Um rataplã\ldots{} \paren{Formam-se duas alas.}

 Coro

 Rataplã! Rataplã! Rataplã!
 Oh, que elegância! que lindo par!\ldots{}
 Todos os outros vêm ofuscar!

\newscenenamed{Cena II}
\stagedir{Os mesmos, Figueiredo e Benvinda}

\paren{Entra Figueiredo, vestido de Radamés, trazendo pela mão Benvinda, vestida
de Aída.}

 Figueiredo

 -I-

 Eis Aída,
 Conduzida
 Pela mão de Radamés!
 Vem chibante,
 Coruscante,
 Da cabeça até os pés!\ldots{}
 Que lindeza!
 Que beleza!
 Meus senhores aqui está
 A trigueira
 Mais faceira
 De São João do Sabará!

 Coro

 A trigueira, etc\ldots{}

 Figueiredo

 - II -

 Diz tolices,
 Parvoíces,
 Se abre a boca pra falar;
 Se se cala,
 Se não fala,
 Pode as pedras encantar!
 Eu a lanço
 Sem descanso!
 Na pontíssima estará
 A trigueira
 Mais faceira
 De São João do Sabará!

 Coro

 A trigueira, etc\ldots{}

\repl{Figueiredo} Minhas senhoras e meus senhores, apresento a Vossas
Excelências e Senhorias, Dona Fredegonda, que -- depois, bem entendido, das damas que se
acham aqui presentes -- é a estrela mais cintilante do demi-monde carioca!

\repl{Todos} \paren{Inclinando-se.} Dona Fredegonda!

\repl{Figueiredo} \paren{Baixo a Benvinda.} Cumprimenta.

\repl{Benvinda} Ó revoá!

\repl{Figueiredo} \paren{Baixo.} Não. Au revoir é quando a gente vai-se embora e não
quando chega.

\repl{Benvinda} Entonces\ldots{}

\repl{Figueiredo} \paren{Baixo.} Cala-te! Não digas nada!\ldots{} \paren{Alto.} Convidado pela
gentilíssima Lola para comparecer a este forrobodó elegante, não quis
perder o magnífico ensejo, que se me oferecia, de iniciar a formosa Fredegonda nos
insondáveis mistérios da galanteria fluminense! Espero que vossas
excelências e senhorias queiram recebê-la com benevolência, dando o necessário desconto
às clássicas emoções da estreia, e ao fato de ser Dona Fredegonda uma simples
roceira, quase tão selvagem como a princesa etíope que o seu vestuário
representa.

\repl{Todos} \paren{Batendo palmas.} Bravo! Bravo! Muito bem!

\repl{Blanchette} \paren{A Figueiredo.} Descanse. A iniciação desta neófita fica por
nossa conta. \paren{Às outras.} Não é assim?

\repl{Dolores e Mercedes} Certamente. \paren{As três cercam Benvinda, que se mostra
muito encafifada.}

\repl{Figueiredo} \paren{Vendo Rodrigues e aproximando-se dele.} Oh! Que vejo! Você
aqui!\ldots{} Você, o homem da família, o moralista retórico e sentimental, o
palmatória do mundo!\ldots{}

\repl{Rodrigues} Sim\ldots{} é que\ldots{} são coisas\ldots{} estou aqui por necessidade\ldots{} por
incidente\ldots{} por uma série de circunstâncias que\ldots{} que\ldots{}

\repl{Figueiredo} Deixe-se disso! Não há nada mais feio que a hipocrisia!
Naquela tarde em que o encontrei no Largo da Carioca, a mulata mostrou-me seu
cartão de visitas\ldots{}

\repl{Rodrigues} O meu?\ldots{} Ah! sim, dei-lhe o meu cartão\ldots{} para\ldots{}

\repl{Figueiredo} Para quê?

\repl{Rodrigues} Para\ldots{}

\repl{Figueiredo} Olhe, cá entre nós que ninguém nos ouve: quer você tomar conta
dela?

\repl{Rodrigues} Quê! Pois já se aborreceu?

\repl{Figueiredo} Todo o meu prazer é lançá-las -- lançá-las e nada mais. Você
viu a Mimi Bilontra?

\repl{Rodrigues} Não.

\repl{Figueiredo} Mas sabe o que é lançar uma mulher?

\repl{Rodrigues} Nesses assuntos sou hóspede\ldots{} você sabe\ldots{} sempre fui um homem
da família\ldots{} mas quer me parecer que lançar uma mulher é como quem diz
atirá-la na vida, iniciá-la neste meio\ldots{}

\repl{Figueiredo} Ah! qui qui! Infelizmente não creio que desta se possa fazer
alguma coisa mais que uma boa companheira. É uma mulher que lhe convinha.

\repl{Rodrigues} Mas eu não preciso de companheiras! Sou casado, e, graças a
Deus, a minha santa esposa\ldots{}

\repl{Figueiredo} \paren{Atalhando.} E o cartão?

\repl{Rodrigues} Que cartão? Ah! sim, o cartão do Largo da Carioca\ldots{} Mas eu não
me comprometi a coisa nenhuma!

\repl{Figueiredo} Bom; então não temos nada feito\ldots{} Mas veja lá! -- se quer\ldots{}

\repl{Rodrigues} Querer; queria\ldots{} mas não com caráter definitivo!

\repl{Figueiredo} Ora, vá pentear macacos!
\paren{Às últimas deixas, Eusébio tem entrado, vestido com uma dessas roupas que
vulgarmente se chamam de princês. Eusébio aperta a mão aos convidados um
por um. Todos se interrogam com os olhos admirados de tão estranho convidado.}

\newscenenamed{Cena III}
\stagedir{Os mesmos, Eusébio}

\repl{Eusébio} \paren{Depois de apertar a mão a muitos dos circunstantes.} Tá tudo
oiando uns pros outro, admirado de me vê aqui! Eu fui convidado pela madama dona
da casa!

\repl{Benvinda} \paren{À parte.} Sinhô Eusébio!\ldots{}

\repl{Figueiredo} \paren{A quem Eusébio aperta a mão, à parte.} Oh! diabo! é o patrão
da Benvinda!\ldots{}

\repl{Blanchette} Donde saiu esta figura?

\repl{Dolores} É um homem da roça!

\repl{Blanchette} Não será um doido?

\repl{Eusébio} \paren{Indo apertar por último a mão de Benvinda, reconhecendo-a.} -
Benvinda!

\repl{Benvinda} Ó revoá!

\repl{Figueiredo} \paren{À parte.} E ela a dar-lhe!\ldots{}

\repl{Eusébio} Tu também tá de fantasia, mulata! O mundo tá perdido!\ldots{}

\repl{Benvinda} Eu vim com seu Figueiredo\ldots{} mas vancê é que me admira!

\repl{Eusébio} Eu vim falá ca madama pro mode seu Gouveia\ldots{} e ela me convidô
pra festa\ldots{} e eu tive que alugá esta vestimenta, mas vim de tilbo porque hoje
é sabo de aleluia e eu não quero embrulho comigo!

\repl{Figueiredo} \paren{À parte.} Oh! bom! foi o seu professor de português!

\repl{Benvinda} Se sinhá soubesse\ldots{}

\repl{Eusébio} Cala a boca! nem pensá nisso é bão! mas onde tá o tar seu
Figueiredo? Eu sempre quero oiá pra cara dele!

\repl{Benvinda} É aquele.

\repl{Eusébio} \paren{Indo a Figueiredo.} Pois foi o sinhô que me desencaminhou a
mulata? O sinhô, um home branco e que já começa a pintá? Agora me alembro de vê o
sinhô lá no hoté só rondando a porta da gente!\ldots{}

\repl{Figueiredo} Estou pronto a dar-lhe todas as satisfações em qualquer
terreno que mas peça\ldots{} mas há de convir que este lugar não é o mais próprio para\ldots{}

\repl{Eusébio} \paren{Atalhando.} Ora viva! Eu não quero satisfação! A mulata não é
minha fia nem parenta minha! mas lá em São João do Sabará há um home chamado seu
Borge, que se soubé\ldots{} um! um!\ldots{} é capaz de vi na Capitá Federá!

\repl{Figueiredo} Pois que venha!\ldots{}

\repl{Mercedes} Aí chega a Lola!

\repl{Todos} Oh! a Lola!\ldots{} Viva a Lola!\ldots{} Viva!\ldots{}

\newscenenamed{Cena IV}
\stagedir{Os mesmos, Lola}

 Coro

 Até que enfim Lola aparece!
 Até que enfim Lola cá está!
 Vem tão bonita que entontece!
 Lola, vem cá! Lola, vem já!\ldots{}

\repl{} \paren{Lola entra ricamente fantasiada à espanhola.}

 Lola

 Querem todos ver a Lola!
 Aqui está ela!

 Coro

 Aqui está ela!

 Lola

 Oh, que esplêndida manola!
 Não há mais bela!

 Coro

 Não há mais bela!

 Lola

 Vejam que graça
 Tem a manola!
 Não é chalaça!
 Não é parola!
 Como se agita!
 Como rebola!
 Isto os excita!
 Isto os consola!
 O olhar brejeiro
 De uma espanhola
 Do mais matreiro
 Transtorna a bola,
 E sem pandeiro,
 Nem castanhola!

 Coro

 Vejam que graça, etc\ldots{} \paren{Dança geral.}

\repl{Figueiredo} Gentilíssima Lola, permite que Radamés te apresente Aída!

\repl{Lola} Folgo muito de conhecê-la. Como se chama?

\repl{Benvinda} Benv\ldots{} \paren{Emendando.} Fredegonda.

\repl{Eusébio} \paren{À parte.} Fredegonda? Uê! Benvinda mudô de nome!\ldots{}

\repl{Figueiredo} Espero que lhe emprestes um raio da tua luz fulgurante!

\repl{Lola} Pode contar com a minha amizade.

\repl{Figueiredo} Agradece.

\repl{Benvinda} Merci.

\repl{Eusébio} \paren{À parte.} Aí, mulata!\ldots{}

\repl{Lola} \paren{Vendo Eusébio.} Bravo! Não imagina como lhe fica bem essa fatiota!

\repl{Eusébio} Diz que é vestuário de conde.

\repl{Lola} Está irresistível!

\repl{Eusébio} Só a madama podia me metê nestas funduras!

\repl{Blanchette} \paren{A Lola.} Onde foste arranjar aquilo?

\repl{Lola} Cala-te! É um tesouro, um roceiro rico\ldots{} e primitivo!

\repl{Blanchette} Tiraste a sorte grande!

\repl{Lola} Meus amigos, espera-os na sala de jantar um ponche, um ponche
monumental, que mandei preparar no intuito de animar as pernas para a dança
e os corações para o amor!

\repl{Todos} Bravo! Bravo!\ldots{}

\repl{Figueiredo} Um ponche! Nesse caso, é preciso apagar as luzes!

\repl{Lola} Já devem estar apagadas. \paren{A Eusébio.} Fica. Preciso falar-te.

\repl{Mercedes} Ao ponche, meus senhores!

\repl{Todos} Ao ponche!\ldots{}

\repl{Blanchette} \paren{A Lola.} Não vens?

\repl{Lola} Vão indo. Eu já vou. Manda-me aqui algumas taças.

\repl{Dolores} Ao ponche!

 Coro

 Vamos ao ponche flamejante!
 Vamos ao ponche sem tardar!
 O ponche aquece um peito amante
 E as cordas da alma faz vibrar!

\repl{} \paren{Saem todos, menos Lola e Eusébio.}

\newscenenamed{Cena V}
\stagedir{Eusébio, Lola}

\repl{Lola} Oh! finalmente estamos sós um instante!

\repl{Eusébio} \paren{Em êxtase.} Como a madama tá bonita!

\repl{Lola} Achas?

\repl{Eusébio} Juro por esta luz que nos alumeia que nunca vi uma muié tão
fermosa!\ldots{}

\repl{Lola} Hei de pedir a Deus que me conserve assim por muito tempo para que
eu nunca te desagrade! \paren{Entra Lourenço com uma bandeja cheia de taças de
ponche chamejante.}

\newscenenamed{Cena VI}
\stagedir{Os mesmos, Lourenço}

\repl{Eusébio} Adeusinho, seu Lourenço. Como passou de indagorinha pra cá?

\repl{Lourenço} \paren{Imperturbável e respeitoso.} Bem; agradecido a vossa
excelência.

\repl{Lola} Deixe a bandeja sobre esta mesa e pode retirar-se. \paren{Lourenço obedece
e vai a retirar-se.}

\repl{Eusébio} Até logo, seu Lourenço. \paren{Aperta-lhe a mão.}

\repl{Lourenço} Oh! excelentíssimo! \paren{Faz uma mesura e sai, lançando um olhar
significativo a Lola.}

\repl{Lola} \paren{À parte.} É um bruto!

\newscenenamed{Cena VII}
\stagedir{Lola, Eusébio}

\repl{Eusébio} Este seu Lourenço é muito delicado. Arruma incelência na gente
que é um gosto!

\repl{Lola} \paren{Oferecendo-lhe uma taça de ponche.} À nossa saúde!

\repl{Eusébio} Bebida de fogo? Não! não é o fio de meu pai!\ldots{}

\repl{Lola} Prova, que hás de gostar. \paren{Eusébio prova.} Então, que tal? \paren{Ele bebe
toda a taça.}

\repl{Eusébio} Home, é muito bão! Cumo chama isto?

\repl{Lola} Ponche.

\repl{Eusébio} Uê! Ponche não é aquela coisa que a gente veste cando amonta a
cavalo?

\repl{Lola} Aqui tens outra taça.

\repl{Eusébio} Isto não faz má? Eu não tenho cabeça forte!

\repl{Lola} Podes beber sem receio.

\repl{Eusébio} Então à nossa, pra que Deus nos livre de alguma coça! \paren{Bebe.}

\repl{Lola} Dize\ldots{} dize que hás de ser meu\ldots{} dá-me a esperança de ser um dia
amada por ti!\ldots{}

\repl{Eusébio} Eu já gosto de madama cumo quê!

\repl{Lola} Não digas a madama. Trata-me por tu.

\repl{Eusébio} Não me ajeito\ldots{} pode sê que despois\ldots{}

\repl{Lola} Depois do quê?

\repl{Eusébio} \paren{Com riso tolo e malicioso.} Ah! ah!

\repl{Lola} \paren{Dando-lhe outra taça.} Bebe!

\repl{Eusébio} Ainda?

\repl{Lola} Esgotemos juntos esta taça! \paren{Bebe um gole e dá a taça a Eusébio.}

\repl{Eusébio} Vou sabê dos seus segredo. \paren{Bebe.}

\repl{Lola} E eu dos teus. \paren{Bebe.} Oh! o teu segredo é delicioso\ldots{} tu gostas
muito de mim\ldots{} da tua Lola\ldots{} mas receias que eu não seja sincera\ldots{} tens medo de
que eu te engane\ldots{}

\repl{Eusébio} \paren{Indo a dar um passo e cambaleando.} Minha Nossa Senhora! Eu tou
fora de mim! parece que tou sonhando!\ldots{} O tar ponche tem feitiço\ldots{} mas é
bão\ldots{} é muito bão!\ldots{} Quero mais!

 Dueto

 Lola

 Dize mais uma vez! Dize que me amas!

 Eusébio

 Eu já disse e arrepito!

 Lola

 O coração me inflama!
 Vem aos meus braços! Vem!
 Assim como eu te amo, ai! nunca amei ninguém!
 Se deste afeto duvidas,
 Se me imaginas perjura,
 Com essas mãos homicidas
 Me cavas a sepultura!
 Será o golpe certeiro,
 A morte será horrenda!
 Tu és o meu fazendeiro
 E eu sou a tua fazenda!

 Eusébio

 Se é moda a bebedeira, tou na moda, pois vejo toda a casa andando à roda!

 Lola

 Bebe ainda uma taça
 Agora pode ser que bem te faça!
 
 Eusébio
 
 \paren{Depois de beber.}
 Não posso mais! \paren{Atira a taça.}
 Oh, Lola, eu tou perdido!

 Lola

 Vem cá, meu bem querido!

 Juntos

 Lola Eusébio

 Vem aos meus braços! Tou nos seus braço!
 Eusébio, vem! Aqui me tem!
 Os meus abraços Mas os abraço
 Te fazem bem! Não me faz bem!

\repl{Eusébio} Oh! tou cuma fogueira aqui dentro! mas é tão bão! \paren{Abraçando
Lola.} Lola, eu sou teu\ldots{} só teu\ldots{} faz de mim o que tu quisé, minha negra!

\repl{Lola} Meu? Isso é verdade? Tu és meu? Meu?

\repl{Eusébio} Sim, sou teu! Tá aí! E agora? Sou teu e de mais ninguém!\ldots{}

\repl{Lola} Então, esta casa é tua! És o meu senhor, o meu dono, e como tal
quero que todos te reconheçam! \paren{Indo à porta e batendo palmas.} Eh! Olá! Venham
todos!\ldots{} venham todos! \paren{Música na orquestra.}

\newscenenamed{Cena VIII}

\paren{Todos os personagens do ato.}

\begin{center}
\textsc{final}
\end{center}

 Coro

 Lola nos chama!
 Que aconteceu?
 Que nos quer Lola?
 Que sucedeu?

 Lola

 Meus amigos, desejo neste instante
 Apresentar-lhes o meu novo amante!
 Ele aqui está! Eu o amo e ele me ama.

\repl{Eusébio} Sim! Aqui tá o home da madama!

\repl{Todos} Ele!\ldots{} \paren{Admiração geral.}

 Lola

 És o meu novo dono!
 Pode dizer-me: És minha!
 É teu, é teu somente
 O meu sincero amor!
 Eu dava-te o meu trono
 Se fosse uma rainha!
 Tu, exclusivamente,
 És hoje o meu senhor!

 Eusébio

 Sou eu seu novo dono!
 Posso dizê: É minha!
 É meu unicamente
 O seu sincero amô!
 Por ela eu me apaixono!
 A Lola é bonitinha!
 Eu, exclusivamente,
 Sou hoje o seu sinhô!

 Lola

 És o meu novo dono! etc\ldots{}

 Coro

 Eis o seu novo dono!
 Pode dizer: É minha!
 É dele unicamente
 O seu sincero amor!
 Gostar assim de um mono
 É sorte bem mesquinha!
 Ele, exclusivamente,
 É hoje o seu senhor!\ldots{}

 Figueiredo

\repl{} \paren{A Eusébio.}
 Nossos cumprimentos,
 Meu amigo, aos centos
 Queira receber!
 E como hoje é trunfo,
 Levado em triunfo
 Agora vai ser!

\paren{Figueiredo e Rodrigues carregam Eusébio. Organiza-se uma pequena marcha,
que faz uma volta pela cena, levando o fazendeiro em triunfo.}

 Coro

 Viva! viva o fazendeiro
 Bonachão e prazenteiro
 Que de um peito bandoleiro
 Os rigores abrandou,
 Conquistando a linda Lola,
 Essa esplêndida espanhola
 Que o país da castanhola
 Generoso nos mandou!

\repl{} \paren{Eusébio é posto sobre uma mesa ao centro da cena.}

 Eusébio

 Obrigado!
 Obrigado!
 Mas eu tô muito chumbado!
 Vejo tudo dobrado!

 Lola

 Dancem! dancem! tudo dance!
 Ninguém canse
 No cancã,
 Pois quem se acha aqui presente
 Tudo é gente
 Folgazã!

 Coro

\repl{Sim! dancemos!} tudo dance!
 Ninguém canse
 No cancã,
 Pois quem se acha aqui presente
 Tudo é gente
 Folgazã!

\paren{Cancã desenfreado em volta da mesa.}

\begin{center}
\textsc{pano}
\end{center}

\newact

\quadro{Quadro VIII}

A saleta de Lola

\newscenenamed{Cena I}
\stagedir{Eusébio, Lola}

\paren{Eusébio, ridiculamente vestido à moda, prepara um enorme cigarro mineiro.
Lola, deitada no sofá, lê um jornal e fuma.}

\repl{Eusébio} Isto tá o diabo! Não sei de Dona Fortunata\ldots{} não sei de
Quinota\ldots{} não sei de Juquinha\ldots{} não sei de seu Gouveia\ldots{} Não tenho corage de entrá em
casa!\ldots{} Se eu me confessá, não encontro um padre que me absorva!\ldots{} Lola, Lola,
que diabo de feitiço foi este?\ldots{} Tu fez de mim o que tu bem quis!

\repl{Lola} Estás arrependido?

\repl{Eusébio} Não, arrependido, não tou, porque a coisa não se pode dizê que
não seje boa\ldots{} Mas minha pobre muié deve está furiosa!\ldots{} E então quando ela
me vi assim todo janota, co’esta roupa de arfaiate francês, feito monsiú da Rua
do Ouvidô\ldots{} Oh! Lola! Lola! as muié é os tormento dos home!\ldots{} \paren{Lola que se
tem levantado e que tem ido, um tanto inquieta, até à porta da esquerda, volta
ao proscênio e vem encostar-se ao ombro de Eusébio.}


\repl{Lola} O tormento! Oh! não!\ldots{}

 Coplas

 -I-

 Meu caro amigo, esta vida
 Sem a mulher nada val!
 É sopa desenxabida,
 Sem uma pedra de sal!
 Se a dor torna um homem triste,
 Tem ele cura, se quer;
 A própria dor não resiste
 Aos beijos de uma mulher!

 - II -

 Ao lado meu, queridinho,
 Serás ditoso e feliz;
 Terás todo o meu carinho,
 É o meu amor que to diz.
 Se tu me amas como eu te amo,
 Se respondes aos meus ais,
 Nada mais de ti reclamo,
 Não te peço nada mais!

\repl{Eusébio} Mas\ldots{} me diz uma coza ; diabo, fala tua verdade\ldots{} Tu tá
inteiramente curada de seu Gouveia?

\repl{Lola} Não me fales mais nisso! Foi um sonho que passou. \paren{Pausa.} A
propósito de sonho\ldots{} foste ver na vitrine do Luís de Resende o tal broche com que eu
sonhei?

\repl{Eusébio} \paren{Coçando a cabeça.} Fui\ldots{} sabe quanto custa?

\repl{Lola} \paren{Com indiferença.} Sei\ldots{} uma bagatela\ldots{} um conto e oitocentos\ldots{}
\paren{Sobe e vai de novo observar à porta da esquerda.}

\repl{Eusébio} \paren{À parte.} Sim, é uma bagatela\ldots{} a espanhola gosta de mim, é
verdade, mas em tão poucos dias já me custa cinco contos de réis! e agora o colá!\ldots{}

\repl{Lola} \paren{À parte.} Que demora! \paren{Alto, descendo.} Mas enfim? o colar? Se é um
sacrifício, não quero!

\repl{Eusébio} O home ficou de fazê um abatimento e me mandá a resposta.

\repl{Lola} \paren{À parte.} É meu!

\repl{Eusébio} Se ele deixá por um conto e quinhento, compro! Não dou nem mais
um vintém.

\repl{Lola} \paren{À parte.} Sobem a escada. É ele!\ldots{}

\repl{Eusébio} Parece que vem gente. \paren{Batem com força à porta.} Quem é?

\repl{Lola} Deixa. Eu vou ver. \paren{Vai abrir a porta. Lourenço entra
arrebatadamente. Traz óculos azuis, barbas postiças, chapéu desabado e veste um sobretudo com a
gola erguida. Lola finge-se assustada.}

\newscenenamed{Cena II}
\stagedir{Os mesmos, Lourenço}

\repl{Lourenço} Minha rica senhora, folgo de encontrá-la!

\repl{Eusébio} Que é isto?

\repl{Lourenço} Fui entrando para não lhe dar tempo de me mandar dizer que não
estava em casa! É esse o seu costume!

\repl{Lola} Senhor!

\repl{Eusébio} Quem é este home danado?

\repl{Lourenço} Quem sou eu?\ldots{} Um credor que quer o seu dinheiro! Quer saber
também quem é esta senhora? Quer saber? É uma caloteira!

\repl{Lola} Que vergonha! \paren{Cai sentada e cobre o rosto com as mãos.}

\repl{Eusébio} O sinhô é um grande marcriado! Não se insurta assim uma fraca
muié que está em sua casa! Faça favô de saí!\ldots{}

\repl{Lourenço} Sair? Eu não saio daqui sem o meu rico dinheiro! O senhor, que
tem cara de homem sério, naturalmente há de julgar que sou grosseirão, um
bruto; mas não imagina a paciência que tenho tido até hoje! \paren{Batendo com a bengala no
chão.} Venho disposto a receber o meu dinheiro!\ldots{}

\repl{Eusébio} Mas dinheiro de quê?

\repl{Lourenço} De quê? Como de quê?\ldots{} Dinheiro que me deve esta senhora!
Dinheiro limpo, que me pediu há quatorze meses para pagar no fim de trinta
dias!\ldots{}

\repl{Lola} \paren{Descobrindo o rosto muito chorosa.} Com juros de sessenta por cento
ao ano!

\repl{Lourenço} Eu dispenso os juros! Isto prova que não sou nenhum agiota! O
que eu quero, o que eu exijo, é o meu capital, os meus dois contos de réis, que me
saíram limpinhos da algibeira e seriam quase o dobro com juros acumulados!

\repl{Lola} \paren{Suplicante.} Senhor, eu pagarei esse dinheiro logo que puder\ldots{}
Poupe-me tamanha vergonha diante deste cavalheiro que estimo e respeito!

\repl{Lourenço} Ora deixe-se de partes! Se a senhora não se quisesse sujeitar a
estas cenas, solveria os seus compromissos! Mas não passa, já disse, de uma reles
caloteira!\ldots{}

\repl{Eusébio} Home, o sinhô arrepare que eu tou aqui! Faça favô de vê como
fala!\ldots{}

\repl{Lourenço} Quem é o senhor? É marido desta senhora? É seu pai? É seu tio? É
seu padrinho? É seu irmão? É seu parente? Com que direito intervém? Eu
tenho ou não tenho razão? Fui ou não fui caloteado?

\repl{Eusébio} Home, o sinhô se cale! Olhe que eu sou mineiro!

\repl{Lourenço} Não me calo, ora aí está! E declaro que não me retiro daqui sem
estar pago e satisfeito! \paren{Senta-se.}

\repl{Eusébio} Seu home, olhe que eu\ldots{}!

\repl{Lourenço} \paren{Erguendo-se.} Eh! lá! Eh! lá! Agora sou eu que lhe digo que se
cale! O senhor não tem o direito de abrir o bico!\ldots{}

\repl{Lola} \paren{Chorando.} Que vergonha! Que vergonha!

\repl{Eusébio} \paren{À parte.} Coitadinha!\ldots{}

\repl{Lourenço} A princípio supus que o senhor fosse o amante desta senhora.
Vejo que me enganei\ldots{} Se o fosse, já teria pago por ela, e não consentiria que
eu a insultasse!\ldots{}

\repl{Eusébio} Hein?

\repl{Lola} \paren{Erguendo-se e correndo a Eusébio.} Não! Não! Sou eu que não
consinto que tu pagues!\ldots{} Não! Não tires a carteira! Eu mesma pagarei essa dívida!

\repl{Lourenço} Mas há de ser hoje, porque eu não me levanto desta cadeira!
\paren{Torna a sentar-se.}

\repl{Eusébio} Mas eu\ldots{}

\repl{Lola} Não! não pagues! Esse dinheiro pedi-o para mandá-lo a minha mãe, que
está em Valladolid\ldots{} Eu é que devo pagá-lo \paren{Voltando suplicante para
Lourenço.}\ldots{} mas não hoje!\ldots{}

\repl{Lourenço} \paren{Batendo com a bengala.} Há de ser hoje!\ldots{}

\repl{Lola} Não posso! não posso!\ldots{}

\repl{Lourenço} Não pode?\ldots{} Dê-me esse par de bichas que traz nas orelhas e
ficarei satisfeito!

\repl{Lola} Estas bichas custaram três contos!

\repl{Lourenço} São os juros.

\repl{Lola} Pois bem! \paren{Vai a tirar as bichas.}

\repl{Eusébio} \paren{Pegando-lhe no braço.} Não tira as bichas, Lola!\ldots{} \paren{Ao credor.}
- Seu desgraçado, não tenho dois conto aqui no borso, mas me acompanha na casa do
meu correspondente, na Rua de São Bento\ldots{} vem recebê o teu mardito
dinheiro!

\repl{Lourenço} \paren{Batendo com a bengala.} Já disse que daqui não saio!

\repl{Lola} \paren{Abraçando Eusébio.} Não, Eusébio, meu querido Eusébio! Não!\ldots{}

\repl{Eusébio} \paren{Sem dar ouvidos a Lola.} Pois não sai, não sai, desgraçado!
\paren{Desvencilhando-se de Lola.} Espera aí sentado, que eu vô buscá teu
dinheiro! \paren{Sai arrebatadamente. Lola, depois de certificar-se de que ele realmente saiu,
volta, e desata a rir às gargalhadas. Lourenço levanta-se, tira os óculos, as barbas
e o chapéu, e também ri às gargalhadas.}

\newscenenamed{Cena III}
\stagedir{Lola, Lourenço}

\repl{Lola} Soberbo! soberbo! Foi uma bela ideia! Toma um beijo! \paren{Dá-lhe um
beijo.}

\repl{Lourenço} Aceito o beijo, mas olhe que não dispenso os vinte por cento.

\repl{Lola} Naturalmente.

\repl{Lourenço} Você há de convir que sou um grande artista!

\repl{Lola} E então eu?

\repl{Lourenço} Você também, mas se eu me houvesse feito cômico em vez de me
fazer cocheiro, estava a estas horas podre de rico!

 Tango

 -I-

 Ai! que jeito pro teatro!
 Que vocação!
 Eu faria o diabo a quatro
 Num dramalhão!
 Mas às rédeas e ao chicote
 Jungido estou!
 Sou cocheiro de cocote!
 Nada mais sou!
 Cumprir o nosso destino
 Nem eu quis nem você quis!
 Fui ator desde menino
 E você foi sempre atriz!

 - II -

 Quando eu era mais mocinho
 \paren{Posso afiançar!}
 Fiz furor num teatrinho
 Particular!
 Talvez outro João Caetano
 Se achasse em mim.
 Mas o fado desumano
 Não quis assim!
 Cumprir o nosso destino, etc\ldots{}

\repl{Lola} Mas por que não acompanhaste o fazendeiro? Era mais seguro!

\repl{Lourenço} Pois eu lá me atrevia a andar por essas ruas de barbas postiças!
Nada, que não queria dar com os ossos no xadrez!

\repl{Lola} Tens agora que esperar aqui a pé firme!

\repl{Lourenço} Estou arrependido de ter perdoado os juros. \paren{Batem à porta.}

\repl{Lola} Quem será?

\repl{Lourenço} \paren{Depois de espreitar.} É o filho-família.

\repl{Lola} Ah! o tal Duquinha! Tomaste as necessárias informações? Que me dizes
desse petiz?

\repl{Lourenço} \paren{Abanando a cabeça com ares de competência.} Digo que no seu
gênero não deixa de ser aproveitável\ldots{} O pai é muito severo, mas a mãe,
que é rica, satisfaz todos os seus caprichos\ldots{} Não digo que você possa esperar dali
mundos e fundos, mas é fácil obrigá-lo a contrair dívidas, se for preciso, para dar
alguns presentes, e ouro é o que ouro vale.

\repl{Lola} Manda-o entrar.

\repl{Lourenço} Não se demore muito, porque o fazendeiro foi a todo o vapor e
não tarda aí.

\repl{Lola} Temos tempo. A Rua de S. Bento é longe. \paren{Sai. Lourenço tira o
sobretudo, a que junta as barbas, os óculos e o chapéu, e vai abrir a porta a Duquinha.}

\newscenenamed{Cena IV}
\stagedir{Duquinha, Lourenço}

\paren{Duquinha tem dezoito anos e é muito tímido.}

\repl{Duquinha} A senhora dona Lola está em casa?

\repl{Lourenço} \paren{Muito respeitoso.} Sim, meu senhor\ldots{} e pede a V. Ex.a que
tenha o obséquio de esperar alguns instantes.

\repl{Duquinha} Muito obrigado. \paren{À parte.} É o cocheiro\ldots{} não sei se deva\ldots{}

\repl{Lourenço} Como diz V. Ex.a?

\repl{Duquinha} Se não fosse ofendê-lo, pedia-lhe que aceitasse\ldots{} \paren{Tira a
carteira.}

\repl{Lourenço} Oh! não!\ldots{} Perdoe V. Ex.a\ldots{} não é orgulho; mas que diria a
patroa se soubesse que eu\ldots{}

\repl{Duquinha} Ah! nesse caso\ldots{} \paren{Guarda a carteira.}

\repl{Lourenço} \paren{Que ia sair, voltando.} Se bem que eu estou certo que V. Ex.a
não diria nada à senhora dona Lola\ldots{}

\repl{Duquinha} \paren{Tirando de novo a carteira.} Ela nunca o saberá. \paren{Dá-lhe
dinheiro.}

\repl{Lourenço} Beijo as mãos de V. Ex.a A senhora dona Lola é tão escrupulosa!
\paren{À parte.} Uma de trinta! O franguinho promete\ldots{} \paren{Sai com muitas mesuras,
levando o sobretudo e demais objetos.}

\newscenenamed{Cena V}

\repl{Duquinha} Estou trêmulo e nervoso\ldots{} É a primeira vez que entro em casa de
uma destas mulheres\ldots{} Não pude resistir!\ldots{} A Lola é tão bonita, e o outro
dia, no Braço de Ouro, me lançou uns olhares tão meigos, tão provocadores, que tenho sonhado
todas as noites com ela! Até versos lhe fiz, e aqui lhos trago\ldots{} Quis
comprar-lhe uma joia, mas receoso de ofendê-la, comprei apenas estas flores\ldots{} Ai, Jesus!
ela aí vem! Que lhe vou dizer?\ldots{}

\newscenenamed{Cena VI}
\stagedir{Duquinha e Lola}

\repl{Lola} Não me engano: é o meu namorado do Braço de Ouro! \paren{Estendendo-lhe a
mão.} Como tem passado?

\repl{Duquinha} Eu\ldots{} sim\ldots{} bem, obrigado; e a senhora?

\repl{Lola} Como tem as mãos frias!

\repl{Duquinha} Estou muito impressionado. É uma coisa esquisita: todas as vezes
que fico impressionado\ldots{} fico também com as mãos frias\ldots{}

\repl{Lola} Mas não se impressione! Esteja à sua vontade! Parece que não lhe
devo meter medo!

\repl{Duquinha} Pelo contrário!

\repl{Lola} \paren{Arremedando-o.} Pelo contrário! \paren{Outro som.} São minhas essas
flores?

\repl{Duquinha} Sim\ldots{} eu não me atrevia\ldots{} \paren{Dá-lhe as flores.}

\repl{Lola} Ora essa! Por quê? \paren{Depois de aspirá-las.} Que lindas são!

\repl{Duquinha} Trago-lhe também umas flores\ldots{} poéticas.

\repl{Lola} Umas quê?\ldots{}

\repl{Duquinha} Uns versos.

\repl{Lola} Versos? Bravo! Não sabia que era poeta!

\repl{Duquinha} Sou poeta, sim, senhora\ldots{} mas poeta moderno, decadente\ldots{}

\repl{Lola} Decadente? nessa idade?

\repl{Duquinha} Nós somos todos muito novos.

\repl{Lola} Nós quem?

\repl{Duquinha} Nós, os decadentes. E só podemos ser compreendidos por gente da
nossa idade. As pessoas de mais de trinta anos não nos entendem.

\repl{Lola} Se o senhor se demorasse mais algum tempo, arriscava-se a não ser
compreendido por mim.

\repl{Duquinha} Se dá licença, leio os meus versos. \paren{Tirando um papel da
algibeira.} Quer ouvi-los?

\repl{Lola} Com todo o prazer.

\repl{Duquinha} \paren{Lendo.}

 Ó flor das flores, linda espanhola,
 Como eu te adoro, como eu te adoro!
 Pelos teus olhos, ó Lola! ó Lola!
 De dia canto, de noite choro,
 Linda espanhola, linda espanhola!

\repl{Lola} Dir-se-ia que o trago de canto chorado!

\repl{Duquinha} Ouça a segunda estrofe:

 És uma santa, santa das santas!
 Como eu te adoro, como eu te adoro!
 Meu peito enlevas, minh’alma encantas!
 Ouve o meu triste canto sonoro,
 Santa das santas, santa das santas!

\repl{Lola} Santa? Eu!\ldots{} Isto é que é liberdade poética!

\repl{Duquinha} A mulher amada pelo poeta é sempre santa para ele! Terceira e
última estrofe\ldots{}

\repl{Lola} Só três? Que pena!

\repl{Duquinha} \paren{Lendo.}

 Ó flor das flores! bela andaluza!
 Como eu te adoro, como eu te adoro!
 Tu és a minha pálida musa!
 Desses teus lábios um beijo imploro,
 Bela andaluza, bela andaluza!

\repl{Lola} Perdão, mas eu não sou da Andaluzia: sou de Valladolid.

\repl{Duquinha} Pois há espanholas bonitas que não sejam andaluzas?

\repl{Lola} Pois não! o que não há são andaluzas bonitas que não sejam
espanholas.

\repl{Duquinha} Hei de fazer uma emenda.

\repl{Lola} E que mais?

\repl{Duquinha} Como?

\repl{Lola} O senhor trouxe-me flores\ldots{} trouxe-me versos\ldots{} e\ldots{} não me trouxe
mais nada?

\repl{Duquinha} Eu?

\repl{Lola} Sim\ldots{} Os versos são bonitos\ldots{} as flores são cheirosas\ldots{} mas há
outras coisas de que as mulheres gostam muito.

\repl{Duquinha} Uma caixinha de marrons glacés?

\repl{Lola} Sim, não digo que não\ldots{} é uma boa gulodice\ldots{} mas não é isso\ldots{}

\repl{Duquinha} Então que é?

\repl{Lola} Faça favor de me dizer para que se inventaram os ourives.

\repl{Duquinha} Ah! já percebo\ldots{} Eu devia trazer-lhe uma joia!

\repl{Lola} Naturalmente. As joias são o “Sésamo, abre-te” destas cavernas de
amor.

\repl{Duquinha} Eu quis trazer-lhe uma joia, quis; mas receei que a senhora se
ofendesse\ldots{}

\repl{Lola} Que me ofendesse?\ldots{} Oh! santa ingenuidade!\ldots{} Em que é que uma joia
poderia ofender? Querem ver que o meu amiguinho me toma por uma respeitável
mãe de família? Creia que um simples grampo de chapéu, com um bonito
brilhante, produziria mais efeito que todo esse:

 Como te adoro, como te adoro!
 Linda espanhola, linda espanhola,
 Santa das santas, santa das santas!

\repl{Duquinha} Vejo que lhe não agrada a escola decadente\ldots{}

\repl{Lola} Confesso que as joias exercem sobre mim uma fascinação maior que a
literatura. E demais, não sou mulher a quem se ofereçam versos\ldots{} Vejo que
o senhor não é de opinião de Bocage\ldots{}

\repl{Duquinha} Oh! Não me fale em Bocage!

\repl{Lola} Que mania essa de não nos tomarem pelo que somos realmente! Guarde
os seus versos para as donzelinhas sentimentais, e, ande, vá buscar o “Sésamo,
abre-te” e volte amanhã. \paren{Empurra-o para o lado da porta. Entra Lourenço.}

\repl{Duquinha} Mas\ldots{}

\repl{Lola} Vá, vá! Não me apareça aqui sem uma joia. \paren{A Lourenço.} Lourenço,
conduza este senhor até a porta. \paren{Sai pela direita.}

\repl{Duquinha} Não, não é preciso, não se incomode. \paren{À parte.} Vou pedir
dinheiro a mamãe. \paren{Sai.}

\newscenenamed{Cena VII}

\repl{Lourenço} Às ordens de Vossa Excelência. \paren{Só.} A Lola saiu-me uma
artista de primeiríssima ordem! -- Bom! vou caracterizar-me de credor, que o fazendeiro
não tarda por aí. Quatrocentos mil-réis cá para o degas! Que bom! Hão de grelar
esta noite no Belódromo, onde conto organizar uma mala onça! \paren{Sai cantarolando o
tango. Mutação.}

\quadro{Quadro IX}

No Belódromo Nacional

\newscenenamed{Cena I}
\stagedir{Lemos, Guedes, um Frequentador do Belódromo, pessoas do povo, depois
amadores, depois S’il vous-plaît, depois Lourenço}

\paren{Durante todo este ato, ouve-se, a intervalos, o som de uma sineta que
chama os compradores à casa das pules, à esquerda, e uma voz que grita: “Vai
fechar!”}

 Coro

 Não há nada como 
 vir ao Belódromo! 
 São estas corridas 
 Muito divertidas! 
 Desgraçadamente 
 Muito raramente 
 O povo, coitado! 
 Não é cá roubado! 
 Mas o cabeçudo, 
 Apesar de tudo, 
 Pules vai comprando, 
 Sempre protestando! 
 Tipos aqui pisam, 
 Mestres em cabalas, 
 E elas organizam 
 As famosas malas! 
 E com artimanha 
 \paren{Manha mais do que arte} 
 Quase sempre ganha 
 Pífio bacamarte! \paren{Entrada dos amadores.}

 Coro de Amadores

 Aqui estamos os melhores 
 Amadores 
 Da elegante bicicleta! 
 Nós corremos, prazenteiros, 
 Mais ligeiros, 
 Mais velozes que uma seta! 
 A todo o público 
 Dos belódromos 
 Muito simpáticos 
 Se diz que somos. 
 O povo aplaude-nos 
 Quando vencemos, 
 Mas também vaia-nos 
 Quando perdemos! 
 Aqui estamos os melhores, etc\ldots{}

\repl{O Frequentador do Belódromo} \paren{A Lemos e Guedes.} Parece impossível!\ldots{} No 
páreo passado joguei no número 17 por ser a data em que minha mulher
morreu, e, por causa das dúvidas, joguei também no número 18, por ser a data em que
ela foi enterrada\ldots{} e ganhou o número 19! Parece impossível!\ldots{}

\repl{Lemos} É verdade! Parece! \paren{A Guedes.} Você já viu velho mais cabuloso?

\repl{O Frequentador} Agora vou jogar no 25\ldots{} Não pode falhar, porque a
sepultura dela tem o número 525.

\repl{Guedes} É\ldots{} é isso\ldots{} vá comprar, vá.

\repl{O Frequentador} Vou jogar uma em primeiro e duas em segundo. \paren{Afasta-se
para o lado da casa das pules.}

\repl{Lemos} E que me dizes a esta, ó Guedes? O S’il vous-plaît foi arranjar
tudo, e do Lourenço nem novas nem mandados!

\repl{Guedes} Quem sabe se ele teve de levar a Lola de carro a algum teatro?\ldots{}

\repl{Lemos} Qual! Não creias! Pois se ele é um cocheiro que faz da patroa o que
bem quer!\ldots{}

\repl{Guedes} Está só pelo diabo! Uma mala segura, e não há dinheiro para o
jogo!\ldots{} Olha, aqui está de volta o S’il vous-plaît.

\repl{S’il vous-plaît} \paren{Aproximando-se, vestido de corredor.} Venho da pista.
Está tudo combinado.

\repl{Lemos} Sim, mais ainda não temos o melhor! O caixa da mala não aparece!

\repl{S’il vous-plaît} Que diz você? Pois o Lourenço\ldots{}

\repl{Guedes} O Lourenço até agora!

\repl{Lourenço} \paren{Aparecendo entre eles.} Que estão vocês aí a falar do Lourenço?

\repl{Os Três} Ora graças!

\repl{Lourenço} Vocês sabem que eu sou de palavra\ldots{} Quando digo que venho é
porque venho!

\repl{Lemos} Estávamos sobre brasas!

\repl{Lourenço} Já estão vendendo?

\repl{Guedes} Há que tempos!

\repl{S’il vous-plaît} Já se fez a segunda apregoação.

\repl{Lourenço} O que está combinado?

\repl{S’il vous-plaît} Ganha o Menelik.

\repl{Lourenço} O Félix Faure não corre?

\repl{S’il vous-plaît} Corre.

\repl{Lourenço} Se tiver boa máquina, pode ganhar sem querer.

\repl{S’il vous-plaît} Está combinado que ele cairá na quinta volta.

\repl{Lourenço} Quantas voltas são?

\repl{S’il vous-plaît} Oito.

\repl{Lourenço} Quem mais corre?

\repl{S’il vous-plaît} O Garibaldi, o Carnot e o Colibri.

\repl{Lourenço} Que Colibri é esse?

\repl{S’il vous-plaît} É um pequenote\ldots{} um bacamarte\ldots{} não vale nada\ldots{} nem eu o meti 
na combinação!

\repl{Lourenço} Os outros quatro quanto recebem?

\repl{S’il vous-plaît} Quinze mil-réis cada um.

\repl{Lourenço} E dez por cento dos lucros para vocês três\ldots{} Bom. \paren{Dando
dinheiro a Lemos.} Tome, seu Lemos; vá comprar dez pules\ldots{} \paren{Dando dinheiro a Guedes.}
Tome, seu Guedes: compre outras dez\ldots{} Vá cada um por sua vez, para
disfarçar\ldots{} Senão, o rateio não dá para o buraco de um dente! Eu compro três cheques.
Vamos. \paren{Afastam-se todos.}

\newscenenamed{Cena II}
\stagedir{Benvinda, Figueiredo}

\repl{Benvinda} Me deixe! Já le disse que não quero mais sabê do sinhô!

\repl{Figueiredo} Por que, rapariga?

\repl{Benvinda} O sinhô co’essa mania de querê me lançá é um cacete insuportave! Tá 
sempre me dando lição e raiando comigo! Pra isso eu não percisava saí de
casa de sinhô Eusébio!

\repl{Figueiredo} Mas é para o teu bem que eu\ldots{}

\repl{Benvinda} Quais pera meu bem nem pera nada! Hei de encontrá quem me queira 
mesmo falando cumo se fala na roça!

\repl{Figueiredo} Estás bem aviada!

\repl{Benvinda} Eu memo posso me lançá sem precisar do sinhô!

\repl{Figueiredo} Oh! mulher, olha que tu não tens nenhuma experiência do mundo! És 
uma tola\ldots{} uma ignorantona\ldots{} não sabes o que é a capital-federal!

\repl{Benvinda} Cumo o sinhô se engana! Eu já tou meia capitalista-federalista!

\repl{Figueiredo} Bom; tu’alma, tua palma! Estou com a minha consciência
tranquila. Mas vê lá: se algum dia precisares de mim, procura-me.

\repl{Benvinda} Merci! \paren{Vai-se afastando.}

\repl{Figueiredo} Adeus, Fredegonda!

\repl{Benvinda} \paren{Parando.} Que Fredegonda! Assim é que o sinhô me lançô! Me deu 
logo um nome tão feio que toda a gente se ri quando ouve ele!

\repl{Figueiredo} É porque não sabem história! Fredegonda foi uma rainha\ldots{} era 
casada com Chilperico\ldots{}

\repl{Benvinda} Pois eu por minha desgraça não sou casada nem com seu Borge. Ó 
revoá. \paren{Afasta-se.}

\repl{Figueiredo} \paren{Só.} No fundo estou satisfeito, porque decididamente não
havia meio de fazer dela alguma coisa\ldots{} Parece que vai chover\ldots{} mas já agora vou
assistir à corrida. \paren{Afasta-se.}

\newscenenamed{Cena III}
\stagedir{Lourenço, Lemos, Guedes, depois O Frequentador do Belódromo}

\repl{Lourenço} Bom! venham as pules. \paren{Lemos e Guedes entregam as pules, que ele 
guarda.}

\repl{Lemos} A mala não transpirou. Félix Faure é o favorito.

\repl{Guedes} Queira Deus que o S’il vous-plaît não dê com a língua nos dentes!

\repl{O Frequentador} \paren{Voltando.} Comprei no 25\ldots{} Mas agora me lembro\ldots{}
somando o número da sepultura dá a soma de 12. 5 e 2-7; e 5-12. Ora 12 e 12 são 24.

\repl{Lemos} 24 é o tal Colibri. Não deite o seu dinheiro fora!

\repl{O Frequentador} Aceito o conselho\ldots{} Já cá tenho o 25\ldots{} e não pode
falhar! O diabo é que parece que vai chover. O tempo está muito entroviscado!
\paren{Afasta-se.}

\repl{Lourenço} \paren{Que tem estado a calcular.} Se o Félix Faure é o favorito, o
Menelik não pode dar menos de sete mil-réis.

\repl{Guedes} Para cima!

\repl{Lourenço} Separemo-nos. Creio que a diretoria já nos traz de olho\ldots{} No
fim da corrida esperá-los-ei no lugar do costume para a divisão dos lúcaros. Até
logo!

\repl{Lemos e Guedes} Até logo. \paren{Afastam-se. Benvinda volta passeando.}

\newscenenamed{Cena IV}
\stagedir{Lourenço e Benvinda}

\repl{Lourenço} \paren{Consigo.} Estes malandretes ganham pela certa\ldots{} não arriscam
um nicolau\ldots{} \paren{Vendo Benvinda.} Não me engano: é a celeste Aída do sábado de 
aleluia\ldots{} Reconhecerá ela na minha fisolostria o cocheiro da Lola?
Vejamos! \paren{Passa e acotovela Benvinda.} Adeus, coração dos outros!

\repl{Benvinda} Vá passando seu caminho e não bula ca gente!

\repl{Lourenço} Tão zangada, meu Deus!

\repl{Benvinda} Que deseja o sinhô?

\repl{Lourenço} Pelo menos saber onde mora.

\repl{Benvinda} Moro na rua das casa.

\repl{Lourenço} Não seja má! Bem sei que é aqui mesmo na Rua do Lavradio.

\repl{Benvinda} Quem le disse?

\repl{Lourenço} Ninguém. Fui eu que lhe vi na janela.

\repl{Benvinda} Pois não vá lá que não lhe arrecebo!

\repl{Lourenço} Por que não me arrecebe, marvada?

\repl{Benvinda} Vou sê franca\ldots{} Só arrecebo quem quisé me tirá desta vida. Não
nasci pra isto. Quero vivê em família.

\repl{Lourenço} Ah, seu benzinho! isso é que não pode ser! Hoje em dia não é
possível viver em família!

\repl{Benvinda} Por quê?

\repl{Lourenço} Por quê? Ainda me perguntas, amor?

 Coplas

 Lourenço

 -I-

 Já não se encontra casa decente, 
 Que custe apenas uns cem mil-réis, 
 E os senhorios constantemente 
 O preço aumentam dos aluguéis! 
 Anda o povinho muito inquieto, 
 E tem -- pudera! -- toda a razão; 
 Não aparece nenhum projeto 
 Que nos arranque desta opressão! 
 Um cidadão neste tempo 
 Não pode andar amarrado\ldots{} 
 A gente vê-se, e adeusinho: 
 Cada um vai pro seu lado!

 - II -

 Das algibeiras some-se o cobre, 
 Como levado por um tufão! 
 Carne de vaca não come o pobre, 
 E qualquer dia não come pão! 
 Fósforos, velas, couve, quiabos, 
 Vinho, aguardente, milho, feijão, 
 Frutas, conservas, cenouras, nabos, 
 Tudo se vende pr’um dinheirão! 
 Um cidadão neste tempo etc\ldots{}

\repl{Benvinda} Tenho sede, venha pagá um copo de cerveja.

\repl{Lourenço} Com muito gosto, mas da Babilônia, que as alamoas estão pela
hora da morte!

\repl{Benvinda} Vamo.

\repl{Lourenço} Como você se chama, seu benzinho?

\repl{Benvinda} Artemisa.

\repl{Lourenço} Que bonito nome! Vamos ali no botequim do Lopes. \paren{Saem.}

\newscenenamed{Cena V}
\stagedir{Eusébio, Lola, Mercedes, Dolores, Blanchette, depois Figueiredo}

\paren{Eusébio entra no meio das mulheres; traz o chapéu atirado para a nuca, e
um enorme charuto. Vêm todos alegres. Acabaram de jantar e lembraram-se de dar
uma volta pelo Belódromo.}

\repl{Eusébio} Não, Lola! Tu hoje há de me deixá i pra casa! Dona Fortunata deve
tá furiosa!

\repl{Lola} Que dona Fortunata nem nada!

\repl{Mercedes} Havemos de acabar a noite num gabinete do Munchen!

\repl{Dolores} Não o deixamos!

\repl{Blanchette} Está preso!\ldots{} E, demais, vamos ter chuva!

\repl{Eusébio} Na chuva já tou eu, se não me engano. Aquele vinho é bão, mas é
veiaco!

\repl{Figueiredo} \paren{Aproximando-se.} Olá! viva a bela sociedade!

\repl{Lola} Olha quem ele é! o Figueiredo!

\repl{Mercedes} O Radamés!

\repl{Dolores} Você no Belódromo!

\repl{Figueiredo} Por mero acaso\ldots{} Não gosto disto\ldots{} No Rio de Janeiro não há
divertimentos que prestem! Não temos nada, nada!

\repl{Eusébio} \paren{Num tom magoado.} Como vai a Fredegonda, seu Figueiredo?

\repl{Figueiredo} A Fredegonda já não é Fredegonda!

\repl{Todos} Ah!\ldots{}

\repl{Figueiredo} Tornou a ser Benvinda, como antigamente. Deixou-me!

\repl{Todos} Deixou-o?

\repl{Figueiredo} Deixou-me, e anda à procura de alguém que saiba lançá-la
melhor do que eu!

\repl{Eusébio} Uê!

\repl{Figueiredo} Deve estar aqui no Belódromo\ldots{} Acompanhei-a até cá para
pedir-lhe que tivesse juízo, mas a sua resolução é inabalável\ldots{} Pobre rapariga!\ldots{}

\repl{Eusébio} \paren{Muito comovido, para o que concorre o vinho que bebeu.} Coitada
da Benvinda!\ldots{} Podia tá casada e agora\ldots{} anda atirada por aí como uma coisa
à toa\ldots{} sem ninguém que tome conta dela\ldots{} \paren{Com lágrimas na voz.} Coitada!\ldots{} não
façum caso\ldots{} Eu vi ela pequena\ldots{} nasceu e cresceu lá em casa\ldots{} \paren{Chorando.}
Minha fia mamou o leite da mãe dela!

\repl{Todos} Que é isso?! Chorando?! Ora esta!\ldots{}

\repl{Eusébio} \paren{Com soluços.} Que chorando que nada! Já passou!\ldots{} Não foi
nada!\ldots{} Que qué vacês! Mineiro tem muito coração!\ldots{}

\repl{Todos} Vamos lá! Que é isso? Então?\ldots{}

\repl{Lola} Há de passar. São efeitos do Chambertin! -- Eusébio, ande\ldots{}
então?\ldots{} vá comprar umas pules para tomar interesse pela corrida.

\repl{Eusébio} Eu não entendo disso!

\repl{Figueiredo} Escolha um nome daqueles. Olhe, ali, na pedra\ldots{} Ligúria,
Carnot, Menelik, Colibri e Félix Faure!

\repl{Eusébio} Colibri! Eu quero Colibri!

\repl{Figueiredo} Ouvi dizer que não vale nada\ldots{} É o que aqui chamam um
bacamarte\ldots{} Não lhe sorri nenhum dos presidentes da República Francesa?

\repl{Eusébio} Não sinhô, não quero outro! Colibri é o nome de um jumento que
tenho lá na fazenda. Dolores, Mercedes e Blanchette \paren{Ao mesmo tempo.} Não faça isso! Se é 
bacamarte, não presta! É dinheiro deitado fora!

\repl{Lola} Deixem-no lá! É um palpite! Vá comprar cinco pules naquele guichê.

\repl{Eusébio} Naquele quê?

\repl{Figueiredo} Naquele buraco.

\repl{Eusébio} Cuanto custa?

\repl{Figueiredo} Cinco pules são dez mil-réis.

\repl{Eusébio} Mas como se faz?

\repl{Figueiredo} Estenda o braço, meta o dinheiro dentro do buraco, abra a mão,
e diga: “Colibri”.

\repl{Eusébio} Sim, sinhô. \paren{Afasta-se.}

\repl{Figueiredo} Pois é o que lhes conto: estou livre como o lindo amor!

\repl{Mercedes} Se me quiser tomar sob a sua valiosa proteção\ldots{}

\repl{Dolores} Se quiser fazer a minha ventura\ldots{}

\repl{Blanchette} Se me quiser lançar\ldots{}

\repl{Lola} Vocês estão a ler! Ele só gosta de\ldots{}

\repl{Figueiredo} \paren{Atalhando.} De trigueiras! Eu digo trigueiras, por ser menos
rebarbativo\ldots{} Acho que as brancas são encantadoras, apetitosas, adoráveis,
lindíssimas, mas que querem? -- tenho cá o meu gênero\ldots{}

\repl{Mercedes} Isso é um crime!

\repl{Dolores} Devia ser preso!

\repl{Blanchette} Deportado!

\repl{Lola} Sim, deportado\ldots{} para a Costa da África!\ldots{}

 Quinteto

 Lola

 Ó Figueiredo, eu cá sou franca:
 Estou com pena de você!

 As Outras

 Nós temos pena de você!

 Figueiredo

 Façam favor, digam por quê!

 Lola

 Por não gostar da mulher branca!

 As Outras

 Por não gostar da mulher branca!

 Figueiredo

 Meu Deus! Deveras? 
 Por isso só?

 Todas

 Somos sinceras!
 Causa-nos dó!

 Figueiredo

 Oh! oh! oh! oh!

 Todas

 Oh! oh! oh! oh!

 Lola

 -I-

 Pele cândida e rosada, 
 Cetinosa e delicada 
 Sempre teve algum valor!

 Figueiredo

 Que tolice!

 Todas

 Sim, senhor!

 Lola

 A cor branca, pelo menos, 
 Era a cor da loura Vênus, 
 Deusa esplêndida do amor.

 Figueiredo

 Quem lhe disse?

 Todas

 Sim, senhor!

 Figueiredo

 Se eu da Mitologia 
 Fosse o reformador, 
 Vênus transformaria 
 Numa mulata!

 Todas

 Horror!\ldots{}

 Figueiredo

 - II -

 A mimosa cor do jambo 
 Pede um meigo ditirambo 
 Cinzelado com primor!

 Lola

 Que tolice!

 Todas

 Não, senhor!

 Figueiredo

 Eu com os ovos, por sistema, 
 Deixo a clara e como a gema, 
 Porque tem melhor sabor.

 Lola

 Quem lhe disse?

 Todas

 Não, senhor!

 Figueiredo

 Se eu da Mitologia 
 Fosse o reformador 
 Vênus transformaria 
 Numa mulata!

 Todas

 Horror!\ldots{}

 Juntos

 Figueiredo As Cocotes

 Gosto do amarelo! Gosta do amarelo! 
 Que prazer me dá! Maus exemplos dá! 
 Nada mais anelo, Vara de marmelo 
 Nem aspiro já! Merecia já!

\repl{Eusébio} \paren{Voltando.} Aqui tá cinco paperzinho do Colibri. Custou! Toda a
gente queria comprá! Eu meti o dinheiro no buraco, e o home lá de dentro
perguntou: “Onde leva?” Eu respondi: “Colibri”, e ele ficou muito espantado, e disse:
“É o premero que compra nesse bacamarte.”

\repl{Figueiredo} Vamos ver a corrida lá de cima. Pedirei um camarote ao
Cartaxo.

\repl{Todos} Vamos! \paren{Saem.}

\newscenenamed{Cena VI}
\stagedir{Benvinda, Lourenço e Povo}

\repl{Lourenço} \paren{Correndo.} Correndo ainda apanho; mas olhe que o Menelik\ldots{} 
\paren{Desaparece.}

\repl{Benvinda} Não sinhô, não sinhô! Não quero Menelik! Compre no que eu disse!
\paren{Só, no proscênio.} Não gosto deste home: tem cara de padre\ldots{} é munto
enjoado\ldots{} Nem deste, nem de nenhum\ldots{} Não gosto de ninguém\ldots{} O que eu tenho a fazê
de mió é vortá para casa e pedi perdão a sinhá veia. \paren{Ouve-se o sinal do
fechamento do jogo.}

\repl{Pessoas do Povo} Fechou! Fechou! Ora! e eu que não comprei! \paren{Dirigem-se 
todos para o fundo: vão assistir à corrida.}

\repl{Lourenço} \paren{Voltando.} Sempre cheguei a tempo de comprar a pule! \paren{Dando a
pule a Benvinda.} Mas que lembrança a sua de jogar no Colibri!

\repl{Benvinda} É porque é o nome de um burrinho que há numa fazenda onde eu fui 
passá uns tempo.

\repl{Lourenço} Ah! é cabula? \paren{Ouve-se um toque de campainha elétrica.} Se ele 
vencesse, você levava a casa das pules! \paren{Ouve-se um tiro de revólver e um
pouco de música.} Começou a corrida! Vamos ver! \paren{Afastam-se para o fundo.}

\newscenenamed{Cena VII}
\stagedir{Gouveia, Fortunata e Quinota}

\repl{Fortunata} \paren{Entrando apressada à frente de Gouveia e Quinota.} Não! não
quero vê meu fio corrê na tar história!\ldots{} E logo que acabá a corrida, levo ele
pra casa, e aqui não vorta!\ldots{} Que coza!\ldots{} Benvinda desaparece\ldots{} Seu Eusébio
desaparece\ldots{} Juquinha não sai do Belódromo\ldots{} Tou vendo cuando Quinota me deixá!\ldots{}

\repl{Quinota} Oh! mamãe! não tenha esse receio!

\repl{Fortunata} Que terra! Eu bem não queria vi no Rio de Janeiro!

\repl{Quinota} Que vida tão diversa da vida da roça! \paren{A Gouveia.} Não ficaremos
aqui depois de casados.

\repl{Gouveia} Por quê?

\repl{Quinota} A vida fluminense é cheia de sobressaltos para as verdadeiras
mães de família!

\repl{Fortunata} Olhe seu Eusébio, um home de cinquenta ano, que teve até agora 
tanto juízo! Arrespirou o á da Capitá Federá, e perdeu a cabeça!

\repl{Gouveia} Apanhou o micróbio da pândega!

\repl{Quinota} Aqui há muita liberdade e pouco escrúpulo\ldots{} faz-se ostentação do 
vício\ldots{} não se respeita ninguém\ldots{} É uma sociedade mal constituída!

\repl{Gouveia} Não a supunha tão observadora\ldots{}

\repl{Quinota} Eu sou roceira, mas não tola que não veja o mal onde se acha.

\repl{Fortunata} Parece que já tá chuviscando\ldots{} Eu senti um pingo\ldots{}

\repl{Quinota} O senhor, por exemplo, o senhor, se pensa que me engana,
engana-se. Conheço perfeitamente os seus defeitos.

\repl{Fortunata} \paren{À parte.} Aí!

\repl{Gouveia} Os meus defeitos?

\repl{Quinota} Oh! são muitíssimos -- e o menor deles não é querer aparentar uma 
fortuna que não existe. Desagradam-me esses visíveis esforços que o senhor
faz para iludir os outros. O melhor partido que o senhor tem a tomar\ldots{} e olhe
que este é o conselho da sua noiva, isto é, da pessoa que mais o estima neste mundo\ldots{}
o melhor partido que o senhor tem a tomar é abrir-se com papai\ldots{}
confessar-lhe que é um jogador arrependido\ldots{}

\repl{Gouveia} Oh! Quinota!\ldots{}

\repl{Fortunata} Não tem -- ó Quinota nem nada! É a verdade!\ldots{}

\repl{Quinota} Irá conosco para a fazenda, onde não lhe faltará ocupação.

\repl{Fortunata} Sim sinhô; é mió trabaiá na roça que fazê vida de vagabundo na 
cidade! -- Outro pingo!

\repl{Quinota} Papai precisa muito associar-se a um moço inteligente, nas suas 
condições. Sacrifique à sua tranquilidade os seus prazeres; case-se,
faça-se agricultor, e sua esposa, que não será muito exigente e terá muito
bom-senso, todos os anos lhe dará licença para vir matar saudades daquilo a que o senhor
chama o micróbio da pândega.

\repl{Gouveia} \paren{À parte.} Sim, senhor, pregou-me uma lição de moral mesmo nas 
bochechas!

\repl{Fortunata} Seu Gouveia, é mió a gente i pro lugá por onde Juquinha tem de
saí!

\repl{Gouveia} Deve sair por acolá\ldots{} Vamos esperá-lo na passagem. \paren{Estendendo o
braço.} É verdade! já está chuviscando.
\paren{Saem. O final da corrida. Um toque de campainha elétrica. Pouco depois de
um pouco de música. Vozeria do povo, que vem todo ao proscênio.}

 Coro

 Oh! Quem diria 
 Que ganharia 
 O Colibri! 
 Ganhou à toa! 
 Pule tão boa 
 Eu nunca vi 
 Aqui!

\newscenenamed{Cena VIII}
\stagedir{Juquinha, depois Fortunata, Quinota, Gouveia, depois Benvinda, depois Lourenço}

\repl{Lemos} Ganhou o Colibri! Quem diria?

\repl{Guedes} o Colibri\ldots{} que pulão!\ldots{}

\repl{Lourenço} Que desgraça!\ldots{} O Félix Faure caiu de propósito, mas por cima
do Félix Faure caiu o Menelik, por cima do Menelik o Ligúria, por cima do
Ligúria, o Carnot, e o Colibri, que vinha na bagagem, não caiu por cima de ninguém e
ganhou o páreo! Que palpite de mulata! Onde estará ela? Vou procurá-la.
\paren{Desaparece.}

\repl{O Frequentador} \paren{A Lemos e Guedes.} Então? eu não dizia? ganhou o 24! Doze
e doze, vinte e quatro. \paren{Com uma ideia.} Ah!

\repl{Os Dois} Que é?

\repl{O Frequentador} Fui um asno! 24 é a data da missa de sétimo dia de minha 
mulher! \paren{Lemos e Guedes afastam-se rindo.} Ora esta! ora esta!\ldots{} E era um pulão!\ldots{} 
\paren{Abre o guarda-chuva.} Chove\ldots{} Naturalmente não há mais corridas hoje\ldots{}
\paren{Afasta-se. Há na cena alguns guarda-chuvas abertos. Aparecem Eusébio, Figueiredo e
as cocotes. Vêm todos de guarda-chuvas abertos.}

\repl{Figueiredo} Bravo! Foi um tiro, seu Eusébio, foi um tiro!\ldots{} O Colibri
vendeu apenas seis pules e o senhor tem cinco!

\repl{S’il vous-plaît} \paren{Metendo-se na conversa, e abrigando-se no guarda-chuva de 
Eusébio.} Dá mais de cem mil-réis cada pule!\ldots{}

\repl{Eusébio} Mais de cem mil-réis? Então? Eu não disse? Co aquele nome, o
menino não podia perdê! O Colibri é um jumento de munta sorte! \paren{A S’il
vous-plaît.} O sinhô conhece ele?

\repl{S’il vous-plaît} Quem? O Colibri? Sim senhor!

\repl{Eusébio} Vá chamá ele. Quero le dá uma lambuge!

\repl{S’il vous-plaît} Nem de propósito! Ele aí vem! \paren{Chamando Juquinha que
aparece.} Ó Colibri! está aqui um senhor que jogou cinco pules em você e quer dar-lhe
uma gratificação.

\repl{Juquinha} \paren{Aproximando-se muito lampeiro.} Aqui estou. Quê dê o home?

\repl{Eusébio} Era o Juquinha!

\repl{Juquinha} Papai! \paren{Deita a correr e foge.}

\repl{Eusébio} Ah! tratante! O Colibri era ele! Alembrou-se do jumento!\ldots{} E
foge do pai! Ora espera lá! \paren{Corre atrás do Juquinha e desaparece. A chuva cresce. O
povo corre todo e abandona a cena.}

\repl{Lola} Onde vai? Espere! \paren{Corre atrás de Eusébio e desaparece.}

\repl{As Mulheres} Vamos também! Vamos também! \paren{Correm atrás de Lola e 
desaparecem.}

\repl{Figueiredo} Então, minhas filhas? Não corram! \paren{Vai atrás delas e
desaparece.}

\repl{Fortunata} \paren{Entrando de guarda-chuva.} É ele! É ele! É seu Eusébio! \paren{Sai 
correndo pelo mesmo lado.}

\repl{Quinota} \paren{Entrando, idem.} Mamãe! Mamãe! \paren{Corre acompanhando Fortunata.}

\repl{Gouveia} \paren{Idem.} Minhas senhoras!\ldots{} Minhas senhoras! \paren{Corre e
desaparece.}

\repl{Benvinda} \paren{Entrando perseguida por Lourenço, ambos de guarda-chuva.} Me 
deixe! Me deixe!\ldots{} \paren{Desaparece.}

\repl{Lourenço} \paren{Só em cena.} Dê cá a pule, seu benzinho, dê cá a pule, que eu
vou receber! \paren{Desaparece. Mutação.}

\quadro{Quadro X}

A Rua do Ouvidor

\newscenenamed{Cena I}
\stagedir{Juquinha}

 Coro

 Não há rua como a rua 
 Que se chama do Ouvidor! 
 Não há outra que possua 
 Certamente o seu valor! 
 Muita gente há que se mace 
 Quando, seja por que for, 
 Passe um dia sem que passe 
 Pela Rua do Ouvidor!

\repl{1° Literato} Tens visto o Duquinha?

\repl{2° Literato} Qual! Depois que se meteu com a Lola, ninguém mais lhe põe a
vista em cima!

\repl{1° Literato} É pena! Um dos primeiros talentos desta geração\ldots{}

\repl{2° Literato} Apaixonado por uma cocote!

\repl{1° Literato} Felizmente a arte lucra alguma coisa com isso. O Duquinha faz 
magníficos versos à Lola. Ainda ontem me deu uns, que são puros Verlaine. Vou
publicá-los no segundo número da minha revista.

\repl{2° Literato} Que está para sair há seis meses?

\repl{1° Literato} Oh! vê que linda rapariga ali vem!

\repl{2° Literato} Parece gente da roça. \paren{Ficam de longe, a examinar Quinota,
que entra com a mãe e o irmão. Vêm todos três carregados de embrulhos.}

\repl{Fortunata} Vamo, minha fia, vamo tomá o bonde no Largo de São Francisco.
As nossa compra tá feita. Amenhã de menhã vamo imbora!

\repl{Quinota} Sem papai?

\repl{Fortunata} Ele que vá cuando quisé! Hei de mostrá que lá em casa não se
percisa de home!

\repl{Quinota} E\ldots{} seu Gouveia?

\repl{Fortunata} Não me fale de seu Gouveia! Há oito dia não aparece! Fez cumo
teu pai! Foi mió assim\ldots{} Havia de sê munto mau marido!

\repl{Juquinha} Eu não quero i pra fazenda!

\repl{Fortunata} Eu te amostro se tu vai ou não vai! Anda pra frente! \paren{Vão
saindo.}

\repl{1° Literato} \paren{A Quinota.} Adeus, teteia!

\repl{Fortunata} Quem é que é teteia? Arrepita a gracinha, seu desavergonhado, e
verá cumo le parto este chapéu de só no lombo!\ldots{} \paren{Risadas.} Vamo! Vamo!\ldots{}
Que terra!\ldots{} Eu bem não queria vi no Rio de Janeiro! \paren{Saem entre risadas.}

\newscenenamed{Cena II}
\stagedir{1° Literato, 2° Literato, Pessoas do Povo, depois Duquinha}

\repl{2° Literato} Tu ainda um dia te sais mal com esse maldito costume de bulir
com as moças!

\repl{1° Literato} Nada disse que a ofendesse. “Adeus, teteia” não é
precisamente um insulto.

\repl{2° Literato} Pois sim, mas que farias tu se dissessem o mesmo à tua irmã?

\repl{1° Literato} Não é a mesma coisa! Minha irmã é\ldots{}

\repl{2° Literato} Não é melhor que as irmãs dos outros. \paren{Entra Duquinha, vem
pálido e com grandes olheiras.}

\repl{Duquinha} Ah! meus amigos! meus amigos! Se soubessem o que me aconteceu?

\repl{Os Dois} Que foi?

\repl{Duquinha} Ainda não estou em mim!

\repl{Os Dois} Fala!

\repl{Duquinha} O fazendeiro\ldots{} aquele fazendeiro de quem lhes falei?\ldots{}

\repl{Os Dois} Sim!

\repl{Duquinha} Apanhou-me com a boca na botija!\ldots{}

\repl{1° Literato} Mas que tem isso?

\repl{Duquinha} Como que tem isso? Aquele homem é rico! Dava tudo à Lola!

\repl{2° Literato} Tu também não lhe davas pouco!

\repl{Duquinha} \paren{Vivamente.} Dinheiro nunca lhe dei --, nem ela o aceitaria\ldots{}

\repl{1° Literato} Pois sim!

\repl{Duquinha} Joias\ldots{} vestidos\ldots{} pares de luvas\ldots{} leques\ldots{} chapéus\ldots{}
Dinheiro, nem vintém! Quem sempre me apanhava algum era o Lourenço, o cocheiro.

\repl{2° Literato} És um pateta! Mas conta-nos isso!

\repl{Duquinha} Estávamos -- ela e eu -- na saleta e o bruto dormia na sala de
jantar. Eu tinha levado à Lola umas pérolas com que ela sonhou\ldots{} Vocês não
imaginam como aquela rapariga sonha com coisas caras!

\repl{1° Literato} Imaginamos! -- Adiante!

\repl{Duquinha} Eu lia para ela ouvir os meus últimos versos\ldots{} aqueles que te
dei ontem para a revista\ldots{} Depois que te amo, depois que és minha, Nado em delícia, nado em delícia\ldots{}

\repl{1° Literato} Eu sei. Verlaine puro.

\repl{Duquinha} Obrigado. -- No fim de cada estrofe, eu dava-lhe um beijo\ldots{} um
beijo quente e apaixonado\ldots{} um beijo de poeta!\ldots{} Pois bem, depois da terceira
estrofe:

 Oh! se algum dia destino fero 
 Nos separasse, nos separasse\ldots{}

\repl{1° Literato} \paren{Continuando.} O que faria contar não quero\ldots{}

\repl{Duquinha} Que se o contasse, que se o contasse\ldots{} 
No fim dessa estrofe, Lola, que esperava a deixa, estende-me a face, eu
beijo-a e o fazendeiro, de pé, na porta da saleta, com os olhos esbugalhados dá este
grito: Ah! seu pelintreca!\ldots{}

\repl{2° Literato} E tu?

\repl{Duquinha} Eu?\ldots{} Eu\ldots{} eu cá estou. Não sei o que mais aconteceu. Quando
dei por mim estava dentro de um bonde elétrico, tocando a toda para a
cidade!\ldots{}

\repl{1° Literato} Fizeste uma bonita figura, não há dúvida! Podes limpar a mão
à parede!

\repl{Duquinha} Por quê?

\repl{1° Literato} Essa mulher não te perdoará nunca tal covardia!

\repl{2° Literato} Olha, o melhor que tens a fazer é não voltares lá!

\repl{Duquinha} Ah! meu amigo! isso é bom de dizer, mas eu estou apaixonado\ldots{}

\repl{2° Literato} Tu estás mas é fazendo asneiras! Onde vais tu buscar dinheiro
para essas loucuras?

\repl{Duquinha} Mamãe tem me dado algum\ldots{} mas confesso que contraí algumas 
dívidas, e não pequenas. -- Ora, adeus! não pensemos em coisas tristes, e
vamos tomar alguma coisa\ldots{} alegre!

\repl{Os Dois} Vamos lá!

\paren{Afastam-se pela direita, cumprimentando Mercedes, Dolores e Blanchette,
que entram por esse lado e se encontram com Lola, que entra da esquerda, muito
ervosa e agitada. Figueiredo entra da direita, observa as cocotes, para, e,
colocado por trás, ouve tudo quanto elas dizem.}

\newscenenamed{Cena III}
\stagedir{Lola, Mercedes, Dolores, Blanchette, Figueiredo, Pessoas do Povo,
depois Duquinha}

\repl{Lola} Ah! venham cá. Estou aflitíssima! Não calculam vocês que série de
desgraças!

\repl{As Outras} Que foi? que foi?

 Lola

 Rondó

 Com o Duquinha a pouco eu estava 
 Na saleta a conversar, 
 E o Eusébio ressonava 
 Lá na sala de jantar. 
 O Duquinha uns versos lia, 
 Mas não lia sem parar, 
 Que a leitura interrompia 
 Para uns beijos me furtar; 
 Mas ao quarto ou quinto beijo, 
 Sem se fazer anunciar, 
 Entra o Eusébio, e o poeta vejo 
 Dar um grito e pôr-se a andar! 
 Pretendi novos enganos, 
 Novas tricas inventar, 
 Mas o Eusébio pôs-se a panos: 
 Não me quis acreditar! 
 Vendo a sorte assim fugir-me, 
 Vendo o Eusébio se escapar, 
 Fui ao quarto pra vestir-me 
 E sair para o apanhar. 
 Mas no quarto vi, de chofre, 
 --’Stive quase a desmaiar! -- 
 Vi as portas do meu cofre 
 Abertas de par em par! 
 O ladrão foi o cocheiro! 
 Nada ali me quis deixar! 
 Levou joias e dinheiro 
 Que nem posso avaliar!

\repl{Blanchette} O cofre aberto!

\repl{Dolores} Joias e dinheiro!

\repl{Mercedes} O cocheiro!

\repl{Lola} Sim, o cocheiro, o Lourenço, que desapareceu!

\repl{Blanchette} Mas como soubeste que foi ele?

\repl{Lola} Por esta carta, a única coisa que encontrei no cofre! Ainda por cima 
escarneceu de mim! \paren{Tem tirado a carta da algibeira.}

\repl{Mercedes} Deixa ver.

\repl{Lola} Depois! Agora vamos à polícia! Não! à polícia não!

\repl{As Três} Por quê?

\repl{Lola} Não convém. Logo saberão por quê. Vamos a um advogado! \paren{Julga
guardar a carta, mas está tão nervosa que deixa-a cair.} Vamos!

\repl{As Três} Vamos! \paren{Vão saindo e encontram com Duquinha.}

\repl{Duquinha} Lola!

\repl{Lola} \paren{Dando-lhe um empurrão.} Vá para o diabo!

\repl{As Três} Vá para o diabo! \paren{Saem as cocotes. Figueiredo disfarça e apanha a
carta que Lola deixou cair.}

\repl{Duquinha} \paren{Consigo.} Estou desmoralizado! Ela não me perdoa o ter saído, 
deixando-a entregue à fúria do fazendeiro! Sou um desgraçado! Que hei de
fazer?\ldots{} Vou desabafar em verso\ldots{} Não! vou tomar uma bebedeira!\ldots{} \paren{Sai.}

\newscenenamed{Cena IV}
\stagedir{Figueiredo, Pessoas do Povo}

\repl{Figueiredo} Ora aqui está como uma pessoa, sem querer, vem ao conhecimento 
de tanta coisa! Vejamos o que o cocheiro lhe deixou escrito. \paren{Põe a luneta
e lê.} “Lola. -- Eu sou um pouco mais artista que tu. Saio da tua casa sem me
despedir de ti, mas levo, como recordação da tua pessoa, as joias e o dinheiro que pude
apanhar no teu cofre. Cala-te; se fazes escândalo, ficas de mal partido, porque eu
te digo: 1°, que de combinação representamos uma comédia pra extorquir dinheiro ao
Eusébio; 2°, que induziste um filho-família a contrair dívidas para presentear-te
com joias; 3°, que nunca foste espanhola e sim ilhoa; 4°, que foste a amante do teu
ex-cocheiro -- Lourenço.” Sim, senhor, é de muita força a tal senhora Dona Lola!\ldots{} Não
há, juro que não há mulata capaz de tanta pouca vergonha! \paren{Sai.}

\newscenenamed{Cena V}
\stagedir{Gouveia, Pessoas do Povo, depois Pinheiro}

\paren{Gouveia traz as botas rotas, a barba por fazer, um aspecto geral da
miséria e desânimo.}

\repl{Gouveia} Ninguém, que me visse ainda há tão pouco tempo tão cheio de
joias, não acreditará que não tenho dinheiro nem crédito para comprar um par de
sapatos! Há oito dias não vou à casa de minha noiva, porque tenho vergonha de lhe
aparecer neste estado!

\repl{Pinheiro} \paren{Aparecendo.} Oh! Gouveia! como vai isso?

\repl{Gouveia} Mal, meu amigo, muito mal\ldots{}

\repl{Pinheiro} Mas que quer isto dizer? Não me pareces o mesmo! Tens a barba
crescida, a roupa no fio\ldots{} Desapareceu do teu dedo aquele esplêndido e 
escandaloso farol, e tens umas botas que riem da tua esbodegação!

\repl{Gouveia} Fala à vontade. Eu mereço os teus remoques.

\repl{Pinheiro} E dizer que já me quiseste pagar, com juros de cento por cento,
dez mil-réis que eu te havia emprestado!

\repl{Gouveia} Por sinal, que disseste, creio, que esses dez mil-réis ficavam ao
meu dispor.

\repl{Pinheiro} E ficaram. \paren{Tirando dinheiro do bolso.} Cá estão eles. -- Mas,
como um par de botinas não se compra com dez mil-réis, aqui tens vinte\ldots{} sem
juros. Pagarás quando quiseres.

\repl{Gouveia} Obrigado, Pinheiro; bem se vê que tens uma alma grande e nunca 
jogaste a roleta.

\repl{Pinheiro} Nada! -- Sempre achei que o jogo, seja ele qual for, não leva
ninguém para diante. -- Adeus, Gouveia\ldots{} aparece! Agora, que estás pobre, isso não
te será difícil!\ldots{} \paren{Sai.}

\newscenenamed{Cena VI}
\stagedir{Gouveia, depois Eusébio}

\repl{Gouveia} Como este tipo faz pagar caro os seus vinte mil-réis! Ah! ele
apanhou-me descalço! Enfim vamos comprar os sapatos! \paren{Vai saindo e encontra-se com 
Eusébio, que entra cabisbaixo.} Oh! o Sr. Eusébio!\ldots{}

\repl{Eusébio} Ora! inda bem que le encontro!\ldots{}

\repl{Gouveia} \paren{À parte.} Naturalmente já voltou à casa\ldots{} Como está sentido!\ldots{}
Vai falar-me de Quinota!\ldots{}

\repl{Eusébio} Hoje de menhã encontrei ela beijando um mocinho!

\repl{Gouveia} Hein?

\repl{Eusébio} É levada do diabo! não sei cumo o sinhô pôde gostá dela!

\repl{Gouveia} Ora essa! a ponto de querer casar-me!

\repl{Eusébio} Era uma burrice!

\repl{Gouveia} Custa-me crer que ela\ldots{}

\repl{Eusébio} Pois creia! Beijando um mocinho, um pelintreca, seu Gouveia!\ldots{}
Veja o sinhô de que serviu gastá tanto dinheiro com ela!\ldots{}

\repl{Gouveia} Sim, o senhor educou-a bem\ldots{} ensinou-lhe muita coisa\ldots{}

\repl{Eusébio} \paren{Vivamente.} Não, sinhô! não ensinei nada!\ldots{} Ela já sabia tudo!
O sinhô, sim! Se arguém ensinou foi o sinhô e não eu! Beijando um pelintreca, seu

Gouveia!\ldots{}

\repl{Gouveia} Dona Fortunata não viu nada?

\repl{Eusébio} Dona Fortunata?\ldots{} Uê!\ldots{} Cumo é que havera de vê?\ldots{} Olhe, eu lá
não vorto!

\repl{Gouveia} Não volta! ora esta!

\repl{Eusébio} Não quero mais sabê dela.

\repl{Gouveia} Deve lembrar-se que é pai!

\repl{Eusébio} Por isso memo! Ah! seu Gouveia, se arrependimento sarvasse\ldots{}
Bem; o sinhô vai me apadrinhá, cumo noutro tempo se fazia cum preto fugido\ldots{} Não
me astrevo a entrá in casa sozinho depois de tantos dias de osença!

\repl{Gouveia} Em casa? Pois o senhor não me acaba de dizer que lá não volta
porque dona Quinota\ldots{}

\repl{Eusébio} Quem le falou de Quinota?

\repl{Gouveia} Quem foi então que o senhor encontrou aos beijos com o
pelintreca? -- Ah! agora percebo! A Lola!\ldots{}

\repl{Eusébio} Pois quem havera de sê?

\repl{Gouveia} E eu supus\ldots{} Onde tinha a cabeça?\ldots{} Perdoa, Quinota, perdoa!\ldots{}

Vamos, senhor Eusébio\ldots{} Eu o apadrinharei, mas com uma condição: o senhor
por sua vez me há de apadrinhar a mim, porque eu também não apareço à minha
noiva há muitos dias!

\repl{Eusébio} Por quê?

\repl{Gouveia} Em caminho tudo lhe direi. \paren{À parte.} Aceito o conselho de
Quinota: vou abrir-me. \paren{Alto.} Tenho ainda que comprar um par de sapatos e fazer a
barba.

\repl{Eusébio} Vamo, seu Gouveia! \paren{Saem. Ao mesmo tempo aparece Lourenço 
perseguido por Lola, Mercedes, Dolores e Blanchette.}

\newscenenamed{Cena VIII}
\stagedir{Lourenço, Lola, Mercedes, Dolores, Blanchette, Pessoas do Povo}

\repl{Lola e os Outros} Pega ladrão! Pega ladrão!\ldots{} \paren{Lourenço é agarrado por
pessoas do povo e dois soldados que aparecem. Grande vozeria e confusão. Apitos.
Mutação.}

\quadro{Quadro XI}

\paren{O sótão ocupado pela família de Eusébio.}

\newscenenamed{Cena I}
\stagedir{Juquinha, depois Fortunata, depois Quinota}

\repl{Juquinha} \paren{Entrando a correr da esquerda.} Mamãe! Mamãe!

\repl{Fortunata} \paren{Entrando da direita.} Que é, menino?

\repl{Juquinha} Papai tá i!

\repl{Fortunata} Tá i?

\repl{Juquinha} Eu encontrei ele ali no canto e ele me disse que viesse vê se
va’mecê tava zangada, que se tivesse, ele não entrava.

\repl{Fortunata} Oh! aquele home, aquele home o que merecia! -- Vai, vai dizê a
ele que não tô zangada!

\repl{Juquinha} Seu Gouveia tá junto co ele.

\repl{Fortunata} Bem! venha todos dois! \paren{Juca sai correndo.} Quinota!
Quinota!\ldots{}

\repl{A voz de Quinota} Senhora?

\repl{Fortunata} Vem cá, minha fia. -- Eu não ganho nada me consumindo. Já tou
veia; não quero me amofiná. \paren{Entra Quinota.} Quinota, teu pai vem aí\ldots{} mas o
que tá arresolvido tá: amenhã de menhã vamo imbora.

\repl{Quinota} E seu Gouveia?

\repl{Fortunata} Também vem aí.

\repl{Quinota} \paren{Contente.} Ah!

\repl{Fortunata} Não quero mais ficá numa terra onde os marido passa dias e notê
fora de casa!\ldots{}

\newscenenamed{Cena II}
\stagedir{Fortunata, Quinota, Juquinha, Eusébio, depois Gouveia}

\repl{Juquinha} \paren{Entrando.} Tá i papai!

\repl{Eusébio} \paren{Da porta.} Posso entrá? Não temo briga?

\repl{Quinota} Estando eu aqui, não há disso!

\repl{Fortunata} Sim, minha fia, tu é o anjo da paz.

\repl{Quinota} \paren{Tomando o pai pela mão.} Venha cá. \paren{Tomando Fortunata pela mão.}
Vamos! Abracem-se!\ldots{}

\repl{Fortunata} \paren{Abraçando-o.} Diabo de home, véio sem juízo!

\repl{Eusébio} Foi uma maluquice que me deu! Raie, raie, Dona Fortunata!

\repl{Fortunata} Pai de fia casadeira!

\repl{Eusébio} Tá bão! tá bão! juro que nunca mais! mas deixe le dizê\ldots{}

\repl{Fortunata} Não! não diga nada! Não se defenda! É mió que as coza fique
cumo tá.

\repl{Juquinha} Seu Gouveia tá no corredô.

\repl{Quinota} Ah! \paren{Vai buscar Gouveia pela mão. Gouveia entra manquejando.}

\repl{Eusébio} Assim é que o sinhô me apadrinhou?

\repl{Gouveia} Deixe-me! Estes sapatos novos fazem-me ver estrelas!

\repl{Fortunata} Seu Gouveia, le participo que amenhã de menhã tamo de viage.

\repl{Eusébio} Já conversei co ele.

\repl{Gouveia} \paren{A Quinota.} Eu abri-me!

\repl{Eusébio} Ele vai coa gente. Não tem que fazê aqui. Tá na pindaíba, mas é o
memo. Casa com Quinota e fica sendo meu sócio na fazenda.

\repl{Quinota} Ah! papai! quanto lhe agradeço!

\repl{Juquinha} A Benvinda tá i.

\repl{Todos} A Benvinda!

\repl{Fortunata} Não quero vê ela! não quero vê ela!

\paren{Quinota vai buscar Benvinda, que entra a chorar, vestida como no 1°
quadro, e ajoelha-se aos pés de Fortunata.}

\newscenenamed{Cena III}
\stagedir{Os mesmos, Benvinda}

\repl{Benvinda} Tô muito arrependida! Não valeu a pena!

\repl{Fortunata} Rua, sua desavergonhada!

\repl{Eusébio} Tenha pena da mulata!

\repl{Fortunata} Rua!

\repl{Quinota} Mamãe, lembre-se de que eu mamei o mesmo leite que ela!

\repl{Fortunata} Este diabo não tem descurpa! Rua!

\repl{Gouveia} Não seja má, dona Fortunata. Ela também apanhou o micróbio da
pândega.

\repl{Fortunata} Pois bem, mas se não se comportá dereto\ldots{} \paren{Benvinda vai para
junto de Juquinha.}

\repl{Eusébio} \paren{Baixo à Fortunata.} Ela há de casá com seu Borge\ldots{} Eu dou o
dote\ldots{}

\repl{Fortunata} Mas seu Borge\ldots{}

\repl{Eusébio} Quem não sabe é como quem não vê. \paren{Alto.} A vida da capitá não se
fez para nós\ldots{} E quem tem isso?\ldots{} É na roça, é no campo, é no sertão, é na
lavoura que tá a vida e o progresso da nossa querida Pátria! \paren{Mutação.}

\quadro{Quadro XII}

\paren{Apoteose à vida rural.}

Toda a música desta peça é composta pelo Senhor Nicolino Milano, à exceção
das coplas do Ato I -- quadro I, cena I e quadro III, cena III --, do coro 
do quadro III, cena I, do duetino do quadro II, cena IV, e do quarteto
do quadro III, cena IV, que foram compostas pelo Senhor Doutor Assis Pacheco, 
e da valsa do Ato I, cena IV, composição do Senhor Luís Moreira.

\begin{center}
\textsc{fim}
\end{center}


