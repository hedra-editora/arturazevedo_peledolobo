\chapter{A pele do lobo}

\textit{Comédia em um ato escrita em 1875 e representada pela primeira vez no Rio de Janeiro, no
Teatro Fênix Dramática, em 10 de abril de 1877.}
\medskip

\hfill A Antonio Fontoura Xavier


\castpage

\cast{Cardoso}{subdelegado}

\cast{Amália}{sua mulher}

\cast{Apolinário}

\cast{Perdigão}

\cast{Jerônimo}

\cast{Manuel Maria}

\cast{Vitorino}

\cast{O Compadre}

\cast{Uma Parte}

\cast{Dois Soldados da polícia}
\vfil

A cena passa"-se no Rio de Janeiro.
Atualidade.
\pagebreak 

\newactnamed{Ato único}
\stagedir{Sala, secretária, relógio de mesa etc. etc.}

\newscenenamed{Cena I}
\stagedir{\textsc{Cardoso, Amália} vestidos para a cerimônia e prontos para sair.}

\repl{Uma Parte\footnote{ Pessoa que faz a queixa de algum crime (dar parte) ao subdelegado. Todas as notas
são da organização.}} \paren{Que logo sai, à porta do fundo.}

\repl{Cardoso} Sim, senhor; sim, senhor! Pode ir com Deus. Descanse, que
hoje mesmo serão dadas as providências que o caso exige.

\repl{Parte} Às ordens de Vossa Senhoria. \paren{Retira"-se.}

\repl{Cardoso} Safa!

\repl{Amália} \paren{Erguendo"-se.}  Deixar"-te"-ão desta vez?

\repl{Cardoso} E metam"-se! \paren{Passeando.}

\repl{Amália}  Hein?

\repl{Cardoso} E metam"-se a servir o país!

\repl{Amália} Para que aceitaste esta maldita subdelegacia?

\repl{Cardoso} \paren{Ainda passeando.} Eu não aceitei: pedi. Mas já tenho dito
um milhão de vezes que os serviços prestados ao país e ao partido pesam muito no
ânimo daqueles que me podem fazer galgar mais um degrau na
escala social.

\repl{Amália} Deixa"-te disso, Cardoso; um degrau dessa tão falada escala social
não vale decerto o sacrifício que te custa essa autoridade de ca"-ca"-ra"-cá. São
uns desfrutadores, eis o que são! Hás de ser pago com um pontapé. Verás!

\repl{Cardoso} Hei de ser promovido na primeira vaga que aparecer. O
Cantidiano está por pouco a bater a bota. Verás se o lugar é ou não é meu!

\repl{Amália} Fia"-te na Virgem e não corras.

\repl{Cardoso} E uma vez que aceitei o cargo\ldots{}

\repl{Amália} A carga, deves dizer.

\repl{Cardoso} Venha com ele o sacrifício. Antes de tudo o dever!

\repl{Amália} Estamos prontos para sair há duas horas.

\repl{Cardoso} \paren{Consultando o relógio de mesa.} Há duas horas e
dois minutos.

\repl{Amália} \paren{Embonecando"-se ao espelho.} Creio que não chegamos a
tempo para o batizado.

\repl{Cardoso} Que remédio terão eles, senão esperar pelos padrinhos?

\repl{Amália} E o carro na porta há tanto tempo!

\repl{Cardoso} Anda com isso, anda com isso! E metam"-se!

\repl{Amália} Hein?

\repl{Cardoso} E metam"-se a servir o país!

\repl{Amália} Vamos. Não percamos mais tempo.

\repl{Cardoso} Vamos. \paren{Vão saindo. Batem palmas.}

\repl{ambos} Bateram.

\repl{Cardoso} Quem é?

\repl{Apolinário} \paren{Fora.} Sou eu.

\repl{Amália} Eu quem?

\repl{Apolinário} \paren{No mesmo.} Um criado de Vossa Senhoria.

\repl{Cardoso} Entre quem é.

\repl{Amália} Temo"-la travada! \paren{Entra Apolinário. Pisa macio e fala
descansado.}

%\vfil\pagebreak 
\newscenenamed{Cena II}

\stagedir{\textsc{Os mesmos} e \textsc{Apolinário}}

\repl{Apolinário} \paren{À porta do fundo.} Dá licença, senhor
subdelegado?

\repl{Cardoso} Entre, senhor. \paren{Vai outra vez pôr o chapéu na
secretária.}

\repl{Apolinário} \paren{Entrando e sentando"-se em uma cadeira que deve estar
no meio da cena.} Não se incomode
Vossa Senhoria. Estou muito bem. Vossa Senhoria como tem passado?

\repl{Cardoso} Bem, obrigado. O que pretende o senhor?

\repl{Apolinário} Sua senhora tem passado bem, senhor subdelegado?

\repl{Amália} Bem, obrigada. O senhor o que pretende?

\repl{Apolinário} Ah! estava aí, minha senhora? Os meninos estão bons?

\repl{Amália} Que meninos, senhor?

\repl{Apolinário} Os seus filhos, minha senhora.

\repl{Amália} Não os tenho. E esta!

\repl{Apolinário} Pois levante as mãos para o céu e dê graças a Nosso Senhor Jesus
Cristo! \paren{Sinais de impaciência em Cardoso e Amália.} Eu tenho três, três!
Todos três machos, felizmente. Mas que consumição! Que canseira! Quando não está
um doente, está outro; quando não está outro, está outro; quando não está
nenhum, está a mãe; quando não está a mãe, está o pai. Às vezes estão, filhos e
pais, todos doentes. É preciso chamar a
vizinha para dar"-nos qualquer coisa. É uma lida, minha rica senhora!
Peça a Deus que não lhe dê filhos. Olhe\ldots{} \paren{Mostra a cabeça.} Não vê?

\repl{Amália} O quê? O quê?

\repl{Apolinário} Já estou pintando\ldots{} Ainda anteontem\ldots{} Anteontem
não\ldots{} Quando foi, Apolinário? Segunda\ldots{} terça\ldots{} Foi anteontem
mesmo\ldots{}  Eu tinha acabado de tomar o meu banhinho e de ouvir minha
missinha\ldots{}

\repl{Cardoso} \paren{Interrompe"-o.}  Meu caro senhor, tomo a liberdade
de preveni"-lo que temos muita pressa e não
podemos perder tempo. Íamos saindo justamente quando o senhor entrou\ldots{}

\repl{Apolinário} \paren{Erguendo"-se.} Nesse caso, senhor doutor\ldots{}

\repl{Cardoso} Perdão; não sou doutor.

\repl{Apolinário} Fica para outro dia\ldots{} Eu vinha dar minha queixa, mas\ldots{}
\paren{Cumprimenta.} Senhor doutor\ldots{} minha
senhora\ldots{} \paren{Vai saindo.}

\repl{Cardoso} Venha cá, senhor: já agora diga o que pretende.

\repl{Apolinário} \paren{Voltando"-se e preparando"-se como para um discurso,
com força.}  Senhor subdelegado\ldots{}

\repl{Cardoso} Não é preciso gritar tanto\ldots{}

\repl{Apolinário} Esta noite fui roubado.

\repl{Cardoso} Diga.

\repl{Apolinário} Dezoito cabeças de criação\ldots{} dezoito ou dezenove\ldots{} Ontem
esteve em nossa casa um cunhado
meu, irmão de minha mulher, empregado no Arsenal de Guerra, e não tenho
certeza de que ele levasse alguma
galinha consigo, mas creio que não. Em todo caso, foram dezoito ou
dezenove cabeças, não falando em um
bonito galo de crista, que comprei no mercado, não há quinze dias.

\repl{Cardoso} Muito bem. O senhor chama"-se\ldots{}

\repl{Apolinário} Apolinário, um criado de Vossa Senhoria.

\repl{Cardoso} Apolinário de quê?

\repl{Apolinário} Apolinário da Rocha Reis Paraguaçu \paren{Dando um
cartão.} Olhe, aqui tem Vossa Senhoria meu
nome e morada.

\repl{Cardoso} Bem; pode ir descansado, que serão dadas as providências que o
caso exige.

\repl{Apolinário} \paren{Preparando"-se outra vez para um discurso e elevando
muito a voz.} Ainda não fica nisso,
senhor doutor!

\repl{Cardoso} Já tive ocasião de dizer"-lhe, primeiro, que não é preciso
gritar tanto; segundo, que não sou doutor.

\repl{Apolinário} \paren{Com a mesma inflexão, porém baixinho.} Não fica
nisso. Eu conheço o gatuno!

\repl{Cardoso} E por que estava calado?

\repl{Amália} \paren{Não se podendo conter.} Com efeito, Senhor
Paraguaçu!

\repl{Apolinário} \paren{Atarantado.} Hein! \paren{Falando com cada
vez mais descanso.} Não conheço eu outra coisa!
Chama"-se Jerônimo de tal, um ilhéu, um vagabundo, que foi há tempo
cocheiro de bondes e agora não sai da
venda de seu Manuel Maria, ao qual dizem que vende por um precinho de
amigo, o que\ldots{} \paren{Ação de furtar.}
Vossa Senhoria sabe qual é a venda de seu Manuel Maria? É a que fica
mesmo em frente à casa do meu
cunhado, do mesmo que esteve ontem em nossa casa, e sobre o qual estou
em dúvida se levou ou não alguma
galinha. \paren{A Amália.} Mas que bonito galinho, senhora! Vossa
Senhoria dava oito mil réis por ele com os olhos
fechados\ldots{} Era branco, branquinho, como aqueles patinhos do Passeio
Público. Uma crista escarlate! Que
bonito galo!

\repl{Cardoso} Vamos! Não temos tempo a perder! Faça o favor de sentar"-se
naquela mesa e dar a queixa por
escrito.

\repl{Apolinário} De muito bom gosto, senhor doutor. \paren{Obedece.}

\repl{Cardoso} E o senhor a dar"-lhe! Já lhe disse que não sou doutor.

\repl{Apolinário} Isso é modéstia de Vossa Senhoria.

\repl{Amália} Parece de propósito, Senhor Paraguaçu.

\repl{Cardoso} Deixa"-o lá. \paren{Vai para junto de Amália.} Que
maçador! E metam"-se!

\repl{Amália} Não chegaremos a tempo.

\repl{Apolinário} \paren{À mesa.} Esta pena está escarrapachada, senhor
subdelegado\ldots{}

\repl{Cardoso} Vou dar"-lhe outra\ldots{} Vou dar"-lhe outra\ldots{}

\repl{Amália} Anda\ldots{} Tem paciência\ldots{} Acaba com isso. \paren{Cardoso vai
abrir a secretaria e muda a pena da caneta.}

\repl{Apolinário} Muito obrigado! Que incômodo tem tomado Vossa Senhoria! Mas
também não há quem diga à
boca cheia: “Aquilo é que é um subdelegado! Zelo até ali\ldots{} É o pai das
Partes!”.

\repl{Cardoso} Faça o favor de escrever o que tem de escrever\ldots{}

\repl{Apolinário} Às ordens de Vossa Senhoria. \paren{Escreve.}

\repl{Cardoso} \paren{Voltando para junto de Amália.} Decididamente peço
a demissão!

\repl{Amália} Isso é o que já devias ter feito há muito tempo.

\repl{Cardoso} Olha que é bem difícil suportar uma maçada assim\ldots{} E
metam"-se!

\repl{Amália} Hein?

\repl{Cardoso} E metam"-se a servir o país!

\repl{Amália} Pede a demissão, Cardoso, pede a demissão.

\repl{Apolinário} \paren{Da mesa.} Senhor subdelegado, faça o favor de
me dizer o modo por que devo principiar este
requerimento\ldots{} Em matéria de polícia sou completamente leigo\ldots{} Diga"-me
só o cabeçalho\ldots{} O cabeçalho!
O resto vai\ldots{}

\repl{Cardoso} Ai, Senhor Paraguaçu! O senhor é maçante! Tenho estado a
aturá"-lo há meia hora!

\repl{Amália} \paren{Olhando o relógio.} Há meia hora e sete minutos.

\repl{Cardoso} Estamos muito apressados, meu caro senhor\ldots{} Não posso estar
com isso\ldots{}

\repl{Apolinário} Eu quis retirar"-me quando Vossa Senhoria disse que\ldots{}

\repl{Cardoso} Vamos lá! Escreva no alto --- Ilustríssimo Senhor.

\repl{Apolinário} O Ilustríssimo Senhor --- já cá está.

\repl{Cardoso} Bem \paren{Ditando.} “O abaixo assinado, morador nesta
freguesia, à rua de tal, número tal\ldots{}”

\repl{Apolinário} \paren{Escrevendo.} \ldots{} número treze\ldots{}

\repl{Cardoso} “Queixa"-se a Vossa Senhoria de que, ontem, às tantas horas da
noite\ldots{}”

\repl{Apolinário} “Queixa"-se” é com \textit{x} ou \textit{ch}?

\repl{Amália} Ó céus! \paren{Rindo"-se.}

\repl{Cardoso} Como quiser! Não faço questão de ortografia.

\repl{Apolinário} Vai com \textit{ch}. \paren{Acabando.}\ldots{} “da noite”\ldots{}

\repl{Cardoso} Como está?! \paren{Vendo.} Fulano de tal, tal, tal. Ah!
\paren{Ditando.} “Furtaram"-lhe tantas galinhas\ldots{}”

\repl{Apolinário} \paren{Escrevendo.} \ldots{}``e um galo de crista”\ldots{}
%luis não falta abertura das aspas?

\repl{Cardoso} “\ldots{} as suspeitas de cujo furto faz recair em Fulano de Tal.”
\paren{Consultando o relógio.} E metam"-se!

\repl{Apolinário} \paren{Escrevendo.} “Fulano de tal, vulgo
Barriga"-cheia.” Pronto!

\repl{Cardoso} Na outra linha: “Deus guarde a Vossa senhoria”.

\repl{Apolinário} \ldots{} “a Vossa Senhora”\ldots{}

\repl{Cardoso} Na outra linha: “Ilustríssimo Senhor Subdelegado de tal
freguesia”.

\repl{Apolinário} Pronto.

\repl{Cardoso} Assine.

\repl{Apolinário} \ldots{} “Apolinário da Rocha Reis Paraguaçu.”
\paren{Erguendo"-se.} Pronto.

\repl{Cardoso} Bem; agora pode ir descansado, que serão dadas as providências
que o caso exige.

\repl{Apolinário} Com licença, senhor subdelegado\ldots{} Às ordens de Vossa
Senhoria\ldots{}

\repl{Cardoso} Passe bem.

\repl{Apolinário} Minha senhora\ldots{}

\repl{Amália} Viva. \paren{Volta"-lhe as costas.}

\repl{Apolinário} Sem mais incômodo. \paren{Saída falsa.}

\repl{Cardoso} Safa!

\repl{Amália} Saiamos, saiamos quanto antes! Pode vir outro\ldots{} \paren{Vão
saindo.}

\repl{Apolinário} \paren{Voltando.} Ia"-me esquecendo, senhor
subdelegado\ldots{}

\repl{Cardoso} Outra vez!

\repl{Amália} Assustou"-me até!

\repl{Cardoso} O que mais deseja?

\repl{Apolinário} Hoje, logo depois do almoço, encontrei"-me cara a cara com o
tal Jerônimo!

\repl{Cardoso} Que Jerônimo, senhor?

\repl{Apolinário} O Barriga"-cheia, o tal que me furtou as galinhas\ldots{}

\repl{Cardoso} E o que tenho eu com isso, não me dirá?

\repl{Apolinário} Direi, sim, senhor. Com licença. \paren{Desce à cena e
senta"-se.} Chamei"-o de ladrão! Disse"-lhe assim: “Você é um ladrão!” --- Com licença da senhora\ldots{}

\repl{Amália} E o que tem meu marido com isso?

\repl{Apolinário} É que o sujeito tomou três testemunhas, e diz que me vai
processar por crime de injúrias verbais.

\repl{Cardoso} Mas, enfim, faz favor de me dizer para que voltou cá?

\repl{Apolinário} Vim prevenir a Vossa Senhoria de que\ldots{}

\repl{Cardoso} Vá prevenir ao diabo que o carregue!

\repl{Apolinário} \paren{Levantando"-se.} Senhor, doutor.

\repl{Cardoso} \paren{Gritando.} Já lhe disse que não sou doutor!

\repl{Apolinário} \paren{Imitando"-o.} Isso é modéstia de Vossa senhoria!

\repl{Cardoso} Saia! Ponha"-se ao fresco! Supõe o senhor que sirvo de joguete?

\repl{Apolinário} Mas Vossa Senhoria\ldots{}

\repl{Cardoso} Saia!

\repl{Apolinário} É que\ldots{}

\repl{Amália} Oh! senhor, já é a terceira vez que se lhe diz --- saia.

\repl{Apolinário} Minha senhora, eu\ldots{} \paren{Tornando a sentar"-se, com todo
o sossego.} Com licença\ldots{}

\repl{Amália} Oh! isto é demais!

\repl{Cardoso} Então, não ouve!

\repl{Apolinário} Quero justificar"-me!

\repl{Cardoso} \paren{Ameaçador.} Cuidado, Senhor Paraguaçu!

\repl{Apolinário} Bem, Vossa Senhoria está em sua casa: manda.
\paren{Levantando"-se e cumprimentando.} Às ordens
de Vossa Senhoria.

\repl{Cardoso} Viva! Há mais tempo! \paren{Passeia agitado.}

\repl{Apolinário} Minha senhora\ldots{}

\repl{Amália} Passe bem. \paren{Saída falsa de Apolinário.} Que inferno!
Que inferno! E metam"-se!

\repl{Apolinário} \paren{Voltando.} Acredite, senhor doutor, que eu não
queria de forma alguma\ldots{}

\repl{Cardoso} \paren{Desesperado.} Ah! ele é isso? \paren{Agarra uma
cadeira e levanta"-a, correndo para Apolinário.}

\repl{Amália} \paren{Muito aflita.} Ah! \paren{Suspende o braço de
Cardoso. Ficam todos numa posição dramática.}

\repl{Apolinário} \paren{Com todo o sangue frio.} Tableau.\footnote{ \textit{Tableau}: O final de uma cena;
``quadro'', em francês.}
\paren{Desaparece.}


\newscenenamed{Cena III} 

\stagedir{\textsc{Cardoso} e \textsc{Amália}}

\repl{Cardoso} Vês, Sinhá, vês como um homem se deita a perder?

\repl{Amália} Sim, sim, mas vamos, anda daí!

\repl{Cardoso} \paren{Caindo na cadeira que tinha nas mãos.} E que dor
de cabeça fez"-me este bruto!\ldots{} E metam"-se.

\repl{Amália} Hein?

\repl{Cardoso} E metam"-se a servir o país!

\repl{Amália} Espera\ldots{} vou buscar a garrafinha de água"-flórida.\footnote{
Água de colônia utilizada no século \textsc{xix} para limpar, perfumar e fazer
descarrego.}
\paren{Sai
e volta com a garrafinha.}

\repl{Cardoso} Depressa\ldots{} depressa, Sinhá! \paren{Amália esfrega"-lhe as
frontes com água"-flórida.} Bem\ldots{} basta\ldots{} está
pronto\ldots{} Ai! que ferroadas! Deita a garrafinha em cima da mesa e vamos,
vamos! \paren{Amália deita a garrafinha sobre a mesa e vai dar o braço a seu marido.}

\repl{Amália} Vamos! \paren{Saem e voltam.} Esqueci"-me do leque.
\paren{Entra à direita baixa.}

\repl{Cardoso} \paren{Falando para dentro.} Que demora, Sinhá, que demora!
Ainda há de vir alguém, verás! \paren{Passeia.}
Então não achas esse leque! Ai! minha cabeça! E metam"-se!
\paren{Quebra"-se alguma coisa dentro.} O que foi
isso?! O que foi isso?! \paren{Corre também para a direita baixa.}

\repl{Amália} \paren{Dentro.} O meu frasco de água da Colônia!

\repl{Cardoso} \paren{Dentro.} Que pena!

\repl{Amália} \paren{Dentro.} Ah! cá está o leque! \paren{Voltam à
cena, de braço dado e dirigem"-se para a porta.}

\repl{Cardoso} Já estou suando. \paren{Procura nos bolsos.} Não tenho
lenço.

\repl{Amália} Oh! que maçada! Quanto mais pressa, mais vagar. \paren{Sai
correndo pela direita baixa.}

\repl{Cardoso} E metam"-se, hein! E metam"-se a servir o país!

\repl{Amália} \paren{Voltando com um par de meias na mão.} Toma, toma\ldots{} Apre! 
\paren{Dá"-lho.}

\repl{Cardoso} Isto é um par de meias, Sinhá! Estás a meter os pés pelas
mãos! \paren{Restitui"-lho.}

\repl{Amália} Como está esta cabeça, meu Deus! \paren{Sai e volta com um
lenço.} Toma\ldots{} Vamos\ldots{} Uf!

\repl{Cardoso} Vamos! \paren{Encaminham"-se para a porta. Batem palmas.}

\repl{Ambos} Ah!

\repl{Cardoso} \paren{Fora de si.} Não estou em casa!

\repl{Jerônimo} \paren{Aparecendo, de chapéu na cabeça.} Licença para
um\ldots{}


\newscenenamed{Cena IV} 

\stagedir{\textsc{Os mesmos} e \textsc{Jerônimo}}

\repl{Cardoso} Então é assim que se entra em casa alheia?

\repl{Jerônimo} \paren{Sombrio.} Assim como? A casa da autoridade é uma
repartição pública. \paren{Deita no chão a cinza de um cachimbo; e escarra na parede.}

\repl{Cardoso} E que tal?

\repl{Amália} Vê o que ele quer, Cardoso?

\repl{Jerônimo} Venho preveni"-lo de que é falso o que lhe veio hoje dizer um
tal Paraguaçu, acerca de um furto de
galinhas. É provável que ele lhe dissesse que eu, Jerônimo Linhares,
vulgo Barriga"-cheia, sou o autor desse
furto, como andou por aí dizendo a quem quis ouvi"-lo. É falso!
\paren{Cospe outra vez na parede.}

\repl{Amália} \paren{Empurrando um escarrador com o pé.} Faz favor de
não cuspir no chão\ldots{} Aqui tem o escarrador\ldots{}
\paren{Jerônimo nem olha para Amália.}

\repl{Cardoso} Era só isso? Estou ciente.

\repl{Jerônimo} Não, senhor; por isto só não vinha eu cá, ora viva! Venho
queixar"-me do queixoso por crime de
injúrias verbais. Chamou"-me de ladrão, e se quiser o mais, mande aquela
mulher para dentro. \paren{Cospe outra vez na parede.}

\repl{Cardoso} Pois apresente a queixa e as testemunhas.

\repl{Jerônimo} A queixa aqui está. \paren{Apresenta um papel sujo, que
Cardoso pega com repugnância. Vai à porta do fundo.} Ó compadre! Ó seu Manuel Maria! Ó seu Vitorino? Podem
entrar\ldots{} Nada de cerimônias!

\repl{Cardoso} \paren{A Amália.} O tratante dispõe desta casa como se
fosse sua!

\newscenenamed{Cena V} 

\stagedir{\textsc{Os mesmos, Manuel Maria}, depois \textsc{O Compadre}, depois \textsc{Vitorino}}

\repl{Manuel Maria} \paren{Entrando.} Aqui estou eu!

\repl{Compadre} \paren{Entrando.} E eu\ldots{}

\repl{Vitorino} \paren{Entrando.} E eu\ldots{}

\repl{Amália} Cardoso, dize"-lhes que venham em outro dia\ldots{} \paren{À
Parte.} Como cheiram a cachaça!

\repl{Cardoso} Meus senhores, tenham a bondade de voltar amanhã.

\repl{Jerônimo} Aí vem o maldito sistema da demora e do papelório.

\repl{Cardoso} Cala"-te daí, insolente, que não tens autoridade para fazer
considerações neste lugar\ldots{} Apareçam
terça"-feira ou mesmo amanhã! Mas terça"-feira é melhor, porque é o dia da
audiência. Não posso estar agora
com isto\ldots{} Estamos prontos para sair há muito tempo!

\repl{Amália} Há três horas!

\repl{Cardoso} \paren{Consultando o relógio.} Há três horas e três
minutos!

\repl{Jerônimo} \paren{Cuspindo na parede.} Então, podiam ter dito logo!
Escusava a gente de estar aqui à espera! É isto
sempre! A autoridade vai para a pândega, e o povo que sofra!

\repl{Cardoso} Insolente! Espera que te ensino! \paren{Agarra numa cadeira
que está perto do toucador.}

\repl{Amália} Cardoso! O que vais fazer?!\ldots{}

\repl{Jerônimo} Ah! Ele é isso? \paren{Tira uma faca e deita a correr atrás
de Cardoso. Amália fecha"-se no quarto. As três testemunhas correm atrás de Jerônimo, para retê"-lo. Cardoso
apita.\footnote{ No século \textsc{xix}, a polícia utilizava um apito para chamar ou reunir
os soldados para ação. As pessoas comuns também podiam apitar para chamar a polícia.}}

\repl{Manuel Maria} O que é isto, seu Jerônimo?!

\repl{Compadre} Compadre, tenha mão!

\repl{Vitorino} Não se deite a perder!
\paren{Cardoso continua a apitar. Confusão.}

\repl{Amália} \paren{Grita de dentro.} Aqui d’el"-rei!


\newscenenamed{Cena VI} 

\stagedir{\textsc{Os mesmos} e \textsc{Dois Soldados}}

\repl{Soldados} O que é isto? O que é isto?\ldots{} \paren{Correm todos em redor da cena.}

\repl{Cardoso} Prendam"-no! Prendam"-no! \paren{Jerônimo é afinal preso.}
Levem"-no! \paren{Os Soldados levam o preso.} 
\paren{Saem também as testemunhas.}


\newscenenamed{Cena VII} 

\stagedir{\textsc{Cardoso} e depois \textsc{Amália}}

\repl{Cardoso} \paren{Caindo extenuado em uma cadeira.} Uf!

\repl{Amália} \paren{Entrando.} Feriu"-te o maldito, feriu"-te?

\repl{Cardoso} Creio que não. \paren{Apalpando"-se.} Não feriu, não, Sinhá!
Se não fossem as ordenanças que estavam na
porta, a estas horas estavas viúva!

\repl{Amália} Credo! Viúva!

\repl{Cardoso} Maldita subdelegacia! Maldita a hora em que aceitei semelhante
cargo!

\repl{Amália} Como estás suando! Esta camisa é incapaz de aparecer no
batizado\ldots{}

\repl{Cardoso} É verdade! O batizado! Vou mudar de camisa\ldots{}

\repl{Amália} Mas isso depressa\ldots{} depressa! \paren{Saída falsa de
Cardoso.} Ó Senhor Deus! Isto contado lá se acredita!
É bem feito, senhor meu marido, é bem feito! Quem não quiser ser
lobo, não lhe vista a pele. \paren{Rolo na rua.
Apitos. Gritos. Pancadaria. Amália vai à janela.} Que vejo! Uma
malta de capoeiras!\footnote{ Grupos de capoeiras formados, em sua maioria, por homens
mulatos e negros, que lutavam entre si e com a polícia, utilizando golpes de
capoeira, navalha e bengalas como armas. As maltas foram duramente reprimidas
pela polícia, no final do século \textsc{xix}. Ver Introdução, p.~\pageref{capoeira}.} Cardoso! Cardoso! Não
tardam a entrar\ldots{}

\repl{Cardoso} \paren{Entra em mangas de camisa e com o fitão de
subdelegado.\footnote{ Fita utilizada por delegados e subdelegados, como símbolo do cargo.}} 
O que é isto? \paren{Espirra.} Atchim!
Constipei"-me\ldots{} Atchim! O que é isto? Atchim! \paren{Sai a correr pelo
fundo.}


\newscenenamed{Cena VIII} 

\stagedir{\textsc{Amália}, depois \textsc{Perdigão}}

\repl{Amália} Meu Deus! Hoje parece ser o dia de São Bartolomeu! Se não anda
o diabo solto na cidade, ao menos
nesta freguesia.

\repl{Perdigão} \paren{Entra apressado pelo fundo, vestido para a cerimônia.}
Ó compadre! Ó comadre!

\repl{Amália} Mais uma Parte!

\repl{Perdigão} Deixe"-se de Partes!

\repl{Amália} Meu marido não está\ldots{} \paren{Reparando.} Ah! é o compadre!

\repl{Perdigão} Estamos até estas horas à espera do padrinho e nada!

\repl{Amália} Queixe"-se da maldita subdelegacia, compadre! Estamos vestidos
há três horas\ldots{} \paren{Consultando o relógio.} Há três horas e um quarto\ldots{}

\repl{Perdigão} Ora! Para que foi o compadre buscar sarna para se coçar\ldots{}

\repl{Amália} O compadre não imagina! Quantas vezes, alta noite, está ele
sossegado a dormir, quando, de
repente, é despertado pelas malditas partes\ldots{}

\repl{Perdigão} Por força!

\repl{Amália} \paren{Indo à janela.} Já está aplacado o rolo\ldots{}
\paren{Voltando.} Hoje quase o matam!

\repl{Perdigão} \paren{Dando um salto.} A quem?

\repl{Amália} Ao Cardoso.

\repl{Perdigão} Ah! Ele descia a escada com tanta impetuosidade! Ia em mangas
de camisa e de fitão\ldots{} Olhem que
figura! Espirrava, que era um Deus nos acuda! “Viva!” lhe disse eu; ele,
porém, não me conheceu, apesar de
responder ``\textit{Dominus tecum”},\footnote{ “Deus esteja contigo”, em latim.}
em vez de ``Obrigado!”

\newscenenamed{Cena IX}

\stagedir{\textsc{Os mesmos} e \textsc{Cardoso.}}

\repl{Cardoso} \paren{Entra e cai espirrando em uma cadeira.} Atchim!

\repl{Perdigão} Viva!

\repl{Cardoso} \textit{Dominus te\ldots{}} Quero dizer: obrigado\ldots{} Atchim! Ah! é o
senhor, compadre? Desculpe.

\repl{Perdigão} Já sei de tudo\ldots{} Está mais que desculpado\ldots{} Mas não perca
tempo!

\repl{Amália} Sim, não percamos tempo!

\repl{Cardoso} Vamos! \paren{Ergue"-se e deita o chapéu.} Estou pronto!

\repl{Perdigão} Em mangas de camisa, compadre?

\repl{Cardoso} É verdade! \paren{Corre ao quarto e volta vestindo a
casaca.}

\repl{Amália} De fitão, Cardoso?

\repl{Cardoso} É verdade! \paren{Despedaça o fitão zangado.} Atchim!

\repl{Perdigão} Já leu o que traz hoje o \textit{Jornal} a seu respeito?

\repl{Cardoso} Já: descompostura bravia! É o pago que dão a tantos
sacrifícios.

\repl{Perdigão} Diga antes: é o castigo que infligem ao erro de aceitá"-los.

\repl{Amália} \paren{Impaciente.} Vamos embora! \paren{Vão todos
saindo.}


\newscenenamed{Cena X}

\stagedir{\textsc{Os mesmos} e \textsc{um soldado.}}

\repl{Soldado} \paren{A Cardoso.} Trouxeram este ofício e esta carta
para Vossa Senhoria. \paren{Entrega a carta e o ofício e sai.}

\repl{Cardoso} Dê cá. \paren{Abrindo a carta.} Com licença. \paren{Lê.}
É um bilhete em que o oficial do gabinete do ministro
me participa haver sido outro nomeado para a vaga do Cantidiano\ldots{} E
metam"-se!

\repl{Perdigão} Hein?

\repl{Cardoso} E metam"-se a servir o país! \paren{Abrindo o ofício.} Com
licença! \paren{Depois de ler o ofício.} Sabem o que
é? Minha demissão.

\repl{Perdigão e Amália} Demissão?

\repl{Cardoso} À vista do que a meu respeito tem aparecido na imprensa
periódica!

\repl{Perdigão} Não falemos mais nisso! Vamos embora.

\repl{Cardoso} Poupou"-me o trabalho de pedi"-la.

\repl{Amália} Quem não quiser ser lobo\ldots{}

\repl{Perdigão} Mas o compadre acaba de despir a pele do
lobo. \paren{Apanhando o fitão.} Ei"-la!

\repl{Cardoso} Atchim! \paren{Saem todos três.}

\vspace{1cm}

\begin{center}
\textsc{Cai o pano}
\end{center}

