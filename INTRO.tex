\hyphenation{tea-tro tea-tral so-cie-da-de}
\renewcommand{\fala}[1]{\noindent\quad\textsc{#1} --}

\chapter[Introdução, por Larissa de Oliveira Neves]{Introdução}
\hedramarkboth{Introdução}{Larissa de Oliveira Neves}

\textsc{A fortuna crítica} relacionada ao dramaturgo, contista, cronista e poeta
Artur Azevedo veio sofrendo uma transformação gradual nos últimos anos.
Batalhador incansável do teatro brasileiro, seu papel como “grande
animador”\footnote{ Expressão de Sábato Magaldi, em \textit{Panorama do teatro
brasileiro}, São Paulo, Global, 1997.} da cena nacional pouco foi contestado. No
entanto, mesmo as avaliações mais favoráveis ao seu talento cômico
mantiveram, por muito tempo, certo ar de condescendência; como se
apreciar o anedótico, o engraçado, o burlesco fosse invariavelmente
desabonador para uma pessoa culta. Felizmente, esse modo de pensar
sofreu uma gradativa reviravolta. Hoje, a obra de Artur Azevedo vem
sendo revisitada, dentro dos meios acadêmicos, a partir de leituras
menos preconceituosas, cujos resultados originaram e continuam a
originar importantes descobertas sobre nosso passado, nosso modo
de vida, nossa cultura -- cultura que, mesmo quando
realizada sob a forma de uma obra erudita, não deixa de conter traços
populares, característicos do que se passou a chamar de brasilidade.

Nesta edição, temos a honra de trazer a público cinco obras carregadas
de brasilidade. Essas peças foram escritas muito antes do surgimento
daquele famoso e polêmico grupo de escritores que surpreendeu o país
com suas obras literárias modernistas. As comédias deste volume
surgiram num momento em que vigorava um padrão literário que
privilegiava a gramática lusitana e os temas ditos elevados; um tempo em
que os intelectuais, nacionalistas a seu modo, ansiavam por ver o
Brasil se tornar um país tão civilizado, elegante e fino quanto os
países europeus, em especial a França. Não se imagine que Artur
Azevedo, por ter escrito essas peças, pensasse de modo diferente, ou
estivesse à frente de seu tempo. Não. Ele fazia parte daquele grupo de
intelectuais atuantes, cujos principais nomes foram Machado de Assis,
Olavo Bilac, Aluísio Azevedo (irmão de Artur) e Coelho Neto, entre
muitos outros. 

Artur Azevedo também se espelhava na França como fonte de uma cultura
mais refinada, e é aí que se constrói a sua figura paradoxal e
polêmica. Criticado pelos colegas fundadores da Academia Brasileira de
Letras por escrever comédias ligeiras, musicadas e populares, ele se
defendia sem proclamar uma igualdade de valor entre os gêneros
teatrais. Seu modo de pensar coadunava"-se com a visão dos demais
intelectuais de seu tempo, todos presos a padrões classicistas, que
repudiavam o teatro musicado. Tal concepção, porém, não o impedia de
realizar, na prática, o teatro para o qual seu talento o conduzia: um
teatro de diversão, em que a mistura de comicidade, de música e de uma
crítica bem"-humorada formam enredos divertidos e bem estruturados; em
que o vaivém das personagens, o ritmo acelerado e as peripécias
delineiam tramas bem urdidas e recheadas de surpresas.

Suas peças foram feitas para serem encenadas e apreciadas por uma
plateia eclética. O público que frequentava os teatros num tempo em que
não havia televisão, rádio ou cinema era composto por praticamente toda
a população do Rio de Janeiro, ricos e pobres (mais pobres do que
ricos, no caso do teatro musicado). Os ingressos mais baratos custavam
menos do que uma passagem de bonde: quase todos podiam ir ao teatro, ao
menos de vez em quando. Artur Azevedo escrevia peças pensando em
agradar toda a população, em fazer as pessoas se divertirem; para
tanto, seus textos são simples, escritos em linguagem brasileira, que
recria as características da fala coloquial da época. Ele utilizava os
próprios acontecimentos da vida real e os costumes da população para
criar as tramas; assim, a plateia identificava"-se imediatamente com as
personagens e as peripécias do enredo, e suas peças faziam grande
sucesso.

\section{A presente edição}

A presente edição contém cinco peças curtas de Artur Azevedo, sendo três
comédias de costumes em um ato: \textit{Amor por anexins}, \textit{A
pele do lobo }e \textit{O Oráculo}. Embora os estudos em torno da obra
do autor privilegiem as revistas de ano e as peças mais longas, são os
textos curtos os campeões em termos de encenação. Sem exigir um grande
número de atores ou aparatos de cena, as comédias em um ato têm sido
constantemente encenadas por grupos amadores e profissionais de todo o
país. Elas demonstram o quanto essa dramaturgia continua viva, atual,
com a capacidade de agradar espectadores mais de cem anos depois de sua
elaboração. Se a perenidade é um dos determinantes do valor de uma obra
de arte, não há dúvida de que esses textos sobreviveram ao seu tempo. 

Hoje em dia, é comum a apresentação de espetáculos teatrais curtos, de
meia hora ou quarenta minutos de duração, principalmente quando acontecem, por exemplo, em ambientes abertos, num espaço cênico
público. Assim, as comédias em um ato de Artur Azevedo têm ganhado
adaptações para a rua, sendo encenadas em praças e escolas. Se a ideia é
produzir um espetáculo mais longo, reúnem"-se dois ou três textos para
compor apresentações de uma a duas horas de duração. Quando foram
escritas, porém, sua função era outra: no século \textsc{xix}, as pecinhas
cômicas tinham espaço garantido como complemento de tragédias, dramas
ou melodramas longuíssimos. Os espetáculos se estendiam por três a
quatro horas, e os textos curtos funcionavam como momentos de
relaxamento entre os atos ou no encerramento da noite. 

Além disso, as peças curtas ganham um novo sabor nas páginas de um
livro, porque, diferentemente de textos cuja teatralidade supera a
literariedade, como é o caso das revistas de ano, é possível desfrutar
plena e prazerosamente dos enredos através de uma leitura solitária.
Não é à toa que tais textos, ainda no século \textsc{xix}, eram publicados em
jornais, revistas, livros ou brochuras, após as encenações.\footnote{
Para informações mais detalhadas, ver Orna Messer Levin, “Teatro de papel – certa dramaturgia de Artur Azevedo”, em Orna M. Levin e Larissa O. Neves, \textit{Teatro,
literatura e imprensa na virada do século}, revista \textit{Remate de
Males }28.1, Campinas, IEL, jan.--jun.
2008.}

Junto às três comédias em um ato, este livro inclui duas outras peças,
cujo tema é o carnaval. \textit{Como eu me diverti!}, curtíssima, foi
publicada no livro \textit{Contos fora de moda}, de 1894. Classificada
pelo autor como conto"-comédia, a peça demonstra o quanto a teatralidade
está presente em toda a obra literária de Artur Azevedo, seja sob a
forma de conto, crônica ou poesia.\footnote{ Sobre a teatralidade na
poesia de Artur Azevedo, ver Pedro Marques, \textit{``Artur
Azevedo e a arte do soneto dramático''}, \textit{idem.}} Isso se comprova,
também, com a peça \textit{O Oráculo}, já que ela é nada mais nada
menos do que a transposição para a cena do conto \textit{Sabina},
escrito em 1894. Já \textit{O Cordão}, embora apenas ligeiramente mais
extensa que as outras comédias, é uma peça de complexidade maior: uma
burleta, gênero que engloba as melhores obras do autor, como \textit{A
Capital Federal} e \textit{O Mambembe}. \textit{Como eu me diverti!} e
\textit{O Cordão} são interessantíssimas por delinearem a ambiguidade
intelectual do escritor mediante uma crítica bonachona relacionada ao
carnaval. 

\section{Três pequenas comédias}

\paragraph{Amor por anexins}
\textit{Amor por anexins} é a primeira peça de Artur Azevedo, se não
levarmos em consideração uma pecinha escrita aos nove anos de idade,
cujo texto está perdido.\footnote{ Raimundo Magalhães Jr.,
\textit{Artur Azevedo e sua época}, São Paulo, Livros
Irradiantes, 1971.} Tinha ele apenas quinze anos e ainda residia
em São Luís do Maranhão, sua cidade natal, quando empreendeu a primeira
incursão séria na arte dramática (séria no sentido de que elaborou
a comédia para ir à cena, representada por duas atrizes, as irmãs
Riosa). Apenas três anos depois, em 1873, ele se mudaria para o Rio de
Janeiro, para onde se dirigiam praticamente todos os jovens com
aspirações literárias no século \textsc{xix}. Sem esquecer o Maranhão, o
comediógrafo tornou"-se um grande satirista dos costumes da capital,
cenário de quase todas as suas peças. \textit{Amor por anexins}, após
ser apresentada pelas irmãs Riosa “em quase todo o Brasil e até em
Portugal”,\footnote{ Citação de Artur Azevedo, \textit{apud } Antonio Martins Araújo, \textit{Teatro de Artur Azevedo}, Rio de Janeiro, Funarte, 2002, v. 4, p. 59.} ganhou sua primeira edição em livro no
ano de 1879, e continuou a ser encenada como entreato de peças maiores
nas décadas seguintes. 

O sucesso dessa pecinha, cujo enredo se baseia no tantas vezes explorado
tema do enamorado velho e endinheirado que persiste na conquista
amorosa de uma jovem, reside no modo como o autor caracterizou o
solteirão Isaías; o cacoete da personagem, que fala o tempo todo por
meio de anexins, torna o texto não só engraçado como extremamente
original. A habilidade com que o escritor soube enfileirar os
provérbios, posicionando"-os uns após os outros de modo a transmitir de
maneira verossímil as ideias da personagem, demonstra o quanto ele,
ainda tão jovem, possuía pleno domínio do uso da língua, além de um forte
talento para a dramaturgia popular. 

A ideia da peça pode ter surgido da leitura da obra \textit{Metáforas ou
feira dos anexins}, do escritor português Francisco Manuel de Mello.
Apesar de a primeira edição desse livro, póstuma, datar de 1875, existe
a possibilidade de o menino Artur ter tido acesso à obra por intermédio
do pai, o cônsul português David Gonçalves de Azevedo, tendo em vista
que em 1840 já se sabia de sua existência e de sua suposta utilidade,
conforme atesta a seguinte citação:

\begin{hedraquote} 
\mbox{}[\ldots] livro curioso, em que estão lançadas metodicamente as metáforas e
locuções populares da língua portuguesa, e que seria quase um manual
para os escritores dramáticos, principalmente do gênero cômico, que
quisessem fazer falar as suas personagens com frase conveniente, e com
as graças e toque próprio da nossa língua portuguesa e do verdadeiro
estilo dramático.\footnote{ Herculano, \textit{apud }Francisco
Manuel Mello, \textit{Metáforas ou a feira dos anexins}, www.dominiopublico.gov.br/download/texto/ub000005.pdf
(consultado em 08/01/2009.}
\end{hedraquote} 

Mesmo que Artur Azevedo tenha lido a obra e, inclusive, o conselho
transcrito acima, tendo decidido segui"-lo (fato que não passa de mera suposição), isso não diminui a importância de sua proeza, porque o
livro \textit{Feira dos anexins}, afora os anexins, pouco, ou quase
nada, tem de semelhante com a comédia \textit{Amor por anexins}. O
primeiro consiste em duas séries de contos dialogados -- nas quais
se discute o valor e a relevância de anexins de todos os tipos e, para
tanto, os contendores utilizam"-se dos próprios provérbios como recursos
de arrazoamento -- e numa série de anedotas temáticas, também
narradas por meio de trocadilhos, metáforas e ditados populares. 

Já em \textit{Amor por anexins}, o impertinente e tagarela
Isaías invade a casa de sua amada, a viuvinha Inês, disposto a não sair
de lá antes de convencê"-la a se casar com ele. A peça nada tem de
romantismo exacerbado: as duas personagens são movidas por interesses
menos nobres do que o amor sincero (este virá depois). Sem arroubos de
paixão, Isaías, cansado da vida de solteiro, simplesmente decidiu se
casar e escolheu Inês por ela ser boa dona de casa e honesta. A jovem
senhora, por seu turno, também não quer permanecer viúva, mas rejeita
Isaías, num primeiro momento, porque já tem um noivo jovem e bonito --
que também não ama, haja vista a rapidez com que se consola após
receber a carta em que o rapaz rompe o compromisso.  Desdenha,
portanto, a oferta do velho, feio e maçante solteirão enquanto supõe
que tem um partido melhor, mas logo muda de ideia, com medo de
permanecer viúva. A despeito de suas imperfeições, as personagens são
carismáticas e a conversa que entretêm conduz, invariavelmente, ao
riso. 

Além do diálogo fluido e engraçado, a pequena comédia também se
enriquece com os versos, escritos para serem cantados pelas personagens
em momentos especialmente escolhidos. Grande mestre do teatro musicado,
cujos gêneros (a revista, a opereta e a burleta) mesclam momentos
dialogados com canções temáticas, nas quais as personagens revelam seus
sentimentos ao público ou umas às outras, Artur Azevedo mostra, em sua
primeira peça, que o talento para compor versos simples, facilmente
musicáveis, sempre fez parte de sua índole artística. A simplicidade
das estrofes de \textit{Amor por anexins} permitiu que fossem criadas
diferentes partituras para as mesmas no decorrer dos anos, sendo que a
primeira, de Leocádio Raiol, está perdida. 

Esse texto de estreia, portanto, contém inúmeros indícios do caminho a
ser percorrido pelo seu autor: a comicidade, a simplicidade, a crítica
zombeteira, o trabalho com a linguagem popular e os números musicados.

\paragraph{A pele do lobo}
Artur Azevedo escreveu \textit{A pele do lobo} em 1875, mas a peça foi
encenada pela primeira vez em 1877, como complemento da opereta
\textit{Abel, Helena}. Esta última é uma adaptação, do próprio Artur,
de \textit{La belle Helène }(de Offenbach), texto francês de gênero
musicado que fazia bastante sucesso à época. Para a festa em homenagem
ao autor, realizada no dia 10 de abril, a companhia do teatro Fênix
Dramática exibiu a inédita comédia em um ato \textit{A pele do lobo},
junto com a opereta, que já estava em cartaz.

Os textos curtos, em geral, não tinham grande repercussão na imprensa,
porque eram considerados apenas adendos do espetáculo principal.
\textit{A pele do lobo }foi publicada na \textit{Revista do Rio de
Janeiro}, periódico para o qual Artur Azevedo escrevia crônicas e
poemas, no mesmo mês em que estreou no teatro. O texto, uma típica
comédia de costumes, retrata, ao mesmo tempo em que critica, uma das
esferas do meio social da época.

Em \textit{A pele do lobo} a crítica social surge diretamente, o que não
faz essa comédia ser menos engraçada para o leitor/\,espectador de hoje.
Pelo contrário, ela demonstra que, se algumas coisas mudaram, outras
continuam sendo as mesmas. Se escarrar em público é hoje uma prática
inconcebível, uma tremenda falta de educação -- fato que torna ainda
mais cômica a cena em que Jerônimo suja a casa de Cardoso porque “a
casa da autoridade é uma repartição pública” --, a falta de
reconhecimento do trabalho público, de um lado, \mbox{e o} interesse pessoal
por trás daqueles que “metem"-se a servir o país”, de outro, continuam
sendo atuais.

O enredo, por si só, é bastante divertido: a personagem principal, o
subdelegado Cardoso, tenta sair com sua esposa, Amália, para ir a uma
cerimônia de batizado, na qual serão os padrinhos, mas são
interrompidos continuamente pelas pessoas que chegam para “dar parte”
de pequenos crimes ou irregularidades. O casal, cada vez mais nervoso,
só consegue sair após três horas de contrariedades, com direito a
correria e confusão.

Na segunda metade do século \textsc{xix}, homens notáveis na sociedade, ao serem
nomeados pelo poder imperial, por meio dos presidentes das províncias,
exerciam, gratuitamente, as funções de delegado e de subdelegado. Cabia
a eles zelar pela manutenção da ordem pública, bem como desempenhar
funções administrativas e, após 1871, preparar inquéritos policiais. A
indicação envolvia jogos de interesse e politicagem. A rotatividade de
delegados e subdelegados era intensa: poucos se mantinham por muito
tempo nos cargos. 

\begin{hedraquote} 
O serviço não"-remunerado muitas vezes não valia as compensações laterais
que o uso do fitão, o símbolo por excelência da autoridade policial,
proporcionavam. Assim, as queixas e pedidos de desligamento eram
igualmente copiosos. E, apesar da obrigatoriedade do aceite, muitas
declinações eram acolhidas pelo governo.\footnote{ André Rosemberg,
\textit{Polícia, policial e policiamento na província de São Paulo, no
final do Império: a instituição, prática cotidiana e cultura} (tese de
doutorado), São Paulo, USP, 2008, p. 44.}  
\end{hedraquote} 

Como se vê, em \textit{A pele do lobo} Artur Azevedo fez uma justa e
cômica crítica aos costumes policiais do Brasil imperial. Cardoso
solicita uma subdelegacia com vistas a uma futura promoção; trabalha
noite e dia atendendo as “partes”, a ponto de levantar"-se de madrugada
e deixar de sair para um batizado a fim de cumprir o dever; ainda
assim, publicam"-se “descomposturas” sobre ele no jornal. Tanta
dedicação de nada lhe adianta, já que perde a promoção, ao mesmo tempo
em que é demitido da subdelegacia.

A comicidade fica por conta da ingenuidade do caipira Apolinário; da
malandragem de Jerônimo; das saídas falsas do casal, a cada momento
impedido por algo ou alguém; e pelo chavão, repetido constantemente por
Cardoso:

\begin{hedraquote}
\fala{Cardoso} E metam"-se!

\fala{Amélia} Hein?

\fala{Cardoso} E metam"-se a servir o país!
\end{hedraquote}

\paragraph{O Oráculo}
O enredo da comédia \textit{O Oráculo} partiu de um conto próprio,
chamado \textit{Sabina} e publicado no jornal \textit{O País}, em 25 de
março de 1894. Em 1903, Eduardo Vitorino, empresário e dramaturgo
português que trabalhou durante muitos anos no Brasil, pediu a Artur
Azevedo uma peça em um ato para compor o programa de um espetáculo. O
comediógrafo, lembrando"-se do conto, por si só bastante teatral,
escreveu \textit{O Oráculo}, cuja personagem central, Helena, foi
criada especialmente para Georgina Pinto, primeira atriz da companhia.
A atriz, no entanto, faleceu de febre amarela antes da encenação, e
\textit{O Oráculo} ficou à espera de uma nova oportunidade. Esta surgiu
com o espetáculo em benefício do autor, ocorrido no dia 2 de abril de
1907, durante as representações de \textit{O dote}, comédia de Azevedo
que fez imenso sucesso naquele ano. Desta vez, o papel principal coube
à atriz brasileira Guilhermina Rocha. O crítico do jornal \textit{O
País} escreveu, à época, o seguinte comentário sobre o texto:

\begin{hedraquote} 
Simples, de urdidura delicada, facilmente apreensível, \textit{O Oráculo} tem 
as qualidades conhecidas da arte dramática de Artur Azevedo, cuja
penetrante observação, fluência e espontaneidade do diálogo e
propriedade de cena fizeram, há muito, dele o nosso melhor
comediógrafo.\footnote{ “Artes e Artistas. Primeiras Representações”,
em \textit{O País}, 04/04/1907.}
\end{hedraquote} 

O comentário descreve adequadamente as qualidades da peça. O enredo,
simples e curto, não permite um aprofundamento emotivo sobre o tema;
como nas outras comédias curtas do autor, a simplicidade fornece o
ritmo acelerado ao desenvolvimento da fábula, o que não significa que
esta peque por falta de continuidade ou verossimilhança: pelo
contrário, trata"-se de um texto de urdidura bastante coesa. 

A comédia é, em verdade, uma bela transposição para a linguagem cênica
de \textit{Sabina}, conto que narra a história do romance entre o
solteiro Figueiredo e a viúva Sabina (Nélson e Helena na comédia); no
começo do namoro, ela recusa o pedido de casamento, pois acredita que
oficializar a relação esfriaria rapidamente o amor de Figueiredo; após
dois anos (três na peça), no entanto, ele se sente cansado da viúva,
independentemente de ter permanecido solteiro e, querendo pôr fim ao
namoro, pede conselhos ao velho solteirão Matos (Frederico, na comédia),
considerado um oráculo em questões de amor. O amigo aconselha
Figueiredo a acusar Sabina de algum “mau passo”, mesmo sabendo da
honestidade de sua amante, abandonando"-a em seguida sem maiores
explicações. Sabina, porém, usa de esperteza para “segurar” o namorado
e casar"-se com ele: ao receber a acusação, aceita e confessa a falta,
embora não o tenha de fato traído; ele, louco de ciúme, pede"-a em
casamento.

Na peça, além da inclusão da personagem José, empregado de Nélson, de
grande importância para a crítica social e para a comicidade, a única
diferença no enredo consiste no fato de que Helena, escondida, ouve a
conversa entre seu amante e Frederico. Sabendo antecipadamente o
artifício a ser utilizado por Nélson para afastar"-se dela, pode, com
argúcia, inventar um erro no passado para prendê"-lo pelo ciúme. Assim,
a fábula ganha consistência no texto dramático, em que as ações das
personagens se justificam -- podemos afirmar, inclusive, que a comédia é
superior ao conto. 

As informações fornecidas pelo narrador, no conto, são permeadas com
naturalidade na peça entre as falas das personagens, sem dar impressão
de artificialidade. Dois monólogos explicam ao leitor/\,espectador a
situação vivida pelo casal: logo na primeira cena, José, “refestelado
na poltrona com um espanador na mão, a saborear um charuto”, conversa
com o público; vangloria"-se de sua vida ociosa como empregado de
Nélson, já que o patrão vive na casa de sua amante, a “viúva de
Laranjeiras”. O monólogo é espirituoso, espontâneo e engraçado,
características que o impedem de se assemelhar a uma mera muleta
narrativa. Logo depois, Helena entra à procura de Nélson; ela está
nervosa com o descaso demonstrado pelo namorado nos últimos tempos;
sozinha, reflete em voz alta:

\begin{hedraquote} 
\fala{Helena} Não há que ver: está farto de mim! Desfez o encanto! Tudo
acabou. Já o esperava: há muitos meses noto a mudança do seu entusiasmo
de outrora. Melhor seria que nos houvéssemos casado. E dizer que fui eu
que não quis! Dei"-me tão mal com o casamento, que não me sorriu
experimentá"-lo de novo. Era bem independente para não me importar com o
que dissessem. (\textit{Senta"-se e ergue"-se logo em seguida, cada vez
mais agitada}.) Mas não! É impossível que Nélson seja ingrato. Há três
anos pertenço"-lhe, e nunca tive outro amor, nunca pensei em outro
homem.
\end{hedraquote} 

O curto monólogo informa sobre a situação do casal -- o passado, a
honestidade e hesitação de Helena, o desinteresse de Nélson -- sem se
caracterizar como um enfadonho relato. A angústia da personagem, que
fala a si mesma sobre os próprios sentimentos, justifica plenamente o
recurso; não há quebra no ritmo, no andamento da trama que se inicia;
por fim, a transposição de fatos narrados no conto em terceira pessoa
para a linguagem dramática ocorre sem qualquer artificialismo.
\textit{O Oráculo} mantém toda a graciosidade da narrativa, chegando
até a superá"-la, não só porque no texto dramático há a explicação sobre
os motivos que levam Helena a urdir seu estratagema, mas também porque
se nota uma crítica social espontânea e cômica, realizada por meio dos
comentários ferinos e recheados de humor proferidos por José.

A personalidade deste tipo cômico baseia"-se na tradição da comédia.
Desde a Antiguidade, até hoje, o empregado da casa é uma personagem
frequente das comédias, capaz de facilitar a representação dos costumes
e geralmente responsável por vários episódios engraçados do enredo. Não
raro, tal personagem centraliza as ações, conduz o enredo e manipula a
vida das demais personagens (devemos lembrar que esse tipo,
desenvolvido na comédia tradicional francesa, surgiu a partir das
personagens da antiga comédia italiana, a \textit{Commedia dell’Arte},
e pode ser encontrado também em peças clássicas de outros países
europeus). Em \textit{O Oráculo}, a influência da tradição surge de
maneira bem explícita, porque José se parece com os criados originais
da \textit{Commedia dell’Arte} (os chamados “zanni”, que compunham o
grupo de personagens mais populares da \textit{Commedia}, o que explica
sua enorme influência na comédia posterior): ele lembra o esperto e
bufão Arlequim, sempre no centro das intrigas.

Empregado de Nélson, José vive refestelado na poltrona, sem nada fazer,
a fumar os charutos de seu amo; português, quem o trouxe ao Brasil foi
o comendador Frederico, o “oráculo do amor”, mas este não aguentou
conviver com um empregado tão ladino. Numa passagem da peça, Frederico
compara seu antigo criado com personagens clássicas do teatro francês,
revelando abertamente ao leitor/\,espectador o diálogo de Artur Azevedo
com a tradição francesa, que ele conhecia tão bem:

\begin{hedraquote} 
\fala{Frederico} Convenci"-me de que tinhas espírito demais para um simples
criado. Os Scapins\footnote{ Scapin, criado na comédia \textit{Les
fourberies de Scapin} (1671), de Molière.}  e os Frontins\footnote{
Frontin, criado na comédia \textit{Marton et Frontin} (1804), de
Jean"-Baptiste Dubois.} só me agradam na \textit{Comédie} ou no Odeon.
Fora dali acho"-os detestáveis. Entretanto, ao saíres de minha casa,
poderias aspirar a coisa melhor\ldots\ Por que não te arranjaste no
comércio?

\fala{José} Não sou ambicioso\ldots\ Agrada"-me esta situação\ldots\ considero"-me
colocado melhor que o meu amo.
\end{hedraquote} 

Satisfeito com sua posição de criado de um homem solteiro, José não
ambiciona qualquer outro trabalho, em que, certamente, não poderia
passar o dia descansando, a passear pela cidade, a comer e beber do
melhor e a fumar bons charutos. A comicidade do tipo apresenta"-se não
só em sua personalidade alegre e prazenteira, mas também em suas
tiradas irônicas, relativas ao namoro do patrão e ao seu escritório de
advocacia sem clientes: 

\begin{hedraquote} 
\fala{José} O amo nunca está em casa, e eu faço de conta que tudo é nosso.
Permita Deus que tão cedo não acabem os seus amores com a tal viúva das
Laranjeiras.

\fala{Helena} Então algum cliente?

\fala{José} Seria um fenômeno, mas\ldots\ quem sabe? Tudo acontece. Não fizeram a
avenida?\footnote{ Referência à avenida Central (hoje Rio Branco),
recém construída durante as famosas reformas encetadas pelo prefeito
Pereira Passos na virada do século \textsc{xix}--\textsc{xx}. As reformas modificaram
profundamente o centro do Rio de Janeiro.}

\fala{José} Logo vi que Vossa Excelência vinha para ser consultado. Para
consultar ainda está para ser o primeiro que aqui venha.
\end{hedraquote} 

A última fala lhe pertence e fecha com graciosidade a comédia; ao saber
do casamento do patrão, cuja consequência seria uma mulher a cuidar da
casa onde ele trabalha, José se dirige à plateia:

\begin{hedraquote} 
\fala{José} (\textit{À parte}) Ele casa"-se!\ldots\ Adeus, beatitude!\ldots
Sua boa vida está com os dias contados\ldots 
\end{hedraquote} 

Nessa curta peça de costumes, Artur Azevedo demonstra toda a sua
habilidade em criar um enredo engraçado, que prende a atenção do
leitor/\,espectador, e ainda reelabora elementos da tradição literária.
Além disso, critica a ociosidade de maneira extremamente divertida:
Nélson, por ser rico, pode passar as horas a namorar, sem se preocupar
com a falta de clientes e sem ser criticado por causa disso pelas
pessoas de seu meio social; já José, o empregado, aproveita a boa vida
do patrão para, ele próprio, viver entre charutos e passeios. Como
pobre que é, no entanto, precisa fingir que trabalha, e é considerado
um malandro pelas pessoas da alta sociedade. A hipocrisia do pensamento
vigente em nossa sociedade até os dias de hoje apresenta"-se clara no
texto teatral: os ricos podem viver entre os prazeres da ociosidade; já
os pobres, se não trabalham, são considerados vagabundos.

A crítica, nesta comediazinha bem estruturada, surge nas entrelinhas,
entre as ações e diálogos, de maneira natural; a ironia presente nas
falas de José propicia o riso, retrata a sociedade e demonstra o poder
de crítica presente na comicidade. O talento do autor nesse sentido
pode ser apreciado tanto em suas melhores comédias de costumes quanto
nas peças musicadas. Grande observador de pessoas e hábitos, arguto
para tirar proveito máximo de aspectos da sociedade capazes de levar o
espectador ao riso, ele soube transpor para a linguagem cênica o Brasil
em que viveu. 

\section{Duas comédias sobre o carnaval}

\paragraph{O carnaval}

O carnaval, a maior festa popular brasileira, sempre uniu o povo nos
dias de folia. No entanto, no Rio de Janeiro do final do século \textsc{xix},
que ansiava por ser a capital de um país moderno e civilizado, havia um
grande preconceito, por parte da elite intelectual e econômica,
relativo ao carnaval brincado pela população mais pobre. Os
intelectuais consideravam os desfiles e bailes organizados pelos ricos
superiores à alegre desordem vista entre a população em geral. No
desejo de “civilizar” a nação brasileira incluía"-se o intuito de
moderar a “balbúrdia” presente na festa dos pobres. O carnaval ideal,
para os escritores, seria aquele comandado pelas sociedades
carnavalescas ricas. Os Tenentes do Diabo, Democráticos e Fenianos eram
os três grupos mais famosos, compostos por pessoas de poder aquisitivo
alto, que pagavam mensalidades caras durante o ano para desfilar com
fantasias e em carros luxuosos durante o carnaval.

Quem não tinha condições, porém, de participar dos desfiles das
sociedades, ou dos bailes bem"-comportados frequentados pelas famílias
ilustres, não deixava de se divertir: os cordões, ou zé"-pereiras,
compostos pelas pessoas pobres ansiosas pela diversão dos dias de
carnaval, pululavam atrás dos desfiles das ricas sociedades. A situação
nos lembra, hoje, os caros abadás, obrigatórios para os acompanhantes
oficiais dos trios elétricos, e os “pipocas”, que seguem os trios sem
pagar nada. 

A maioria dos literatos, fazendo"-se de porta"-voz das damas e senhores da
alta sociedade, criticava a dança gingada, a música de origem africana,
a bagunça, as brigas, a bebedeira dos cordões, em oposição aos belos
carros alegóricos das sociedades. A mistura das diferentes classes
sociais durante o carnaval expunha a todos a pobreza e o modo de vida
das pessoas marginalizadas. Nesses dias, tornava"-se evidente a todos,
em desfiles nas ruas centrais da capital, as expressões culturais
(música, dança, luta da capoeira) que a intelectualidade procurava
“regenerar”, ou seja, suprimir.\footnote{ Para maiores informações, ver Leonardo Affonso M. Pereira, \textit{Carnaval
das letras}, Rio de Janeiro, Secretaria Municipal de Cultura, 1994, e Nicolau Sevcenko, \textit{Literatura como missão:
tensões sociais e criação cultural na Primeira República}, São Paulo, Companhia das Letras, 2003.}

Artur Azevedo, apesar de concordar em parte com esse posicionamento,
diferenciava"-se dos demais escritores por escrever para um público
popular, para aquelas mesmas pessoas cujos hábitos eram tão criticados.
Ao aproximar"-se do povo através do teatro ligeiro, a sua posição
frente aos costumes considerados “bárbaros” adquire ares extremamente
simpáticos. Sua afinidade com a cultura popular brasileira apresenta"-se
na empatia com que caracterizou o desacordado folião de \textit{Como eu
me diverti!} e o povo da lira, isto é, os componentes do grupo “Foliões
do Itapiru”, da burleta \textit{O Cordão}.

\paragraph{Como eu me diverti!}

O conto"-comédia \textit{Como eu me diverti!}, pela extensão, é o que
hoje se denomina esquete teatral. Se encenado junto com outras peças
curtas de Artur Azevedo, ou de outros autores, pode originar um
espetáculo dinâmico e interessante.\footnote{ Em 2008, o projeto
Clássicos do Teatro, da cidade do Rio de Janeiro, apresentou a leitura
dramatizada de quatro peças de Artur Azevedo: \textit{Amor por
anexins}, \textit{Entre o vermute e a sopa}, \textit{Uma noite em claro
}e \textit{Como eu me diverti!}, realizada pelo grupo Teatro de Roda.}
Tudo indica, porém, que o autor escreveu o texto para ser publicado, e
não encenado, não obstante sua teatralidade -- presente, aliás, em
diversos de seus contos (dialogados ou narrados).

Além de ser engraçada e ter um final inusitado, essa pecinha revela a
ambiguidade do escritor em relação ao carnaval. Um rapaz, após uma
noite de diversão durante o carnaval, retorna, bêbado e passando mal, à
pensão onde mora. Em virtude de seus excessos, acaba perdendo a noiva e
o emprego, porque seu chefe e o padrinho da moça (o médico que, por
coincidência do destino, vem examiná"-lo a pedido da dona da pensão)
descobrem o deslize. O jovem em torno do qual o esquete se desenvolve
tem apenas uma fala, aquela que fecha o enredo. Essa frase -- por,
ao mesmo tempo, encerrar e intitular a obra -- propicia um sorriso
no rosto do leitor/\,espectador e transmite, ainda que nas entrelinhas, a
simpatia do autor pela festa popular. 

A crítica severa dos dois senhores moralistas que, com tanta indignação,
repreendem as loucuras cometidas pelo jovem nos dias de carnaval perde
a força e torna"-se, apenas, cômica, quando o jovem, ao final, entreabre
os olhos, sorri e “exclama com voz rouca e sumida”, sem saber das
dificuldades que o aguardam quando a ressaca passar: “Como eu me
diverti!”. Não há como não simpatizarmos com o protagonista, tão
defendido por Dona Maria, a dona da pensão, até porque são poucos os
brasileiros que nunca cometeram suas pequenas ou grandes loucuras
durante o carnaval. E, por transmitir essa simpatia, o mais provável é
que o escritor também a sentisse.

\paragraph{O Cordão}
A hipótese acima ganha força quando analisamos a burleta \textit{O
Cordão}, encenada durante o carnaval de 1908. As burletas são peças de
um gênero que engloba, num mesmo texto, as diferentes manifestações do
teatro musicado: enredo cômico de temática brasileira, permeado por
números musicais, com muitas personagens e variadas mudanças de
cenário. \textit{O Cordão}, apesar de ser a burleta mais curta e uma
das menos conhecidas de Artur Azevedo, chama a atenção por diversos
motivos: a inclusão, em primeiro plano, de personagens oriundas da
periferia urbana do Rio de Janeiro; a linguagem repleta de gírias e
oralidades, que caracteriza cada personagem; a ambiguidade na abordagem
da temática do carnaval, que, de modo mais saliente que em \textit{Como
eu me diverti!}, demonstra o perfil intelectual de seu autor.

O enredo dessa burleta teve origem em um episódio da revista de ano
\textit{Comeu!}, de 1902. O segundo quadro de \textit{O Cordão}
consiste na transposição quase idêntica do quarto quadro de
\textit{Comeu!}, publicado no jornal \textit{O País}, em 28 de julho de
1904. Artur Azevedo acrescentou alguns episódios, de modo a formar um
enredo coeso: Remígio, um praça reformado do exército, participa
assiduamente dos cordões nos dias de folia; suas filhas, Florinda e
Rosa, educadas por um padrinho, são avessas à festa. Os namorados das
moças, Alfredo e Gastão, funcionários de repartição, desejam afastá"-las
desse ambiente -- considerado, por eles, maléfico à sua “pureza”. O
namoro serve de pretexto para a exposição dos costumes e para a crítica
social, que decorre da diferença nos modos de pensar, agir e falar
entre as personagens instruídas (os namorados e o chefe da repartição
na qual trabalham, chamado de Conselheiro) e as personagens da
periferia (os participantes costumeiros do cordão \textit{Foliões do
Itapiru}, do qual faz parte Remígio).

A partir da oposição existente entre as personagens da burleta, é
possível dividi"-las em dois grupos. No primeiro incluem"-se as
personagens formalmente educadas, com emprego fixo: os dois rapazes,
Gastão e Alfredo; as moças, Florinda e Rosa (educadas por um padrinho
General), e o Conselheiro. Do “grupo do Cordão” fazem parte os
membros assíduos do “Cordão Carnavalesco Foliões do Itapiru”:
Remígio e seus companheiros de folia, pobres e analfabetos. À primeira
vista, sobressai no texto a crítica aos costumes do “grupo do Cordão”,
expressa pelas personagens externas àquele meio -- Alfredo e Gastão, os
rapazes instruídos, empregados, com renda fixa, representam os olhos da
sociedade culta frente aos “bárbaros” do Catumbi. No entanto, um outro
posicionamento transparece à medida que a fábula se desenvolve,
conforme veremos.

Alfredo e Gastão participam de um ensaio do cordão carnavalesco, a fim
de observar o ambiente e retirar suas namoradas dali. As moças também
acham incorreto participar do ensaio, e o fazem apenas porque são
obrigadas pelo pai. Ao chegar à casa de Salustiano, o presidente do
Cordão, os rapazes espantam"-se com as atitudes dos membros do grupo.
Eles bebem vorazmente cachaça pelo gargalo de uma mesma garrafa, narram
brigas em que seus companheiros de folia foram presos, dançam e cantam
em ritmo alegre: 
``Ai ai ai! Eu ai!/
Deixa as cadeira/
Da negra boli!''.

Ao ver Salustiano pela primeira vez (descrito, nas rubricas, como
“pernóstico, pardavasco, grande carapinha, pretensa elegância,
procurando os termos e sibilandos”), Alfredo expressa sua opinião nada
positiva:

\begin{hedraquote} 
\fala{Alfredo} A figura é de um verdadeiro cafajeste.
\end{hedraquote} 

Durante o ensaio, a primeira impressão do rapaz se intensifica; o
contato com os demais participantes do Cordão apenas reforça a opinião
negativa que Alfredo e Gastão têm ao conversar com Salustiano pela
primeira vez. Cada componente do grupo reflete o modo de ser de um
cidadão carioca marginalizado: a veracidade obtida a partir da
caracterização cuidadosa das personagens faz surgir aos olhos do
espectador um quadro vivo e animado; suas falas, seus gestos, o modo
como tratam uns aos outros compõem um quadro autêntico e verossímil de
representação social. A crítica direta revela"-se mediante os olhares e
comentários dos dois rapazes, moradores de uma outra região da cidade,
estranhos à vida de pessoas com as quais dividem as mesmas ruas da
capital. Tal crítica, no entanto, não é endossada pelo autor; em
diferentes momentos da peça podemos vislumbrar um outro ponto de vista.

Após serem recebidos pelo presidente do Cordão, o primeiro componente a
aparecer diante dos olhos dos namorados é Cazuza, um típico capoeira,
cujas atividades na sociedade carioca do fim do século \textsc{xix} faziam as
autoridades tremer. Esses ex"-escravos ou descendentes de escravos
exibiam suas habilidades de capoeiragem pelas ruas durante o ano todo.
Eles provocavam confusões com a polícia e brigas que não raras vezes
terminavam em morte; os capoeiras formavam maltas distintas e competiam
entre si e contra a polícia. 

No início do século \textsc{xx} a perseguição aos capoeiras\label{capoeira} aumentou, porque suas
atitudes figuravam como um símbolo de toda a brasilidade que a elite
desejava esconder e suprimir. No carnaval, a situação se agravava,
porque o clima de festa e desordem tornava o ambiente propício às
exibições e lutas dos capoeiras.\footnote{ Para maiores informações, ver Carlos Eugênio Líbanos Soares, \textit{Festa e
violência: os capoeiras e as festas populares na corte do Rio de
Janeiro (1809–1890)}, em Maria Clementina Pereira Cunha,
\textit{idem}.} Na peça, Cazuza entra em cena
“esbaforido, como que perseguido por alguém”, e assusta Gastão e
Alfredo. Ele narra e encena, através de gírias e golpes, uma confusão
iniciada por causa de uma mulata, na qual a polícia interferiu e alguns
de seus companheiros foram presos, enquanto outros, junto com Cazuza,
conseguiram fugir. Há a inserção, na peça, de personagens
“indesejáveis” da periferia social urbana, inéditas até então no teatro
brasileiro. E a maneira como ocorre essa introdução, simples e ao mesmo
tempo verdadeira, demonstra a capacidade de observação de Artur Azevedo
em relação aos habitantes de sua cidade.

Não apenas verossímil, o episódio delineia o olhar crítico do escritor:
apesar da clara mensagem de aversão ao modo de vida dessas personagens,
expressa ao espectador franca e abertamente por Gastão e Alfredo, a
esfera de alegria e espontaneidade que rodeia os participantes do
Cordão não deixa dúvidas sobre o posicionamento ambíguo -- que deseja se
mostrar contrário à falta de educação dos malandros do Catumbi, mas
apresenta, não obstante, grande simpatia por eles. O resultado final
pende para a festa do povo da lira e contagia o leitor/\,espectador, que
se vê inclinado a compartilhar dos arroubos esfuziantes do “grupo do
Cordão”.

A cena entre Cazuza e os rapazes apresenta imensa graça, devido à reação
dos últimos frente à extroversão do mulato. A comicidade se dá em
consequência do estranhamento causado em Alfredo e Gastão pelas
atitudes do membro de um grupo social que lhes é desconhecido: eles se
sentem completamente desconfortáveis e temerosos perante o modo de
viver e falar dos membros do Cordão; sem naturalidade, não sabem como
agir. As indicações nas rubricas servem para elevar o teor cômico do
episódio: Cazuza praticamente ataca os funcionários de repartição ao
relatar como se desvencilhou da polícia após a briga. Depois da
violenta demonstração, os dois pensam em fugir dali, mas mudam de ideia
quando se lembram das namoradas, que chegarão ao ensaio a qualquer
momento:
\medskip

\begin{hedraquote} 
\fala{Gastão} (\textit{A Alfredo}) Acho prudente irmos embora.

\fala{Alfredo} Deixá"-las com essa cáfila, nunca.
\end{hedraquote} 

Como se vê, a crítica ao “grupo do Cordão” mostra"-se evidente no repúdio
ao modo de vida dos malandros. Os olhos de Alfredo e Gastão são os
olhos dos habitantes cultos frente à grande massa iletrada e pobre
circulante no Rio de Janeiro: os ex"-escravos, os imigrantes sem
dinheiro. Daquele reduto não deveriam fazer parte as moças e os rapazes
de “boa família” -- essa mensagem, explícita no texto, agradaria os
espectadores da elite que porventura assistissem à peça, porque revela
a moral comum à parcela letrada da população. A crítica, porém, não
constitui a mensagem final da burleta, porque a exaltação dos costumes
populares prevalece no enredo, a despeito dos comentários ferinos das
personagens ilustradas.

Apesar das críticas diretas expressas no decorrer do texto por
Alfredo e Gastão, corroboradas, no final do texto, pela autoridade do
Conselheiro, o desfecho, bem como alguns episódios e o tom alegre da
burleta, permitem entrever um outro ponto de vista, que, embora não
seja expresso abertamente, predomina por ser mais envolvente e conduzir
o ritmo da peça. A impressão agradável despertada pela alegria que
envolve o “grupo do Cordão” prevalece sobre qualquer opinião negativa
proferida com todas as letras pelas personagens externas ao grupo.
Ademais, há a defesa explícita do carnaval do cordão, denominado pelas
personagens como “o verdadeiro carnaval carioca”.

A fim de conseguir ser convidado para o ensaio, Alfredo afirma:
``O amigo e eu somos doidos pelo carnaval\ldots\ mas o verdadeiramente
popular, o carnaval bem entendido, o carnaval de cordão''.

O carnaval “bem entendido”, o “verdadeiramente popular”, não seria o
desfile das sociedades, preferido pelos literatos, mas o espontâneo e
festivo zé"-pereira. A confirmação da afirmativa ocorre através das
coplas cantadas por Salustiano, Alfredo e Gastão:

\begin{hedraquote} 
\begin{verse}
\hspace{-1cm}\fala{Salustiano}\\ 
Deixem lá falar quem fala,\\
Pois o melhor carnaval\\
Não é carnaval de sala\\
Nem da avenida Central.\\
O verdadeiro carioca\\
Nascido nesse torrão\\
Por nenhum carnaval troca\\
O do cordão!\\
Os três -- O do cordão!\\
O do cordão!\\
Dão, dão, dão, dão!\\!
\hspace{-1cm}\fala{Salustiano} \\
Ninguém nestes belos dias\\
Se diverte como nós,\\
Que odiamos as sombrias\\
Figuras dos dominós!\\
Carnaval que, como vinho,\\
Torna alegre o coração\\
É o carnaval do povinho,\\
Os três -- O do cordão! 
\end{verse}
\end{hedraquote} 

O verdadeiro carnaval, portanto, não seria aquele dos ricos e bem
produzidos desfiles das sociedades na avenida Central, ou o dos bailes
comportados, frequentados pela elite com suas fantasias de dominós,
arlequins e colombinas. O legítimo carnaval de raiz consistia naquele
em que participava a população mais pobre, livre, alegre, sem regras,
impulsionado pela música de descendência africana: o lundu, o jongo, o
futuro samba -- o originalmente brasileiro, o autêntico carnaval era
aquele do qual Florinda e Rosa não deveriam fazer parte. 

O desfecho da comédia ratifica esse modo de pensar: os recém"-casados
assistem aos desfiles na avenida Central; em certo momento, Alfredo
sugere que eles voltem para a casa, e Florinda responde:
``Decerto. Estas festas não se inventaram para os noivos. Depois, já
vimos passar as sociedades''.

A fala de Florinda revela que o carnaval do povo não foi feito para as
“pessoas de bem”; desse modo, os recém"-casados partem logo depois do
desfile das sociedades ricas. Remígio, convertido em homem de sociedade
pelo casamento das filhas, fica a assistir o carnaval; na última cena,
encontra seu antigo grupo e, sem conseguir negar suas origens, é
arrastado para a folia:

\begin{hedraquote} 
\fala{Salustiano} Olhem, é ele! O nosso incomensurável Remígio! Viva o
Remígio! (\textit{O cordão entra cantando e dançando}.)

\fala{Todos} Viva! Entra! Entra! Fecha! (\textit{Põem Remígio no centro e
dançam todos}.)

\fala{Remígio} Não \textit{arresisto}! Oh, o cordão! O cordão do povo!
(\textit{Dança}.)
\end{hedraquote} 

Após as severas críticas ao cordão realizadas no decorrer do texto, o
desfecho coroa o carnaval do povo; tal posicionamento contrariava as
ideias dos literatos do período, porque estava em franca oposição ao
seu ideal civilizador. Fazendo parte desse grupo de intelectuais e
compartilhando parcialmente de seus posicionamentos, Artur Azevedo
conseguiu demonstrar em sua obra dramática a simpatia pelo popular -- e
é nesse sentido que sua obra ganha relevo. 

As personagens do “grupo do Cordão” representam a grande maioria da
população carioca, pouco presente nos meios literários até então. O
relevo dado às personagens da periferia urbana garante a
verossimilhança externa, filtrada pela comicidade; assim, a burleta se
aproxima dos contos humorísticos e das crônicas, em virtude de sua
simplicidade e veracidade. A comédia de costumes surge aqui em sua
melhor forma: além de criticar os hábitos e os problemas sociais muitas
vezes esquecidos pelos demais gêneros literários, ela nos oferece
diversos indícios sobre a vida de uma época, sem, por isso, deixar de
sobreviver ao tempo, como esta nova edição vem atestar.
