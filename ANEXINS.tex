\chapter[Amor por Anexins]{Amor por Anexins\subtitulo{Entreato cômico}}
\hedramarkboth{Amor por Anexins}{Artur Azevedo}
\medskip

\begin{linenumbers}

\textit{Esta farsa, entremez, entreato, ou que melhor nome tenha
em juízo, o meu primeiro trabalho teatral, foi escrito há mais de
sete anos, no Maranhão, para as meninas Riosa, que a
representaram em quase todo o Brasil e até em Portugal. Pô"-la em
música e em boa música, Leocádio Raiol; mas ultimamente
representaram"-na sem ela Helena Cavalier e Silva Pereira:
desencaminhara"-se a partitura. Tem agora nova música, e não
inferior, de Carlos Cavalier.}

\castpage

\cast{Isaías}{solteirão}

\cast{Inês}{viúva}

\cast{Um Carteiro}

\vfill
A cena passa"-se no Rio de Janeiro.

Época, atualidade.

\pagebreak 
\newactnamed{Ato único}

\stagedir{Sala simples, janela à esquerda, portas ao fundo e à direita. Mesa à esquerda
com preparos de costura. Num dos cantos da sala uma talha
d’água. Cadeiras.}

\newscenenamed{Cena I}

\stagedir{\textsc{Inês}}

\repl{Inês} \paren{Cose sentada à mesa, e olha para a rua, pela janela.}  Lá
está parado à esquina o homem dos anexins! Não há meio de ver"-me livre
de semelhante cáustico. Ora eu, uma viúva, e, de mais a mais com promessa de
casamento, havia de aceitar para marido aquele velho! Não vê! E ninguém o tira
dali! Isto até dá que falar à vizinhança\ldots{} \paren{Desce à boca de cena.}

{\smallskip\raggedleft\itshape Copla\footnote{ Versos a serem cantados pelos atores em cena.}\par}
\begin{verse}
Eu, que gosto, perdido\\
Tenho casamentos mil,\\
Com mais de um belo marido,\\
Garboso, rico e gentil,\\
De um velho agora a proposta,\\
Meu Deus! Devia aceitar?\\
Demais um velho que gosta\\
De assim tão jarreta andar!\\
\quad Nada! Nada!\\
\quad Não me agrada!\\
Quero um marido melhor!\\
É bem mau não ser casada,\\
Mas mal casada é pior.\\
\end{verse}

Ainda hoje escreveu"-me uma cartinha, a terceira em que me fala de amor, e a
segunda em que me pede em casamento. \paren{Tira uma carta da algibeira.}
Ela aqui está. \paren{Lê.} “Minha bela senhora. Estimo que estas duas regras vão
encontrá"-la no gozo da mais perfeita saúde. Eu vou indo como Deus é servido.
Antes assim que amortalhado. Venho pedi"-la em casamento pela segunda vez. Ruim é
quem em ruim conta se tem, e eu que não me tenho nessa conta. Jamais senti por
outra o que sinto pela senhora; mas uma \mbox{vez é} a primeira.” \paren{Declamando.}
Que enfiada de anexins! Pois é o mesmo homem a falar! \paren{Continua a
ler.} “Tenho uns cobres a render; são poucos, é verdade, mas de hora em hora
Deus melhora, e mais tem Deus para dar do que o diabo para levar. Não devo nada a ninguém, e quem não
deve não teme. Tenho boa casa e boa mesa, e onde come um comem dois. Irei saber
da resposta hoje mesmo. Todo seu,
Isaías.” \paren{Guardando a carta.} Está bem aviado, Senhor Isaías!  Vou às
compras; é um excelente meio de me ver livre de vossemecê e de seus
anexins. Vou preparar"-me. \paren{Sai pela porta da direita.  Pausa.}

\newscenenamed{Cena II} 

\stagedir{\textsc{[Isaías]}}

\repl{Isaías} \paren{Deita com precaução a cabeça pela porta do fundo.} Porta
aberta, o justo peca.
\paren{Avançando na ponta dos pés.} A ocasião faz o ladrão. Preciso estudar o
gênio desta mulher: antes que cases, olha o que fazes. Dois gênios iguais não
fazem liga; se a pequena não me sai ao pintar, para cá vem de carrinho. É
preciso olhar para o futuro: quem para adiante não olha atrás fica; quem cospe
para o ar cai"-lhe na cara, e quem boa cama faz nela se deita. Resolvi casar"-me,
mas bem sei que casar não é casaca. Alguém dirá que resolvi um pouco tarde,
porém, mais vale tarde que nunca. Deus ajuda a quem madruga, é verdade; mas nem
por muito madrugar se amanhece mais cedo. Procurei uma mulher como quem procura
ouro. Infeliz até ali! Vi"-as a dar com um pau: bonitas, que era um louvar a Deus
de gatinhas; mas\ldots{} nem tudo o que luz é ouro; feias também que era um Deus
nos acuda; mas muitas vezes donde não se espera daí é que vem. Quem porfia mata
caça dizia com meus botões, e não foi nada, que enquanto o diabo esfrega um
olho, cá a dona encheu"-me\ldots{} o olho. Pois olhem que não me passou camarão
pela malha\ldots{} Esta é viúva e costureira\ldots{} Estou pelo beicinho, e creio
que estou servido. Quem já deu não tem para dar, é certo; mas, ora adeus!
quem muito quer muito perde. Já tomei informações a seu respeito: foram as
melhores possíveis; mas como o saber não ocupa lugar, e mais vale um tolo no seu
que um avisado no alheio, observei"-a. Eu sou como
São Tomé: ver para crer. Vi"-a andar sempre sozinha\ldots{} e nada de pândegas!
Dize"-me com quem andas, dir"-te"-ei as manhas que tens.
\paren{Examinando a casa.} Boa dona"-de"-casa parece ser! Asseio e simplicidade.
Pelo dedo se conhece o gigante. Há de ser o que Deus quiser: o casamento e a
mortalha no céu se talham. \paren{Reparando.} Ai, que ela aí vem!
\paren{Perfilando"-se.} Coragem, Isaías! Lembra"-te de que um homem\ldots{}
\paren{Atrapalhando"-se.} é um gato e um bicho é um homem!
Disse asneira\ldots{}


\newscenenamed{Cena III}

\stagedir{\textsc{Isaías} e \textsc{Inês}}

\repl{Inês} \paren{Vem pronta para sair, ao ver Isaías assusta"-se e quer fugir.} Ai!

\repl{Isaías} \paren{Embargando"-lhe a passagem.}  Ninguém deve correr sem ver
de quê.

\repl{Inês}  Que quer o senhor aqui?

\repl{Isaías}  Vim em pessoa saber da resposta de minha carta: quem quer vai e
quem não quer manda; quem nunca arriscou nunca perdeu nem ganhou; cautela e
caldo de galinha\ldots{}

\repl{Inês} \paren{Interrompendo"-o.} Não tenho resposta alguma que dar!
Saia, senhor!

\repl{Isaías}  Não há carta sem resposta\ldots{}

\repl{Inês} \paren{Correndo à talha e trazendo um púcaro cheio d’água.} Saia,
quando não\ldots{}

\repl{Isaías} \paren{Impassível.}  Se me molhar, mais tempo passarei a seu
lado; não hei de sair molhado à rua. Eh! Eh! Foi buscar lã e saiu
tosquiada\ldots{}

\repl{Inês}  Eu grito!

\repl{Isaías}  Não faça tal! Não seja tola, que quem o é para si \mbox{pede a} Deus
que o mate e ao diabo que o carregue! Não exponha a sua boa reputação! Veja que
sou um rapaz; a um rapaz nada fica mal\ldots{}

\repl{Inês}  O senhor, um rapaz?! O senhor é um velho muito idiota e muito
impertinente!

\repl{Isaías}  O diabo não é tão feio como se pinta\ldots{}

\repl{Inês}  É feio, é!\ldots{}

\repl{Isaías}  Quem o feio ama bonito lhe parece.

\repl{Inês}  Amá"-lo eu?! Nunca\ldots{}

\repl{Isaías}  Ninguém diga: desta água não beberei\ldots{}

\repl{Inês}  É abominável! Irra!

\repl{Isaías}  Água mole em pedra dura, tanto dá\ldots{}

\repl{Inês}  Repugnante!

\repl{Isaías}  Quem espera sempre alcança.

\repl{Inês}  Desengane"-se!

\repl{Isaías}  O futuro a Deus pertence!

\repl{Inês}  Há alguém que me estima deveras\ldots{}

\repl{Isaías}  Esse alguém \paren{naturalmente} sou eu.

\repl{Inês}  Isso era o que faltava! \paren{Suspirando.} Esse alguém\ldots{}

\repl{Isaías}  Quem conta um conto, acrescenta um ponto\ldots{}

\repl{Inês}  Esse alguém é um moço tão bonito\ldots{} de tão boas
qualidades\ldots{}

\repl{Isaías}  Quem elogia a noiva\ldots{}

\repl{Inês}  O senhor forma com ele um verdadeiro contraste.

\repl{Isaías}  Quem desdenha quer comprar\ldots{}

\repl{Inês}  Comprar! Um homem tão feio!\ldots{}

\repl{Isaías}  Feio no corpo, bonito na alma.

\repl{Inês} \paren{Sentando"-se.}  Deus me livre de semelhante marido!

\repl{Isaías}  Presunção e água benta cada qual toma a que quer\ldots{}
\paren{Senta"-se também.}

\repl{Inês} \paren{Erguendo"-se.}  Ah, o senhor senta"-se? Dispõe"-se a ficar! Meu
Deus, isto foi um mal que me entrou pela porta!

\repl{Isaías} \paren{Sempre impassível.}  Há males que vêm para bem.

\repl{Inês}  Temo"-la travada.

\repl{Isaías}  Venha sentar"-se a meu lado. \paren{Vendo que Inês senta"-se longe
dele.} Se não quiser, vou eu\ldots{} \paren{Dispõe"-se a aproximar a cadeira.}

\repl{Inês}  Pois sim! Não se incomode! \paren{Faz"-lhe a vontade.} Não há
remédio!

\repl{Isaías} \paren{Chegando mais a cadeira.}  O que não tem remédio remediado
está.

\repl{Inês} \paren{Afastando a sua.} O que mais deseja?

\repl{Isaías}  Diga"-me cá: o seu noivo? \ldots{} \paren{Faz"-lhe uma cara.}

\repl{Inês}  Não entendo.

\repl{Isaías}  Para bom entendedor meia palavra basta\ldots{}

\repl{Inês}  Mas o senhor nem meia palavra disse!

\repl{Isaías}  Pergunto se\ldots{} fala francês\ldots{}

\repl{Inês}  Como?

\repl{Isaías}  Ora bolas! Quem é surdo não conversa!

\repl{Inês}  Mas a que vem essa pergunta?

\repl{Isaías} \paren{Naturalmente.}  Quem pergunta quer saber.

\repl{Inês}  Ora!

\repl{Isaías} \paren{Sentencioso.}  Dois sacos vazios não se podem ter de pé.

\repl{Inês}  Essa teoria parece"-se muito com o senhor.

\repl{Isaías}  Por quê?

\repl{Inês}  Porque já caducou também.

\repl{Isaías} \paren{Formalizado.}  Então eu já caduquei, menina?  Isso é
mentira.

\repl{Inês}  É verdade.

\repl{Isaías}  Não é.

\repl{Inês}  É.

\repl{Isaías}  Pois se é, nem todas as verdades se dizem.  \paren{Ergue"-se e
passeia.}

\repl{Inês}  Ah! O senhor zanga"-se? É porque quer; não me viesse dizer tolices!
\paren{Ergue"-se.}

\repl{Isaías} \paren{Interrompendo o seu passeio, solenemente.} Na casa em
que não há pão, todos ralham, ninguém tem razão.

\repl{Inês}  Ora! Somos ainda muito moços!

\repl{Isaías}  Quem? Nós?

\repl{Inês} \paren{De mau humor.}  Não falo do senhor: falo dele\ldots{}

\repl{Isaías}  Ah! Fala dele\ldots{}

\repl{Inês}  Havemos de trabalhar um para o outro\ldots{}

\repl{Isaías}  É bom, é: Deus ajuda a quem trabalha.

{\smallskip\raggedleft\itshape Canto\par}
\begin{verse}
\fala{Inês}\\
Sem desgosto viveremos,\\
Seremos ricos, talvez;\\
Muitos morgados teremos\ldots{}

%\vfil\pagebreak
\fala{Isaías}\\
Mas um só de cada vez\ldots{}\\
\paren{Zangado.} A faceira\\
Talvez convidar"-me queira\\
Para padrinho de algum!
\end{verse}

\repl{Inês}  E não suponha que, apesar de pobre, não me faça bonitos presentes
o meu noivo.

\repl{Isaías}  É! Quem cabras não tem e cabritos\ldots{}

\repl{Inês}  Insulta"-o?

\repl{Isaías}  Cão danado, todos a ele! Pois eu havia de insultá"-lo, senhora?

\repl{Inês}  Se estivesse calado\ldots{}

\repl{Isaías}  Sim, senhora: em boca fechada não entram mosquitos\ldots{} Mas é
que o seu futurozinho me interessa\ldots{}

\repl{Inês}  Muito obrigada. \paren{Senta"-se.}

\repl{Isaías}  Não há de quê. Se bem que eu não seja nenhum Matusalém, estou no
caso de lhe dar conselhos. Ouça"-me; quem me avisa meu amigo é; quem à boa árvore
se chega boa sombra o cobre.

\repl{Inês}  Mesmo por já estar no caso de me dar conselhos, é que o não quero
para marido.

\repl{Isaías}  Se eu fosse jovem, não me havia de aceitar, por estar no caso de
os receber. Preso por ter cão e preso por não ter!\ldots{}

\repl{Inês}  Não desejo enviuvar de novo\ldots{}

\repl{Isaías}  Vaso ruim não quebra\ldots{}

\repl{Inês}  Desengana"-se, senhor: não são os seus ditados que me hão de fazer
mudar de resolução! \paren{Passeia.} Oh!

\repl{Isaías} \paren{Acompanhando"-a.} Talvez façam, talvez!\ldots{} Devagar se
vai ao longe\ldots{} Muito tolo é quem se cansa\ldots{} \paren{Inês volta"-se, param defronte um do outro.} Menina,
antes só do que mal acompanhado\ldots{} Olhe que o pior cego é aquele que não
quer ver\ldots{}

\repl{Inês} \paren{À parte.}  Vou pregar"-lhe uma peta.  \paren{Alto.} Mas se me
faltasse esse noivo, outros rapazes há que me têm feito pé"-de"-alferes.

\repl{Isaías}  Águas passadas não movem moinhos!

\repl{Inês}  E entre eles\ldots{}

\repl{Isaías} O passado! passado!

\repl{Inês}  Não me interrompa!\ldots{} E entre eles há um ricaço que em outro
tempo\ldots{}

\repl{Isaías} O tempo que vai não volta!

\repl{Inês}  Não me interrompa, já disse! E entre eles há um ricaço que noutro
tempo se esqueceu da promessa\ldots{}

\repl{Isaías}  O prometido é devido!

\repl{Inês}  Ai, mau!\ldots{} se esqueceu da promessa que me havia feito; mas
que está outra vez pelo beicinho\ldots{}

\repl{Isaías}  Cesteiro que faz um cesto faz um cento\ldots{} \paren{Movimento
de Inês. Com força.} Se tiver verga e tempo! E quem é esse\ldots{} ricaço?

\repl{Inês}  É segredo.

\repl{Isaías}  Segredo em boca de mulher é manteiga em nariz\ldots{} \paren{A um
gesto de Inês} de homem! Mas faz bem, faz bem: o segredo é a alma do
negócio\ldots{}

\repl{Inês}  O senhor tem na cabeça um moinho de adágios! Passa!\ldots{}

\repl{Isaías}  O que abunda não prejudica.

\repl{Inês}  Bem! Para maçadas basta. Mude"-se!

\repl{Isaías}  Os incomodados é que se mudam.

\repl{Inês}  Mas eu estou em minha casa, senhor!

\repl{Isaías}  Descobriu mel de pau!

\repl{Inês}  Irra! Que homem sem"-vergonha!

\repl{Isaías} \paren{Examinando cinicamente a costura.}  Quem não tem vergonha
todo o mundo é seu.

\repl{Inês}  Se o meu noivo o visse aqui! Ele, que jurou dar cabo do primeiro
rival que\ldots{}

\repl{Isaías}  Cão que ladra não morde\ldots{} E eu sou homem!\ldots{} tenho
força\ldots{} E contra a força não há resistência!\ldots{}

\repl{Inês} \paren{Irônica.}  Ora, por quem é, não faça mal ao pobre moço, sim?

\repl{Isaías}  Faço!\ldots{} Quem o seu inimigo poupa às mãos lhe morre.  Julga
que não estou falando sério? Uma coisa é ver a outra\ldots{}

\repl{Inês} \paren{No mesmo.}  Ora não faça tal.

\repl{Isaías}  Faço! Isto é tão certo como dois e três serem cinco. São favas
contadas.
Quem não quiser ser lobo não lhe vista a pele!

\repl{Inês}  Mas sabe que ele é valente?

\repl{Isaías}  Também eu sou! Cá e lá más fadas há! Duro com duro não faz bom
muro, e dois bicudos não se beijam!

\repl{Inês}  Ponha"-se ao fresco, preciso sair; tenho que fazer lá fora.

\repl{Isaías}  E eu tenho que fazer cá dentro. Um dia bom mete"-se em casa.
\paren{Pausa.} Olhe, senhora, olhe bem para mim, acha"-me feio: não acha?

\repl{Inês}  Ai, ai, ai!\ldots{}

\repl{Isaías}  Eu também acho, e feliz é o doente que se conhece. Mas muitas
vezes as aparências enganam e o hábito não faz o monge. Experimente e verá.
\paren{Suplicante.} Case comigo.

\repl{Inês}  Gentes!

\repl{Isaías}  Ah! se fôssemos casadinhos, outro galo cantaria! Por exemplo: em
vez de sair agora à rua, com este sol de matar passarinho, mandava"-me a mim, ao
seu maridinho\ldots{}

\repl{Inês} \paren{Arremedando"-o.} Ao seu maridinho\ldots{} \paren{À parte.}
Oh! Que ideia! Vou me ver livre dele. \paren{Alto.} Então, sem sermos casados,
não pode prestar"-me um pequeno serviço?

\repl{Isaías}  Conforme o serviço: ponha os pontos nos \textit{ii}.

\repl{Inês}  Se me fosse comprar três metros de escumilha. Olhe\ldots{} Aqui tem
a amostra\ldots{} No armarinho do Godinho\ldots{} Sabe onde é?

\repl{Isaías}  Sei; mas quando não soubesse? Quem tem boca vai a Roma.

\repl{Inês}  Está contrariado?

\repl{Isaías}  O que vai por gosto regala a vida.

\repl{Inês}  Tome o dinheiro.

\repl{Isaías}  Nada\ldots{} não é preciso\ldots{} \paren{Vai saindo e estaca.}
Diabo! não me lembra um ditado a propósito! \paren{Sai.}

\newscenenamed{Cena IV} 

\stagedir{\textsc{[Inês]}}

\repl{Inês}  Está bem aviado\ldots{} Quando voltares, hás de achar a porta
fechada. 
Safa! Que maçador! Agora, tratemos de sair: são mais que horas. \paren{Aparece à
porta um carteiro.}

\newscenenamed{Cena V}

\stagedir{\textsc{Inês, o Carteiro}}

\repl{O Carteiro}  Boa tarde, minha senhora.

\repl{Inês}  Boa tarde. O que deseja?

\repl{O Carteiro}  Aqui tem esta carta\ldots{} é da caixa urbana\ldots{}

\repl{Inês}  Uma carta? \paren{Recebendo a carta, consigo.} De quem será?
\paren{Ao carteiro.} Obrigada.

\repl{O Carteiro}  Não há de quê, minha senhora. Passe muito bem!

\repl{Inês} Adeus. \paren{O Carteiro sai.}

\newscenenamed{Cena VI} 

\stagedir{\textsc{[Inês]}}

\repl{Inês}  Ah! A letra é de Filipe. Faz bem em escrever"-me o ingrato! Há doze
dias que nos não vemos\ldots{} \paren{Abre a carta e lê. Jogo de fisionomia.}
“Inês.  Peço"-te perdão por ter dado causa a que perdesses comigo o teu tempo.
Ofereceram"-me um casamento vantajoso, e não soube recusar. Ainda uma vez perdão!
Falta"-me o ânimo para dizer"-te mais alguma coisa.
Dentro em uma semana estarei casado. Esquece"-te de mim --- Filipe.”
\paren{Declamando.} Será possível! Oh! Meu Deus! \paren{Relendo.} Sim\ldots{} cá
está\ldots{} é a sua letra\ldots{} \paren{Depois de ter ficado pensativa um
momento.} Ora, adeus. Eu também não gostava dele lá essas coisas\ldots{} Digo
mais, antes o
Isaías; é mais velho, mais sensato, tem dinheiro a render, e Filipe acaba de me
provar que o dinheiro é tudo nestes tempos. Espero aqui o Isaías com o meu “sim”
perfeitamente engatilhado! Oh! O dinheiro\ldots{}

\begin{verse} 
Louro dinheiro, soberano esplêndido,\\*
Força, Direito, Rei dos reis, Razão.\\*
Que ao trono teu auriluzente e fúlgido\\*
Meus pobres hinos proclamar"-te vão.\\! 

Do teu poder universal, enérgico,\\*
Ninguém se atreve a duvidar! Ninguém!\\*
Rígida mola desta imensa máquina,\\*
Fácil conduto para o eterno bem!\\!

Aos teus acenos, Deus antigo e déspota,\\*
Aos teus acenos, Deus moderno e bom,\\*
Caem virtudes e se exaltam vícios!\\*
Todos te almejam precioso dom!\\!

Inda hás de ser o derradeiro ídolo,\\
Inda hás de ser a só religião,\\
Louro dinheiro, soberano esplêndido,\\
Força, Dinheiro, Rei dos reis, Razão!\ldots{}
\end{verse}

\newscenenamed{Cena VII}

\stagedir{\textsc{Inês, Isaías}}

\repl{Isaías} \paren{Entrando.}  Quem canta seus males espanta.

\repl{Inês}  Já de volta! O senhor foi a correr!

\repl{Isaías}  Nada! Quem corre cansa. Encontrei outro armarinho mais
perto\ldots{}

\repl{Inês} \paren{Tomando a fazenda.}  Muito obrigada. Quanto custou?

\repl{Isaías}  Um pau por um olho. Mil e duzentos o metro\ldots{}

\repl{Inês}  Pois olhe: o outro vende mais barato.

\repl{Isaías}  O barato sai caro, e mais vale um gosto do que quatro vinténs.

\repl{Inês}  Regateou?

\repl{Isaías}  Regatear! Para quê? Mais tem Deus para dar do que o diabo para
tomar.

\repl{Inês}  Já vejo que é tão pródigo de dinheiro como de anexins!

\repl{Isaías}  Da pataca do sovina o diabo tem três tostões e dez réis. Poupado
sim, sovina não.
Eu cá sou assim! Nem tanto ao mar nem tanto à terra. Tenho um só defeito: quero
casar"-me. Cada louco com sua mania.

{\smallskip\raggedleft\itshape Canto\par}
\begin{verse} 
Há sido um gato sapato;\\
Preciso do casamento!\\
O maldito celibato\\
Não é viver, é tormento.

Quero honesta rapariga\\
Entre as belas procurar,\\
Muito embora o mundo diga:\\
Quem já andou não tem pra andar\ldots{}

A existência de casado\\
Talvez venturas me traga,\\
Se diz verdade o ditado:\\
Amor com amor se paga.

Se eu for constante e fervente,\\
Ela tudo isso será;\\
Se eu amá"-la eternamente,\\
Ela também me amará!

Eu escravo e a esposa escrava,\\
Viveremos sem desgosto;\\
Uma mão a outra lava\\
E ambas lavam o rosto!\ldots{}
\end{verse}

Faço"-lhe pela milésima vez o meu pedido. Nem todos os dias há carne gorda. A
senhora falou"-me em um apaixonado. Por onde andará ele? Eu estou aqui, e mais
vale um pássaro na mão do que dois a voar.

\repl{Inês} \paren{À parte.}  Levemos a coisa com jeito.  \paren{Alto.} O
senhor\ldots{} \paren{Com uma ideia.} Ah!

\repl{Isaías}  Oh!

\repl{Inês}  Já viu representar \textit{As pragas do Capitão}?

\repl{Isaías}  Não, senhora. De pragas ando eu farto.

\repl{Inês}  Era um militar que praguejava muito. A senhora que ele amava deu"-lhe a mão de esposa, mas depois de estabelecer"-lhe a condição de não praguejar
durante meia hora.

\repl{Isaías}  Falo em alhos, a senhora responde com bugalhos!

\repl{Inês}  Já lá vamos aos alhos: aceito a sua proposta.

\repl{Isaías} \paren{Impetuosamente.}  Aceita?

\repl{Inês}  Sim, senhor.

\repl{Isaías} \paren{Incrédulo.}  Qual! Quando a esmola é muita, o pobre
desconfia\ldots{}

\repl{Inês}  Mas imponho também a minha condição\ldots{}

\repl{Isaías}  Imponha: manda quem pode.

\repl{Inês}  Se conseguir levar meia hora sem\ldots{}

\repl{Isaías}  Sem praguejar?\ldots{}

\repl{Inês}  Não! Sem dizer um anexim! Se conseguir, é sua a minha mão.

\repl{Isaías}  Deveras?

\repl{Inês} \paren{Sentando"-se.}  Deveras.

\repl{Isaías}  Mas eu posso estar calado?

\repl{Inês}  Como assim?! Era o que faltava! Há de falar pelos cotovelos!

\repl{Isaías}  Isso é um pouco difícil: o costume faz lei\ldots{}

\repl{Inês}  Ai, que escapou"-lhe um!

\repl{Isaías}  Pois o que quer? A continuação do cachimbo\ldots{}

\repl{Inês}  Faz a boca torta, já duas vezes.

\repl{Isaías}  Nas três o diabo as fez.

\repl{Inês}  Ai, ai, ai! Vamos muito mal!

\repl{Isaías}  Mas não tínhamos ainda entrado em campo\ldots{} Aqueles foram
ditos de propósito. Agora sim! Agora é que são elas!

\repl{Inês}  Outro!

\repl{Isaías}  Protesto! “Agora é que são elas” nunca foi anexim. A César o que
é de César!

\repl{Inês}  O senhor vai perder\ldots{} Olhe: são duas horas. \paren{Aponta para um
relógio que deve estar sobre a mesa.} Aceita o desafio? \paren{Pausa.} Bem. Quem
cala consente\ldots{}

\repl{Isaías}  Ah! Agora é a senhora quem os diz! Virou"-se o feitiço contra o
feiticeiro\ldots{}

\repl{Inês}  Ai, ai!

\repl{Isaías}  Foi engano.

\repl{Inês}  Dos enganos comem os escrivães. \paren{Pausa.} Então?  Diga alguma
coisa\ldots{}

\repl{Isaías}  O que hei de dizer\ldots{} senão\ldots{} que gosto muito da
senhora\ldots{} e\ldots{}

\repl{Inês}  Pois diga: vai tantas vezes o cântaro à fonte, que lá fica.

\repl{Isaías}  Não me provoque, senhora, não me provoque!

\repl{Inês}  Cada qual puxa a brasa para sua sardinha\ldots{}

\repl{Isaías} \paren{Agitado.}  Brasa! Sardinha! Oh! que suplício!

\repl{Inês}  O que tem o senhor?

\repl{Isaías}  Nada\ldots{} não tenho nada\ldots{} é que esta proibição me
incomoda\ldots{} Este maldito costume\ldots{} parece que não estou em mim\ldots{}

\repl{Inês}  Sabe o que mais?

\repl{Isaías}  Vou saber.

%Jorge: conf. esta fala (o ? final) com o original (mudamos!)
\repl{Inês}  Diga o que quiser! Abra a torneira dos anexins, ditados,
rifões, sentenças, adágios e provérbios\ldots{} Fale, fale para aí!

\repl{Isaías}  E a condição?

\repl{Inês}  Caducou. \paren{Dando"-lhe a mão.} Aqui tem: sou sua.

\repl{Isaías} \paren{Contente.}  Minha! \paren{Em outro tom.} E os outros?

\repl{Inês}  Não existem, nunca existiram!

\repl{Isaías}  Pois estou acordado? Se estiver dormindo, deixa"-me estar: não me
acordes.

\repl{Inês}  Está bem acordado.

\repl{Isaías}  Estou?! \paren{Pulando de contente.} Então viva Deus! Viva o
prazer!\ldots{} Trá lá lá rá lá! \paren{Quer abraçá"-la.}

\repl{Inês} \paren{Gritando.}  Alto lá! Mais amor e menor confiança!

\repl{Isaías}  É que o rato nunca comeu mel, quando come\ldots{} \paren{Outro tom.}
Pode"-se dizer este ditadozinho?\ldots{}

\repl{Inês}  Quantos quiser!

\repl{Isaías} \paren{Concluindo.}  \ldots{} se lambuza!  \paren{Tomando"-lhe as
mãos.} E tu? Amas"-me, meu bem?

\repl{Inês}  Sossegue: o amor virá depois. Seja bom marido e deixe o
barco andar!

\repl{Isaías}  Apoiado. Roma não se fez num dia!

\repl{Inês}  E tenha sempre muita fé nos seus anexins.

\repl{Isaías}  É verdade: o que tem de ser tem muita força. O homem põe\ldots{}
e a mulher dispõe!\ldots{}

\repl{Inês}  Basta! Despeça"-se destes senhores, e vá tratar dos papéis\ldots{}

\repl{Isaías}  Quem tem boca não manda\ldots{} cantar. Mas, enfim\ldots{} 
\smallskip

\paren{Ao público.} 

{\smallskip\raggedleft\itshape Copla final\par}
\begin{verse} 
Antes que daqui nos vamos,\\
Inês vos dirá quais são\\
Os votos que alimentamos\\
No fundo do coração.  
\end{verse}

\begin{verse}
\fala{Inês}\\
Os votos que neste instante\\
Fazemos nestes confins\\
\paren{Deita a mão sobre o coração.}\\
É que nos ameis bastante\\
Embora por anexins.

\fala{Ambos}\\
Muitas palmas esperamos\\
\quad De vós:\\
Metade para o autor, metade para nós.
\end{verse}

\vspace{1cm}

\begin{center}
\textsc{Cai o pano}
\end{center}

\end{linenumbers}

